\documentclass[11pt,a4paper]{article}
\usepackage[a4paper, margin=0.8in]{geometry}
%\usepackage{article}
\usepackage{amsmath}
\usepackage{amssymb}
\usepackage{amsthm}
\usepackage{amsfonts}
\usepackage{natbib}
\usepackage{wrapfig}
\usepackage{graphicx}
\usepackage{url}
\usepackage{hyperref}
\usepackage{pgf}
\hypersetup{
	colorlinks,
	citecolor=black,
	filecolor=black,
	linkcolor=black,
	urlcolor=black
}

\makeatletter
\def\@maketitle{%
	\newpage
	\null
	\vskip 2em%
	\begin{center}%
		\let \footnote \thanks
		{\LARGE \@title \par}%
		\vskip 1em%
		{\Large Fast Iterative Solvers, Project 1\par}%
		\vskip 1.5em%
		{\large
			\lineskip .5em%
			\begin{tabular}[t]{c}%
				\@author
			\end{tabular}\par}%
		\vskip 1em%
		{\large \@date}%
	\end{center}%
	\par
	\vskip 1.5em}
\makeatother

\title{Krylov Subspace Methods}
\author{Johannes Leonard Grafen}
\makeindex

\ifpdf  
\hypersetup{pdfauthor={Johannes Leonard Grafen},%
	pdftitle={Fast Iterative Solvers, Project 1},%
}
\fi

%new commands
\newcommand{\refFig}[1]{Fig. \ref{#1}}
\newcommand{\refEq}[1]{Eq. (\ref{#1})}
\newcommand{\refTab}[1]{Tab. \ref{#1}}

\begin{document}
\renewcommand\baselinestretch{1.0}
\baselineskip=18pt plus1pt	
	
\maketitle
\tableofcontents
\listoffigures	% print list of figures
\listoftables  % print list of tables

\section{Remarks on used architecture and compiler options}
The code was compiled using Clang-compiler and ran on an Apple Silicon M1 Pro Chip with ARM64 architecture. All timings were conducted with the a CPU-timer object of the Boost timer library, which was specifically compiled for the previously mentioned CPU architecture. If not stated otherwise, no optimization using -O0 flag was used. For timings, the output was disabled at specific regions of the code, using a specific preprocessor directive "DISABLEIO" which can be passed to the compiler using -D option. 

\section{Generalized Minimal Residual Method - GMRES}
\label{chapter:GMRES}

\subsection{Relative residuals for GMRES with and without preconditioning}

For the full GMRES method, the relative residuals are plotted against the iterations index of the top loop in the full GMRES algorithm in \refFig{fig::Residuals}. The iterations index in this case corresponds to the Krylov vectors. The number of necessary Krylov vectors to reduce the 2-norm of the initial residual by 8 orders of magnitude ($||\mathbf{r}_k|| / ||\mathbf{r}_0|| = 10^{-8}$) is given by $\tilde{m}$ in the plot's legend. All iterations were started with parameter $m = 600$ in case the convergence criterion was fulfilled before reaching the prescribed max number of Krylov vectors, the loop was exited prematurely with $\tilde{m}$. Compared to the full GMRES method without preconditioning ($\tilde{m} = 479$ ) the Gauss-Seidel preconditioning performed best with just $\tilde{m} = 143$ Krylov vectors that are required to established convergence (according to the aforementioned convergence criterion). I obtained $\tilde{m} = 256$ required Krylov vectors for Jacobi preconditioning and $\tilde{m} = 465$ Krylov vectors for Incomplete LU Factorization (ILU(0)) to reach convergence.
%
\begin{figure}[!htbp]
	\centering
	\hspace*{0.8cm}
	\leavevmode
	\resizebox{0.9\width}{!}{%% Creator: Matplotlib, PGF backend
%%
%% To include the figure in your LaTeX document, write
%%   \input{<filename>.pgf}
%%
%% Make sure the required packages are loaded in your preamble
%%   \usepackage{pgf}
%%
%% Also ensure that all the required font packages are loaded; for instance,
%% the lmodern package is sometimes necessary when using math font.
%%   \usepackage{lmodern}
%%
%% Figures using additional raster images can only be included by \input if
%% they are in the same directory as the main LaTeX file. For loading figures
%% from other directories you can use the `import` package
%%   \usepackage{import}
%%
%% and then include the figures with
%%   \import{<path to file>}{<filename>.pgf}
%%
%% Matplotlib used the following preamble
%%   
%%   \makeatletter\@ifpackageloaded{underscore}{}{\usepackage[strings]{underscore}}\makeatother
%%
\begingroup%
\makeatletter%
\begin{pgfpicture}%
\pgfpathrectangle{\pgfpointorigin}{\pgfqpoint{6.565064in}{4.725328in}}%
\pgfusepath{use as bounding box, clip}%
\begin{pgfscope}%
\pgfsetbuttcap%
\pgfsetmiterjoin%
\definecolor{currentfill}{rgb}{1.000000,1.000000,1.000000}%
\pgfsetfillcolor{currentfill}%
\pgfsetlinewidth{0.000000pt}%
\definecolor{currentstroke}{rgb}{1.000000,1.000000,1.000000}%
\pgfsetstrokecolor{currentstroke}%
\pgfsetdash{}{0pt}%
\pgfpathmoveto{\pgfqpoint{0.000000in}{0.000000in}}%
\pgfpathlineto{\pgfqpoint{6.565064in}{0.000000in}}%
\pgfpathlineto{\pgfqpoint{6.565064in}{4.725328in}}%
\pgfpathlineto{\pgfqpoint{0.000000in}{4.725328in}}%
\pgfpathlineto{\pgfqpoint{0.000000in}{0.000000in}}%
\pgfpathclose%
\pgfusepath{fill}%
\end{pgfscope}%
\begin{pgfscope}%
\pgfsetbuttcap%
\pgfsetmiterjoin%
\definecolor{currentfill}{rgb}{1.000000,1.000000,1.000000}%
\pgfsetfillcolor{currentfill}%
\pgfsetlinewidth{0.000000pt}%
\definecolor{currentstroke}{rgb}{0.000000,0.000000,0.000000}%
\pgfsetstrokecolor{currentstroke}%
\pgfsetstrokeopacity{0.000000}%
\pgfsetdash{}{0pt}%
\pgfpathmoveto{\pgfqpoint{0.977263in}{0.549691in}}%
\pgfpathlineto{\pgfqpoint{6.342364in}{0.549691in}}%
\pgfpathlineto{\pgfqpoint{6.342364in}{4.575328in}}%
\pgfpathlineto{\pgfqpoint{0.977263in}{4.575328in}}%
\pgfpathlineto{\pgfqpoint{0.977263in}{0.549691in}}%
\pgfpathclose%
\pgfusepath{fill}%
\end{pgfscope}%
\begin{pgfscope}%
\pgfpathrectangle{\pgfqpoint{0.977263in}{0.549691in}}{\pgfqpoint{5.365101in}{4.025637in}}%
\pgfusepath{clip}%
\pgfsetrectcap%
\pgfsetroundjoin%
\pgfsetlinewidth{0.803000pt}%
\definecolor{currentstroke}{rgb}{0.690196,0.690196,0.690196}%
\pgfsetstrokecolor{currentstroke}%
\pgfsetdash{}{0pt}%
\pgfpathmoveto{\pgfqpoint{0.977263in}{0.549691in}}%
\pgfpathlineto{\pgfqpoint{0.977263in}{4.575328in}}%
\pgfusepath{stroke}%
\end{pgfscope}%
\begin{pgfscope}%
\pgfsetbuttcap%
\pgfsetroundjoin%
\definecolor{currentfill}{rgb}{0.000000,0.000000,0.000000}%
\pgfsetfillcolor{currentfill}%
\pgfsetlinewidth{0.803000pt}%
\definecolor{currentstroke}{rgb}{0.000000,0.000000,0.000000}%
\pgfsetstrokecolor{currentstroke}%
\pgfsetdash{}{0pt}%
\pgfsys@defobject{currentmarker}{\pgfqpoint{0.000000in}{-0.048611in}}{\pgfqpoint{0.000000in}{0.000000in}}{%
\pgfpathmoveto{\pgfqpoint{0.000000in}{0.000000in}}%
\pgfpathlineto{\pgfqpoint{0.000000in}{-0.048611in}}%
\pgfusepath{stroke,fill}%
}%
\begin{pgfscope}%
\pgfsys@transformshift{0.977263in}{0.549691in}%
\pgfsys@useobject{currentmarker}{}%
\end{pgfscope}%
\end{pgfscope}%
\begin{pgfscope}%
\definecolor{textcolor}{rgb}{0.000000,0.000000,0.000000}%
\pgfsetstrokecolor{textcolor}%
\pgfsetfillcolor{textcolor}%
\pgftext[x=0.977263in,y=0.452469in,,top]{\color{textcolor}\rmfamily\fontsize{10.000000}{12.000000}\selectfont \(\displaystyle {0}\)}%
\end{pgfscope}%
\begin{pgfscope}%
\pgfpathrectangle{\pgfqpoint{0.977263in}{0.549691in}}{\pgfqpoint{5.365101in}{4.025637in}}%
\pgfusepath{clip}%
\pgfsetrectcap%
\pgfsetroundjoin%
\pgfsetlinewidth{0.803000pt}%
\definecolor{currentstroke}{rgb}{0.690196,0.690196,0.690196}%
\pgfsetstrokecolor{currentstroke}%
\pgfsetdash{}{0pt}%
\pgfpathmoveto{\pgfqpoint{2.043990in}{0.549691in}}%
\pgfpathlineto{\pgfqpoint{2.043990in}{4.575328in}}%
\pgfusepath{stroke}%
\end{pgfscope}%
\begin{pgfscope}%
\pgfsetbuttcap%
\pgfsetroundjoin%
\definecolor{currentfill}{rgb}{0.000000,0.000000,0.000000}%
\pgfsetfillcolor{currentfill}%
\pgfsetlinewidth{0.803000pt}%
\definecolor{currentstroke}{rgb}{0.000000,0.000000,0.000000}%
\pgfsetstrokecolor{currentstroke}%
\pgfsetdash{}{0pt}%
\pgfsys@defobject{currentmarker}{\pgfqpoint{0.000000in}{-0.048611in}}{\pgfqpoint{0.000000in}{0.000000in}}{%
\pgfpathmoveto{\pgfqpoint{0.000000in}{0.000000in}}%
\pgfpathlineto{\pgfqpoint{0.000000in}{-0.048611in}}%
\pgfusepath{stroke,fill}%
}%
\begin{pgfscope}%
\pgfsys@transformshift{2.043990in}{0.549691in}%
\pgfsys@useobject{currentmarker}{}%
\end{pgfscope}%
\end{pgfscope}%
\begin{pgfscope}%
\definecolor{textcolor}{rgb}{0.000000,0.000000,0.000000}%
\pgfsetstrokecolor{textcolor}%
\pgfsetfillcolor{textcolor}%
\pgftext[x=2.043990in,y=0.452469in,,top]{\color{textcolor}\rmfamily\fontsize{10.000000}{12.000000}\selectfont \(\displaystyle {100}\)}%
\end{pgfscope}%
\begin{pgfscope}%
\pgfpathrectangle{\pgfqpoint{0.977263in}{0.549691in}}{\pgfqpoint{5.365101in}{4.025637in}}%
\pgfusepath{clip}%
\pgfsetrectcap%
\pgfsetroundjoin%
\pgfsetlinewidth{0.803000pt}%
\definecolor{currentstroke}{rgb}{0.690196,0.690196,0.690196}%
\pgfsetstrokecolor{currentstroke}%
\pgfsetdash{}{0pt}%
\pgfpathmoveto{\pgfqpoint{3.110716in}{0.549691in}}%
\pgfpathlineto{\pgfqpoint{3.110716in}{4.575328in}}%
\pgfusepath{stroke}%
\end{pgfscope}%
\begin{pgfscope}%
\pgfsetbuttcap%
\pgfsetroundjoin%
\definecolor{currentfill}{rgb}{0.000000,0.000000,0.000000}%
\pgfsetfillcolor{currentfill}%
\pgfsetlinewidth{0.803000pt}%
\definecolor{currentstroke}{rgb}{0.000000,0.000000,0.000000}%
\pgfsetstrokecolor{currentstroke}%
\pgfsetdash{}{0pt}%
\pgfsys@defobject{currentmarker}{\pgfqpoint{0.000000in}{-0.048611in}}{\pgfqpoint{0.000000in}{0.000000in}}{%
\pgfpathmoveto{\pgfqpoint{0.000000in}{0.000000in}}%
\pgfpathlineto{\pgfqpoint{0.000000in}{-0.048611in}}%
\pgfusepath{stroke,fill}%
}%
\begin{pgfscope}%
\pgfsys@transformshift{3.110716in}{0.549691in}%
\pgfsys@useobject{currentmarker}{}%
\end{pgfscope}%
\end{pgfscope}%
\begin{pgfscope}%
\definecolor{textcolor}{rgb}{0.000000,0.000000,0.000000}%
\pgfsetstrokecolor{textcolor}%
\pgfsetfillcolor{textcolor}%
\pgftext[x=3.110716in,y=0.452469in,,top]{\color{textcolor}\rmfamily\fontsize{10.000000}{12.000000}\selectfont \(\displaystyle {200}\)}%
\end{pgfscope}%
\begin{pgfscope}%
\pgfpathrectangle{\pgfqpoint{0.977263in}{0.549691in}}{\pgfqpoint{5.365101in}{4.025637in}}%
\pgfusepath{clip}%
\pgfsetrectcap%
\pgfsetroundjoin%
\pgfsetlinewidth{0.803000pt}%
\definecolor{currentstroke}{rgb}{0.690196,0.690196,0.690196}%
\pgfsetstrokecolor{currentstroke}%
\pgfsetdash{}{0pt}%
\pgfpathmoveto{\pgfqpoint{4.177443in}{0.549691in}}%
\pgfpathlineto{\pgfqpoint{4.177443in}{4.575328in}}%
\pgfusepath{stroke}%
\end{pgfscope}%
\begin{pgfscope}%
\pgfsetbuttcap%
\pgfsetroundjoin%
\definecolor{currentfill}{rgb}{0.000000,0.000000,0.000000}%
\pgfsetfillcolor{currentfill}%
\pgfsetlinewidth{0.803000pt}%
\definecolor{currentstroke}{rgb}{0.000000,0.000000,0.000000}%
\pgfsetstrokecolor{currentstroke}%
\pgfsetdash{}{0pt}%
\pgfsys@defobject{currentmarker}{\pgfqpoint{0.000000in}{-0.048611in}}{\pgfqpoint{0.000000in}{0.000000in}}{%
\pgfpathmoveto{\pgfqpoint{0.000000in}{0.000000in}}%
\pgfpathlineto{\pgfqpoint{0.000000in}{-0.048611in}}%
\pgfusepath{stroke,fill}%
}%
\begin{pgfscope}%
\pgfsys@transformshift{4.177443in}{0.549691in}%
\pgfsys@useobject{currentmarker}{}%
\end{pgfscope}%
\end{pgfscope}%
\begin{pgfscope}%
\definecolor{textcolor}{rgb}{0.000000,0.000000,0.000000}%
\pgfsetstrokecolor{textcolor}%
\pgfsetfillcolor{textcolor}%
\pgftext[x=4.177443in,y=0.452469in,,top]{\color{textcolor}\rmfamily\fontsize{10.000000}{12.000000}\selectfont \(\displaystyle {300}\)}%
\end{pgfscope}%
\begin{pgfscope}%
\pgfpathrectangle{\pgfqpoint{0.977263in}{0.549691in}}{\pgfqpoint{5.365101in}{4.025637in}}%
\pgfusepath{clip}%
\pgfsetrectcap%
\pgfsetroundjoin%
\pgfsetlinewidth{0.803000pt}%
\definecolor{currentstroke}{rgb}{0.690196,0.690196,0.690196}%
\pgfsetstrokecolor{currentstroke}%
\pgfsetdash{}{0pt}%
\pgfpathmoveto{\pgfqpoint{5.244169in}{0.549691in}}%
\pgfpathlineto{\pgfqpoint{5.244169in}{4.575328in}}%
\pgfusepath{stroke}%
\end{pgfscope}%
\begin{pgfscope}%
\pgfsetbuttcap%
\pgfsetroundjoin%
\definecolor{currentfill}{rgb}{0.000000,0.000000,0.000000}%
\pgfsetfillcolor{currentfill}%
\pgfsetlinewidth{0.803000pt}%
\definecolor{currentstroke}{rgb}{0.000000,0.000000,0.000000}%
\pgfsetstrokecolor{currentstroke}%
\pgfsetdash{}{0pt}%
\pgfsys@defobject{currentmarker}{\pgfqpoint{0.000000in}{-0.048611in}}{\pgfqpoint{0.000000in}{0.000000in}}{%
\pgfpathmoveto{\pgfqpoint{0.000000in}{0.000000in}}%
\pgfpathlineto{\pgfqpoint{0.000000in}{-0.048611in}}%
\pgfusepath{stroke,fill}%
}%
\begin{pgfscope}%
\pgfsys@transformshift{5.244169in}{0.549691in}%
\pgfsys@useobject{currentmarker}{}%
\end{pgfscope}%
\end{pgfscope}%
\begin{pgfscope}%
\definecolor{textcolor}{rgb}{0.000000,0.000000,0.000000}%
\pgfsetstrokecolor{textcolor}%
\pgfsetfillcolor{textcolor}%
\pgftext[x=5.244169in,y=0.452469in,,top]{\color{textcolor}\rmfamily\fontsize{10.000000}{12.000000}\selectfont \(\displaystyle {400}\)}%
\end{pgfscope}%
\begin{pgfscope}%
\pgfpathrectangle{\pgfqpoint{0.977263in}{0.549691in}}{\pgfqpoint{5.365101in}{4.025637in}}%
\pgfusepath{clip}%
\pgfsetrectcap%
\pgfsetroundjoin%
\pgfsetlinewidth{0.803000pt}%
\definecolor{currentstroke}{rgb}{0.690196,0.690196,0.690196}%
\pgfsetstrokecolor{currentstroke}%
\pgfsetdash{}{0pt}%
\pgfpathmoveto{\pgfqpoint{6.310896in}{0.549691in}}%
\pgfpathlineto{\pgfqpoint{6.310896in}{4.575328in}}%
\pgfusepath{stroke}%
\end{pgfscope}%
\begin{pgfscope}%
\pgfsetbuttcap%
\pgfsetroundjoin%
\definecolor{currentfill}{rgb}{0.000000,0.000000,0.000000}%
\pgfsetfillcolor{currentfill}%
\pgfsetlinewidth{0.803000pt}%
\definecolor{currentstroke}{rgb}{0.000000,0.000000,0.000000}%
\pgfsetstrokecolor{currentstroke}%
\pgfsetdash{}{0pt}%
\pgfsys@defobject{currentmarker}{\pgfqpoint{0.000000in}{-0.048611in}}{\pgfqpoint{0.000000in}{0.000000in}}{%
\pgfpathmoveto{\pgfqpoint{0.000000in}{0.000000in}}%
\pgfpathlineto{\pgfqpoint{0.000000in}{-0.048611in}}%
\pgfusepath{stroke,fill}%
}%
\begin{pgfscope}%
\pgfsys@transformshift{6.310896in}{0.549691in}%
\pgfsys@useobject{currentmarker}{}%
\end{pgfscope}%
\end{pgfscope}%
\begin{pgfscope}%
\definecolor{textcolor}{rgb}{0.000000,0.000000,0.000000}%
\pgfsetstrokecolor{textcolor}%
\pgfsetfillcolor{textcolor}%
\pgftext[x=6.310896in,y=0.452469in,,top]{\color{textcolor}\rmfamily\fontsize{10.000000}{12.000000}\selectfont \(\displaystyle {500}\)}%
\end{pgfscope}%
\begin{pgfscope}%
\definecolor{textcolor}{rgb}{0.000000,0.000000,0.000000}%
\pgfsetstrokecolor{textcolor}%
\pgfsetfillcolor{textcolor}%
\pgftext[x=3.659814in,y=0.273457in,,top]{\color{textcolor}\rmfamily\fontsize{10.000000}{12.000000}\selectfont \(\displaystyle k\)}%
\end{pgfscope}%
\begin{pgfscope}%
\pgfpathrectangle{\pgfqpoint{0.977263in}{0.549691in}}{\pgfqpoint{5.365101in}{4.025637in}}%
\pgfusepath{clip}%
\pgfsetrectcap%
\pgfsetroundjoin%
\pgfsetlinewidth{0.803000pt}%
\definecolor{currentstroke}{rgb}{0.690196,0.690196,0.690196}%
\pgfsetstrokecolor{currentstroke}%
\pgfsetdash{}{0pt}%
\pgfpathmoveto{\pgfqpoint{0.977263in}{0.785515in}}%
\pgfpathlineto{\pgfqpoint{6.342364in}{0.785515in}}%
\pgfusepath{stroke}%
\end{pgfscope}%
\begin{pgfscope}%
\pgfsetbuttcap%
\pgfsetroundjoin%
\definecolor{currentfill}{rgb}{0.000000,0.000000,0.000000}%
\pgfsetfillcolor{currentfill}%
\pgfsetlinewidth{0.803000pt}%
\definecolor{currentstroke}{rgb}{0.000000,0.000000,0.000000}%
\pgfsetstrokecolor{currentstroke}%
\pgfsetdash{}{0pt}%
\pgfsys@defobject{currentmarker}{\pgfqpoint{-0.048611in}{0.000000in}}{\pgfqpoint{-0.000000in}{0.000000in}}{%
\pgfpathmoveto{\pgfqpoint{-0.000000in}{0.000000in}}%
\pgfpathlineto{\pgfqpoint{-0.048611in}{0.000000in}}%
\pgfusepath{stroke,fill}%
}%
\begin{pgfscope}%
\pgfsys@transformshift{0.977263in}{0.785515in}%
\pgfsys@useobject{currentmarker}{}%
\end{pgfscope}%
\end{pgfscope}%
\begin{pgfscope}%
\definecolor{textcolor}{rgb}{0.000000,0.000000,0.000000}%
\pgfsetstrokecolor{textcolor}%
\pgfsetfillcolor{textcolor}%
\pgftext[x=0.592038in, y=0.737289in, left, base]{\color{textcolor}\rmfamily\fontsize{10.000000}{12.000000}\selectfont \(\displaystyle {10^{-7}}\)}%
\end{pgfscope}%
\begin{pgfscope}%
\pgfpathrectangle{\pgfqpoint{0.977263in}{0.549691in}}{\pgfqpoint{5.365101in}{4.025637in}}%
\pgfusepath{clip}%
\pgfsetrectcap%
\pgfsetroundjoin%
\pgfsetlinewidth{0.803000pt}%
\definecolor{currentstroke}{rgb}{0.690196,0.690196,0.690196}%
\pgfsetstrokecolor{currentstroke}%
\pgfsetdash{}{0pt}%
\pgfpathmoveto{\pgfqpoint{0.977263in}{1.300776in}}%
\pgfpathlineto{\pgfqpoint{6.342364in}{1.300776in}}%
\pgfusepath{stroke}%
\end{pgfscope}%
\begin{pgfscope}%
\pgfsetbuttcap%
\pgfsetroundjoin%
\definecolor{currentfill}{rgb}{0.000000,0.000000,0.000000}%
\pgfsetfillcolor{currentfill}%
\pgfsetlinewidth{0.803000pt}%
\definecolor{currentstroke}{rgb}{0.000000,0.000000,0.000000}%
\pgfsetstrokecolor{currentstroke}%
\pgfsetdash{}{0pt}%
\pgfsys@defobject{currentmarker}{\pgfqpoint{-0.048611in}{0.000000in}}{\pgfqpoint{-0.000000in}{0.000000in}}{%
\pgfpathmoveto{\pgfqpoint{-0.000000in}{0.000000in}}%
\pgfpathlineto{\pgfqpoint{-0.048611in}{0.000000in}}%
\pgfusepath{stroke,fill}%
}%
\begin{pgfscope}%
\pgfsys@transformshift{0.977263in}{1.300776in}%
\pgfsys@useobject{currentmarker}{}%
\end{pgfscope}%
\end{pgfscope}%
\begin{pgfscope}%
\definecolor{textcolor}{rgb}{0.000000,0.000000,0.000000}%
\pgfsetstrokecolor{textcolor}%
\pgfsetfillcolor{textcolor}%
\pgftext[x=0.592038in, y=1.252551in, left, base]{\color{textcolor}\rmfamily\fontsize{10.000000}{12.000000}\selectfont \(\displaystyle {10^{-6}}\)}%
\end{pgfscope}%
\begin{pgfscope}%
\pgfpathrectangle{\pgfqpoint{0.977263in}{0.549691in}}{\pgfqpoint{5.365101in}{4.025637in}}%
\pgfusepath{clip}%
\pgfsetrectcap%
\pgfsetroundjoin%
\pgfsetlinewidth{0.803000pt}%
\definecolor{currentstroke}{rgb}{0.690196,0.690196,0.690196}%
\pgfsetstrokecolor{currentstroke}%
\pgfsetdash{}{0pt}%
\pgfpathmoveto{\pgfqpoint{0.977263in}{1.816037in}}%
\pgfpathlineto{\pgfqpoint{6.342364in}{1.816037in}}%
\pgfusepath{stroke}%
\end{pgfscope}%
\begin{pgfscope}%
\pgfsetbuttcap%
\pgfsetroundjoin%
\definecolor{currentfill}{rgb}{0.000000,0.000000,0.000000}%
\pgfsetfillcolor{currentfill}%
\pgfsetlinewidth{0.803000pt}%
\definecolor{currentstroke}{rgb}{0.000000,0.000000,0.000000}%
\pgfsetstrokecolor{currentstroke}%
\pgfsetdash{}{0pt}%
\pgfsys@defobject{currentmarker}{\pgfqpoint{-0.048611in}{0.000000in}}{\pgfqpoint{-0.000000in}{0.000000in}}{%
\pgfpathmoveto{\pgfqpoint{-0.000000in}{0.000000in}}%
\pgfpathlineto{\pgfqpoint{-0.048611in}{0.000000in}}%
\pgfusepath{stroke,fill}%
}%
\begin{pgfscope}%
\pgfsys@transformshift{0.977263in}{1.816037in}%
\pgfsys@useobject{currentmarker}{}%
\end{pgfscope}%
\end{pgfscope}%
\begin{pgfscope}%
\definecolor{textcolor}{rgb}{0.000000,0.000000,0.000000}%
\pgfsetstrokecolor{textcolor}%
\pgfsetfillcolor{textcolor}%
\pgftext[x=0.592038in, y=1.767812in, left, base]{\color{textcolor}\rmfamily\fontsize{10.000000}{12.000000}\selectfont \(\displaystyle {10^{-5}}\)}%
\end{pgfscope}%
\begin{pgfscope}%
\pgfpathrectangle{\pgfqpoint{0.977263in}{0.549691in}}{\pgfqpoint{5.365101in}{4.025637in}}%
\pgfusepath{clip}%
\pgfsetrectcap%
\pgfsetroundjoin%
\pgfsetlinewidth{0.803000pt}%
\definecolor{currentstroke}{rgb}{0.690196,0.690196,0.690196}%
\pgfsetstrokecolor{currentstroke}%
\pgfsetdash{}{0pt}%
\pgfpathmoveto{\pgfqpoint{0.977263in}{2.331299in}}%
\pgfpathlineto{\pgfqpoint{6.342364in}{2.331299in}}%
\pgfusepath{stroke}%
\end{pgfscope}%
\begin{pgfscope}%
\pgfsetbuttcap%
\pgfsetroundjoin%
\definecolor{currentfill}{rgb}{0.000000,0.000000,0.000000}%
\pgfsetfillcolor{currentfill}%
\pgfsetlinewidth{0.803000pt}%
\definecolor{currentstroke}{rgb}{0.000000,0.000000,0.000000}%
\pgfsetstrokecolor{currentstroke}%
\pgfsetdash{}{0pt}%
\pgfsys@defobject{currentmarker}{\pgfqpoint{-0.048611in}{0.000000in}}{\pgfqpoint{-0.000000in}{0.000000in}}{%
\pgfpathmoveto{\pgfqpoint{-0.000000in}{0.000000in}}%
\pgfpathlineto{\pgfqpoint{-0.048611in}{0.000000in}}%
\pgfusepath{stroke,fill}%
}%
\begin{pgfscope}%
\pgfsys@transformshift{0.977263in}{2.331299in}%
\pgfsys@useobject{currentmarker}{}%
\end{pgfscope}%
\end{pgfscope}%
\begin{pgfscope}%
\definecolor{textcolor}{rgb}{0.000000,0.000000,0.000000}%
\pgfsetstrokecolor{textcolor}%
\pgfsetfillcolor{textcolor}%
\pgftext[x=0.592038in, y=2.283074in, left, base]{\color{textcolor}\rmfamily\fontsize{10.000000}{12.000000}\selectfont \(\displaystyle {10^{-4}}\)}%
\end{pgfscope}%
\begin{pgfscope}%
\pgfpathrectangle{\pgfqpoint{0.977263in}{0.549691in}}{\pgfqpoint{5.365101in}{4.025637in}}%
\pgfusepath{clip}%
\pgfsetrectcap%
\pgfsetroundjoin%
\pgfsetlinewidth{0.803000pt}%
\definecolor{currentstroke}{rgb}{0.690196,0.690196,0.690196}%
\pgfsetstrokecolor{currentstroke}%
\pgfsetdash{}{0pt}%
\pgfpathmoveto{\pgfqpoint{0.977263in}{2.846560in}}%
\pgfpathlineto{\pgfqpoint{6.342364in}{2.846560in}}%
\pgfusepath{stroke}%
\end{pgfscope}%
\begin{pgfscope}%
\pgfsetbuttcap%
\pgfsetroundjoin%
\definecolor{currentfill}{rgb}{0.000000,0.000000,0.000000}%
\pgfsetfillcolor{currentfill}%
\pgfsetlinewidth{0.803000pt}%
\definecolor{currentstroke}{rgb}{0.000000,0.000000,0.000000}%
\pgfsetstrokecolor{currentstroke}%
\pgfsetdash{}{0pt}%
\pgfsys@defobject{currentmarker}{\pgfqpoint{-0.048611in}{0.000000in}}{\pgfqpoint{-0.000000in}{0.000000in}}{%
\pgfpathmoveto{\pgfqpoint{-0.000000in}{0.000000in}}%
\pgfpathlineto{\pgfqpoint{-0.048611in}{0.000000in}}%
\pgfusepath{stroke,fill}%
}%
\begin{pgfscope}%
\pgfsys@transformshift{0.977263in}{2.846560in}%
\pgfsys@useobject{currentmarker}{}%
\end{pgfscope}%
\end{pgfscope}%
\begin{pgfscope}%
\definecolor{textcolor}{rgb}{0.000000,0.000000,0.000000}%
\pgfsetstrokecolor{textcolor}%
\pgfsetfillcolor{textcolor}%
\pgftext[x=0.592038in, y=2.798335in, left, base]{\color{textcolor}\rmfamily\fontsize{10.000000}{12.000000}\selectfont \(\displaystyle {10^{-3}}\)}%
\end{pgfscope}%
\begin{pgfscope}%
\pgfpathrectangle{\pgfqpoint{0.977263in}{0.549691in}}{\pgfqpoint{5.365101in}{4.025637in}}%
\pgfusepath{clip}%
\pgfsetrectcap%
\pgfsetroundjoin%
\pgfsetlinewidth{0.803000pt}%
\definecolor{currentstroke}{rgb}{0.690196,0.690196,0.690196}%
\pgfsetstrokecolor{currentstroke}%
\pgfsetdash{}{0pt}%
\pgfpathmoveto{\pgfqpoint{0.977263in}{3.361822in}}%
\pgfpathlineto{\pgfqpoint{6.342364in}{3.361822in}}%
\pgfusepath{stroke}%
\end{pgfscope}%
\begin{pgfscope}%
\pgfsetbuttcap%
\pgfsetroundjoin%
\definecolor{currentfill}{rgb}{0.000000,0.000000,0.000000}%
\pgfsetfillcolor{currentfill}%
\pgfsetlinewidth{0.803000pt}%
\definecolor{currentstroke}{rgb}{0.000000,0.000000,0.000000}%
\pgfsetstrokecolor{currentstroke}%
\pgfsetdash{}{0pt}%
\pgfsys@defobject{currentmarker}{\pgfqpoint{-0.048611in}{0.000000in}}{\pgfqpoint{-0.000000in}{0.000000in}}{%
\pgfpathmoveto{\pgfqpoint{-0.000000in}{0.000000in}}%
\pgfpathlineto{\pgfqpoint{-0.048611in}{0.000000in}}%
\pgfusepath{stroke,fill}%
}%
\begin{pgfscope}%
\pgfsys@transformshift{0.977263in}{3.361822in}%
\pgfsys@useobject{currentmarker}{}%
\end{pgfscope}%
\end{pgfscope}%
\begin{pgfscope}%
\definecolor{textcolor}{rgb}{0.000000,0.000000,0.000000}%
\pgfsetstrokecolor{textcolor}%
\pgfsetfillcolor{textcolor}%
\pgftext[x=0.592038in, y=3.313596in, left, base]{\color{textcolor}\rmfamily\fontsize{10.000000}{12.000000}\selectfont \(\displaystyle {10^{-2}}\)}%
\end{pgfscope}%
\begin{pgfscope}%
\pgfpathrectangle{\pgfqpoint{0.977263in}{0.549691in}}{\pgfqpoint{5.365101in}{4.025637in}}%
\pgfusepath{clip}%
\pgfsetrectcap%
\pgfsetroundjoin%
\pgfsetlinewidth{0.803000pt}%
\definecolor{currentstroke}{rgb}{0.690196,0.690196,0.690196}%
\pgfsetstrokecolor{currentstroke}%
\pgfsetdash{}{0pt}%
\pgfpathmoveto{\pgfqpoint{0.977263in}{3.877083in}}%
\pgfpathlineto{\pgfqpoint{6.342364in}{3.877083in}}%
\pgfusepath{stroke}%
\end{pgfscope}%
\begin{pgfscope}%
\pgfsetbuttcap%
\pgfsetroundjoin%
\definecolor{currentfill}{rgb}{0.000000,0.000000,0.000000}%
\pgfsetfillcolor{currentfill}%
\pgfsetlinewidth{0.803000pt}%
\definecolor{currentstroke}{rgb}{0.000000,0.000000,0.000000}%
\pgfsetstrokecolor{currentstroke}%
\pgfsetdash{}{0pt}%
\pgfsys@defobject{currentmarker}{\pgfqpoint{-0.048611in}{0.000000in}}{\pgfqpoint{-0.000000in}{0.000000in}}{%
\pgfpathmoveto{\pgfqpoint{-0.000000in}{0.000000in}}%
\pgfpathlineto{\pgfqpoint{-0.048611in}{0.000000in}}%
\pgfusepath{stroke,fill}%
}%
\begin{pgfscope}%
\pgfsys@transformshift{0.977263in}{3.877083in}%
\pgfsys@useobject{currentmarker}{}%
\end{pgfscope}%
\end{pgfscope}%
\begin{pgfscope}%
\definecolor{textcolor}{rgb}{0.000000,0.000000,0.000000}%
\pgfsetstrokecolor{textcolor}%
\pgfsetfillcolor{textcolor}%
\pgftext[x=0.592038in, y=3.828858in, left, base]{\color{textcolor}\rmfamily\fontsize{10.000000}{12.000000}\selectfont \(\displaystyle {10^{-1}}\)}%
\end{pgfscope}%
\begin{pgfscope}%
\pgfpathrectangle{\pgfqpoint{0.977263in}{0.549691in}}{\pgfqpoint{5.365101in}{4.025637in}}%
\pgfusepath{clip}%
\pgfsetrectcap%
\pgfsetroundjoin%
\pgfsetlinewidth{0.803000pt}%
\definecolor{currentstroke}{rgb}{0.690196,0.690196,0.690196}%
\pgfsetstrokecolor{currentstroke}%
\pgfsetdash{}{0pt}%
\pgfpathmoveto{\pgfqpoint{0.977263in}{4.392345in}}%
\pgfpathlineto{\pgfqpoint{6.342364in}{4.392345in}}%
\pgfusepath{stroke}%
\end{pgfscope}%
\begin{pgfscope}%
\pgfsetbuttcap%
\pgfsetroundjoin%
\definecolor{currentfill}{rgb}{0.000000,0.000000,0.000000}%
\pgfsetfillcolor{currentfill}%
\pgfsetlinewidth{0.803000pt}%
\definecolor{currentstroke}{rgb}{0.000000,0.000000,0.000000}%
\pgfsetstrokecolor{currentstroke}%
\pgfsetdash{}{0pt}%
\pgfsys@defobject{currentmarker}{\pgfqpoint{-0.048611in}{0.000000in}}{\pgfqpoint{-0.000000in}{0.000000in}}{%
\pgfpathmoveto{\pgfqpoint{-0.000000in}{0.000000in}}%
\pgfpathlineto{\pgfqpoint{-0.048611in}{0.000000in}}%
\pgfusepath{stroke,fill}%
}%
\begin{pgfscope}%
\pgfsys@transformshift{0.977263in}{4.392345in}%
\pgfsys@useobject{currentmarker}{}%
\end{pgfscope}%
\end{pgfscope}%
\begin{pgfscope}%
\definecolor{textcolor}{rgb}{0.000000,0.000000,0.000000}%
\pgfsetstrokecolor{textcolor}%
\pgfsetfillcolor{textcolor}%
\pgftext[x=0.678844in, y=4.344119in, left, base]{\color{textcolor}\rmfamily\fontsize{10.000000}{12.000000}\selectfont \(\displaystyle {10^{0}}\)}%
\end{pgfscope}%
\begin{pgfscope}%
\pgfsetbuttcap%
\pgfsetroundjoin%
\definecolor{currentfill}{rgb}{0.000000,0.000000,0.000000}%
\pgfsetfillcolor{currentfill}%
\pgfsetlinewidth{0.602250pt}%
\definecolor{currentstroke}{rgb}{0.000000,0.000000,0.000000}%
\pgfsetstrokecolor{currentstroke}%
\pgfsetdash{}{0pt}%
\pgfsys@defobject{currentmarker}{\pgfqpoint{-0.027778in}{0.000000in}}{\pgfqpoint{-0.000000in}{0.000000in}}{%
\pgfpathmoveto{\pgfqpoint{-0.000000in}{0.000000in}}%
\pgfpathlineto{\pgfqpoint{-0.027778in}{0.000000in}}%
\pgfusepath{stroke,fill}%
}%
\begin{pgfscope}%
\pgfsys@transformshift{0.977263in}{0.580472in}%
\pgfsys@useobject{currentmarker}{}%
\end{pgfscope}%
\end{pgfscope}%
\begin{pgfscope}%
\pgfsetbuttcap%
\pgfsetroundjoin%
\definecolor{currentfill}{rgb}{0.000000,0.000000,0.000000}%
\pgfsetfillcolor{currentfill}%
\pgfsetlinewidth{0.602250pt}%
\definecolor{currentstroke}{rgb}{0.000000,0.000000,0.000000}%
\pgfsetstrokecolor{currentstroke}%
\pgfsetdash{}{0pt}%
\pgfsys@defobject{currentmarker}{\pgfqpoint{-0.027778in}{0.000000in}}{\pgfqpoint{-0.000000in}{0.000000in}}{%
\pgfpathmoveto{\pgfqpoint{-0.000000in}{0.000000in}}%
\pgfpathlineto{\pgfqpoint{-0.027778in}{0.000000in}}%
\pgfusepath{stroke,fill}%
}%
\begin{pgfscope}%
\pgfsys@transformshift{0.977263in}{0.630406in}%
\pgfsys@useobject{currentmarker}{}%
\end{pgfscope}%
\end{pgfscope}%
\begin{pgfscope}%
\pgfsetbuttcap%
\pgfsetroundjoin%
\definecolor{currentfill}{rgb}{0.000000,0.000000,0.000000}%
\pgfsetfillcolor{currentfill}%
\pgfsetlinewidth{0.602250pt}%
\definecolor{currentstroke}{rgb}{0.000000,0.000000,0.000000}%
\pgfsetstrokecolor{currentstroke}%
\pgfsetdash{}{0pt}%
\pgfsys@defobject{currentmarker}{\pgfqpoint{-0.027778in}{0.000000in}}{\pgfqpoint{-0.000000in}{0.000000in}}{%
\pgfpathmoveto{\pgfqpoint{-0.000000in}{0.000000in}}%
\pgfpathlineto{\pgfqpoint{-0.027778in}{0.000000in}}%
\pgfusepath{stroke,fill}%
}%
\begin{pgfscope}%
\pgfsys@transformshift{0.977263in}{0.671205in}%
\pgfsys@useobject{currentmarker}{}%
\end{pgfscope}%
\end{pgfscope}%
\begin{pgfscope}%
\pgfsetbuttcap%
\pgfsetroundjoin%
\definecolor{currentfill}{rgb}{0.000000,0.000000,0.000000}%
\pgfsetfillcolor{currentfill}%
\pgfsetlinewidth{0.602250pt}%
\definecolor{currentstroke}{rgb}{0.000000,0.000000,0.000000}%
\pgfsetstrokecolor{currentstroke}%
\pgfsetdash{}{0pt}%
\pgfsys@defobject{currentmarker}{\pgfqpoint{-0.027778in}{0.000000in}}{\pgfqpoint{-0.000000in}{0.000000in}}{%
\pgfpathmoveto{\pgfqpoint{-0.000000in}{0.000000in}}%
\pgfpathlineto{\pgfqpoint{-0.027778in}{0.000000in}}%
\pgfusepath{stroke,fill}%
}%
\begin{pgfscope}%
\pgfsys@transformshift{0.977263in}{0.705700in}%
\pgfsys@useobject{currentmarker}{}%
\end{pgfscope}%
\end{pgfscope}%
\begin{pgfscope}%
\pgfsetbuttcap%
\pgfsetroundjoin%
\definecolor{currentfill}{rgb}{0.000000,0.000000,0.000000}%
\pgfsetfillcolor{currentfill}%
\pgfsetlinewidth{0.602250pt}%
\definecolor{currentstroke}{rgb}{0.000000,0.000000,0.000000}%
\pgfsetstrokecolor{currentstroke}%
\pgfsetdash{}{0pt}%
\pgfsys@defobject{currentmarker}{\pgfqpoint{-0.027778in}{0.000000in}}{\pgfqpoint{-0.000000in}{0.000000in}}{%
\pgfpathmoveto{\pgfqpoint{-0.000000in}{0.000000in}}%
\pgfpathlineto{\pgfqpoint{-0.027778in}{0.000000in}}%
\pgfusepath{stroke,fill}%
}%
\begin{pgfscope}%
\pgfsys@transformshift{0.977263in}{0.735581in}%
\pgfsys@useobject{currentmarker}{}%
\end{pgfscope}%
\end{pgfscope}%
\begin{pgfscope}%
\pgfsetbuttcap%
\pgfsetroundjoin%
\definecolor{currentfill}{rgb}{0.000000,0.000000,0.000000}%
\pgfsetfillcolor{currentfill}%
\pgfsetlinewidth{0.602250pt}%
\definecolor{currentstroke}{rgb}{0.000000,0.000000,0.000000}%
\pgfsetstrokecolor{currentstroke}%
\pgfsetdash{}{0pt}%
\pgfsys@defobject{currentmarker}{\pgfqpoint{-0.027778in}{0.000000in}}{\pgfqpoint{-0.000000in}{0.000000in}}{%
\pgfpathmoveto{\pgfqpoint{-0.000000in}{0.000000in}}%
\pgfpathlineto{\pgfqpoint{-0.027778in}{0.000000in}}%
\pgfusepath{stroke,fill}%
}%
\begin{pgfscope}%
\pgfsys@transformshift{0.977263in}{0.761938in}%
\pgfsys@useobject{currentmarker}{}%
\end{pgfscope}%
\end{pgfscope}%
\begin{pgfscope}%
\pgfsetbuttcap%
\pgfsetroundjoin%
\definecolor{currentfill}{rgb}{0.000000,0.000000,0.000000}%
\pgfsetfillcolor{currentfill}%
\pgfsetlinewidth{0.602250pt}%
\definecolor{currentstroke}{rgb}{0.000000,0.000000,0.000000}%
\pgfsetstrokecolor{currentstroke}%
\pgfsetdash{}{0pt}%
\pgfsys@defobject{currentmarker}{\pgfqpoint{-0.027778in}{0.000000in}}{\pgfqpoint{-0.000000in}{0.000000in}}{%
\pgfpathmoveto{\pgfqpoint{-0.000000in}{0.000000in}}%
\pgfpathlineto{\pgfqpoint{-0.027778in}{0.000000in}}%
\pgfusepath{stroke,fill}%
}%
\begin{pgfscope}%
\pgfsys@transformshift{0.977263in}{0.940624in}%
\pgfsys@useobject{currentmarker}{}%
\end{pgfscope}%
\end{pgfscope}%
\begin{pgfscope}%
\pgfsetbuttcap%
\pgfsetroundjoin%
\definecolor{currentfill}{rgb}{0.000000,0.000000,0.000000}%
\pgfsetfillcolor{currentfill}%
\pgfsetlinewidth{0.602250pt}%
\definecolor{currentstroke}{rgb}{0.000000,0.000000,0.000000}%
\pgfsetstrokecolor{currentstroke}%
\pgfsetdash{}{0pt}%
\pgfsys@defobject{currentmarker}{\pgfqpoint{-0.027778in}{0.000000in}}{\pgfqpoint{-0.000000in}{0.000000in}}{%
\pgfpathmoveto{\pgfqpoint{-0.000000in}{0.000000in}}%
\pgfpathlineto{\pgfqpoint{-0.027778in}{0.000000in}}%
\pgfusepath{stroke,fill}%
}%
\begin{pgfscope}%
\pgfsys@transformshift{0.977263in}{1.031357in}%
\pgfsys@useobject{currentmarker}{}%
\end{pgfscope}%
\end{pgfscope}%
\begin{pgfscope}%
\pgfsetbuttcap%
\pgfsetroundjoin%
\definecolor{currentfill}{rgb}{0.000000,0.000000,0.000000}%
\pgfsetfillcolor{currentfill}%
\pgfsetlinewidth{0.602250pt}%
\definecolor{currentstroke}{rgb}{0.000000,0.000000,0.000000}%
\pgfsetstrokecolor{currentstroke}%
\pgfsetdash{}{0pt}%
\pgfsys@defobject{currentmarker}{\pgfqpoint{-0.027778in}{0.000000in}}{\pgfqpoint{-0.000000in}{0.000000in}}{%
\pgfpathmoveto{\pgfqpoint{-0.000000in}{0.000000in}}%
\pgfpathlineto{\pgfqpoint{-0.027778in}{0.000000in}}%
\pgfusepath{stroke,fill}%
}%
\begin{pgfscope}%
\pgfsys@transformshift{0.977263in}{1.095733in}%
\pgfsys@useobject{currentmarker}{}%
\end{pgfscope}%
\end{pgfscope}%
\begin{pgfscope}%
\pgfsetbuttcap%
\pgfsetroundjoin%
\definecolor{currentfill}{rgb}{0.000000,0.000000,0.000000}%
\pgfsetfillcolor{currentfill}%
\pgfsetlinewidth{0.602250pt}%
\definecolor{currentstroke}{rgb}{0.000000,0.000000,0.000000}%
\pgfsetstrokecolor{currentstroke}%
\pgfsetdash{}{0pt}%
\pgfsys@defobject{currentmarker}{\pgfqpoint{-0.027778in}{0.000000in}}{\pgfqpoint{-0.000000in}{0.000000in}}{%
\pgfpathmoveto{\pgfqpoint{-0.000000in}{0.000000in}}%
\pgfpathlineto{\pgfqpoint{-0.027778in}{0.000000in}}%
\pgfusepath{stroke,fill}%
}%
\begin{pgfscope}%
\pgfsys@transformshift{0.977263in}{1.145667in}%
\pgfsys@useobject{currentmarker}{}%
\end{pgfscope}%
\end{pgfscope}%
\begin{pgfscope}%
\pgfsetbuttcap%
\pgfsetroundjoin%
\definecolor{currentfill}{rgb}{0.000000,0.000000,0.000000}%
\pgfsetfillcolor{currentfill}%
\pgfsetlinewidth{0.602250pt}%
\definecolor{currentstroke}{rgb}{0.000000,0.000000,0.000000}%
\pgfsetstrokecolor{currentstroke}%
\pgfsetdash{}{0pt}%
\pgfsys@defobject{currentmarker}{\pgfqpoint{-0.027778in}{0.000000in}}{\pgfqpoint{-0.000000in}{0.000000in}}{%
\pgfpathmoveto{\pgfqpoint{-0.000000in}{0.000000in}}%
\pgfpathlineto{\pgfqpoint{-0.027778in}{0.000000in}}%
\pgfusepath{stroke,fill}%
}%
\begin{pgfscope}%
\pgfsys@transformshift{0.977263in}{1.186466in}%
\pgfsys@useobject{currentmarker}{}%
\end{pgfscope}%
\end{pgfscope}%
\begin{pgfscope}%
\pgfsetbuttcap%
\pgfsetroundjoin%
\definecolor{currentfill}{rgb}{0.000000,0.000000,0.000000}%
\pgfsetfillcolor{currentfill}%
\pgfsetlinewidth{0.602250pt}%
\definecolor{currentstroke}{rgb}{0.000000,0.000000,0.000000}%
\pgfsetstrokecolor{currentstroke}%
\pgfsetdash{}{0pt}%
\pgfsys@defobject{currentmarker}{\pgfqpoint{-0.027778in}{0.000000in}}{\pgfqpoint{-0.000000in}{0.000000in}}{%
\pgfpathmoveto{\pgfqpoint{-0.000000in}{0.000000in}}%
\pgfpathlineto{\pgfqpoint{-0.027778in}{0.000000in}}%
\pgfusepath{stroke,fill}%
}%
\begin{pgfscope}%
\pgfsys@transformshift{0.977263in}{1.220961in}%
\pgfsys@useobject{currentmarker}{}%
\end{pgfscope}%
\end{pgfscope}%
\begin{pgfscope}%
\pgfsetbuttcap%
\pgfsetroundjoin%
\definecolor{currentfill}{rgb}{0.000000,0.000000,0.000000}%
\pgfsetfillcolor{currentfill}%
\pgfsetlinewidth{0.602250pt}%
\definecolor{currentstroke}{rgb}{0.000000,0.000000,0.000000}%
\pgfsetstrokecolor{currentstroke}%
\pgfsetdash{}{0pt}%
\pgfsys@defobject{currentmarker}{\pgfqpoint{-0.027778in}{0.000000in}}{\pgfqpoint{-0.000000in}{0.000000in}}{%
\pgfpathmoveto{\pgfqpoint{-0.000000in}{0.000000in}}%
\pgfpathlineto{\pgfqpoint{-0.027778in}{0.000000in}}%
\pgfusepath{stroke,fill}%
}%
\begin{pgfscope}%
\pgfsys@transformshift{0.977263in}{1.250842in}%
\pgfsys@useobject{currentmarker}{}%
\end{pgfscope}%
\end{pgfscope}%
\begin{pgfscope}%
\pgfsetbuttcap%
\pgfsetroundjoin%
\definecolor{currentfill}{rgb}{0.000000,0.000000,0.000000}%
\pgfsetfillcolor{currentfill}%
\pgfsetlinewidth{0.602250pt}%
\definecolor{currentstroke}{rgb}{0.000000,0.000000,0.000000}%
\pgfsetstrokecolor{currentstroke}%
\pgfsetdash{}{0pt}%
\pgfsys@defobject{currentmarker}{\pgfqpoint{-0.027778in}{0.000000in}}{\pgfqpoint{-0.000000in}{0.000000in}}{%
\pgfpathmoveto{\pgfqpoint{-0.000000in}{0.000000in}}%
\pgfpathlineto{\pgfqpoint{-0.027778in}{0.000000in}}%
\pgfusepath{stroke,fill}%
}%
\begin{pgfscope}%
\pgfsys@transformshift{0.977263in}{1.277199in}%
\pgfsys@useobject{currentmarker}{}%
\end{pgfscope}%
\end{pgfscope}%
\begin{pgfscope}%
\pgfsetbuttcap%
\pgfsetroundjoin%
\definecolor{currentfill}{rgb}{0.000000,0.000000,0.000000}%
\pgfsetfillcolor{currentfill}%
\pgfsetlinewidth{0.602250pt}%
\definecolor{currentstroke}{rgb}{0.000000,0.000000,0.000000}%
\pgfsetstrokecolor{currentstroke}%
\pgfsetdash{}{0pt}%
\pgfsys@defobject{currentmarker}{\pgfqpoint{-0.027778in}{0.000000in}}{\pgfqpoint{-0.000000in}{0.000000in}}{%
\pgfpathmoveto{\pgfqpoint{-0.000000in}{0.000000in}}%
\pgfpathlineto{\pgfqpoint{-0.027778in}{0.000000in}}%
\pgfusepath{stroke,fill}%
}%
\begin{pgfscope}%
\pgfsys@transformshift{0.977263in}{1.455885in}%
\pgfsys@useobject{currentmarker}{}%
\end{pgfscope}%
\end{pgfscope}%
\begin{pgfscope}%
\pgfsetbuttcap%
\pgfsetroundjoin%
\definecolor{currentfill}{rgb}{0.000000,0.000000,0.000000}%
\pgfsetfillcolor{currentfill}%
\pgfsetlinewidth{0.602250pt}%
\definecolor{currentstroke}{rgb}{0.000000,0.000000,0.000000}%
\pgfsetstrokecolor{currentstroke}%
\pgfsetdash{}{0pt}%
\pgfsys@defobject{currentmarker}{\pgfqpoint{-0.027778in}{0.000000in}}{\pgfqpoint{-0.000000in}{0.000000in}}{%
\pgfpathmoveto{\pgfqpoint{-0.000000in}{0.000000in}}%
\pgfpathlineto{\pgfqpoint{-0.027778in}{0.000000in}}%
\pgfusepath{stroke,fill}%
}%
\begin{pgfscope}%
\pgfsys@transformshift{0.977263in}{1.546618in}%
\pgfsys@useobject{currentmarker}{}%
\end{pgfscope}%
\end{pgfscope}%
\begin{pgfscope}%
\pgfsetbuttcap%
\pgfsetroundjoin%
\definecolor{currentfill}{rgb}{0.000000,0.000000,0.000000}%
\pgfsetfillcolor{currentfill}%
\pgfsetlinewidth{0.602250pt}%
\definecolor{currentstroke}{rgb}{0.000000,0.000000,0.000000}%
\pgfsetstrokecolor{currentstroke}%
\pgfsetdash{}{0pt}%
\pgfsys@defobject{currentmarker}{\pgfqpoint{-0.027778in}{0.000000in}}{\pgfqpoint{-0.000000in}{0.000000in}}{%
\pgfpathmoveto{\pgfqpoint{-0.000000in}{0.000000in}}%
\pgfpathlineto{\pgfqpoint{-0.027778in}{0.000000in}}%
\pgfusepath{stroke,fill}%
}%
\begin{pgfscope}%
\pgfsys@transformshift{0.977263in}{1.610994in}%
\pgfsys@useobject{currentmarker}{}%
\end{pgfscope}%
\end{pgfscope}%
\begin{pgfscope}%
\pgfsetbuttcap%
\pgfsetroundjoin%
\definecolor{currentfill}{rgb}{0.000000,0.000000,0.000000}%
\pgfsetfillcolor{currentfill}%
\pgfsetlinewidth{0.602250pt}%
\definecolor{currentstroke}{rgb}{0.000000,0.000000,0.000000}%
\pgfsetstrokecolor{currentstroke}%
\pgfsetdash{}{0pt}%
\pgfsys@defobject{currentmarker}{\pgfqpoint{-0.027778in}{0.000000in}}{\pgfqpoint{-0.000000in}{0.000000in}}{%
\pgfpathmoveto{\pgfqpoint{-0.000000in}{0.000000in}}%
\pgfpathlineto{\pgfqpoint{-0.027778in}{0.000000in}}%
\pgfusepath{stroke,fill}%
}%
\begin{pgfscope}%
\pgfsys@transformshift{0.977263in}{1.660928in}%
\pgfsys@useobject{currentmarker}{}%
\end{pgfscope}%
\end{pgfscope}%
\begin{pgfscope}%
\pgfsetbuttcap%
\pgfsetroundjoin%
\definecolor{currentfill}{rgb}{0.000000,0.000000,0.000000}%
\pgfsetfillcolor{currentfill}%
\pgfsetlinewidth{0.602250pt}%
\definecolor{currentstroke}{rgb}{0.000000,0.000000,0.000000}%
\pgfsetstrokecolor{currentstroke}%
\pgfsetdash{}{0pt}%
\pgfsys@defobject{currentmarker}{\pgfqpoint{-0.027778in}{0.000000in}}{\pgfqpoint{-0.000000in}{0.000000in}}{%
\pgfpathmoveto{\pgfqpoint{-0.000000in}{0.000000in}}%
\pgfpathlineto{\pgfqpoint{-0.027778in}{0.000000in}}%
\pgfusepath{stroke,fill}%
}%
\begin{pgfscope}%
\pgfsys@transformshift{0.977263in}{1.701727in}%
\pgfsys@useobject{currentmarker}{}%
\end{pgfscope}%
\end{pgfscope}%
\begin{pgfscope}%
\pgfsetbuttcap%
\pgfsetroundjoin%
\definecolor{currentfill}{rgb}{0.000000,0.000000,0.000000}%
\pgfsetfillcolor{currentfill}%
\pgfsetlinewidth{0.602250pt}%
\definecolor{currentstroke}{rgb}{0.000000,0.000000,0.000000}%
\pgfsetstrokecolor{currentstroke}%
\pgfsetdash{}{0pt}%
\pgfsys@defobject{currentmarker}{\pgfqpoint{-0.027778in}{0.000000in}}{\pgfqpoint{-0.000000in}{0.000000in}}{%
\pgfpathmoveto{\pgfqpoint{-0.000000in}{0.000000in}}%
\pgfpathlineto{\pgfqpoint{-0.027778in}{0.000000in}}%
\pgfusepath{stroke,fill}%
}%
\begin{pgfscope}%
\pgfsys@transformshift{0.977263in}{1.736222in}%
\pgfsys@useobject{currentmarker}{}%
\end{pgfscope}%
\end{pgfscope}%
\begin{pgfscope}%
\pgfsetbuttcap%
\pgfsetroundjoin%
\definecolor{currentfill}{rgb}{0.000000,0.000000,0.000000}%
\pgfsetfillcolor{currentfill}%
\pgfsetlinewidth{0.602250pt}%
\definecolor{currentstroke}{rgb}{0.000000,0.000000,0.000000}%
\pgfsetstrokecolor{currentstroke}%
\pgfsetdash{}{0pt}%
\pgfsys@defobject{currentmarker}{\pgfqpoint{-0.027778in}{0.000000in}}{\pgfqpoint{-0.000000in}{0.000000in}}{%
\pgfpathmoveto{\pgfqpoint{-0.000000in}{0.000000in}}%
\pgfpathlineto{\pgfqpoint{-0.027778in}{0.000000in}}%
\pgfusepath{stroke,fill}%
}%
\begin{pgfscope}%
\pgfsys@transformshift{0.977263in}{1.766103in}%
\pgfsys@useobject{currentmarker}{}%
\end{pgfscope}%
\end{pgfscope}%
\begin{pgfscope}%
\pgfsetbuttcap%
\pgfsetroundjoin%
\definecolor{currentfill}{rgb}{0.000000,0.000000,0.000000}%
\pgfsetfillcolor{currentfill}%
\pgfsetlinewidth{0.602250pt}%
\definecolor{currentstroke}{rgb}{0.000000,0.000000,0.000000}%
\pgfsetstrokecolor{currentstroke}%
\pgfsetdash{}{0pt}%
\pgfsys@defobject{currentmarker}{\pgfqpoint{-0.027778in}{0.000000in}}{\pgfqpoint{-0.000000in}{0.000000in}}{%
\pgfpathmoveto{\pgfqpoint{-0.000000in}{0.000000in}}%
\pgfpathlineto{\pgfqpoint{-0.027778in}{0.000000in}}%
\pgfusepath{stroke,fill}%
}%
\begin{pgfscope}%
\pgfsys@transformshift{0.977263in}{1.792460in}%
\pgfsys@useobject{currentmarker}{}%
\end{pgfscope}%
\end{pgfscope}%
\begin{pgfscope}%
\pgfsetbuttcap%
\pgfsetroundjoin%
\definecolor{currentfill}{rgb}{0.000000,0.000000,0.000000}%
\pgfsetfillcolor{currentfill}%
\pgfsetlinewidth{0.602250pt}%
\definecolor{currentstroke}{rgb}{0.000000,0.000000,0.000000}%
\pgfsetstrokecolor{currentstroke}%
\pgfsetdash{}{0pt}%
\pgfsys@defobject{currentmarker}{\pgfqpoint{-0.027778in}{0.000000in}}{\pgfqpoint{-0.000000in}{0.000000in}}{%
\pgfpathmoveto{\pgfqpoint{-0.000000in}{0.000000in}}%
\pgfpathlineto{\pgfqpoint{-0.027778in}{0.000000in}}%
\pgfusepath{stroke,fill}%
}%
\begin{pgfscope}%
\pgfsys@transformshift{0.977263in}{1.971147in}%
\pgfsys@useobject{currentmarker}{}%
\end{pgfscope}%
\end{pgfscope}%
\begin{pgfscope}%
\pgfsetbuttcap%
\pgfsetroundjoin%
\definecolor{currentfill}{rgb}{0.000000,0.000000,0.000000}%
\pgfsetfillcolor{currentfill}%
\pgfsetlinewidth{0.602250pt}%
\definecolor{currentstroke}{rgb}{0.000000,0.000000,0.000000}%
\pgfsetstrokecolor{currentstroke}%
\pgfsetdash{}{0pt}%
\pgfsys@defobject{currentmarker}{\pgfqpoint{-0.027778in}{0.000000in}}{\pgfqpoint{-0.000000in}{0.000000in}}{%
\pgfpathmoveto{\pgfqpoint{-0.000000in}{0.000000in}}%
\pgfpathlineto{\pgfqpoint{-0.027778in}{0.000000in}}%
\pgfusepath{stroke,fill}%
}%
\begin{pgfscope}%
\pgfsys@transformshift{0.977263in}{2.061880in}%
\pgfsys@useobject{currentmarker}{}%
\end{pgfscope}%
\end{pgfscope}%
\begin{pgfscope}%
\pgfsetbuttcap%
\pgfsetroundjoin%
\definecolor{currentfill}{rgb}{0.000000,0.000000,0.000000}%
\pgfsetfillcolor{currentfill}%
\pgfsetlinewidth{0.602250pt}%
\definecolor{currentstroke}{rgb}{0.000000,0.000000,0.000000}%
\pgfsetstrokecolor{currentstroke}%
\pgfsetdash{}{0pt}%
\pgfsys@defobject{currentmarker}{\pgfqpoint{-0.027778in}{0.000000in}}{\pgfqpoint{-0.000000in}{0.000000in}}{%
\pgfpathmoveto{\pgfqpoint{-0.000000in}{0.000000in}}%
\pgfpathlineto{\pgfqpoint{-0.027778in}{0.000000in}}%
\pgfusepath{stroke,fill}%
}%
\begin{pgfscope}%
\pgfsys@transformshift{0.977263in}{2.126256in}%
\pgfsys@useobject{currentmarker}{}%
\end{pgfscope}%
\end{pgfscope}%
\begin{pgfscope}%
\pgfsetbuttcap%
\pgfsetroundjoin%
\definecolor{currentfill}{rgb}{0.000000,0.000000,0.000000}%
\pgfsetfillcolor{currentfill}%
\pgfsetlinewidth{0.602250pt}%
\definecolor{currentstroke}{rgb}{0.000000,0.000000,0.000000}%
\pgfsetstrokecolor{currentstroke}%
\pgfsetdash{}{0pt}%
\pgfsys@defobject{currentmarker}{\pgfqpoint{-0.027778in}{0.000000in}}{\pgfqpoint{-0.000000in}{0.000000in}}{%
\pgfpathmoveto{\pgfqpoint{-0.000000in}{0.000000in}}%
\pgfpathlineto{\pgfqpoint{-0.027778in}{0.000000in}}%
\pgfusepath{stroke,fill}%
}%
\begin{pgfscope}%
\pgfsys@transformshift{0.977263in}{2.176190in}%
\pgfsys@useobject{currentmarker}{}%
\end{pgfscope}%
\end{pgfscope}%
\begin{pgfscope}%
\pgfsetbuttcap%
\pgfsetroundjoin%
\definecolor{currentfill}{rgb}{0.000000,0.000000,0.000000}%
\pgfsetfillcolor{currentfill}%
\pgfsetlinewidth{0.602250pt}%
\definecolor{currentstroke}{rgb}{0.000000,0.000000,0.000000}%
\pgfsetstrokecolor{currentstroke}%
\pgfsetdash{}{0pt}%
\pgfsys@defobject{currentmarker}{\pgfqpoint{-0.027778in}{0.000000in}}{\pgfqpoint{-0.000000in}{0.000000in}}{%
\pgfpathmoveto{\pgfqpoint{-0.000000in}{0.000000in}}%
\pgfpathlineto{\pgfqpoint{-0.027778in}{0.000000in}}%
\pgfusepath{stroke,fill}%
}%
\begin{pgfscope}%
\pgfsys@transformshift{0.977263in}{2.216989in}%
\pgfsys@useobject{currentmarker}{}%
\end{pgfscope}%
\end{pgfscope}%
\begin{pgfscope}%
\pgfsetbuttcap%
\pgfsetroundjoin%
\definecolor{currentfill}{rgb}{0.000000,0.000000,0.000000}%
\pgfsetfillcolor{currentfill}%
\pgfsetlinewidth{0.602250pt}%
\definecolor{currentstroke}{rgb}{0.000000,0.000000,0.000000}%
\pgfsetstrokecolor{currentstroke}%
\pgfsetdash{}{0pt}%
\pgfsys@defobject{currentmarker}{\pgfqpoint{-0.027778in}{0.000000in}}{\pgfqpoint{-0.000000in}{0.000000in}}{%
\pgfpathmoveto{\pgfqpoint{-0.000000in}{0.000000in}}%
\pgfpathlineto{\pgfqpoint{-0.027778in}{0.000000in}}%
\pgfusepath{stroke,fill}%
}%
\begin{pgfscope}%
\pgfsys@transformshift{0.977263in}{2.251484in}%
\pgfsys@useobject{currentmarker}{}%
\end{pgfscope}%
\end{pgfscope}%
\begin{pgfscope}%
\pgfsetbuttcap%
\pgfsetroundjoin%
\definecolor{currentfill}{rgb}{0.000000,0.000000,0.000000}%
\pgfsetfillcolor{currentfill}%
\pgfsetlinewidth{0.602250pt}%
\definecolor{currentstroke}{rgb}{0.000000,0.000000,0.000000}%
\pgfsetstrokecolor{currentstroke}%
\pgfsetdash{}{0pt}%
\pgfsys@defobject{currentmarker}{\pgfqpoint{-0.027778in}{0.000000in}}{\pgfqpoint{-0.000000in}{0.000000in}}{%
\pgfpathmoveto{\pgfqpoint{-0.000000in}{0.000000in}}%
\pgfpathlineto{\pgfqpoint{-0.027778in}{0.000000in}}%
\pgfusepath{stroke,fill}%
}%
\begin{pgfscope}%
\pgfsys@transformshift{0.977263in}{2.281365in}%
\pgfsys@useobject{currentmarker}{}%
\end{pgfscope}%
\end{pgfscope}%
\begin{pgfscope}%
\pgfsetbuttcap%
\pgfsetroundjoin%
\definecolor{currentfill}{rgb}{0.000000,0.000000,0.000000}%
\pgfsetfillcolor{currentfill}%
\pgfsetlinewidth{0.602250pt}%
\definecolor{currentstroke}{rgb}{0.000000,0.000000,0.000000}%
\pgfsetstrokecolor{currentstroke}%
\pgfsetdash{}{0pt}%
\pgfsys@defobject{currentmarker}{\pgfqpoint{-0.027778in}{0.000000in}}{\pgfqpoint{-0.000000in}{0.000000in}}{%
\pgfpathmoveto{\pgfqpoint{-0.000000in}{0.000000in}}%
\pgfpathlineto{\pgfqpoint{-0.027778in}{0.000000in}}%
\pgfusepath{stroke,fill}%
}%
\begin{pgfscope}%
\pgfsys@transformshift{0.977263in}{2.307722in}%
\pgfsys@useobject{currentmarker}{}%
\end{pgfscope}%
\end{pgfscope}%
\begin{pgfscope}%
\pgfsetbuttcap%
\pgfsetroundjoin%
\definecolor{currentfill}{rgb}{0.000000,0.000000,0.000000}%
\pgfsetfillcolor{currentfill}%
\pgfsetlinewidth{0.602250pt}%
\definecolor{currentstroke}{rgb}{0.000000,0.000000,0.000000}%
\pgfsetstrokecolor{currentstroke}%
\pgfsetdash{}{0pt}%
\pgfsys@defobject{currentmarker}{\pgfqpoint{-0.027778in}{0.000000in}}{\pgfqpoint{-0.000000in}{0.000000in}}{%
\pgfpathmoveto{\pgfqpoint{-0.000000in}{0.000000in}}%
\pgfpathlineto{\pgfqpoint{-0.027778in}{0.000000in}}%
\pgfusepath{stroke,fill}%
}%
\begin{pgfscope}%
\pgfsys@transformshift{0.977263in}{2.486408in}%
\pgfsys@useobject{currentmarker}{}%
\end{pgfscope}%
\end{pgfscope}%
\begin{pgfscope}%
\pgfsetbuttcap%
\pgfsetroundjoin%
\definecolor{currentfill}{rgb}{0.000000,0.000000,0.000000}%
\pgfsetfillcolor{currentfill}%
\pgfsetlinewidth{0.602250pt}%
\definecolor{currentstroke}{rgb}{0.000000,0.000000,0.000000}%
\pgfsetstrokecolor{currentstroke}%
\pgfsetdash{}{0pt}%
\pgfsys@defobject{currentmarker}{\pgfqpoint{-0.027778in}{0.000000in}}{\pgfqpoint{-0.000000in}{0.000000in}}{%
\pgfpathmoveto{\pgfqpoint{-0.000000in}{0.000000in}}%
\pgfpathlineto{\pgfqpoint{-0.027778in}{0.000000in}}%
\pgfusepath{stroke,fill}%
}%
\begin{pgfscope}%
\pgfsys@transformshift{0.977263in}{2.577141in}%
\pgfsys@useobject{currentmarker}{}%
\end{pgfscope}%
\end{pgfscope}%
\begin{pgfscope}%
\pgfsetbuttcap%
\pgfsetroundjoin%
\definecolor{currentfill}{rgb}{0.000000,0.000000,0.000000}%
\pgfsetfillcolor{currentfill}%
\pgfsetlinewidth{0.602250pt}%
\definecolor{currentstroke}{rgb}{0.000000,0.000000,0.000000}%
\pgfsetstrokecolor{currentstroke}%
\pgfsetdash{}{0pt}%
\pgfsys@defobject{currentmarker}{\pgfqpoint{-0.027778in}{0.000000in}}{\pgfqpoint{-0.000000in}{0.000000in}}{%
\pgfpathmoveto{\pgfqpoint{-0.000000in}{0.000000in}}%
\pgfpathlineto{\pgfqpoint{-0.027778in}{0.000000in}}%
\pgfusepath{stroke,fill}%
}%
\begin{pgfscope}%
\pgfsys@transformshift{0.977263in}{2.641517in}%
\pgfsys@useobject{currentmarker}{}%
\end{pgfscope}%
\end{pgfscope}%
\begin{pgfscope}%
\pgfsetbuttcap%
\pgfsetroundjoin%
\definecolor{currentfill}{rgb}{0.000000,0.000000,0.000000}%
\pgfsetfillcolor{currentfill}%
\pgfsetlinewidth{0.602250pt}%
\definecolor{currentstroke}{rgb}{0.000000,0.000000,0.000000}%
\pgfsetstrokecolor{currentstroke}%
\pgfsetdash{}{0pt}%
\pgfsys@defobject{currentmarker}{\pgfqpoint{-0.027778in}{0.000000in}}{\pgfqpoint{-0.000000in}{0.000000in}}{%
\pgfpathmoveto{\pgfqpoint{-0.000000in}{0.000000in}}%
\pgfpathlineto{\pgfqpoint{-0.027778in}{0.000000in}}%
\pgfusepath{stroke,fill}%
}%
\begin{pgfscope}%
\pgfsys@transformshift{0.977263in}{2.691451in}%
\pgfsys@useobject{currentmarker}{}%
\end{pgfscope}%
\end{pgfscope}%
\begin{pgfscope}%
\pgfsetbuttcap%
\pgfsetroundjoin%
\definecolor{currentfill}{rgb}{0.000000,0.000000,0.000000}%
\pgfsetfillcolor{currentfill}%
\pgfsetlinewidth{0.602250pt}%
\definecolor{currentstroke}{rgb}{0.000000,0.000000,0.000000}%
\pgfsetstrokecolor{currentstroke}%
\pgfsetdash{}{0pt}%
\pgfsys@defobject{currentmarker}{\pgfqpoint{-0.027778in}{0.000000in}}{\pgfqpoint{-0.000000in}{0.000000in}}{%
\pgfpathmoveto{\pgfqpoint{-0.000000in}{0.000000in}}%
\pgfpathlineto{\pgfqpoint{-0.027778in}{0.000000in}}%
\pgfusepath{stroke,fill}%
}%
\begin{pgfscope}%
\pgfsys@transformshift{0.977263in}{2.732250in}%
\pgfsys@useobject{currentmarker}{}%
\end{pgfscope}%
\end{pgfscope}%
\begin{pgfscope}%
\pgfsetbuttcap%
\pgfsetroundjoin%
\definecolor{currentfill}{rgb}{0.000000,0.000000,0.000000}%
\pgfsetfillcolor{currentfill}%
\pgfsetlinewidth{0.602250pt}%
\definecolor{currentstroke}{rgb}{0.000000,0.000000,0.000000}%
\pgfsetstrokecolor{currentstroke}%
\pgfsetdash{}{0pt}%
\pgfsys@defobject{currentmarker}{\pgfqpoint{-0.027778in}{0.000000in}}{\pgfqpoint{-0.000000in}{0.000000in}}{%
\pgfpathmoveto{\pgfqpoint{-0.000000in}{0.000000in}}%
\pgfpathlineto{\pgfqpoint{-0.027778in}{0.000000in}}%
\pgfusepath{stroke,fill}%
}%
\begin{pgfscope}%
\pgfsys@transformshift{0.977263in}{2.766745in}%
\pgfsys@useobject{currentmarker}{}%
\end{pgfscope}%
\end{pgfscope}%
\begin{pgfscope}%
\pgfsetbuttcap%
\pgfsetroundjoin%
\definecolor{currentfill}{rgb}{0.000000,0.000000,0.000000}%
\pgfsetfillcolor{currentfill}%
\pgfsetlinewidth{0.602250pt}%
\definecolor{currentstroke}{rgb}{0.000000,0.000000,0.000000}%
\pgfsetstrokecolor{currentstroke}%
\pgfsetdash{}{0pt}%
\pgfsys@defobject{currentmarker}{\pgfqpoint{-0.027778in}{0.000000in}}{\pgfqpoint{-0.000000in}{0.000000in}}{%
\pgfpathmoveto{\pgfqpoint{-0.000000in}{0.000000in}}%
\pgfpathlineto{\pgfqpoint{-0.027778in}{0.000000in}}%
\pgfusepath{stroke,fill}%
}%
\begin{pgfscope}%
\pgfsys@transformshift{0.977263in}{2.796626in}%
\pgfsys@useobject{currentmarker}{}%
\end{pgfscope}%
\end{pgfscope}%
\begin{pgfscope}%
\pgfsetbuttcap%
\pgfsetroundjoin%
\definecolor{currentfill}{rgb}{0.000000,0.000000,0.000000}%
\pgfsetfillcolor{currentfill}%
\pgfsetlinewidth{0.602250pt}%
\definecolor{currentstroke}{rgb}{0.000000,0.000000,0.000000}%
\pgfsetstrokecolor{currentstroke}%
\pgfsetdash{}{0pt}%
\pgfsys@defobject{currentmarker}{\pgfqpoint{-0.027778in}{0.000000in}}{\pgfqpoint{-0.000000in}{0.000000in}}{%
\pgfpathmoveto{\pgfqpoint{-0.000000in}{0.000000in}}%
\pgfpathlineto{\pgfqpoint{-0.027778in}{0.000000in}}%
\pgfusepath{stroke,fill}%
}%
\begin{pgfscope}%
\pgfsys@transformshift{0.977263in}{2.822983in}%
\pgfsys@useobject{currentmarker}{}%
\end{pgfscope}%
\end{pgfscope}%
\begin{pgfscope}%
\pgfsetbuttcap%
\pgfsetroundjoin%
\definecolor{currentfill}{rgb}{0.000000,0.000000,0.000000}%
\pgfsetfillcolor{currentfill}%
\pgfsetlinewidth{0.602250pt}%
\definecolor{currentstroke}{rgb}{0.000000,0.000000,0.000000}%
\pgfsetstrokecolor{currentstroke}%
\pgfsetdash{}{0pt}%
\pgfsys@defobject{currentmarker}{\pgfqpoint{-0.027778in}{0.000000in}}{\pgfqpoint{-0.000000in}{0.000000in}}{%
\pgfpathmoveto{\pgfqpoint{-0.000000in}{0.000000in}}%
\pgfpathlineto{\pgfqpoint{-0.027778in}{0.000000in}}%
\pgfusepath{stroke,fill}%
}%
\begin{pgfscope}%
\pgfsys@transformshift{0.977263in}{3.001669in}%
\pgfsys@useobject{currentmarker}{}%
\end{pgfscope}%
\end{pgfscope}%
\begin{pgfscope}%
\pgfsetbuttcap%
\pgfsetroundjoin%
\definecolor{currentfill}{rgb}{0.000000,0.000000,0.000000}%
\pgfsetfillcolor{currentfill}%
\pgfsetlinewidth{0.602250pt}%
\definecolor{currentstroke}{rgb}{0.000000,0.000000,0.000000}%
\pgfsetstrokecolor{currentstroke}%
\pgfsetdash{}{0pt}%
\pgfsys@defobject{currentmarker}{\pgfqpoint{-0.027778in}{0.000000in}}{\pgfqpoint{-0.000000in}{0.000000in}}{%
\pgfpathmoveto{\pgfqpoint{-0.000000in}{0.000000in}}%
\pgfpathlineto{\pgfqpoint{-0.027778in}{0.000000in}}%
\pgfusepath{stroke,fill}%
}%
\begin{pgfscope}%
\pgfsys@transformshift{0.977263in}{3.092402in}%
\pgfsys@useobject{currentmarker}{}%
\end{pgfscope}%
\end{pgfscope}%
\begin{pgfscope}%
\pgfsetbuttcap%
\pgfsetroundjoin%
\definecolor{currentfill}{rgb}{0.000000,0.000000,0.000000}%
\pgfsetfillcolor{currentfill}%
\pgfsetlinewidth{0.602250pt}%
\definecolor{currentstroke}{rgb}{0.000000,0.000000,0.000000}%
\pgfsetstrokecolor{currentstroke}%
\pgfsetdash{}{0pt}%
\pgfsys@defobject{currentmarker}{\pgfqpoint{-0.027778in}{0.000000in}}{\pgfqpoint{-0.000000in}{0.000000in}}{%
\pgfpathmoveto{\pgfqpoint{-0.000000in}{0.000000in}}%
\pgfpathlineto{\pgfqpoint{-0.027778in}{0.000000in}}%
\pgfusepath{stroke,fill}%
}%
\begin{pgfscope}%
\pgfsys@transformshift{0.977263in}{3.156779in}%
\pgfsys@useobject{currentmarker}{}%
\end{pgfscope}%
\end{pgfscope}%
\begin{pgfscope}%
\pgfsetbuttcap%
\pgfsetroundjoin%
\definecolor{currentfill}{rgb}{0.000000,0.000000,0.000000}%
\pgfsetfillcolor{currentfill}%
\pgfsetlinewidth{0.602250pt}%
\definecolor{currentstroke}{rgb}{0.000000,0.000000,0.000000}%
\pgfsetstrokecolor{currentstroke}%
\pgfsetdash{}{0pt}%
\pgfsys@defobject{currentmarker}{\pgfqpoint{-0.027778in}{0.000000in}}{\pgfqpoint{-0.000000in}{0.000000in}}{%
\pgfpathmoveto{\pgfqpoint{-0.000000in}{0.000000in}}%
\pgfpathlineto{\pgfqpoint{-0.027778in}{0.000000in}}%
\pgfusepath{stroke,fill}%
}%
\begin{pgfscope}%
\pgfsys@transformshift{0.977263in}{3.206713in}%
\pgfsys@useobject{currentmarker}{}%
\end{pgfscope}%
\end{pgfscope}%
\begin{pgfscope}%
\pgfsetbuttcap%
\pgfsetroundjoin%
\definecolor{currentfill}{rgb}{0.000000,0.000000,0.000000}%
\pgfsetfillcolor{currentfill}%
\pgfsetlinewidth{0.602250pt}%
\definecolor{currentstroke}{rgb}{0.000000,0.000000,0.000000}%
\pgfsetstrokecolor{currentstroke}%
\pgfsetdash{}{0pt}%
\pgfsys@defobject{currentmarker}{\pgfqpoint{-0.027778in}{0.000000in}}{\pgfqpoint{-0.000000in}{0.000000in}}{%
\pgfpathmoveto{\pgfqpoint{-0.000000in}{0.000000in}}%
\pgfpathlineto{\pgfqpoint{-0.027778in}{0.000000in}}%
\pgfusepath{stroke,fill}%
}%
\begin{pgfscope}%
\pgfsys@transformshift{0.977263in}{3.247512in}%
\pgfsys@useobject{currentmarker}{}%
\end{pgfscope}%
\end{pgfscope}%
\begin{pgfscope}%
\pgfsetbuttcap%
\pgfsetroundjoin%
\definecolor{currentfill}{rgb}{0.000000,0.000000,0.000000}%
\pgfsetfillcolor{currentfill}%
\pgfsetlinewidth{0.602250pt}%
\definecolor{currentstroke}{rgb}{0.000000,0.000000,0.000000}%
\pgfsetstrokecolor{currentstroke}%
\pgfsetdash{}{0pt}%
\pgfsys@defobject{currentmarker}{\pgfqpoint{-0.027778in}{0.000000in}}{\pgfqpoint{-0.000000in}{0.000000in}}{%
\pgfpathmoveto{\pgfqpoint{-0.000000in}{0.000000in}}%
\pgfpathlineto{\pgfqpoint{-0.027778in}{0.000000in}}%
\pgfusepath{stroke,fill}%
}%
\begin{pgfscope}%
\pgfsys@transformshift{0.977263in}{3.282007in}%
\pgfsys@useobject{currentmarker}{}%
\end{pgfscope}%
\end{pgfscope}%
\begin{pgfscope}%
\pgfsetbuttcap%
\pgfsetroundjoin%
\definecolor{currentfill}{rgb}{0.000000,0.000000,0.000000}%
\pgfsetfillcolor{currentfill}%
\pgfsetlinewidth{0.602250pt}%
\definecolor{currentstroke}{rgb}{0.000000,0.000000,0.000000}%
\pgfsetstrokecolor{currentstroke}%
\pgfsetdash{}{0pt}%
\pgfsys@defobject{currentmarker}{\pgfqpoint{-0.027778in}{0.000000in}}{\pgfqpoint{-0.000000in}{0.000000in}}{%
\pgfpathmoveto{\pgfqpoint{-0.000000in}{0.000000in}}%
\pgfpathlineto{\pgfqpoint{-0.027778in}{0.000000in}}%
\pgfusepath{stroke,fill}%
}%
\begin{pgfscope}%
\pgfsys@transformshift{0.977263in}{3.311888in}%
\pgfsys@useobject{currentmarker}{}%
\end{pgfscope}%
\end{pgfscope}%
\begin{pgfscope}%
\pgfsetbuttcap%
\pgfsetroundjoin%
\definecolor{currentfill}{rgb}{0.000000,0.000000,0.000000}%
\pgfsetfillcolor{currentfill}%
\pgfsetlinewidth{0.602250pt}%
\definecolor{currentstroke}{rgb}{0.000000,0.000000,0.000000}%
\pgfsetstrokecolor{currentstroke}%
\pgfsetdash{}{0pt}%
\pgfsys@defobject{currentmarker}{\pgfqpoint{-0.027778in}{0.000000in}}{\pgfqpoint{-0.000000in}{0.000000in}}{%
\pgfpathmoveto{\pgfqpoint{-0.000000in}{0.000000in}}%
\pgfpathlineto{\pgfqpoint{-0.027778in}{0.000000in}}%
\pgfusepath{stroke,fill}%
}%
\begin{pgfscope}%
\pgfsys@transformshift{0.977263in}{3.338245in}%
\pgfsys@useobject{currentmarker}{}%
\end{pgfscope}%
\end{pgfscope}%
\begin{pgfscope}%
\pgfsetbuttcap%
\pgfsetroundjoin%
\definecolor{currentfill}{rgb}{0.000000,0.000000,0.000000}%
\pgfsetfillcolor{currentfill}%
\pgfsetlinewidth{0.602250pt}%
\definecolor{currentstroke}{rgb}{0.000000,0.000000,0.000000}%
\pgfsetstrokecolor{currentstroke}%
\pgfsetdash{}{0pt}%
\pgfsys@defobject{currentmarker}{\pgfqpoint{-0.027778in}{0.000000in}}{\pgfqpoint{-0.000000in}{0.000000in}}{%
\pgfpathmoveto{\pgfqpoint{-0.000000in}{0.000000in}}%
\pgfpathlineto{\pgfqpoint{-0.027778in}{0.000000in}}%
\pgfusepath{stroke,fill}%
}%
\begin{pgfscope}%
\pgfsys@transformshift{0.977263in}{3.516931in}%
\pgfsys@useobject{currentmarker}{}%
\end{pgfscope}%
\end{pgfscope}%
\begin{pgfscope}%
\pgfsetbuttcap%
\pgfsetroundjoin%
\definecolor{currentfill}{rgb}{0.000000,0.000000,0.000000}%
\pgfsetfillcolor{currentfill}%
\pgfsetlinewidth{0.602250pt}%
\definecolor{currentstroke}{rgb}{0.000000,0.000000,0.000000}%
\pgfsetstrokecolor{currentstroke}%
\pgfsetdash{}{0pt}%
\pgfsys@defobject{currentmarker}{\pgfqpoint{-0.027778in}{0.000000in}}{\pgfqpoint{-0.000000in}{0.000000in}}{%
\pgfpathmoveto{\pgfqpoint{-0.000000in}{0.000000in}}%
\pgfpathlineto{\pgfqpoint{-0.027778in}{0.000000in}}%
\pgfusepath{stroke,fill}%
}%
\begin{pgfscope}%
\pgfsys@transformshift{0.977263in}{3.607664in}%
\pgfsys@useobject{currentmarker}{}%
\end{pgfscope}%
\end{pgfscope}%
\begin{pgfscope}%
\pgfsetbuttcap%
\pgfsetroundjoin%
\definecolor{currentfill}{rgb}{0.000000,0.000000,0.000000}%
\pgfsetfillcolor{currentfill}%
\pgfsetlinewidth{0.602250pt}%
\definecolor{currentstroke}{rgb}{0.000000,0.000000,0.000000}%
\pgfsetstrokecolor{currentstroke}%
\pgfsetdash{}{0pt}%
\pgfsys@defobject{currentmarker}{\pgfqpoint{-0.027778in}{0.000000in}}{\pgfqpoint{-0.000000in}{0.000000in}}{%
\pgfpathmoveto{\pgfqpoint{-0.000000in}{0.000000in}}%
\pgfpathlineto{\pgfqpoint{-0.027778in}{0.000000in}}%
\pgfusepath{stroke,fill}%
}%
\begin{pgfscope}%
\pgfsys@transformshift{0.977263in}{3.672040in}%
\pgfsys@useobject{currentmarker}{}%
\end{pgfscope}%
\end{pgfscope}%
\begin{pgfscope}%
\pgfsetbuttcap%
\pgfsetroundjoin%
\definecolor{currentfill}{rgb}{0.000000,0.000000,0.000000}%
\pgfsetfillcolor{currentfill}%
\pgfsetlinewidth{0.602250pt}%
\definecolor{currentstroke}{rgb}{0.000000,0.000000,0.000000}%
\pgfsetstrokecolor{currentstroke}%
\pgfsetdash{}{0pt}%
\pgfsys@defobject{currentmarker}{\pgfqpoint{-0.027778in}{0.000000in}}{\pgfqpoint{-0.000000in}{0.000000in}}{%
\pgfpathmoveto{\pgfqpoint{-0.000000in}{0.000000in}}%
\pgfpathlineto{\pgfqpoint{-0.027778in}{0.000000in}}%
\pgfusepath{stroke,fill}%
}%
\begin{pgfscope}%
\pgfsys@transformshift{0.977263in}{3.721974in}%
\pgfsys@useobject{currentmarker}{}%
\end{pgfscope}%
\end{pgfscope}%
\begin{pgfscope}%
\pgfsetbuttcap%
\pgfsetroundjoin%
\definecolor{currentfill}{rgb}{0.000000,0.000000,0.000000}%
\pgfsetfillcolor{currentfill}%
\pgfsetlinewidth{0.602250pt}%
\definecolor{currentstroke}{rgb}{0.000000,0.000000,0.000000}%
\pgfsetstrokecolor{currentstroke}%
\pgfsetdash{}{0pt}%
\pgfsys@defobject{currentmarker}{\pgfqpoint{-0.027778in}{0.000000in}}{\pgfqpoint{-0.000000in}{0.000000in}}{%
\pgfpathmoveto{\pgfqpoint{-0.000000in}{0.000000in}}%
\pgfpathlineto{\pgfqpoint{-0.027778in}{0.000000in}}%
\pgfusepath{stroke,fill}%
}%
\begin{pgfscope}%
\pgfsys@transformshift{0.977263in}{3.762773in}%
\pgfsys@useobject{currentmarker}{}%
\end{pgfscope}%
\end{pgfscope}%
\begin{pgfscope}%
\pgfsetbuttcap%
\pgfsetroundjoin%
\definecolor{currentfill}{rgb}{0.000000,0.000000,0.000000}%
\pgfsetfillcolor{currentfill}%
\pgfsetlinewidth{0.602250pt}%
\definecolor{currentstroke}{rgb}{0.000000,0.000000,0.000000}%
\pgfsetstrokecolor{currentstroke}%
\pgfsetdash{}{0pt}%
\pgfsys@defobject{currentmarker}{\pgfqpoint{-0.027778in}{0.000000in}}{\pgfqpoint{-0.000000in}{0.000000in}}{%
\pgfpathmoveto{\pgfqpoint{-0.000000in}{0.000000in}}%
\pgfpathlineto{\pgfqpoint{-0.027778in}{0.000000in}}%
\pgfusepath{stroke,fill}%
}%
\begin{pgfscope}%
\pgfsys@transformshift{0.977263in}{3.797268in}%
\pgfsys@useobject{currentmarker}{}%
\end{pgfscope}%
\end{pgfscope}%
\begin{pgfscope}%
\pgfsetbuttcap%
\pgfsetroundjoin%
\definecolor{currentfill}{rgb}{0.000000,0.000000,0.000000}%
\pgfsetfillcolor{currentfill}%
\pgfsetlinewidth{0.602250pt}%
\definecolor{currentstroke}{rgb}{0.000000,0.000000,0.000000}%
\pgfsetstrokecolor{currentstroke}%
\pgfsetdash{}{0pt}%
\pgfsys@defobject{currentmarker}{\pgfqpoint{-0.027778in}{0.000000in}}{\pgfqpoint{-0.000000in}{0.000000in}}{%
\pgfpathmoveto{\pgfqpoint{-0.000000in}{0.000000in}}%
\pgfpathlineto{\pgfqpoint{-0.027778in}{0.000000in}}%
\pgfusepath{stroke,fill}%
}%
\begin{pgfscope}%
\pgfsys@transformshift{0.977263in}{3.827149in}%
\pgfsys@useobject{currentmarker}{}%
\end{pgfscope}%
\end{pgfscope}%
\begin{pgfscope}%
\pgfsetbuttcap%
\pgfsetroundjoin%
\definecolor{currentfill}{rgb}{0.000000,0.000000,0.000000}%
\pgfsetfillcolor{currentfill}%
\pgfsetlinewidth{0.602250pt}%
\definecolor{currentstroke}{rgb}{0.000000,0.000000,0.000000}%
\pgfsetstrokecolor{currentstroke}%
\pgfsetdash{}{0pt}%
\pgfsys@defobject{currentmarker}{\pgfqpoint{-0.027778in}{0.000000in}}{\pgfqpoint{-0.000000in}{0.000000in}}{%
\pgfpathmoveto{\pgfqpoint{-0.000000in}{0.000000in}}%
\pgfpathlineto{\pgfqpoint{-0.027778in}{0.000000in}}%
\pgfusepath{stroke,fill}%
}%
\begin{pgfscope}%
\pgfsys@transformshift{0.977263in}{3.853506in}%
\pgfsys@useobject{currentmarker}{}%
\end{pgfscope}%
\end{pgfscope}%
\begin{pgfscope}%
\pgfsetbuttcap%
\pgfsetroundjoin%
\definecolor{currentfill}{rgb}{0.000000,0.000000,0.000000}%
\pgfsetfillcolor{currentfill}%
\pgfsetlinewidth{0.602250pt}%
\definecolor{currentstroke}{rgb}{0.000000,0.000000,0.000000}%
\pgfsetstrokecolor{currentstroke}%
\pgfsetdash{}{0pt}%
\pgfsys@defobject{currentmarker}{\pgfqpoint{-0.027778in}{0.000000in}}{\pgfqpoint{-0.000000in}{0.000000in}}{%
\pgfpathmoveto{\pgfqpoint{-0.000000in}{0.000000in}}%
\pgfpathlineto{\pgfqpoint{-0.027778in}{0.000000in}}%
\pgfusepath{stroke,fill}%
}%
\begin{pgfscope}%
\pgfsys@transformshift{0.977263in}{4.032192in}%
\pgfsys@useobject{currentmarker}{}%
\end{pgfscope}%
\end{pgfscope}%
\begin{pgfscope}%
\pgfsetbuttcap%
\pgfsetroundjoin%
\definecolor{currentfill}{rgb}{0.000000,0.000000,0.000000}%
\pgfsetfillcolor{currentfill}%
\pgfsetlinewidth{0.602250pt}%
\definecolor{currentstroke}{rgb}{0.000000,0.000000,0.000000}%
\pgfsetstrokecolor{currentstroke}%
\pgfsetdash{}{0pt}%
\pgfsys@defobject{currentmarker}{\pgfqpoint{-0.027778in}{0.000000in}}{\pgfqpoint{-0.000000in}{0.000000in}}{%
\pgfpathmoveto{\pgfqpoint{-0.000000in}{0.000000in}}%
\pgfpathlineto{\pgfqpoint{-0.027778in}{0.000000in}}%
\pgfusepath{stroke,fill}%
}%
\begin{pgfscope}%
\pgfsys@transformshift{0.977263in}{4.122925in}%
\pgfsys@useobject{currentmarker}{}%
\end{pgfscope}%
\end{pgfscope}%
\begin{pgfscope}%
\pgfsetbuttcap%
\pgfsetroundjoin%
\definecolor{currentfill}{rgb}{0.000000,0.000000,0.000000}%
\pgfsetfillcolor{currentfill}%
\pgfsetlinewidth{0.602250pt}%
\definecolor{currentstroke}{rgb}{0.000000,0.000000,0.000000}%
\pgfsetstrokecolor{currentstroke}%
\pgfsetdash{}{0pt}%
\pgfsys@defobject{currentmarker}{\pgfqpoint{-0.027778in}{0.000000in}}{\pgfqpoint{-0.000000in}{0.000000in}}{%
\pgfpathmoveto{\pgfqpoint{-0.000000in}{0.000000in}}%
\pgfpathlineto{\pgfqpoint{-0.027778in}{0.000000in}}%
\pgfusepath{stroke,fill}%
}%
\begin{pgfscope}%
\pgfsys@transformshift{0.977263in}{4.187301in}%
\pgfsys@useobject{currentmarker}{}%
\end{pgfscope}%
\end{pgfscope}%
\begin{pgfscope}%
\pgfsetbuttcap%
\pgfsetroundjoin%
\definecolor{currentfill}{rgb}{0.000000,0.000000,0.000000}%
\pgfsetfillcolor{currentfill}%
\pgfsetlinewidth{0.602250pt}%
\definecolor{currentstroke}{rgb}{0.000000,0.000000,0.000000}%
\pgfsetstrokecolor{currentstroke}%
\pgfsetdash{}{0pt}%
\pgfsys@defobject{currentmarker}{\pgfqpoint{-0.027778in}{0.000000in}}{\pgfqpoint{-0.000000in}{0.000000in}}{%
\pgfpathmoveto{\pgfqpoint{-0.000000in}{0.000000in}}%
\pgfpathlineto{\pgfqpoint{-0.027778in}{0.000000in}}%
\pgfusepath{stroke,fill}%
}%
\begin{pgfscope}%
\pgfsys@transformshift{0.977263in}{4.237235in}%
\pgfsys@useobject{currentmarker}{}%
\end{pgfscope}%
\end{pgfscope}%
\begin{pgfscope}%
\pgfsetbuttcap%
\pgfsetroundjoin%
\definecolor{currentfill}{rgb}{0.000000,0.000000,0.000000}%
\pgfsetfillcolor{currentfill}%
\pgfsetlinewidth{0.602250pt}%
\definecolor{currentstroke}{rgb}{0.000000,0.000000,0.000000}%
\pgfsetstrokecolor{currentstroke}%
\pgfsetdash{}{0pt}%
\pgfsys@defobject{currentmarker}{\pgfqpoint{-0.027778in}{0.000000in}}{\pgfqpoint{-0.000000in}{0.000000in}}{%
\pgfpathmoveto{\pgfqpoint{-0.000000in}{0.000000in}}%
\pgfpathlineto{\pgfqpoint{-0.027778in}{0.000000in}}%
\pgfusepath{stroke,fill}%
}%
\begin{pgfscope}%
\pgfsys@transformshift{0.977263in}{4.278034in}%
\pgfsys@useobject{currentmarker}{}%
\end{pgfscope}%
\end{pgfscope}%
\begin{pgfscope}%
\pgfsetbuttcap%
\pgfsetroundjoin%
\definecolor{currentfill}{rgb}{0.000000,0.000000,0.000000}%
\pgfsetfillcolor{currentfill}%
\pgfsetlinewidth{0.602250pt}%
\definecolor{currentstroke}{rgb}{0.000000,0.000000,0.000000}%
\pgfsetstrokecolor{currentstroke}%
\pgfsetdash{}{0pt}%
\pgfsys@defobject{currentmarker}{\pgfqpoint{-0.027778in}{0.000000in}}{\pgfqpoint{-0.000000in}{0.000000in}}{%
\pgfpathmoveto{\pgfqpoint{-0.000000in}{0.000000in}}%
\pgfpathlineto{\pgfqpoint{-0.027778in}{0.000000in}}%
\pgfusepath{stroke,fill}%
}%
\begin{pgfscope}%
\pgfsys@transformshift{0.977263in}{4.312530in}%
\pgfsys@useobject{currentmarker}{}%
\end{pgfscope}%
\end{pgfscope}%
\begin{pgfscope}%
\pgfsetbuttcap%
\pgfsetroundjoin%
\definecolor{currentfill}{rgb}{0.000000,0.000000,0.000000}%
\pgfsetfillcolor{currentfill}%
\pgfsetlinewidth{0.602250pt}%
\definecolor{currentstroke}{rgb}{0.000000,0.000000,0.000000}%
\pgfsetstrokecolor{currentstroke}%
\pgfsetdash{}{0pt}%
\pgfsys@defobject{currentmarker}{\pgfqpoint{-0.027778in}{0.000000in}}{\pgfqpoint{-0.000000in}{0.000000in}}{%
\pgfpathmoveto{\pgfqpoint{-0.000000in}{0.000000in}}%
\pgfpathlineto{\pgfqpoint{-0.027778in}{0.000000in}}%
\pgfusepath{stroke,fill}%
}%
\begin{pgfscope}%
\pgfsys@transformshift{0.977263in}{4.342411in}%
\pgfsys@useobject{currentmarker}{}%
\end{pgfscope}%
\end{pgfscope}%
\begin{pgfscope}%
\pgfsetbuttcap%
\pgfsetroundjoin%
\definecolor{currentfill}{rgb}{0.000000,0.000000,0.000000}%
\pgfsetfillcolor{currentfill}%
\pgfsetlinewidth{0.602250pt}%
\definecolor{currentstroke}{rgb}{0.000000,0.000000,0.000000}%
\pgfsetstrokecolor{currentstroke}%
\pgfsetdash{}{0pt}%
\pgfsys@defobject{currentmarker}{\pgfqpoint{-0.027778in}{0.000000in}}{\pgfqpoint{-0.000000in}{0.000000in}}{%
\pgfpathmoveto{\pgfqpoint{-0.000000in}{0.000000in}}%
\pgfpathlineto{\pgfqpoint{-0.027778in}{0.000000in}}%
\pgfusepath{stroke,fill}%
}%
\begin{pgfscope}%
\pgfsys@transformshift{0.977263in}{4.368767in}%
\pgfsys@useobject{currentmarker}{}%
\end{pgfscope}%
\end{pgfscope}%
\begin{pgfscope}%
\pgfsetbuttcap%
\pgfsetroundjoin%
\definecolor{currentfill}{rgb}{0.000000,0.000000,0.000000}%
\pgfsetfillcolor{currentfill}%
\pgfsetlinewidth{0.602250pt}%
\definecolor{currentstroke}{rgb}{0.000000,0.000000,0.000000}%
\pgfsetstrokecolor{currentstroke}%
\pgfsetdash{}{0pt}%
\pgfsys@defobject{currentmarker}{\pgfqpoint{-0.027778in}{0.000000in}}{\pgfqpoint{-0.000000in}{0.000000in}}{%
\pgfpathmoveto{\pgfqpoint{-0.000000in}{0.000000in}}%
\pgfpathlineto{\pgfqpoint{-0.027778in}{0.000000in}}%
\pgfusepath{stroke,fill}%
}%
\begin{pgfscope}%
\pgfsys@transformshift{0.977263in}{4.547454in}%
\pgfsys@useobject{currentmarker}{}%
\end{pgfscope}%
\end{pgfscope}%
\begin{pgfscope}%
\definecolor{textcolor}{rgb}{0.000000,0.000000,0.000000}%
\pgfsetstrokecolor{textcolor}%
\pgfsetfillcolor{textcolor}%
\pgftext[x=0.342038in,y=2.562510in,,bottom]{\color{textcolor}\rmfamily\fontsize{10.000000}{12.000000}\selectfont \(\displaystyle \frac{||\mathbf{r}_k||_2}{||\mathbf{r}_0||_2}\)}%
\end{pgfscope}%
\begin{pgfscope}%
\pgfpathrectangle{\pgfqpoint{0.977263in}{0.549691in}}{\pgfqpoint{5.365101in}{4.025637in}}%
\pgfusepath{clip}%
\pgfsetrectcap%
\pgfsetroundjoin%
\pgfsetlinewidth{1.505625pt}%
\definecolor{currentstroke}{rgb}{0.121569,0.466667,0.705882}%
\pgfsetstrokecolor{currentstroke}%
\pgfsetdash{}{0pt}%
\pgfpathmoveto{\pgfqpoint{0.977263in}{4.392345in}}%
\pgfpathlineto{\pgfqpoint{0.998598in}{4.391192in}}%
\pgfpathlineto{\pgfqpoint{1.009265in}{4.390916in}}%
\pgfpathlineto{\pgfqpoint{1.030599in}{4.379305in}}%
\pgfpathlineto{\pgfqpoint{1.041267in}{4.369644in}}%
\pgfpathlineto{\pgfqpoint{1.051934in}{4.363251in}}%
\pgfpathlineto{\pgfqpoint{1.062601in}{4.359463in}}%
\pgfpathlineto{\pgfqpoint{1.073269in}{4.350653in}}%
\pgfpathlineto{\pgfqpoint{1.105270in}{4.349688in}}%
\pgfpathlineto{\pgfqpoint{1.115938in}{4.346748in}}%
\pgfpathlineto{\pgfqpoint{1.126605in}{4.345806in}}%
\pgfpathlineto{\pgfqpoint{1.137272in}{4.340976in}}%
\pgfpathlineto{\pgfqpoint{1.147939in}{4.339724in}}%
\pgfpathlineto{\pgfqpoint{1.158607in}{4.336416in}}%
\pgfpathlineto{\pgfqpoint{1.169274in}{4.330736in}}%
\pgfpathlineto{\pgfqpoint{1.179941in}{4.326565in}}%
\pgfpathlineto{\pgfqpoint{1.190608in}{4.321186in}}%
\pgfpathlineto{\pgfqpoint{1.201276in}{4.317506in}}%
\pgfpathlineto{\pgfqpoint{1.233277in}{4.310317in}}%
\pgfpathlineto{\pgfqpoint{1.243945in}{4.305512in}}%
\pgfpathlineto{\pgfqpoint{1.254612in}{4.298144in}}%
\pgfpathlineto{\pgfqpoint{1.265279in}{4.294031in}}%
\pgfpathlineto{\pgfqpoint{1.275947in}{4.293007in}}%
\pgfpathlineto{\pgfqpoint{1.286614in}{4.290516in}}%
\pgfpathlineto{\pgfqpoint{1.307948in}{4.288403in}}%
\pgfpathlineto{\pgfqpoint{1.318616in}{4.285996in}}%
\pgfpathlineto{\pgfqpoint{1.329283in}{4.282262in}}%
\pgfpathlineto{\pgfqpoint{1.339950in}{4.280434in}}%
\pgfpathlineto{\pgfqpoint{1.350617in}{4.267284in}}%
\pgfpathlineto{\pgfqpoint{1.361285in}{4.260779in}}%
\pgfpathlineto{\pgfqpoint{1.371952in}{4.252297in}}%
\pgfpathlineto{\pgfqpoint{1.393286in}{4.243336in}}%
\pgfpathlineto{\pgfqpoint{1.425288in}{4.241822in}}%
\pgfpathlineto{\pgfqpoint{1.446623in}{4.212226in}}%
\pgfpathlineto{\pgfqpoint{1.467957in}{4.204949in}}%
\pgfpathlineto{\pgfqpoint{1.478625in}{4.203313in}}%
\pgfpathlineto{\pgfqpoint{1.521294in}{4.191105in}}%
\pgfpathlineto{\pgfqpoint{1.542628in}{4.189103in}}%
\pgfpathlineto{\pgfqpoint{1.563963in}{4.187939in}}%
\pgfpathlineto{\pgfqpoint{1.585297in}{4.181221in}}%
\pgfpathlineto{\pgfqpoint{1.606632in}{4.169192in}}%
\pgfpathlineto{\pgfqpoint{1.617299in}{4.161397in}}%
\pgfpathlineto{\pgfqpoint{1.638634in}{4.155549in}}%
\pgfpathlineto{\pgfqpoint{1.659968in}{4.152903in}}%
\pgfpathlineto{\pgfqpoint{1.691970in}{4.148035in}}%
\pgfpathlineto{\pgfqpoint{1.702637in}{4.146003in}}%
\pgfpathlineto{\pgfqpoint{1.713304in}{4.141834in}}%
\pgfpathlineto{\pgfqpoint{1.723972in}{4.139622in}}%
\pgfpathlineto{\pgfqpoint{1.734639in}{4.133531in}}%
\pgfpathlineto{\pgfqpoint{1.745306in}{4.122645in}}%
\pgfpathlineto{\pgfqpoint{1.755974in}{4.118481in}}%
\pgfpathlineto{\pgfqpoint{1.766641in}{4.116477in}}%
\pgfpathlineto{\pgfqpoint{1.798643in}{4.114345in}}%
\pgfpathlineto{\pgfqpoint{1.809310in}{4.111473in}}%
\pgfpathlineto{\pgfqpoint{1.819977in}{4.101070in}}%
\pgfpathlineto{\pgfqpoint{1.830644in}{4.097820in}}%
\pgfpathlineto{\pgfqpoint{1.841312in}{4.087396in}}%
\pgfpathlineto{\pgfqpoint{1.873313in}{4.077425in}}%
\pgfpathlineto{\pgfqpoint{1.937317in}{4.068813in}}%
\pgfpathlineto{\pgfqpoint{1.947984in}{4.061416in}}%
\pgfpathlineto{\pgfqpoint{1.958652in}{4.049062in}}%
\pgfpathlineto{\pgfqpoint{1.969319in}{4.025502in}}%
\pgfpathlineto{\pgfqpoint{1.979986in}{4.017521in}}%
\pgfpathlineto{\pgfqpoint{1.990653in}{4.012444in}}%
\pgfpathlineto{\pgfqpoint{2.011988in}{4.009459in}}%
\pgfpathlineto{\pgfqpoint{2.022655in}{4.001582in}}%
\pgfpathlineto{\pgfqpoint{2.033322in}{3.986664in}}%
\pgfpathlineto{\pgfqpoint{2.043990in}{3.984565in}}%
\pgfpathlineto{\pgfqpoint{2.065324in}{3.959729in}}%
\pgfpathlineto{\pgfqpoint{2.075991in}{3.953906in}}%
\pgfpathlineto{\pgfqpoint{2.129328in}{3.942457in}}%
\pgfpathlineto{\pgfqpoint{2.139995in}{3.936071in}}%
\pgfpathlineto{\pgfqpoint{2.150662in}{3.916995in}}%
\pgfpathlineto{\pgfqpoint{2.161330in}{3.901432in}}%
\pgfpathlineto{\pgfqpoint{2.171997in}{3.891815in}}%
\pgfpathlineto{\pgfqpoint{2.182664in}{3.884119in}}%
\pgfpathlineto{\pgfqpoint{2.193331in}{3.882988in}}%
\pgfpathlineto{\pgfqpoint{2.203999in}{3.879219in}}%
\pgfpathlineto{\pgfqpoint{2.214666in}{3.873456in}}%
\pgfpathlineto{\pgfqpoint{2.225333in}{3.871173in}}%
\pgfpathlineto{\pgfqpoint{2.236000in}{3.866247in}}%
\pgfpathlineto{\pgfqpoint{2.246668in}{3.859023in}}%
\pgfpathlineto{\pgfqpoint{2.257335in}{3.845714in}}%
\pgfpathlineto{\pgfqpoint{2.278670in}{3.836183in}}%
\pgfpathlineto{\pgfqpoint{2.289337in}{3.834932in}}%
\pgfpathlineto{\pgfqpoint{2.300004in}{3.834847in}}%
\pgfpathlineto{\pgfqpoint{2.321339in}{3.827872in}}%
\pgfpathlineto{\pgfqpoint{2.332006in}{3.823112in}}%
\pgfpathlineto{\pgfqpoint{2.342673in}{3.814389in}}%
\pgfpathlineto{\pgfqpoint{2.353340in}{3.802114in}}%
\pgfpathlineto{\pgfqpoint{2.385342in}{3.788868in}}%
\pgfpathlineto{\pgfqpoint{2.396009in}{3.786307in}}%
\pgfpathlineto{\pgfqpoint{2.417344in}{3.784607in}}%
\pgfpathlineto{\pgfqpoint{2.428011in}{3.782956in}}%
\pgfpathlineto{\pgfqpoint{2.438679in}{3.778170in}}%
\pgfpathlineto{\pgfqpoint{2.449346in}{3.761558in}}%
\pgfpathlineto{\pgfqpoint{2.460013in}{3.741837in}}%
\pgfpathlineto{\pgfqpoint{2.481348in}{3.737888in}}%
\pgfpathlineto{\pgfqpoint{2.492015in}{3.734232in}}%
\pgfpathlineto{\pgfqpoint{2.513349in}{3.730754in}}%
\pgfpathlineto{\pgfqpoint{2.534684in}{3.716458in}}%
\pgfpathlineto{\pgfqpoint{2.556018in}{3.693133in}}%
\pgfpathlineto{\pgfqpoint{2.566686in}{3.674715in}}%
\pgfpathlineto{\pgfqpoint{2.577353in}{3.667592in}}%
\pgfpathlineto{\pgfqpoint{2.588020in}{3.663966in}}%
\pgfpathlineto{\pgfqpoint{2.598687in}{3.663278in}}%
\pgfpathlineto{\pgfqpoint{2.620022in}{3.659751in}}%
\pgfpathlineto{\pgfqpoint{2.630689in}{3.647469in}}%
\pgfpathlineto{\pgfqpoint{2.641357in}{3.630965in}}%
\pgfpathlineto{\pgfqpoint{2.652024in}{3.625014in}}%
\pgfpathlineto{\pgfqpoint{2.662691in}{3.620391in}}%
\pgfpathlineto{\pgfqpoint{2.705360in}{3.607988in}}%
\pgfpathlineto{\pgfqpoint{2.716027in}{3.601657in}}%
\pgfpathlineto{\pgfqpoint{2.726695in}{3.584204in}}%
\pgfpathlineto{\pgfqpoint{2.748029in}{3.565584in}}%
\pgfpathlineto{\pgfqpoint{2.758696in}{3.559331in}}%
\pgfpathlineto{\pgfqpoint{2.780031in}{3.556696in}}%
\pgfpathlineto{\pgfqpoint{2.790698in}{3.555685in}}%
\pgfpathlineto{\pgfqpoint{2.801366in}{3.550485in}}%
\pgfpathlineto{\pgfqpoint{2.822700in}{3.533674in}}%
\pgfpathlineto{\pgfqpoint{2.833367in}{3.520343in}}%
\pgfpathlineto{\pgfqpoint{2.854702in}{3.499260in}}%
\pgfpathlineto{\pgfqpoint{2.865369in}{3.494834in}}%
\pgfpathlineto{\pgfqpoint{2.886704in}{3.492897in}}%
\pgfpathlineto{\pgfqpoint{2.897371in}{3.489964in}}%
\pgfpathlineto{\pgfqpoint{2.908038in}{3.482766in}}%
\pgfpathlineto{\pgfqpoint{2.918705in}{3.469163in}}%
\pgfpathlineto{\pgfqpoint{2.929373in}{3.463082in}}%
\pgfpathlineto{\pgfqpoint{2.940040in}{3.452734in}}%
\pgfpathlineto{\pgfqpoint{2.950707in}{3.447712in}}%
\pgfpathlineto{\pgfqpoint{2.961375in}{3.444887in}}%
\pgfpathlineto{\pgfqpoint{2.972042in}{3.440849in}}%
\pgfpathlineto{\pgfqpoint{2.993376in}{3.435471in}}%
\pgfpathlineto{\pgfqpoint{3.004044in}{3.431398in}}%
\pgfpathlineto{\pgfqpoint{3.025378in}{3.416148in}}%
\pgfpathlineto{\pgfqpoint{3.046713in}{3.389464in}}%
\pgfpathlineto{\pgfqpoint{3.068047in}{3.387912in}}%
\pgfpathlineto{\pgfqpoint{3.078714in}{3.382963in}}%
\pgfpathlineto{\pgfqpoint{3.089382in}{3.359046in}}%
\pgfpathlineto{\pgfqpoint{3.100049in}{3.350427in}}%
\pgfpathlineto{\pgfqpoint{3.110716in}{3.333943in}}%
\pgfpathlineto{\pgfqpoint{3.121383in}{3.328097in}}%
\pgfpathlineto{\pgfqpoint{3.132051in}{3.314027in}}%
\pgfpathlineto{\pgfqpoint{3.142718in}{3.309255in}}%
\pgfpathlineto{\pgfqpoint{3.174720in}{3.304832in}}%
\pgfpathlineto{\pgfqpoint{3.196054in}{3.297728in}}%
\pgfpathlineto{\pgfqpoint{3.206722in}{3.286440in}}%
\pgfpathlineto{\pgfqpoint{3.217389in}{3.271867in}}%
\pgfpathlineto{\pgfqpoint{3.228056in}{3.267057in}}%
\pgfpathlineto{\pgfqpoint{3.249391in}{3.264721in}}%
\pgfpathlineto{\pgfqpoint{3.260058in}{3.260389in}}%
\pgfpathlineto{\pgfqpoint{3.270725in}{3.258643in}}%
\pgfpathlineto{\pgfqpoint{3.281392in}{3.253694in}}%
\pgfpathlineto{\pgfqpoint{3.292060in}{3.233702in}}%
\pgfpathlineto{\pgfqpoint{3.302727in}{3.224194in}}%
\pgfpathlineto{\pgfqpoint{3.324062in}{3.216600in}}%
\pgfpathlineto{\pgfqpoint{3.345396in}{3.214578in}}%
\pgfpathlineto{\pgfqpoint{3.377398in}{3.205334in}}%
\pgfpathlineto{\pgfqpoint{3.388065in}{3.192022in}}%
\pgfpathlineto{\pgfqpoint{3.409400in}{3.181086in}}%
\pgfpathlineto{\pgfqpoint{3.420067in}{3.177989in}}%
\pgfpathlineto{\pgfqpoint{3.430734in}{3.173272in}}%
\pgfpathlineto{\pgfqpoint{3.452069in}{3.171229in}}%
\pgfpathlineto{\pgfqpoint{3.462736in}{3.165587in}}%
\pgfpathlineto{\pgfqpoint{3.473403in}{3.147115in}}%
\pgfpathlineto{\pgfqpoint{3.484071in}{3.139191in}}%
\pgfpathlineto{\pgfqpoint{3.537407in}{3.130684in}}%
\pgfpathlineto{\pgfqpoint{3.558741in}{3.124029in}}%
\pgfpathlineto{\pgfqpoint{3.569409in}{3.116307in}}%
\pgfpathlineto{\pgfqpoint{3.580076in}{3.105409in}}%
\pgfpathlineto{\pgfqpoint{3.590743in}{3.103837in}}%
\pgfpathlineto{\pgfqpoint{3.601410in}{3.099658in}}%
\pgfpathlineto{\pgfqpoint{3.622745in}{3.096208in}}%
\pgfpathlineto{\pgfqpoint{3.633412in}{3.090464in}}%
\pgfpathlineto{\pgfqpoint{3.644080in}{3.081965in}}%
\pgfpathlineto{\pgfqpoint{3.654747in}{3.065938in}}%
\pgfpathlineto{\pgfqpoint{3.665414in}{3.059538in}}%
\pgfpathlineto{\pgfqpoint{3.686749in}{3.050917in}}%
\pgfpathlineto{\pgfqpoint{3.718750in}{3.040938in}}%
\pgfpathlineto{\pgfqpoint{3.729418in}{3.036607in}}%
\pgfpathlineto{\pgfqpoint{3.740085in}{3.014966in}}%
\pgfpathlineto{\pgfqpoint{3.750752in}{2.999126in}}%
\pgfpathlineto{\pgfqpoint{3.761419in}{2.995029in}}%
\pgfpathlineto{\pgfqpoint{3.772087in}{2.992828in}}%
\pgfpathlineto{\pgfqpoint{3.782754in}{2.988937in}}%
\pgfpathlineto{\pgfqpoint{3.814756in}{2.986560in}}%
\pgfpathlineto{\pgfqpoint{3.825423in}{2.980508in}}%
\pgfpathlineto{\pgfqpoint{3.836090in}{2.963344in}}%
\pgfpathlineto{\pgfqpoint{3.857425in}{2.956424in}}%
\pgfpathlineto{\pgfqpoint{3.868092in}{2.952053in}}%
\pgfpathlineto{\pgfqpoint{3.889427in}{2.950830in}}%
\pgfpathlineto{\pgfqpoint{3.900094in}{2.946053in}}%
\pgfpathlineto{\pgfqpoint{3.910761in}{2.933002in}}%
\pgfpathlineto{\pgfqpoint{3.921428in}{2.925999in}}%
\pgfpathlineto{\pgfqpoint{3.932096in}{2.922851in}}%
\pgfpathlineto{\pgfqpoint{3.942763in}{2.909768in}}%
\pgfpathlineto{\pgfqpoint{3.953430in}{2.907127in}}%
\pgfpathlineto{\pgfqpoint{3.964097in}{2.906123in}}%
\pgfpathlineto{\pgfqpoint{3.974765in}{2.898910in}}%
\pgfpathlineto{\pgfqpoint{3.985432in}{2.893976in}}%
\pgfpathlineto{\pgfqpoint{3.996099in}{2.878129in}}%
\pgfpathlineto{\pgfqpoint{4.006767in}{2.868073in}}%
\pgfpathlineto{\pgfqpoint{4.017434in}{2.860661in}}%
\pgfpathlineto{\pgfqpoint{4.038768in}{2.853977in}}%
\pgfpathlineto{\pgfqpoint{4.070770in}{2.849417in}}%
\pgfpathlineto{\pgfqpoint{4.081437in}{2.836667in}}%
\pgfpathlineto{\pgfqpoint{4.092105in}{2.813609in}}%
\pgfpathlineto{\pgfqpoint{4.102772in}{2.810009in}}%
\pgfpathlineto{\pgfqpoint{4.113439in}{2.804700in}}%
\pgfpathlineto{\pgfqpoint{4.124106in}{2.802431in}}%
\pgfpathlineto{\pgfqpoint{4.145441in}{2.800398in}}%
\pgfpathlineto{\pgfqpoint{4.166776in}{2.780826in}}%
\pgfpathlineto{\pgfqpoint{4.177443in}{2.775066in}}%
\pgfpathlineto{\pgfqpoint{4.188110in}{2.752959in}}%
\pgfpathlineto{\pgfqpoint{4.198777in}{2.745967in}}%
\pgfpathlineto{\pgfqpoint{4.209445in}{2.743074in}}%
\pgfpathlineto{\pgfqpoint{4.230779in}{2.733332in}}%
\pgfpathlineto{\pgfqpoint{4.241446in}{2.718034in}}%
\pgfpathlineto{\pgfqpoint{4.252114in}{2.710530in}}%
\pgfpathlineto{\pgfqpoint{4.262781in}{2.704880in}}%
\pgfpathlineto{\pgfqpoint{4.273448in}{2.694489in}}%
\pgfpathlineto{\pgfqpoint{4.294783in}{2.693599in}}%
\pgfpathlineto{\pgfqpoint{4.305450in}{2.690709in}}%
\pgfpathlineto{\pgfqpoint{4.316117in}{2.666297in}}%
\pgfpathlineto{\pgfqpoint{4.326785in}{2.627404in}}%
\pgfpathlineto{\pgfqpoint{4.337452in}{2.620799in}}%
\pgfpathlineto{\pgfqpoint{4.358786in}{2.617137in}}%
\pgfpathlineto{\pgfqpoint{4.380121in}{2.600095in}}%
\pgfpathlineto{\pgfqpoint{4.390788in}{2.591242in}}%
\pgfpathlineto{\pgfqpoint{4.412123in}{2.545875in}}%
\pgfpathlineto{\pgfqpoint{4.422790in}{2.543419in}}%
\pgfpathlineto{\pgfqpoint{4.433457in}{2.539525in}}%
\pgfpathlineto{\pgfqpoint{4.444124in}{2.521131in}}%
\pgfpathlineto{\pgfqpoint{4.454792in}{2.488920in}}%
\pgfpathlineto{\pgfqpoint{4.465459in}{2.465864in}}%
\pgfpathlineto{\pgfqpoint{4.476126in}{2.454924in}}%
\pgfpathlineto{\pgfqpoint{4.497461in}{2.450081in}}%
\pgfpathlineto{\pgfqpoint{4.508128in}{2.436721in}}%
\pgfpathlineto{\pgfqpoint{4.518795in}{2.414903in}}%
\pgfpathlineto{\pgfqpoint{4.529463in}{2.411900in}}%
\pgfpathlineto{\pgfqpoint{4.540130in}{2.404365in}}%
\pgfpathlineto{\pgfqpoint{4.572132in}{2.394515in}}%
\pgfpathlineto{\pgfqpoint{4.582799in}{2.383259in}}%
\pgfpathlineto{\pgfqpoint{4.593466in}{2.349150in}}%
\pgfpathlineto{\pgfqpoint{4.604133in}{2.338096in}}%
\pgfpathlineto{\pgfqpoint{4.614801in}{2.337404in}}%
\pgfpathlineto{\pgfqpoint{4.625468in}{2.333377in}}%
\pgfpathlineto{\pgfqpoint{4.636135in}{2.323349in}}%
\pgfpathlineto{\pgfqpoint{4.668137in}{2.285598in}}%
\pgfpathlineto{\pgfqpoint{4.678804in}{2.283341in}}%
\pgfpathlineto{\pgfqpoint{4.689472in}{2.277251in}}%
\pgfpathlineto{\pgfqpoint{4.700139in}{2.240779in}}%
\pgfpathlineto{\pgfqpoint{4.721473in}{2.221031in}}%
\pgfpathlineto{\pgfqpoint{4.732141in}{2.209971in}}%
\pgfpathlineto{\pgfqpoint{4.742808in}{2.206230in}}%
\pgfpathlineto{\pgfqpoint{4.753475in}{2.200961in}}%
\pgfpathlineto{\pgfqpoint{4.764142in}{2.192012in}}%
\pgfpathlineto{\pgfqpoint{4.774810in}{2.167328in}}%
\pgfpathlineto{\pgfqpoint{4.796144in}{2.162383in}}%
\pgfpathlineto{\pgfqpoint{4.806811in}{2.161997in}}%
\pgfpathlineto{\pgfqpoint{4.817479in}{2.160279in}}%
\pgfpathlineto{\pgfqpoint{4.828146in}{2.156313in}}%
\pgfpathlineto{\pgfqpoint{4.838813in}{2.143792in}}%
\pgfpathlineto{\pgfqpoint{4.849481in}{2.127901in}}%
\pgfpathlineto{\pgfqpoint{4.870815in}{2.126208in}}%
\pgfpathlineto{\pgfqpoint{4.881482in}{2.114288in}}%
\pgfpathlineto{\pgfqpoint{4.892150in}{2.096262in}}%
\pgfpathlineto{\pgfqpoint{4.902817in}{2.081046in}}%
\pgfpathlineto{\pgfqpoint{4.913484in}{2.071125in}}%
\pgfpathlineto{\pgfqpoint{4.924151in}{2.065633in}}%
\pgfpathlineto{\pgfqpoint{4.934819in}{2.063644in}}%
\pgfpathlineto{\pgfqpoint{4.945486in}{2.052138in}}%
\pgfpathlineto{\pgfqpoint{4.956153in}{2.037051in}}%
\pgfpathlineto{\pgfqpoint{4.966820in}{2.033141in}}%
\pgfpathlineto{\pgfqpoint{4.977488in}{2.026106in}}%
\pgfpathlineto{\pgfqpoint{4.998822in}{2.022352in}}%
\pgfpathlineto{\pgfqpoint{5.009489in}{2.000054in}}%
\pgfpathlineto{\pgfqpoint{5.020157in}{1.980678in}}%
\pgfpathlineto{\pgfqpoint{5.030824in}{1.971185in}}%
\pgfpathlineto{\pgfqpoint{5.041491in}{1.965081in}}%
\pgfpathlineto{\pgfqpoint{5.052159in}{1.961803in}}%
\pgfpathlineto{\pgfqpoint{5.062826in}{1.959703in}}%
\pgfpathlineto{\pgfqpoint{5.073493in}{1.946921in}}%
\pgfpathlineto{\pgfqpoint{5.084160in}{1.922601in}}%
\pgfpathlineto{\pgfqpoint{5.094828in}{1.920175in}}%
\pgfpathlineto{\pgfqpoint{5.126829in}{1.900930in}}%
\pgfpathlineto{\pgfqpoint{5.137497in}{1.887252in}}%
\pgfpathlineto{\pgfqpoint{5.148164in}{1.859758in}}%
\pgfpathlineto{\pgfqpoint{5.158831in}{1.854879in}}%
\pgfpathlineto{\pgfqpoint{5.169498in}{1.852504in}}%
\pgfpathlineto{\pgfqpoint{5.180166in}{1.836607in}}%
\pgfpathlineto{\pgfqpoint{5.190833in}{1.816578in}}%
\pgfpathlineto{\pgfqpoint{5.212168in}{1.795320in}}%
\pgfpathlineto{\pgfqpoint{5.222835in}{1.792338in}}%
\pgfpathlineto{\pgfqpoint{5.233502in}{1.791130in}}%
\pgfpathlineto{\pgfqpoint{5.244169in}{1.755496in}}%
\pgfpathlineto{\pgfqpoint{5.254837in}{1.736095in}}%
\pgfpathlineto{\pgfqpoint{5.276171in}{1.722284in}}%
\pgfpathlineto{\pgfqpoint{5.286838in}{1.720060in}}%
\pgfpathlineto{\pgfqpoint{5.297506in}{1.695589in}}%
\pgfpathlineto{\pgfqpoint{5.308173in}{1.683131in}}%
\pgfpathlineto{\pgfqpoint{5.318840in}{1.680515in}}%
\pgfpathlineto{\pgfqpoint{5.329507in}{1.676499in}}%
\pgfpathlineto{\pgfqpoint{5.340175in}{1.670651in}}%
\pgfpathlineto{\pgfqpoint{5.350842in}{1.641231in}}%
\pgfpathlineto{\pgfqpoint{5.361509in}{1.625796in}}%
\pgfpathlineto{\pgfqpoint{5.372177in}{1.623526in}}%
\pgfpathlineto{\pgfqpoint{5.382844in}{1.616601in}}%
\pgfpathlineto{\pgfqpoint{5.393511in}{1.601789in}}%
\pgfpathlineto{\pgfqpoint{5.404178in}{1.584853in}}%
\pgfpathlineto{\pgfqpoint{5.414846in}{1.548554in}}%
\pgfpathlineto{\pgfqpoint{5.425513in}{1.547368in}}%
\pgfpathlineto{\pgfqpoint{5.436180in}{1.544165in}}%
\pgfpathlineto{\pgfqpoint{5.457515in}{1.504609in}}%
\pgfpathlineto{\pgfqpoint{5.468182in}{1.470104in}}%
\pgfpathlineto{\pgfqpoint{5.478849in}{1.468551in}}%
\pgfpathlineto{\pgfqpoint{5.489516in}{1.461073in}}%
\pgfpathlineto{\pgfqpoint{5.500184in}{1.431193in}}%
\pgfpathlineto{\pgfqpoint{5.510851in}{1.425966in}}%
\pgfpathlineto{\pgfqpoint{5.542853in}{1.416424in}}%
\pgfpathlineto{\pgfqpoint{5.553520in}{1.371852in}}%
\pgfpathlineto{\pgfqpoint{5.574855in}{1.365499in}}%
\pgfpathlineto{\pgfqpoint{5.596189in}{1.352625in}}%
\pgfpathlineto{\pgfqpoint{5.606856in}{1.326621in}}%
\pgfpathlineto{\pgfqpoint{5.617524in}{1.322693in}}%
\pgfpathlineto{\pgfqpoint{5.628191in}{1.320563in}}%
\pgfpathlineto{\pgfqpoint{5.638858in}{1.301192in}}%
\pgfpathlineto{\pgfqpoint{5.649525in}{1.287999in}}%
\pgfpathlineto{\pgfqpoint{5.660193in}{1.257684in}}%
\pgfpathlineto{\pgfqpoint{5.670860in}{1.252035in}}%
\pgfpathlineto{\pgfqpoint{5.681527in}{1.248445in}}%
\pgfpathlineto{\pgfqpoint{5.692194in}{1.213154in}}%
\pgfpathlineto{\pgfqpoint{5.702862in}{1.204662in}}%
\pgfpathlineto{\pgfqpoint{5.713529in}{1.197698in}}%
\pgfpathlineto{\pgfqpoint{5.734864in}{1.192267in}}%
\pgfpathlineto{\pgfqpoint{5.745531in}{1.171754in}}%
\pgfpathlineto{\pgfqpoint{5.756198in}{1.138786in}}%
\pgfpathlineto{\pgfqpoint{5.766865in}{1.135442in}}%
\pgfpathlineto{\pgfqpoint{5.777533in}{1.130216in}}%
\pgfpathlineto{\pgfqpoint{5.788200in}{1.126429in}}%
\pgfpathlineto{\pgfqpoint{5.798867in}{1.104706in}}%
\pgfpathlineto{\pgfqpoint{5.809534in}{1.086953in}}%
\pgfpathlineto{\pgfqpoint{5.820202in}{1.085221in}}%
\pgfpathlineto{\pgfqpoint{5.841536in}{1.077333in}}%
\pgfpathlineto{\pgfqpoint{5.852203in}{1.052163in}}%
\pgfpathlineto{\pgfqpoint{5.862871in}{1.023374in}}%
\pgfpathlineto{\pgfqpoint{5.873538in}{1.020145in}}%
\pgfpathlineto{\pgfqpoint{5.884205in}{1.011619in}}%
\pgfpathlineto{\pgfqpoint{5.894873in}{0.991355in}}%
\pgfpathlineto{\pgfqpoint{5.916207in}{0.963767in}}%
\pgfpathlineto{\pgfqpoint{5.926874in}{0.958973in}}%
\pgfpathlineto{\pgfqpoint{5.937542in}{0.943463in}}%
\pgfpathlineto{\pgfqpoint{5.948209in}{0.917008in}}%
\pgfpathlineto{\pgfqpoint{5.958876in}{0.908455in}}%
\pgfpathlineto{\pgfqpoint{5.969543in}{0.903877in}}%
\pgfpathlineto{\pgfqpoint{5.980211in}{0.901874in}}%
\pgfpathlineto{\pgfqpoint{5.990878in}{0.892868in}}%
\pgfpathlineto{\pgfqpoint{6.001545in}{0.864813in}}%
\pgfpathlineto{\pgfqpoint{6.012212in}{0.858490in}}%
\pgfpathlineto{\pgfqpoint{6.022880in}{0.853684in}}%
\pgfpathlineto{\pgfqpoint{6.033547in}{0.850237in}}%
\pgfpathlineto{\pgfqpoint{6.054882in}{0.800085in}}%
\pgfpathlineto{\pgfqpoint{6.065549in}{0.796580in}}%
\pgfpathlineto{\pgfqpoint{6.086883in}{0.778497in}}%
\pgfpathlineto{\pgfqpoint{6.086883in}{0.778497in}}%
\pgfusepath{stroke}%
\end{pgfscope}%
\begin{pgfscope}%
\pgfpathrectangle{\pgfqpoint{0.977263in}{0.549691in}}{\pgfqpoint{5.365101in}{4.025637in}}%
\pgfusepath{clip}%
\pgfsetrectcap%
\pgfsetroundjoin%
\pgfsetlinewidth{1.505625pt}%
\definecolor{currentstroke}{rgb}{1.000000,0.498039,0.054902}%
\pgfsetstrokecolor{currentstroke}%
\pgfsetdash{}{0pt}%
\pgfpathmoveto{\pgfqpoint{0.977263in}{4.392345in}}%
\pgfpathlineto{\pgfqpoint{0.987930in}{4.389598in}}%
\pgfpathlineto{\pgfqpoint{0.998598in}{4.389462in}}%
\pgfpathlineto{\pgfqpoint{1.009265in}{4.381405in}}%
\pgfpathlineto{\pgfqpoint{1.019932in}{4.275820in}}%
\pgfpathlineto{\pgfqpoint{1.030599in}{4.028417in}}%
\pgfpathlineto{\pgfqpoint{1.041267in}{4.022637in}}%
\pgfpathlineto{\pgfqpoint{1.062601in}{3.893807in}}%
\pgfpathlineto{\pgfqpoint{1.073269in}{3.849854in}}%
\pgfpathlineto{\pgfqpoint{1.083936in}{3.822195in}}%
\pgfpathlineto{\pgfqpoint{1.094603in}{3.816477in}}%
\pgfpathlineto{\pgfqpoint{1.105270in}{3.691454in}}%
\pgfpathlineto{\pgfqpoint{1.115938in}{3.670882in}}%
\pgfpathlineto{\pgfqpoint{1.126605in}{3.668679in}}%
\pgfpathlineto{\pgfqpoint{1.137272in}{3.658478in}}%
\pgfpathlineto{\pgfqpoint{1.147939in}{3.655895in}}%
\pgfpathlineto{\pgfqpoint{1.158607in}{3.563409in}}%
\pgfpathlineto{\pgfqpoint{1.169274in}{3.555511in}}%
\pgfpathlineto{\pgfqpoint{1.179941in}{3.552604in}}%
\pgfpathlineto{\pgfqpoint{1.190608in}{3.544346in}}%
\pgfpathlineto{\pgfqpoint{1.201276in}{3.538558in}}%
\pgfpathlineto{\pgfqpoint{1.211943in}{3.491864in}}%
\pgfpathlineto{\pgfqpoint{1.222610in}{3.490533in}}%
\pgfpathlineto{\pgfqpoint{1.233277in}{3.483231in}}%
\pgfpathlineto{\pgfqpoint{1.243945in}{3.437411in}}%
\pgfpathlineto{\pgfqpoint{1.254612in}{3.412438in}}%
\pgfpathlineto{\pgfqpoint{1.265279in}{3.359361in}}%
\pgfpathlineto{\pgfqpoint{1.275947in}{3.357981in}}%
\pgfpathlineto{\pgfqpoint{1.286614in}{3.354402in}}%
\pgfpathlineto{\pgfqpoint{1.297281in}{3.339496in}}%
\pgfpathlineto{\pgfqpoint{1.307948in}{3.337211in}}%
\pgfpathlineto{\pgfqpoint{1.318616in}{3.330501in}}%
\pgfpathlineto{\pgfqpoint{1.329283in}{3.329886in}}%
\pgfpathlineto{\pgfqpoint{1.339950in}{3.308049in}}%
\pgfpathlineto{\pgfqpoint{1.350617in}{3.250615in}}%
\pgfpathlineto{\pgfqpoint{1.361285in}{3.243113in}}%
\pgfpathlineto{\pgfqpoint{1.371952in}{3.222424in}}%
\pgfpathlineto{\pgfqpoint{1.382619in}{3.221321in}}%
\pgfpathlineto{\pgfqpoint{1.403954in}{3.177561in}}%
\pgfpathlineto{\pgfqpoint{1.414621in}{3.174872in}}%
\pgfpathlineto{\pgfqpoint{1.425288in}{3.169963in}}%
\pgfpathlineto{\pgfqpoint{1.435956in}{3.167686in}}%
\pgfpathlineto{\pgfqpoint{1.446623in}{3.138958in}}%
\pgfpathlineto{\pgfqpoint{1.457290in}{3.133583in}}%
\pgfpathlineto{\pgfqpoint{1.467957in}{3.130303in}}%
\pgfpathlineto{\pgfqpoint{1.489292in}{3.109649in}}%
\pgfpathlineto{\pgfqpoint{1.499959in}{3.065577in}}%
\pgfpathlineto{\pgfqpoint{1.521294in}{3.061480in}}%
\pgfpathlineto{\pgfqpoint{1.531961in}{3.028957in}}%
\pgfpathlineto{\pgfqpoint{1.542628in}{3.018814in}}%
\pgfpathlineto{\pgfqpoint{1.553295in}{3.001100in}}%
\pgfpathlineto{\pgfqpoint{1.574630in}{2.989392in}}%
\pgfpathlineto{\pgfqpoint{1.585297in}{2.946601in}}%
\pgfpathlineto{\pgfqpoint{1.606632in}{2.940447in}}%
\pgfpathlineto{\pgfqpoint{1.617299in}{2.939316in}}%
\pgfpathlineto{\pgfqpoint{1.627966in}{2.925821in}}%
\pgfpathlineto{\pgfqpoint{1.638634in}{2.908621in}}%
\pgfpathlineto{\pgfqpoint{1.649301in}{2.905483in}}%
\pgfpathlineto{\pgfqpoint{1.659968in}{2.898723in}}%
\pgfpathlineto{\pgfqpoint{1.670635in}{2.894802in}}%
\pgfpathlineto{\pgfqpoint{1.681303in}{2.854634in}}%
\pgfpathlineto{\pgfqpoint{1.691970in}{2.842371in}}%
\pgfpathlineto{\pgfqpoint{1.702637in}{2.839150in}}%
\pgfpathlineto{\pgfqpoint{1.713304in}{2.822516in}}%
\pgfpathlineto{\pgfqpoint{1.723972in}{2.815932in}}%
\pgfpathlineto{\pgfqpoint{1.734639in}{2.791586in}}%
\pgfpathlineto{\pgfqpoint{1.745306in}{2.785077in}}%
\pgfpathlineto{\pgfqpoint{1.755974in}{2.782956in}}%
\pgfpathlineto{\pgfqpoint{1.766641in}{2.760445in}}%
\pgfpathlineto{\pgfqpoint{1.777308in}{2.756102in}}%
\pgfpathlineto{\pgfqpoint{1.787975in}{2.745858in}}%
\pgfpathlineto{\pgfqpoint{1.809310in}{2.743822in}}%
\pgfpathlineto{\pgfqpoint{1.819977in}{2.724064in}}%
\pgfpathlineto{\pgfqpoint{1.830644in}{2.719726in}}%
\pgfpathlineto{\pgfqpoint{1.841312in}{2.714160in}}%
\pgfpathlineto{\pgfqpoint{1.851979in}{2.713220in}}%
\pgfpathlineto{\pgfqpoint{1.862646in}{2.707339in}}%
\pgfpathlineto{\pgfqpoint{1.873313in}{2.657516in}}%
\pgfpathlineto{\pgfqpoint{1.894648in}{2.643466in}}%
\pgfpathlineto{\pgfqpoint{1.905315in}{2.641534in}}%
\pgfpathlineto{\pgfqpoint{1.915982in}{2.631805in}}%
\pgfpathlineto{\pgfqpoint{1.926650in}{2.620357in}}%
\pgfpathlineto{\pgfqpoint{1.937317in}{2.617788in}}%
\pgfpathlineto{\pgfqpoint{1.947984in}{2.607928in}}%
\pgfpathlineto{\pgfqpoint{1.958652in}{2.600120in}}%
\pgfpathlineto{\pgfqpoint{1.969319in}{2.581819in}}%
\pgfpathlineto{\pgfqpoint{1.979986in}{2.571568in}}%
\pgfpathlineto{\pgfqpoint{1.990653in}{2.570550in}}%
\pgfpathlineto{\pgfqpoint{2.001321in}{2.550846in}}%
\pgfpathlineto{\pgfqpoint{2.011988in}{2.542265in}}%
\pgfpathlineto{\pgfqpoint{2.022655in}{2.525289in}}%
\pgfpathlineto{\pgfqpoint{2.033322in}{2.521629in}}%
\pgfpathlineto{\pgfqpoint{2.043990in}{2.520679in}}%
\pgfpathlineto{\pgfqpoint{2.054657in}{2.489075in}}%
\pgfpathlineto{\pgfqpoint{2.065324in}{2.484457in}}%
\pgfpathlineto{\pgfqpoint{2.075991in}{2.476533in}}%
\pgfpathlineto{\pgfqpoint{2.086659in}{2.474294in}}%
\pgfpathlineto{\pgfqpoint{2.097326in}{2.469752in}}%
\pgfpathlineto{\pgfqpoint{2.107993in}{2.434829in}}%
\pgfpathlineto{\pgfqpoint{2.139995in}{2.423956in}}%
\pgfpathlineto{\pgfqpoint{2.150662in}{2.415996in}}%
\pgfpathlineto{\pgfqpoint{2.161330in}{2.373555in}}%
\pgfpathlineto{\pgfqpoint{2.171997in}{2.370813in}}%
\pgfpathlineto{\pgfqpoint{2.182664in}{2.364232in}}%
\pgfpathlineto{\pgfqpoint{2.193331in}{2.360810in}}%
\pgfpathlineto{\pgfqpoint{2.203999in}{2.347618in}}%
\pgfpathlineto{\pgfqpoint{2.214666in}{2.325791in}}%
\pgfpathlineto{\pgfqpoint{2.225333in}{2.322482in}}%
\pgfpathlineto{\pgfqpoint{2.236000in}{2.298450in}}%
\pgfpathlineto{\pgfqpoint{2.246668in}{2.290931in}}%
\pgfpathlineto{\pgfqpoint{2.257335in}{2.276589in}}%
\pgfpathlineto{\pgfqpoint{2.268002in}{2.268757in}}%
\pgfpathlineto{\pgfqpoint{2.278670in}{2.265831in}}%
\pgfpathlineto{\pgfqpoint{2.289337in}{2.209591in}}%
\pgfpathlineto{\pgfqpoint{2.300004in}{2.203163in}}%
\pgfpathlineto{\pgfqpoint{2.310671in}{2.199260in}}%
\pgfpathlineto{\pgfqpoint{2.321339in}{2.196582in}}%
\pgfpathlineto{\pgfqpoint{2.332006in}{2.195123in}}%
\pgfpathlineto{\pgfqpoint{2.342673in}{2.173862in}}%
\pgfpathlineto{\pgfqpoint{2.353340in}{2.171674in}}%
\pgfpathlineto{\pgfqpoint{2.364008in}{2.167275in}}%
\pgfpathlineto{\pgfqpoint{2.385342in}{2.155539in}}%
\pgfpathlineto{\pgfqpoint{2.396009in}{2.106168in}}%
\pgfpathlineto{\pgfqpoint{2.406677in}{2.103865in}}%
\pgfpathlineto{\pgfqpoint{2.438679in}{2.084331in}}%
\pgfpathlineto{\pgfqpoint{2.449346in}{2.074450in}}%
\pgfpathlineto{\pgfqpoint{2.460013in}{2.073357in}}%
\pgfpathlineto{\pgfqpoint{2.470680in}{2.055492in}}%
\pgfpathlineto{\pgfqpoint{2.481348in}{2.040965in}}%
\pgfpathlineto{\pgfqpoint{2.492015in}{2.028616in}}%
\pgfpathlineto{\pgfqpoint{2.502682in}{2.021097in}}%
\pgfpathlineto{\pgfqpoint{2.513349in}{2.019698in}}%
\pgfpathlineto{\pgfqpoint{2.524017in}{1.995476in}}%
\pgfpathlineto{\pgfqpoint{2.534684in}{1.977904in}}%
\pgfpathlineto{\pgfqpoint{2.545351in}{1.965027in}}%
\pgfpathlineto{\pgfqpoint{2.556018in}{1.956558in}}%
\pgfpathlineto{\pgfqpoint{2.566686in}{1.953741in}}%
\pgfpathlineto{\pgfqpoint{2.577353in}{1.889966in}}%
\pgfpathlineto{\pgfqpoint{2.598687in}{1.881073in}}%
\pgfpathlineto{\pgfqpoint{2.609355in}{1.873105in}}%
\pgfpathlineto{\pgfqpoint{2.620022in}{1.867096in}}%
\pgfpathlineto{\pgfqpoint{2.630689in}{1.834237in}}%
\pgfpathlineto{\pgfqpoint{2.652024in}{1.827641in}}%
\pgfpathlineto{\pgfqpoint{2.662691in}{1.815198in}}%
\pgfpathlineto{\pgfqpoint{2.673358in}{1.805448in}}%
\pgfpathlineto{\pgfqpoint{2.684026in}{1.775624in}}%
\pgfpathlineto{\pgfqpoint{2.694693in}{1.773770in}}%
\pgfpathlineto{\pgfqpoint{2.716027in}{1.744087in}}%
\pgfpathlineto{\pgfqpoint{2.726695in}{1.738786in}}%
\pgfpathlineto{\pgfqpoint{2.737362in}{1.732240in}}%
\pgfpathlineto{\pgfqpoint{2.748029in}{1.731010in}}%
\pgfpathlineto{\pgfqpoint{2.758696in}{1.706174in}}%
\pgfpathlineto{\pgfqpoint{2.769364in}{1.692500in}}%
\pgfpathlineto{\pgfqpoint{2.780031in}{1.689958in}}%
\pgfpathlineto{\pgfqpoint{2.790698in}{1.684980in}}%
\pgfpathlineto{\pgfqpoint{2.801366in}{1.683483in}}%
\pgfpathlineto{\pgfqpoint{2.812033in}{1.653751in}}%
\pgfpathlineto{\pgfqpoint{2.822700in}{1.648578in}}%
\pgfpathlineto{\pgfqpoint{2.833367in}{1.645993in}}%
\pgfpathlineto{\pgfqpoint{2.844035in}{1.642184in}}%
\pgfpathlineto{\pgfqpoint{2.854702in}{1.639914in}}%
\pgfpathlineto{\pgfqpoint{2.865369in}{1.606943in}}%
\pgfpathlineto{\pgfqpoint{2.876036in}{1.604499in}}%
\pgfpathlineto{\pgfqpoint{2.886704in}{1.600875in}}%
\pgfpathlineto{\pgfqpoint{2.897371in}{1.586674in}}%
\pgfpathlineto{\pgfqpoint{2.908038in}{1.576888in}}%
\pgfpathlineto{\pgfqpoint{2.918705in}{1.550022in}}%
\pgfpathlineto{\pgfqpoint{2.929373in}{1.548290in}}%
\pgfpathlineto{\pgfqpoint{2.940040in}{1.544955in}}%
\pgfpathlineto{\pgfqpoint{2.950707in}{1.521053in}}%
\pgfpathlineto{\pgfqpoint{2.961375in}{1.508151in}}%
\pgfpathlineto{\pgfqpoint{2.972042in}{1.477662in}}%
\pgfpathlineto{\pgfqpoint{2.982709in}{1.475040in}}%
\pgfpathlineto{\pgfqpoint{2.993376in}{1.465749in}}%
\pgfpathlineto{\pgfqpoint{3.004044in}{1.442853in}}%
\pgfpathlineto{\pgfqpoint{3.014711in}{1.441252in}}%
\pgfpathlineto{\pgfqpoint{3.025378in}{1.438241in}}%
\pgfpathlineto{\pgfqpoint{3.036045in}{1.437230in}}%
\pgfpathlineto{\pgfqpoint{3.046713in}{1.421438in}}%
\pgfpathlineto{\pgfqpoint{3.057380in}{1.413879in}}%
\pgfpathlineto{\pgfqpoint{3.068047in}{1.411688in}}%
\pgfpathlineto{\pgfqpoint{3.078714in}{1.405283in}}%
\pgfpathlineto{\pgfqpoint{3.089382in}{1.400197in}}%
\pgfpathlineto{\pgfqpoint{3.100049in}{1.366917in}}%
\pgfpathlineto{\pgfqpoint{3.110716in}{1.364014in}}%
\pgfpathlineto{\pgfqpoint{3.121383in}{1.362610in}}%
\pgfpathlineto{\pgfqpoint{3.132051in}{1.357010in}}%
\pgfpathlineto{\pgfqpoint{3.142718in}{1.352665in}}%
\pgfpathlineto{\pgfqpoint{3.153385in}{1.329137in}}%
\pgfpathlineto{\pgfqpoint{3.174720in}{1.323293in}}%
\pgfpathlineto{\pgfqpoint{3.185387in}{1.299098in}}%
\pgfpathlineto{\pgfqpoint{3.196054in}{1.288882in}}%
\pgfpathlineto{\pgfqpoint{3.206722in}{1.263988in}}%
\pgfpathlineto{\pgfqpoint{3.217389in}{1.261988in}}%
\pgfpathlineto{\pgfqpoint{3.228056in}{1.255870in}}%
\pgfpathlineto{\pgfqpoint{3.238723in}{1.221040in}}%
\pgfpathlineto{\pgfqpoint{3.249391in}{1.216200in}}%
\pgfpathlineto{\pgfqpoint{3.260058in}{1.209222in}}%
\pgfpathlineto{\pgfqpoint{3.270725in}{1.208026in}}%
\pgfpathlineto{\pgfqpoint{3.281392in}{1.195047in}}%
\pgfpathlineto{\pgfqpoint{3.292060in}{1.167036in}}%
\pgfpathlineto{\pgfqpoint{3.302727in}{1.164639in}}%
\pgfpathlineto{\pgfqpoint{3.313394in}{1.160570in}}%
\pgfpathlineto{\pgfqpoint{3.324062in}{1.159033in}}%
\pgfpathlineto{\pgfqpoint{3.345396in}{1.136945in}}%
\pgfpathlineto{\pgfqpoint{3.356063in}{1.135817in}}%
\pgfpathlineto{\pgfqpoint{3.366731in}{1.125505in}}%
\pgfpathlineto{\pgfqpoint{3.377398in}{1.118139in}}%
\pgfpathlineto{\pgfqpoint{3.388065in}{1.095911in}}%
\pgfpathlineto{\pgfqpoint{3.398732in}{1.092478in}}%
\pgfpathlineto{\pgfqpoint{3.409400in}{1.091038in}}%
\pgfpathlineto{\pgfqpoint{3.420067in}{1.066923in}}%
\pgfpathlineto{\pgfqpoint{3.430734in}{1.059535in}}%
\pgfpathlineto{\pgfqpoint{3.441401in}{1.049079in}}%
\pgfpathlineto{\pgfqpoint{3.462736in}{1.046270in}}%
\pgfpathlineto{\pgfqpoint{3.473403in}{1.014062in}}%
\pgfpathlineto{\pgfqpoint{3.484071in}{1.007367in}}%
\pgfpathlineto{\pgfqpoint{3.494738in}{0.996595in}}%
\pgfpathlineto{\pgfqpoint{3.505405in}{0.995359in}}%
\pgfpathlineto{\pgfqpoint{3.516072in}{0.992677in}}%
\pgfpathlineto{\pgfqpoint{3.526740in}{0.963028in}}%
\pgfpathlineto{\pgfqpoint{3.537407in}{0.961625in}}%
\pgfpathlineto{\pgfqpoint{3.548074in}{0.956495in}}%
\pgfpathlineto{\pgfqpoint{3.558741in}{0.953930in}}%
\pgfpathlineto{\pgfqpoint{3.569409in}{0.940985in}}%
\pgfpathlineto{\pgfqpoint{3.580076in}{0.909012in}}%
\pgfpathlineto{\pgfqpoint{3.590743in}{0.906803in}}%
\pgfpathlineto{\pgfqpoint{3.601410in}{0.894626in}}%
\pgfpathlineto{\pgfqpoint{3.612078in}{0.884670in}}%
\pgfpathlineto{\pgfqpoint{3.622745in}{0.858679in}}%
\pgfpathlineto{\pgfqpoint{3.633412in}{0.849905in}}%
\pgfpathlineto{\pgfqpoint{3.644080in}{0.848945in}}%
\pgfpathlineto{\pgfqpoint{3.654747in}{0.826755in}}%
\pgfpathlineto{\pgfqpoint{3.665414in}{0.819180in}}%
\pgfpathlineto{\pgfqpoint{3.676081in}{0.809825in}}%
\pgfpathlineto{\pgfqpoint{3.686749in}{0.807835in}}%
\pgfpathlineto{\pgfqpoint{3.697416in}{0.807024in}}%
\pgfpathlineto{\pgfqpoint{3.708083in}{0.773711in}}%
\pgfpathlineto{\pgfqpoint{3.708083in}{0.773711in}}%
\pgfusepath{stroke}%
\end{pgfscope}%
\begin{pgfscope}%
\pgfpathrectangle{\pgfqpoint{0.977263in}{0.549691in}}{\pgfqpoint{5.365101in}{4.025637in}}%
\pgfusepath{clip}%
\pgfsetrectcap%
\pgfsetroundjoin%
\pgfsetlinewidth{1.505625pt}%
\definecolor{currentstroke}{rgb}{0.172549,0.627451,0.172549}%
\pgfsetstrokecolor{currentstroke}%
\pgfsetdash{}{0pt}%
\pgfpathmoveto{\pgfqpoint{0.977263in}{4.392345in}}%
\pgfpathlineto{\pgfqpoint{0.998598in}{4.391892in}}%
\pgfpathlineto{\pgfqpoint{1.009265in}{4.390833in}}%
\pgfpathlineto{\pgfqpoint{1.019932in}{4.382748in}}%
\pgfpathlineto{\pgfqpoint{1.030599in}{4.318406in}}%
\pgfpathlineto{\pgfqpoint{1.041267in}{4.094874in}}%
\pgfpathlineto{\pgfqpoint{1.051934in}{4.064246in}}%
\pgfpathlineto{\pgfqpoint{1.062601in}{3.900914in}}%
\pgfpathlineto{\pgfqpoint{1.073269in}{3.878137in}}%
\pgfpathlineto{\pgfqpoint{1.083936in}{3.870596in}}%
\pgfpathlineto{\pgfqpoint{1.094603in}{3.719340in}}%
\pgfpathlineto{\pgfqpoint{1.105270in}{3.699678in}}%
\pgfpathlineto{\pgfqpoint{1.126605in}{3.589016in}}%
\pgfpathlineto{\pgfqpoint{1.147939in}{3.553353in}}%
\pgfpathlineto{\pgfqpoint{1.158607in}{3.518353in}}%
\pgfpathlineto{\pgfqpoint{1.169274in}{3.478954in}}%
\pgfpathlineto{\pgfqpoint{1.179941in}{3.420348in}}%
\pgfpathlineto{\pgfqpoint{1.190608in}{3.411431in}}%
\pgfpathlineto{\pgfqpoint{1.201276in}{3.370256in}}%
\pgfpathlineto{\pgfqpoint{1.211943in}{3.366216in}}%
\pgfpathlineto{\pgfqpoint{1.222610in}{3.359864in}}%
\pgfpathlineto{\pgfqpoint{1.233277in}{3.314598in}}%
\pgfpathlineto{\pgfqpoint{1.243945in}{3.307233in}}%
\pgfpathlineto{\pgfqpoint{1.254612in}{3.255786in}}%
\pgfpathlineto{\pgfqpoint{1.265279in}{3.220206in}}%
\pgfpathlineto{\pgfqpoint{1.275947in}{3.175141in}}%
\pgfpathlineto{\pgfqpoint{1.297281in}{3.165175in}}%
\pgfpathlineto{\pgfqpoint{1.307948in}{3.112663in}}%
\pgfpathlineto{\pgfqpoint{1.318616in}{3.106338in}}%
\pgfpathlineto{\pgfqpoint{1.339950in}{3.060796in}}%
\pgfpathlineto{\pgfqpoint{1.350617in}{3.047293in}}%
\pgfpathlineto{\pgfqpoint{1.361285in}{3.021169in}}%
\pgfpathlineto{\pgfqpoint{1.371952in}{3.016186in}}%
\pgfpathlineto{\pgfqpoint{1.382619in}{2.972604in}}%
\pgfpathlineto{\pgfqpoint{1.393286in}{2.965921in}}%
\pgfpathlineto{\pgfqpoint{1.403954in}{2.930440in}}%
\pgfpathlineto{\pgfqpoint{1.414621in}{2.926744in}}%
\pgfpathlineto{\pgfqpoint{1.425288in}{2.919329in}}%
\pgfpathlineto{\pgfqpoint{1.435956in}{2.899856in}}%
\pgfpathlineto{\pgfqpoint{1.446623in}{2.888299in}}%
\pgfpathlineto{\pgfqpoint{1.457290in}{2.818512in}}%
\pgfpathlineto{\pgfqpoint{1.467957in}{2.785045in}}%
\pgfpathlineto{\pgfqpoint{1.478625in}{2.766846in}}%
\pgfpathlineto{\pgfqpoint{1.489292in}{2.755011in}}%
\pgfpathlineto{\pgfqpoint{1.499959in}{2.747765in}}%
\pgfpathlineto{\pgfqpoint{1.510626in}{2.714868in}}%
\pgfpathlineto{\pgfqpoint{1.521294in}{2.707954in}}%
\pgfpathlineto{\pgfqpoint{1.531961in}{2.663482in}}%
\pgfpathlineto{\pgfqpoint{1.542628in}{2.657573in}}%
\pgfpathlineto{\pgfqpoint{1.553295in}{2.647042in}}%
\pgfpathlineto{\pgfqpoint{1.563963in}{2.613580in}}%
\pgfpathlineto{\pgfqpoint{1.574630in}{2.601672in}}%
\pgfpathlineto{\pgfqpoint{1.585297in}{2.586635in}}%
\pgfpathlineto{\pgfqpoint{1.595965in}{2.582588in}}%
\pgfpathlineto{\pgfqpoint{1.606632in}{2.559511in}}%
\pgfpathlineto{\pgfqpoint{1.617299in}{2.549276in}}%
\pgfpathlineto{\pgfqpoint{1.627966in}{2.498156in}}%
\pgfpathlineto{\pgfqpoint{1.638634in}{2.492195in}}%
\pgfpathlineto{\pgfqpoint{1.649301in}{2.483428in}}%
\pgfpathlineto{\pgfqpoint{1.659968in}{2.430853in}}%
\pgfpathlineto{\pgfqpoint{1.670635in}{2.421717in}}%
\pgfpathlineto{\pgfqpoint{1.681303in}{2.383117in}}%
\pgfpathlineto{\pgfqpoint{1.691970in}{2.380036in}}%
\pgfpathlineto{\pgfqpoint{1.702637in}{2.357952in}}%
\pgfpathlineto{\pgfqpoint{1.713304in}{2.341213in}}%
\pgfpathlineto{\pgfqpoint{1.723972in}{2.276435in}}%
\pgfpathlineto{\pgfqpoint{1.734639in}{2.245958in}}%
\pgfpathlineto{\pgfqpoint{1.745306in}{2.239299in}}%
\pgfpathlineto{\pgfqpoint{1.755974in}{2.204537in}}%
\pgfpathlineto{\pgfqpoint{1.766641in}{2.197907in}}%
\pgfpathlineto{\pgfqpoint{1.777308in}{2.166193in}}%
\pgfpathlineto{\pgfqpoint{1.787975in}{2.157691in}}%
\pgfpathlineto{\pgfqpoint{1.798643in}{2.145818in}}%
\pgfpathlineto{\pgfqpoint{1.809310in}{2.109451in}}%
\pgfpathlineto{\pgfqpoint{1.819977in}{2.092150in}}%
\pgfpathlineto{\pgfqpoint{1.830644in}{2.071475in}}%
\pgfpathlineto{\pgfqpoint{1.841312in}{2.064038in}}%
\pgfpathlineto{\pgfqpoint{1.851979in}{2.038819in}}%
\pgfpathlineto{\pgfqpoint{1.862646in}{2.032977in}}%
\pgfpathlineto{\pgfqpoint{1.873313in}{2.024565in}}%
\pgfpathlineto{\pgfqpoint{1.883981in}{1.966771in}}%
\pgfpathlineto{\pgfqpoint{1.894648in}{1.951892in}}%
\pgfpathlineto{\pgfqpoint{1.905315in}{1.891434in}}%
\pgfpathlineto{\pgfqpoint{1.915982in}{1.884422in}}%
\pgfpathlineto{\pgfqpoint{1.937317in}{1.823142in}}%
\pgfpathlineto{\pgfqpoint{1.947984in}{1.793618in}}%
\pgfpathlineto{\pgfqpoint{1.958652in}{1.772339in}}%
\pgfpathlineto{\pgfqpoint{1.969319in}{1.768831in}}%
\pgfpathlineto{\pgfqpoint{1.979986in}{1.738597in}}%
\pgfpathlineto{\pgfqpoint{1.990653in}{1.732957in}}%
\pgfpathlineto{\pgfqpoint{2.001321in}{1.687305in}}%
\pgfpathlineto{\pgfqpoint{2.011988in}{1.683665in}}%
\pgfpathlineto{\pgfqpoint{2.022655in}{1.676916in}}%
\pgfpathlineto{\pgfqpoint{2.033322in}{1.652828in}}%
\pgfpathlineto{\pgfqpoint{2.043990in}{1.616809in}}%
\pgfpathlineto{\pgfqpoint{2.054657in}{1.600947in}}%
\pgfpathlineto{\pgfqpoint{2.065324in}{1.598761in}}%
\pgfpathlineto{\pgfqpoint{2.075991in}{1.548053in}}%
\pgfpathlineto{\pgfqpoint{2.086659in}{1.530887in}}%
\pgfpathlineto{\pgfqpoint{2.097326in}{1.487177in}}%
\pgfpathlineto{\pgfqpoint{2.107993in}{1.476963in}}%
\pgfpathlineto{\pgfqpoint{2.118661in}{1.468506in}}%
\pgfpathlineto{\pgfqpoint{2.129328in}{1.440580in}}%
\pgfpathlineto{\pgfqpoint{2.139995in}{1.433438in}}%
\pgfpathlineto{\pgfqpoint{2.150662in}{1.405406in}}%
\pgfpathlineto{\pgfqpoint{2.161330in}{1.397847in}}%
\pgfpathlineto{\pgfqpoint{2.171997in}{1.379000in}}%
\pgfpathlineto{\pgfqpoint{2.182664in}{1.367779in}}%
\pgfpathlineto{\pgfqpoint{2.193331in}{1.350825in}}%
\pgfpathlineto{\pgfqpoint{2.203999in}{1.338562in}}%
\pgfpathlineto{\pgfqpoint{2.214666in}{1.334763in}}%
\pgfpathlineto{\pgfqpoint{2.225333in}{1.301923in}}%
\pgfpathlineto{\pgfqpoint{2.236000in}{1.285626in}}%
\pgfpathlineto{\pgfqpoint{2.246668in}{1.204979in}}%
\pgfpathlineto{\pgfqpoint{2.268002in}{1.189028in}}%
\pgfpathlineto{\pgfqpoint{2.278670in}{1.171914in}}%
\pgfpathlineto{\pgfqpoint{2.289337in}{1.159564in}}%
\pgfpathlineto{\pgfqpoint{2.300004in}{1.134125in}}%
\pgfpathlineto{\pgfqpoint{2.310671in}{1.124174in}}%
\pgfpathlineto{\pgfqpoint{2.321339in}{1.087432in}}%
\pgfpathlineto{\pgfqpoint{2.332006in}{1.083628in}}%
\pgfpathlineto{\pgfqpoint{2.342673in}{1.068216in}}%
\pgfpathlineto{\pgfqpoint{2.353340in}{1.038317in}}%
\pgfpathlineto{\pgfqpoint{2.364008in}{1.024793in}}%
\pgfpathlineto{\pgfqpoint{2.374675in}{1.002069in}}%
\pgfpathlineto{\pgfqpoint{2.385342in}{0.990854in}}%
\pgfpathlineto{\pgfqpoint{2.396009in}{0.958142in}}%
\pgfpathlineto{\pgfqpoint{2.406677in}{0.937694in}}%
\pgfpathlineto{\pgfqpoint{2.417344in}{0.896431in}}%
\pgfpathlineto{\pgfqpoint{2.428011in}{0.881269in}}%
\pgfpathlineto{\pgfqpoint{2.438679in}{0.875080in}}%
\pgfpathlineto{\pgfqpoint{2.449346in}{0.835418in}}%
\pgfpathlineto{\pgfqpoint{2.460013in}{0.827385in}}%
\pgfpathlineto{\pgfqpoint{2.470680in}{0.803838in}}%
\pgfpathlineto{\pgfqpoint{2.481348in}{0.802052in}}%
\pgfpathlineto{\pgfqpoint{2.492015in}{0.786566in}}%
\pgfpathlineto{\pgfqpoint{2.502682in}{0.732675in}}%
\pgfpathlineto{\pgfqpoint{2.502682in}{0.732675in}}%
\pgfusepath{stroke}%
\end{pgfscope}%
\begin{pgfscope}%
\pgfpathrectangle{\pgfqpoint{0.977263in}{0.549691in}}{\pgfqpoint{5.365101in}{4.025637in}}%
\pgfusepath{clip}%
\pgfsetrectcap%
\pgfsetroundjoin%
\pgfsetlinewidth{1.505625pt}%
\definecolor{currentstroke}{rgb}{0.839216,0.152941,0.156863}%
\pgfsetstrokecolor{currentstroke}%
\pgfsetdash{}{0pt}%
\pgfpathmoveto{\pgfqpoint{0.977263in}{4.392345in}}%
\pgfpathlineto{\pgfqpoint{0.987930in}{4.360970in}}%
\pgfpathlineto{\pgfqpoint{0.998598in}{4.357662in}}%
\pgfpathlineto{\pgfqpoint{1.009265in}{4.322027in}}%
\pgfpathlineto{\pgfqpoint{1.019932in}{4.293685in}}%
\pgfpathlineto{\pgfqpoint{1.041267in}{4.263769in}}%
\pgfpathlineto{\pgfqpoint{1.051934in}{4.228105in}}%
\pgfpathlineto{\pgfqpoint{1.062601in}{4.216345in}}%
\pgfpathlineto{\pgfqpoint{1.073269in}{4.200903in}}%
\pgfpathlineto{\pgfqpoint{1.083936in}{4.168943in}}%
\pgfpathlineto{\pgfqpoint{1.105270in}{4.131771in}}%
\pgfpathlineto{\pgfqpoint{1.115938in}{4.118560in}}%
\pgfpathlineto{\pgfqpoint{1.137272in}{4.074272in}}%
\pgfpathlineto{\pgfqpoint{1.147939in}{4.053701in}}%
\pgfpathlineto{\pgfqpoint{1.158607in}{4.050576in}}%
\pgfpathlineto{\pgfqpoint{1.169274in}{4.045145in}}%
\pgfpathlineto{\pgfqpoint{1.190608in}{4.008045in}}%
\pgfpathlineto{\pgfqpoint{1.201276in}{3.995106in}}%
\pgfpathlineto{\pgfqpoint{1.222610in}{3.977923in}}%
\pgfpathlineto{\pgfqpoint{1.233277in}{3.973464in}}%
\pgfpathlineto{\pgfqpoint{1.243945in}{3.967241in}}%
\pgfpathlineto{\pgfqpoint{1.254612in}{3.959718in}}%
\pgfpathlineto{\pgfqpoint{1.265279in}{3.955362in}}%
\pgfpathlineto{\pgfqpoint{1.275947in}{3.940800in}}%
\pgfpathlineto{\pgfqpoint{1.286614in}{3.935635in}}%
\pgfpathlineto{\pgfqpoint{1.297281in}{3.928486in}}%
\pgfpathlineto{\pgfqpoint{1.307948in}{3.916556in}}%
\pgfpathlineto{\pgfqpoint{1.318616in}{3.907866in}}%
\pgfpathlineto{\pgfqpoint{1.329283in}{3.893335in}}%
\pgfpathlineto{\pgfqpoint{1.339950in}{3.884093in}}%
\pgfpathlineto{\pgfqpoint{1.350617in}{3.876561in}}%
\pgfpathlineto{\pgfqpoint{1.361285in}{3.870494in}}%
\pgfpathlineto{\pgfqpoint{1.371952in}{3.860745in}}%
\pgfpathlineto{\pgfqpoint{1.382619in}{3.853824in}}%
\pgfpathlineto{\pgfqpoint{1.393286in}{3.849104in}}%
\pgfpathlineto{\pgfqpoint{1.414621in}{3.827510in}}%
\pgfpathlineto{\pgfqpoint{1.425288in}{3.812943in}}%
\pgfpathlineto{\pgfqpoint{1.435956in}{3.803288in}}%
\pgfpathlineto{\pgfqpoint{1.446623in}{3.795823in}}%
\pgfpathlineto{\pgfqpoint{1.457290in}{3.789881in}}%
\pgfpathlineto{\pgfqpoint{1.478625in}{3.782430in}}%
\pgfpathlineto{\pgfqpoint{1.489292in}{3.775052in}}%
\pgfpathlineto{\pgfqpoint{1.499959in}{3.766204in}}%
\pgfpathlineto{\pgfqpoint{1.510626in}{3.763737in}}%
\pgfpathlineto{\pgfqpoint{1.521294in}{3.747398in}}%
\pgfpathlineto{\pgfqpoint{1.531961in}{3.739207in}}%
\pgfpathlineto{\pgfqpoint{1.542628in}{3.735276in}}%
\pgfpathlineto{\pgfqpoint{1.553295in}{3.728876in}}%
\pgfpathlineto{\pgfqpoint{1.563963in}{3.724224in}}%
\pgfpathlineto{\pgfqpoint{1.574630in}{3.721001in}}%
\pgfpathlineto{\pgfqpoint{1.585297in}{3.715124in}}%
\pgfpathlineto{\pgfqpoint{1.606632in}{3.700874in}}%
\pgfpathlineto{\pgfqpoint{1.638634in}{3.669319in}}%
\pgfpathlineto{\pgfqpoint{1.649301in}{3.663068in}}%
\pgfpathlineto{\pgfqpoint{1.659968in}{3.660771in}}%
\pgfpathlineto{\pgfqpoint{1.681303in}{3.641855in}}%
\pgfpathlineto{\pgfqpoint{1.702637in}{3.630101in}}%
\pgfpathlineto{\pgfqpoint{1.713304in}{3.626937in}}%
\pgfpathlineto{\pgfqpoint{1.745306in}{3.611927in}}%
\pgfpathlineto{\pgfqpoint{1.755974in}{3.608985in}}%
\pgfpathlineto{\pgfqpoint{1.766641in}{3.601157in}}%
\pgfpathlineto{\pgfqpoint{1.787975in}{3.588715in}}%
\pgfpathlineto{\pgfqpoint{1.798643in}{3.567125in}}%
\pgfpathlineto{\pgfqpoint{1.809310in}{3.556100in}}%
\pgfpathlineto{\pgfqpoint{1.819977in}{3.549586in}}%
\pgfpathlineto{\pgfqpoint{1.830644in}{3.545107in}}%
\pgfpathlineto{\pgfqpoint{1.841312in}{3.537897in}}%
\pgfpathlineto{\pgfqpoint{1.851979in}{3.533500in}}%
\pgfpathlineto{\pgfqpoint{1.862646in}{3.526508in}}%
\pgfpathlineto{\pgfqpoint{1.873313in}{3.522631in}}%
\pgfpathlineto{\pgfqpoint{1.894648in}{3.504988in}}%
\pgfpathlineto{\pgfqpoint{1.905315in}{3.500849in}}%
\pgfpathlineto{\pgfqpoint{1.915982in}{3.491201in}}%
\pgfpathlineto{\pgfqpoint{1.979986in}{3.469486in}}%
\pgfpathlineto{\pgfqpoint{1.990653in}{3.462036in}}%
\pgfpathlineto{\pgfqpoint{2.001321in}{3.459339in}}%
\pgfpathlineto{\pgfqpoint{2.022655in}{3.447427in}}%
\pgfpathlineto{\pgfqpoint{2.033322in}{3.442290in}}%
\pgfpathlineto{\pgfqpoint{2.043990in}{3.430718in}}%
\pgfpathlineto{\pgfqpoint{2.054657in}{3.417419in}}%
\pgfpathlineto{\pgfqpoint{2.075991in}{3.410038in}}%
\pgfpathlineto{\pgfqpoint{2.086659in}{3.402458in}}%
\pgfpathlineto{\pgfqpoint{2.097326in}{3.400750in}}%
\pgfpathlineto{\pgfqpoint{2.107993in}{3.391654in}}%
\pgfpathlineto{\pgfqpoint{2.129328in}{3.378224in}}%
\pgfpathlineto{\pgfqpoint{2.139995in}{3.369889in}}%
\pgfpathlineto{\pgfqpoint{2.150662in}{3.363287in}}%
\pgfpathlineto{\pgfqpoint{2.161330in}{3.353697in}}%
\pgfpathlineto{\pgfqpoint{2.171997in}{3.347015in}}%
\pgfpathlineto{\pgfqpoint{2.182664in}{3.342640in}}%
\pgfpathlineto{\pgfqpoint{2.193331in}{3.339506in}}%
\pgfpathlineto{\pgfqpoint{2.225333in}{3.319504in}}%
\pgfpathlineto{\pgfqpoint{2.236000in}{3.312227in}}%
\pgfpathlineto{\pgfqpoint{2.246668in}{3.309268in}}%
\pgfpathlineto{\pgfqpoint{2.257335in}{3.302709in}}%
\pgfpathlineto{\pgfqpoint{2.278670in}{3.297914in}}%
\pgfpathlineto{\pgfqpoint{2.300004in}{3.282791in}}%
\pgfpathlineto{\pgfqpoint{2.310671in}{3.278231in}}%
\pgfpathlineto{\pgfqpoint{2.321339in}{3.272438in}}%
\pgfpathlineto{\pgfqpoint{2.332006in}{3.262920in}}%
\pgfpathlineto{\pgfqpoint{2.342673in}{3.258965in}}%
\pgfpathlineto{\pgfqpoint{2.353340in}{3.252305in}}%
\pgfpathlineto{\pgfqpoint{2.364008in}{3.249946in}}%
\pgfpathlineto{\pgfqpoint{2.385342in}{3.238844in}}%
\pgfpathlineto{\pgfqpoint{2.406677in}{3.224726in}}%
\pgfpathlineto{\pgfqpoint{2.417344in}{3.219835in}}%
\pgfpathlineto{\pgfqpoint{2.428011in}{3.210773in}}%
\pgfpathlineto{\pgfqpoint{2.438679in}{3.206506in}}%
\pgfpathlineto{\pgfqpoint{2.449346in}{3.200470in}}%
\pgfpathlineto{\pgfqpoint{2.460013in}{3.196814in}}%
\pgfpathlineto{\pgfqpoint{2.470680in}{3.191318in}}%
\pgfpathlineto{\pgfqpoint{2.481348in}{3.181076in}}%
\pgfpathlineto{\pgfqpoint{2.492015in}{3.175744in}}%
\pgfpathlineto{\pgfqpoint{2.524017in}{3.152114in}}%
\pgfpathlineto{\pgfqpoint{2.534684in}{3.140236in}}%
\pgfpathlineto{\pgfqpoint{2.556018in}{3.129528in}}%
\pgfpathlineto{\pgfqpoint{2.566686in}{3.126370in}}%
\pgfpathlineto{\pgfqpoint{2.577353in}{3.117282in}}%
\pgfpathlineto{\pgfqpoint{2.588020in}{3.114471in}}%
\pgfpathlineto{\pgfqpoint{2.609355in}{3.104727in}}%
\pgfpathlineto{\pgfqpoint{2.620022in}{3.101686in}}%
\pgfpathlineto{\pgfqpoint{2.662691in}{3.083110in}}%
\pgfpathlineto{\pgfqpoint{2.673358in}{3.075835in}}%
\pgfpathlineto{\pgfqpoint{2.684026in}{3.070970in}}%
\pgfpathlineto{\pgfqpoint{2.694693in}{3.061104in}}%
\pgfpathlineto{\pgfqpoint{2.748029in}{3.033151in}}%
\pgfpathlineto{\pgfqpoint{2.758696in}{3.029346in}}%
\pgfpathlineto{\pgfqpoint{2.769364in}{3.024266in}}%
\pgfpathlineto{\pgfqpoint{2.780031in}{3.021197in}}%
\pgfpathlineto{\pgfqpoint{2.801366in}{3.000589in}}%
\pgfpathlineto{\pgfqpoint{2.812033in}{2.980328in}}%
\pgfpathlineto{\pgfqpoint{2.822700in}{2.968423in}}%
\pgfpathlineto{\pgfqpoint{2.833367in}{2.962600in}}%
\pgfpathlineto{\pgfqpoint{2.844035in}{2.952730in}}%
\pgfpathlineto{\pgfqpoint{2.865369in}{2.940310in}}%
\pgfpathlineto{\pgfqpoint{2.876036in}{2.936356in}}%
\pgfpathlineto{\pgfqpoint{2.886704in}{2.927875in}}%
\pgfpathlineto{\pgfqpoint{2.908038in}{2.915913in}}%
\pgfpathlineto{\pgfqpoint{2.918705in}{2.896523in}}%
\pgfpathlineto{\pgfqpoint{2.940040in}{2.880108in}}%
\pgfpathlineto{\pgfqpoint{2.950707in}{2.874968in}}%
\pgfpathlineto{\pgfqpoint{2.961375in}{2.872316in}}%
\pgfpathlineto{\pgfqpoint{2.972042in}{2.865834in}}%
\pgfpathlineto{\pgfqpoint{2.982709in}{2.855221in}}%
\pgfpathlineto{\pgfqpoint{2.993376in}{2.839976in}}%
\pgfpathlineto{\pgfqpoint{3.004044in}{2.835260in}}%
\pgfpathlineto{\pgfqpoint{3.014711in}{2.821335in}}%
\pgfpathlineto{\pgfqpoint{3.025378in}{2.814147in}}%
\pgfpathlineto{\pgfqpoint{3.036045in}{2.803881in}}%
\pgfpathlineto{\pgfqpoint{3.046713in}{2.798962in}}%
\pgfpathlineto{\pgfqpoint{3.068047in}{2.781012in}}%
\pgfpathlineto{\pgfqpoint{3.078714in}{2.770312in}}%
\pgfpathlineto{\pgfqpoint{3.089382in}{2.754837in}}%
\pgfpathlineto{\pgfqpoint{3.110716in}{2.728789in}}%
\pgfpathlineto{\pgfqpoint{3.142718in}{2.715966in}}%
\pgfpathlineto{\pgfqpoint{3.153385in}{2.712350in}}%
\pgfpathlineto{\pgfqpoint{3.164053in}{2.706897in}}%
\pgfpathlineto{\pgfqpoint{3.196054in}{2.694907in}}%
\pgfpathlineto{\pgfqpoint{3.228056in}{2.680059in}}%
\pgfpathlineto{\pgfqpoint{3.238723in}{2.666637in}}%
\pgfpathlineto{\pgfqpoint{3.249391in}{2.663732in}}%
\pgfpathlineto{\pgfqpoint{3.260058in}{2.657527in}}%
\pgfpathlineto{\pgfqpoint{3.281392in}{2.652554in}}%
\pgfpathlineto{\pgfqpoint{3.292060in}{2.648420in}}%
\pgfpathlineto{\pgfqpoint{3.302727in}{2.637150in}}%
\pgfpathlineto{\pgfqpoint{3.313394in}{2.629197in}}%
\pgfpathlineto{\pgfqpoint{3.334729in}{2.618344in}}%
\pgfpathlineto{\pgfqpoint{3.345396in}{2.615659in}}%
\pgfpathlineto{\pgfqpoint{3.356063in}{2.609358in}}%
\pgfpathlineto{\pgfqpoint{3.366731in}{2.605656in}}%
\pgfpathlineto{\pgfqpoint{3.377398in}{2.598909in}}%
\pgfpathlineto{\pgfqpoint{3.398732in}{2.580750in}}%
\pgfpathlineto{\pgfqpoint{3.409400in}{2.566920in}}%
\pgfpathlineto{\pgfqpoint{3.420067in}{2.562630in}}%
\pgfpathlineto{\pgfqpoint{3.430734in}{2.557025in}}%
\pgfpathlineto{\pgfqpoint{3.441401in}{2.552840in}}%
\pgfpathlineto{\pgfqpoint{3.452069in}{2.546792in}}%
\pgfpathlineto{\pgfqpoint{3.462736in}{2.543865in}}%
\pgfpathlineto{\pgfqpoint{3.473403in}{2.536643in}}%
\pgfpathlineto{\pgfqpoint{3.484071in}{2.531032in}}%
\pgfpathlineto{\pgfqpoint{3.494738in}{2.528106in}}%
\pgfpathlineto{\pgfqpoint{3.505405in}{2.522008in}}%
\pgfpathlineto{\pgfqpoint{3.526740in}{2.515078in}}%
\pgfpathlineto{\pgfqpoint{3.537407in}{2.505943in}}%
\pgfpathlineto{\pgfqpoint{3.548074in}{2.486929in}}%
\pgfpathlineto{\pgfqpoint{3.590743in}{2.448411in}}%
\pgfpathlineto{\pgfqpoint{3.601410in}{2.432998in}}%
\pgfpathlineto{\pgfqpoint{3.622745in}{2.419732in}}%
\pgfpathlineto{\pgfqpoint{3.644080in}{2.411024in}}%
\pgfpathlineto{\pgfqpoint{3.654747in}{2.406633in}}%
\pgfpathlineto{\pgfqpoint{3.676081in}{2.386685in}}%
\pgfpathlineto{\pgfqpoint{3.686749in}{2.382057in}}%
\pgfpathlineto{\pgfqpoint{3.697416in}{2.374518in}}%
\pgfpathlineto{\pgfqpoint{3.729418in}{2.368343in}}%
\pgfpathlineto{\pgfqpoint{3.740085in}{2.362780in}}%
\pgfpathlineto{\pgfqpoint{3.793421in}{2.346673in}}%
\pgfpathlineto{\pgfqpoint{3.814756in}{2.343776in}}%
\pgfpathlineto{\pgfqpoint{3.825423in}{2.336171in}}%
\pgfpathlineto{\pgfqpoint{3.836090in}{2.332655in}}%
\pgfpathlineto{\pgfqpoint{3.846758in}{2.323306in}}%
\pgfpathlineto{\pgfqpoint{3.857425in}{2.318338in}}%
\pgfpathlineto{\pgfqpoint{3.878759in}{2.310728in}}%
\pgfpathlineto{\pgfqpoint{3.889427in}{2.298193in}}%
\pgfpathlineto{\pgfqpoint{3.900094in}{2.294275in}}%
\pgfpathlineto{\pgfqpoint{3.910761in}{2.283462in}}%
\pgfpathlineto{\pgfqpoint{3.921428in}{2.276471in}}%
\pgfpathlineto{\pgfqpoint{3.932096in}{2.273551in}}%
\pgfpathlineto{\pgfqpoint{3.953430in}{2.257368in}}%
\pgfpathlineto{\pgfqpoint{3.964097in}{2.247323in}}%
\pgfpathlineto{\pgfqpoint{3.974765in}{2.233092in}}%
\pgfpathlineto{\pgfqpoint{3.985432in}{2.215790in}}%
\pgfpathlineto{\pgfqpoint{3.996099in}{2.201185in}}%
\pgfpathlineto{\pgfqpoint{4.006767in}{2.191555in}}%
\pgfpathlineto{\pgfqpoint{4.017434in}{2.184997in}}%
\pgfpathlineto{\pgfqpoint{4.028101in}{2.181858in}}%
\pgfpathlineto{\pgfqpoint{4.038768in}{2.170417in}}%
\pgfpathlineto{\pgfqpoint{4.060103in}{2.153284in}}%
\pgfpathlineto{\pgfqpoint{4.070770in}{2.141181in}}%
\pgfpathlineto{\pgfqpoint{4.092105in}{2.127510in}}%
\pgfpathlineto{\pgfqpoint{4.102772in}{2.122613in}}%
\pgfpathlineto{\pgfqpoint{4.113439in}{2.114040in}}%
\pgfpathlineto{\pgfqpoint{4.124106in}{2.109616in}}%
\pgfpathlineto{\pgfqpoint{4.134774in}{2.101707in}}%
\pgfpathlineto{\pgfqpoint{4.145441in}{2.095468in}}%
\pgfpathlineto{\pgfqpoint{4.156108in}{2.092085in}}%
\pgfpathlineto{\pgfqpoint{4.166776in}{2.084331in}}%
\pgfpathlineto{\pgfqpoint{4.177443in}{2.081727in}}%
\pgfpathlineto{\pgfqpoint{4.198777in}{2.072316in}}%
\pgfpathlineto{\pgfqpoint{4.220112in}{2.059902in}}%
\pgfpathlineto{\pgfqpoint{4.230779in}{2.052105in}}%
\pgfpathlineto{\pgfqpoint{4.252114in}{2.043898in}}%
\pgfpathlineto{\pgfqpoint{4.262781in}{2.036537in}}%
\pgfpathlineto{\pgfqpoint{4.273448in}{2.031076in}}%
\pgfpathlineto{\pgfqpoint{4.294783in}{2.007035in}}%
\pgfpathlineto{\pgfqpoint{4.305450in}{2.000189in}}%
\pgfpathlineto{\pgfqpoint{4.316117in}{1.995887in}}%
\pgfpathlineto{\pgfqpoint{4.326785in}{1.985972in}}%
\pgfpathlineto{\pgfqpoint{4.337452in}{1.972633in}}%
\pgfpathlineto{\pgfqpoint{4.348119in}{1.964366in}}%
\pgfpathlineto{\pgfqpoint{4.358786in}{1.944653in}}%
\pgfpathlineto{\pgfqpoint{4.369454in}{1.929109in}}%
\pgfpathlineto{\pgfqpoint{4.380121in}{1.917082in}}%
\pgfpathlineto{\pgfqpoint{4.401455in}{1.908533in}}%
\pgfpathlineto{\pgfqpoint{4.412123in}{1.899802in}}%
\pgfpathlineto{\pgfqpoint{4.422790in}{1.892675in}}%
\pgfpathlineto{\pgfqpoint{4.454792in}{1.860697in}}%
\pgfpathlineto{\pgfqpoint{4.465459in}{1.855349in}}%
\pgfpathlineto{\pgfqpoint{4.476126in}{1.844840in}}%
\pgfpathlineto{\pgfqpoint{4.486793in}{1.841853in}}%
\pgfpathlineto{\pgfqpoint{4.497461in}{1.836289in}}%
\pgfpathlineto{\pgfqpoint{4.508128in}{1.825855in}}%
\pgfpathlineto{\pgfqpoint{4.518795in}{1.820370in}}%
\pgfpathlineto{\pgfqpoint{4.529463in}{1.798239in}}%
\pgfpathlineto{\pgfqpoint{4.550797in}{1.778276in}}%
\pgfpathlineto{\pgfqpoint{4.561464in}{1.772856in}}%
\pgfpathlineto{\pgfqpoint{4.572132in}{1.762497in}}%
\pgfpathlineto{\pgfqpoint{4.593466in}{1.750556in}}%
\pgfpathlineto{\pgfqpoint{4.604133in}{1.741580in}}%
\pgfpathlineto{\pgfqpoint{4.614801in}{1.736321in}}%
\pgfpathlineto{\pgfqpoint{4.625468in}{1.722219in}}%
\pgfpathlineto{\pgfqpoint{4.636135in}{1.716207in}}%
\pgfpathlineto{\pgfqpoint{4.646802in}{1.702362in}}%
\pgfpathlineto{\pgfqpoint{4.668137in}{1.664627in}}%
\pgfpathlineto{\pgfqpoint{4.678804in}{1.653458in}}%
\pgfpathlineto{\pgfqpoint{4.700139in}{1.643091in}}%
\pgfpathlineto{\pgfqpoint{4.710806in}{1.638164in}}%
\pgfpathlineto{\pgfqpoint{4.721473in}{1.623433in}}%
\pgfpathlineto{\pgfqpoint{4.732141in}{1.616673in}}%
\pgfpathlineto{\pgfqpoint{4.742808in}{1.602560in}}%
\pgfpathlineto{\pgfqpoint{4.753475in}{1.583353in}}%
\pgfpathlineto{\pgfqpoint{4.764142in}{1.577796in}}%
\pgfpathlineto{\pgfqpoint{4.774810in}{1.566878in}}%
\pgfpathlineto{\pgfqpoint{4.806811in}{1.560232in}}%
\pgfpathlineto{\pgfqpoint{4.817479in}{1.548330in}}%
\pgfpathlineto{\pgfqpoint{4.828146in}{1.539595in}}%
\pgfpathlineto{\pgfqpoint{4.849481in}{1.516837in}}%
\pgfpathlineto{\pgfqpoint{4.860148in}{1.512715in}}%
\pgfpathlineto{\pgfqpoint{4.881482in}{1.500729in}}%
\pgfpathlineto{\pgfqpoint{4.892150in}{1.494552in}}%
\pgfpathlineto{\pgfqpoint{4.902817in}{1.486519in}}%
\pgfpathlineto{\pgfqpoint{4.913484in}{1.458507in}}%
\pgfpathlineto{\pgfqpoint{4.924151in}{1.451759in}}%
\pgfpathlineto{\pgfqpoint{4.945486in}{1.432889in}}%
\pgfpathlineto{\pgfqpoint{4.956153in}{1.426451in}}%
\pgfpathlineto{\pgfqpoint{4.966820in}{1.415048in}}%
\pgfpathlineto{\pgfqpoint{4.977488in}{1.412072in}}%
\pgfpathlineto{\pgfqpoint{4.988155in}{1.399907in}}%
\pgfpathlineto{\pgfqpoint{4.998822in}{1.393991in}}%
\pgfpathlineto{\pgfqpoint{5.009489in}{1.390573in}}%
\pgfpathlineto{\pgfqpoint{5.020157in}{1.381893in}}%
\pgfpathlineto{\pgfqpoint{5.030824in}{1.371288in}}%
\pgfpathlineto{\pgfqpoint{5.041491in}{1.362996in}}%
\pgfpathlineto{\pgfqpoint{5.052159in}{1.350862in}}%
\pgfpathlineto{\pgfqpoint{5.062826in}{1.341927in}}%
\pgfpathlineto{\pgfqpoint{5.073493in}{1.338615in}}%
\pgfpathlineto{\pgfqpoint{5.084160in}{1.331757in}}%
\pgfpathlineto{\pgfqpoint{5.105495in}{1.323613in}}%
\pgfpathlineto{\pgfqpoint{5.116162in}{1.315674in}}%
\pgfpathlineto{\pgfqpoint{5.126829in}{1.311331in}}%
\pgfpathlineto{\pgfqpoint{5.148164in}{1.296669in}}%
\pgfpathlineto{\pgfqpoint{5.158831in}{1.287964in}}%
\pgfpathlineto{\pgfqpoint{5.169498in}{1.286565in}}%
\pgfpathlineto{\pgfqpoint{5.180166in}{1.278062in}}%
\pgfpathlineto{\pgfqpoint{5.212168in}{1.262956in}}%
\pgfpathlineto{\pgfqpoint{5.222835in}{1.259984in}}%
\pgfpathlineto{\pgfqpoint{5.233502in}{1.254450in}}%
\pgfpathlineto{\pgfqpoint{5.244169in}{1.246539in}}%
\pgfpathlineto{\pgfqpoint{5.254837in}{1.235642in}}%
\pgfpathlineto{\pgfqpoint{5.265504in}{1.231135in}}%
\pgfpathlineto{\pgfqpoint{5.276171in}{1.224551in}}%
\pgfpathlineto{\pgfqpoint{5.297506in}{1.218172in}}%
\pgfpathlineto{\pgfqpoint{5.318840in}{1.203276in}}%
\pgfpathlineto{\pgfqpoint{5.329507in}{1.184220in}}%
\pgfpathlineto{\pgfqpoint{5.350842in}{1.178732in}}%
\pgfpathlineto{\pgfqpoint{5.361509in}{1.174286in}}%
\pgfpathlineto{\pgfqpoint{5.372177in}{1.168627in}}%
\pgfpathlineto{\pgfqpoint{5.382844in}{1.161216in}}%
\pgfpathlineto{\pgfqpoint{5.393511in}{1.155383in}}%
\pgfpathlineto{\pgfqpoint{5.404178in}{1.148178in}}%
\pgfpathlineto{\pgfqpoint{5.414846in}{1.143218in}}%
\pgfpathlineto{\pgfqpoint{5.425513in}{1.131450in}}%
\pgfpathlineto{\pgfqpoint{5.446847in}{1.126533in}}%
\pgfpathlineto{\pgfqpoint{5.457515in}{1.121068in}}%
\pgfpathlineto{\pgfqpoint{5.468182in}{1.117570in}}%
\pgfpathlineto{\pgfqpoint{5.478849in}{1.102940in}}%
\pgfpathlineto{\pgfqpoint{5.489516in}{1.084823in}}%
\pgfpathlineto{\pgfqpoint{5.510851in}{1.068151in}}%
\pgfpathlineto{\pgfqpoint{5.521518in}{1.061069in}}%
\pgfpathlineto{\pgfqpoint{5.532186in}{1.058470in}}%
\pgfpathlineto{\pgfqpoint{5.553520in}{1.036908in}}%
\pgfpathlineto{\pgfqpoint{5.564187in}{1.031753in}}%
\pgfpathlineto{\pgfqpoint{5.574855in}{1.012555in}}%
\pgfpathlineto{\pgfqpoint{5.596189in}{1.004477in}}%
\pgfpathlineto{\pgfqpoint{5.628191in}{0.980750in}}%
\pgfpathlineto{\pgfqpoint{5.638858in}{0.977342in}}%
\pgfpathlineto{\pgfqpoint{5.660193in}{0.968028in}}%
\pgfpathlineto{\pgfqpoint{5.670860in}{0.958716in}}%
\pgfpathlineto{\pgfqpoint{5.681527in}{0.953160in}}%
\pgfpathlineto{\pgfqpoint{5.692194in}{0.937440in}}%
\pgfpathlineto{\pgfqpoint{5.713529in}{0.916342in}}%
\pgfpathlineto{\pgfqpoint{5.724196in}{0.903409in}}%
\pgfpathlineto{\pgfqpoint{5.734864in}{0.895398in}}%
\pgfpathlineto{\pgfqpoint{5.745531in}{0.892551in}}%
\pgfpathlineto{\pgfqpoint{5.756198in}{0.887891in}}%
\pgfpathlineto{\pgfqpoint{5.766865in}{0.874197in}}%
\pgfpathlineto{\pgfqpoint{5.777533in}{0.868842in}}%
\pgfpathlineto{\pgfqpoint{5.788200in}{0.856767in}}%
\pgfpathlineto{\pgfqpoint{5.798867in}{0.853379in}}%
\pgfpathlineto{\pgfqpoint{5.841536in}{0.844483in}}%
\pgfpathlineto{\pgfqpoint{5.852203in}{0.840639in}}%
\pgfpathlineto{\pgfqpoint{5.862871in}{0.832724in}}%
\pgfpathlineto{\pgfqpoint{5.873538in}{0.815353in}}%
\pgfpathlineto{\pgfqpoint{5.905540in}{0.804030in}}%
\pgfpathlineto{\pgfqpoint{5.916207in}{0.793472in}}%
\pgfpathlineto{\pgfqpoint{5.926874in}{0.791191in}}%
\pgfpathlineto{\pgfqpoint{5.937542in}{0.784816in}}%
\pgfpathlineto{\pgfqpoint{5.937542in}{0.784816in}}%
\pgfusepath{stroke}%
\end{pgfscope}%
\begin{pgfscope}%
\pgfsetrectcap%
\pgfsetmiterjoin%
\pgfsetlinewidth{0.803000pt}%
\definecolor{currentstroke}{rgb}{0.000000,0.000000,0.000000}%
\pgfsetstrokecolor{currentstroke}%
\pgfsetdash{}{0pt}%
\pgfpathmoveto{\pgfqpoint{0.977263in}{0.549691in}}%
\pgfpathlineto{\pgfqpoint{0.977263in}{4.575328in}}%
\pgfusepath{stroke}%
\end{pgfscope}%
\begin{pgfscope}%
\pgfsetrectcap%
\pgfsetmiterjoin%
\pgfsetlinewidth{0.803000pt}%
\definecolor{currentstroke}{rgb}{0.000000,0.000000,0.000000}%
\pgfsetstrokecolor{currentstroke}%
\pgfsetdash{}{0pt}%
\pgfpathmoveto{\pgfqpoint{6.342364in}{0.549691in}}%
\pgfpathlineto{\pgfqpoint{6.342364in}{4.575328in}}%
\pgfusepath{stroke}%
\end{pgfscope}%
\begin{pgfscope}%
\pgfsetrectcap%
\pgfsetmiterjoin%
\pgfsetlinewidth{0.803000pt}%
\definecolor{currentstroke}{rgb}{0.000000,0.000000,0.000000}%
\pgfsetstrokecolor{currentstroke}%
\pgfsetdash{}{0pt}%
\pgfpathmoveto{\pgfqpoint{0.977263in}{0.549691in}}%
\pgfpathlineto{\pgfqpoint{6.342364in}{0.549691in}}%
\pgfusepath{stroke}%
\end{pgfscope}%
\begin{pgfscope}%
\pgfsetrectcap%
\pgfsetmiterjoin%
\pgfsetlinewidth{0.803000pt}%
\definecolor{currentstroke}{rgb}{0.000000,0.000000,0.000000}%
\pgfsetstrokecolor{currentstroke}%
\pgfsetdash{}{0pt}%
\pgfpathmoveto{\pgfqpoint{0.977263in}{4.575328in}}%
\pgfpathlineto{\pgfqpoint{6.342364in}{4.575328in}}%
\pgfusepath{stroke}%
\end{pgfscope}%
\begin{pgfscope}%
\pgfsetbuttcap%
\pgfsetmiterjoin%
\definecolor{currentfill}{rgb}{1.000000,1.000000,1.000000}%
\pgfsetfillcolor{currentfill}%
\pgfsetlinewidth{1.003750pt}%
\definecolor{currentstroke}{rgb}{0.000000,0.000000,0.000000}%
\pgfsetstrokecolor{currentstroke}%
\pgfsetdash{}{0pt}%
\pgfpathmoveto{\pgfqpoint{3.829768in}{3.512366in}}%
\pgfpathlineto{\pgfqpoint{6.225698in}{3.512366in}}%
\pgfpathquadraticcurveto{\pgfqpoint{6.259031in}{3.512366in}}{\pgfqpoint{6.259031in}{3.545700in}}%
\pgfpathlineto{\pgfqpoint{6.259031in}{4.458661in}}%
\pgfpathquadraticcurveto{\pgfqpoint{6.259031in}{4.491995in}}{\pgfqpoint{6.225698in}{4.491995in}}%
\pgfpathlineto{\pgfqpoint{3.829768in}{4.491995in}}%
\pgfpathquadraticcurveto{\pgfqpoint{3.796434in}{4.491995in}}{\pgfqpoint{3.796434in}{4.458661in}}%
\pgfpathlineto{\pgfqpoint{3.796434in}{3.545700in}}%
\pgfpathquadraticcurveto{\pgfqpoint{3.796434in}{3.512366in}}{\pgfqpoint{3.829768in}{3.512366in}}%
\pgfpathlineto{\pgfqpoint{3.829768in}{3.512366in}}%
\pgfpathclose%
\pgfusepath{stroke,fill}%
\end{pgfscope}%
\begin{pgfscope}%
\pgfsetrectcap%
\pgfsetroundjoin%
\pgfsetlinewidth{1.505625pt}%
\definecolor{currentstroke}{rgb}{0.121569,0.466667,0.705882}%
\pgfsetstrokecolor{currentstroke}%
\pgfsetdash{}{0pt}%
\pgfpathmoveto{\pgfqpoint{3.863101in}{4.366995in}}%
\pgfpathlineto{\pgfqpoint{4.029768in}{4.366995in}}%
\pgfpathlineto{\pgfqpoint{4.196434in}{4.366995in}}%
\pgfusepath{stroke}%
\end{pgfscope}%
\begin{pgfscope}%
\definecolor{textcolor}{rgb}{0.000000,0.000000,0.000000}%
\pgfsetstrokecolor{textcolor}%
\pgfsetfillcolor{textcolor}%
\pgftext[x=4.329768in,y=4.308661in,left,base]{\color{textcolor}\rmfamily\fontsize{12.000000}{14.400000}\selectfont NONE, \(\displaystyle \tilde{m} = \) 480}%
\end{pgfscope}%
\begin{pgfscope}%
\pgfsetrectcap%
\pgfsetroundjoin%
\pgfsetlinewidth{1.505625pt}%
\definecolor{currentstroke}{rgb}{1.000000,0.498039,0.054902}%
\pgfsetstrokecolor{currentstroke}%
\pgfsetdash{}{0pt}%
\pgfpathmoveto{\pgfqpoint{3.863101in}{4.134588in}}%
\pgfpathlineto{\pgfqpoint{4.029768in}{4.134588in}}%
\pgfpathlineto{\pgfqpoint{4.196434in}{4.134588in}}%
\pgfusepath{stroke}%
\end{pgfscope}%
\begin{pgfscope}%
\definecolor{textcolor}{rgb}{0.000000,0.000000,0.000000}%
\pgfsetstrokecolor{textcolor}%
\pgfsetfillcolor{textcolor}%
\pgftext[x=4.329768in,y=4.076254in,left,base]{\color{textcolor}\rmfamily\fontsize{12.000000}{14.400000}\selectfont JACOBI, \(\displaystyle \tilde{m} = \) 257}%
\end{pgfscope}%
\begin{pgfscope}%
\pgfsetrectcap%
\pgfsetroundjoin%
\pgfsetlinewidth{1.505625pt}%
\definecolor{currentstroke}{rgb}{0.172549,0.627451,0.172549}%
\pgfsetstrokecolor{currentstroke}%
\pgfsetdash{}{0pt}%
\pgfpathmoveto{\pgfqpoint{3.863101in}{3.902180in}}%
\pgfpathlineto{\pgfqpoint{4.029768in}{3.902180in}}%
\pgfpathlineto{\pgfqpoint{4.196434in}{3.902180in}}%
\pgfusepath{stroke}%
\end{pgfscope}%
\begin{pgfscope}%
\definecolor{textcolor}{rgb}{0.000000,0.000000,0.000000}%
\pgfsetstrokecolor{textcolor}%
\pgfsetfillcolor{textcolor}%
\pgftext[x=4.329768in,y=3.843847in,left,base]{\color{textcolor}\rmfamily\fontsize{12.000000}{14.400000}\selectfont GAUSSSEIDEL, \(\displaystyle \tilde{m} = \) 144}%
\end{pgfscope}%
\begin{pgfscope}%
\pgfsetrectcap%
\pgfsetroundjoin%
\pgfsetlinewidth{1.505625pt}%
\definecolor{currentstroke}{rgb}{0.839216,0.152941,0.156863}%
\pgfsetstrokecolor{currentstroke}%
\pgfsetdash{}{0pt}%
\pgfpathmoveto{\pgfqpoint{3.863101in}{3.669773in}}%
\pgfpathlineto{\pgfqpoint{4.029768in}{3.669773in}}%
\pgfpathlineto{\pgfqpoint{4.196434in}{3.669773in}}%
\pgfusepath{stroke}%
\end{pgfscope}%
\begin{pgfscope}%
\definecolor{textcolor}{rgb}{0.000000,0.000000,0.000000}%
\pgfsetstrokecolor{textcolor}%
\pgfsetfillcolor{textcolor}%
\pgftext[x=4.329768in,y=3.611440in,left,base]{\color{textcolor}\rmfamily\fontsize{12.000000}{14.400000}\selectfont ILU, \(\displaystyle \tilde{m} = \) 466}%
\end{pgfscope}%
\end{pgfpicture}%
\makeatother%
\endgroup%
}
	\caption{Comparison of the different preconditioning options for the GMRES procedure to the full GMRES Method without preconditioning using m = 600 Krylov vectors as an input.}
	\label{fig::Residuals}
\end{figure}
%

\subsection{Optimization of restart parameter for restarted GMRES}
In an effort to find a "good" restart parameter the parameters $m=30, m=50$ and $m=100$ are compared to the runtime of full GMRES in table \refTab{tab:TimingsO0} for no optimization and in table \refTab{tab:TimingsO2} for moderate optimization level.
% 
\renewcommand{\arraystretch}{2}
\begin{table}[h!]
	\begin{center}
		\begin{tabular}{ p{2cm} p{2cm} p{1.5cm} p{1.8cm} p{2.5cm}}
			\hline
			\hline
			$m$ & Iterations & User $[s]$ & System $[s]$ & CPU-Time $[s]$ \\
			\hline
			\hline
			479 (full) & 1 & 5.04 & 0.11 & 5.15\\
			\hline
			\hline
			30 & 144 & 7.25 & 0.08 & 7.33\\
			\hline
			50 & 43 & 4.43 & 0.06 & 4.49\\
			\hline
			100 & 14 & 4.23 & 0.09 & 4.32\\
			\hline
			\hline
		\end{tabular}
		\caption{\label{tab:TimingsO0}  Comparison of timings for different restart parameters using clang compiler with no optimization (-O0 flag) + disabled Output (-DDISABLEIO) and without preconditioning}
	\end{center}
\end{table}
\renewcommand{\arraystretch}{1}
%
\renewcommand{\arraystretch}{2}
\begin{table}[h!]
	\begin{center}
		\begin{tabular}{ p{2cm} p{2cm} p{1.5cm} p{1.8cm} p{2.5cm}}
			\hline
			\hline
			$m$ & Iterations & User $[s]$ & System $[s]$ & CPU-Time $[s]$ \\
			\hline
			\hline
			479 (full) & 1 & 0.29 & 0.1 & 0.39\\
			\hline
			\hline
			30 & 143 & 0.46 & 0.06 & 0.52\\
			\hline
			50 & 43 & 0.29 & 0.07 & 0.36\\
			\hline
			100 & 14 & 0.27 & 0.07 & 0.34\\
			\hline
			\hline
		\end{tabular}
		\caption{\label{tab:TimingsO2}  Comparison of timings for different restart parameters using clang compiler with optimization (-O2 flag) + disabled Output (-DDISABLEIO) and without preconditioning}
	\end{center}
\end{table}
\renewcommand{\arraystretch}{1}
%
Restart parameters of $m=50$ and $m=100$ showed a decrease in CPU-Time of 12.8\% and 16.1\% for restarted GMRES compared to full GMRES \footnote{in the following context "full" GMRES refers to a restarted GMRES method were only one loop iteration of the restarted GMRES method is necessary to establish convergence according to the given convergence criterion.} respectively, where as a restart parameter of $m=30$ showed an increase of 43\%, if compiler optimization is disabled. Activating a moderate optimization level (-O2) restart parameters of $m=50$ and $m=100$ decreased the CPU-Time only by 7.7\% and 12.82\%, compared to full GMRES. 
Furthermore, I conducted an investigation to find the globally "best" restart parameter, which is presented in \refFig{fig::Timings}. 
%
\begin{figure}[!htbp]
	\centering
	\hspace*{0.8cm}
	\leavevmode
	\resizebox{0.9\width}{!}{%% Creator: Matplotlib, PGF backend
%%
%% To include the figure in your LaTeX document, write
%%   \input{<filename>.pgf}
%%
%% Make sure the required packages are loaded in your preamble
%%   \usepackage{pgf}
%%
%% Also ensure that all the required font packages are loaded; for instance,
%% the lmodern package is sometimes necessary when using math font.
%%   \usepackage{lmodern}
%%
%% Figures using additional raster images can only be included by \input if
%% they are in the same directory as the main LaTeX file. For loading figures
%% from other directories you can use the `import` package
%%   \usepackage{import}
%%
%% and then include the figures with
%%   \import{<path to file>}{<filename>.pgf}
%%
%% Matplotlib used the following preamble
%%   
%%   \makeatletter\@ifpackageloaded{underscore}{}{\usepackage[strings]{underscore}}\makeatother
%%
\begingroup%
\makeatletter%
\begin{pgfpicture}%
\pgfpathrectangle{\pgfpointorigin}{\pgfqpoint{6.565064in}{4.725328in}}%
\pgfusepath{use as bounding box, clip}%
\begin{pgfscope}%
\pgfsetbuttcap%
\pgfsetmiterjoin%
\definecolor{currentfill}{rgb}{1.000000,1.000000,1.000000}%
\pgfsetfillcolor{currentfill}%
\pgfsetlinewidth{0.000000pt}%
\definecolor{currentstroke}{rgb}{1.000000,1.000000,1.000000}%
\pgfsetstrokecolor{currentstroke}%
\pgfsetdash{}{0pt}%
\pgfpathmoveto{\pgfqpoint{0.000000in}{0.000000in}}%
\pgfpathlineto{\pgfqpoint{6.565064in}{0.000000in}}%
\pgfpathlineto{\pgfqpoint{6.565064in}{4.725328in}}%
\pgfpathlineto{\pgfqpoint{0.000000in}{4.725328in}}%
\pgfpathlineto{\pgfqpoint{0.000000in}{0.000000in}}%
\pgfpathclose%
\pgfusepath{fill}%
\end{pgfscope}%
\begin{pgfscope}%
\pgfsetbuttcap%
\pgfsetmiterjoin%
\definecolor{currentfill}{rgb}{1.000000,1.000000,1.000000}%
\pgfsetfillcolor{currentfill}%
\pgfsetlinewidth{0.000000pt}%
\definecolor{currentstroke}{rgb}{0.000000,0.000000,0.000000}%
\pgfsetstrokecolor{currentstroke}%
\pgfsetstrokeopacity{0.000000}%
\pgfsetdash{}{0pt}%
\pgfpathmoveto{\pgfqpoint{1.748007in}{0.604012in}}%
\pgfpathlineto{\pgfqpoint{6.415064in}{0.604012in}}%
\pgfpathlineto{\pgfqpoint{6.415064in}{4.575328in}}%
\pgfpathlineto{\pgfqpoint{1.748007in}{4.575328in}}%
\pgfpathlineto{\pgfqpoint{1.748007in}{0.604012in}}%
\pgfpathclose%
\pgfusepath{fill}%
\end{pgfscope}%
\begin{pgfscope}%
\pgfpathrectangle{\pgfqpoint{1.748007in}{0.604012in}}{\pgfqpoint{4.667057in}{3.971316in}}%
\pgfusepath{clip}%
\pgfsetrectcap%
\pgfsetroundjoin%
\pgfsetlinewidth{0.803000pt}%
\definecolor{currentstroke}{rgb}{0.690196,0.690196,0.690196}%
\pgfsetstrokecolor{currentstroke}%
\pgfsetdash{}{0pt}%
\pgfpathmoveto{\pgfqpoint{1.748007in}{0.604012in}}%
\pgfpathlineto{\pgfqpoint{1.748007in}{4.575328in}}%
\pgfusepath{stroke}%
\end{pgfscope}%
\begin{pgfscope}%
\pgfsetbuttcap%
\pgfsetroundjoin%
\definecolor{currentfill}{rgb}{0.000000,0.000000,0.000000}%
\pgfsetfillcolor{currentfill}%
\pgfsetlinewidth{0.803000pt}%
\definecolor{currentstroke}{rgb}{0.000000,0.000000,0.000000}%
\pgfsetstrokecolor{currentstroke}%
\pgfsetdash{}{0pt}%
\pgfsys@defobject{currentmarker}{\pgfqpoint{0.000000in}{-0.048611in}}{\pgfqpoint{0.000000in}{0.000000in}}{%
\pgfpathmoveto{\pgfqpoint{0.000000in}{0.000000in}}%
\pgfpathlineto{\pgfqpoint{0.000000in}{-0.048611in}}%
\pgfusepath{stroke,fill}%
}%
\begin{pgfscope}%
\pgfsys@transformshift{1.748007in}{0.604012in}%
\pgfsys@useobject{currentmarker}{}%
\end{pgfscope}%
\end{pgfscope}%
\begin{pgfscope}%
\definecolor{textcolor}{rgb}{0.000000,0.000000,0.000000}%
\pgfsetstrokecolor{textcolor}%
\pgfsetfillcolor{textcolor}%
\pgftext[x=1.748007in,y=0.506790in,,top]{\color{textcolor}\rmfamily\fontsize{10.000000}{12.000000}\selectfont \(\displaystyle {0}\)}%
\end{pgfscope}%
\begin{pgfscope}%
\pgfpathrectangle{\pgfqpoint{1.748007in}{0.604012in}}{\pgfqpoint{4.667057in}{3.971316in}}%
\pgfusepath{clip}%
\pgfsetrectcap%
\pgfsetroundjoin%
\pgfsetlinewidth{0.803000pt}%
\definecolor{currentstroke}{rgb}{0.690196,0.690196,0.690196}%
\pgfsetstrokecolor{currentstroke}%
\pgfsetdash{}{0pt}%
\pgfpathmoveto{\pgfqpoint{1.942468in}{0.604012in}}%
\pgfpathlineto{\pgfqpoint{1.942468in}{4.575328in}}%
\pgfusepath{stroke}%
\end{pgfscope}%
\begin{pgfscope}%
\pgfsetbuttcap%
\pgfsetroundjoin%
\definecolor{currentfill}{rgb}{0.000000,0.000000,0.000000}%
\pgfsetfillcolor{currentfill}%
\pgfsetlinewidth{0.803000pt}%
\definecolor{currentstroke}{rgb}{0.000000,0.000000,0.000000}%
\pgfsetstrokecolor{currentstroke}%
\pgfsetdash{}{0pt}%
\pgfsys@defobject{currentmarker}{\pgfqpoint{0.000000in}{-0.048611in}}{\pgfqpoint{0.000000in}{0.000000in}}{%
\pgfpathmoveto{\pgfqpoint{0.000000in}{0.000000in}}%
\pgfpathlineto{\pgfqpoint{0.000000in}{-0.048611in}}%
\pgfusepath{stroke,fill}%
}%
\begin{pgfscope}%
\pgfsys@transformshift{1.942468in}{0.604012in}%
\pgfsys@useobject{currentmarker}{}%
\end{pgfscope}%
\end{pgfscope}%
\begin{pgfscope}%
\definecolor{textcolor}{rgb}{0.000000,0.000000,0.000000}%
\pgfsetstrokecolor{textcolor}%
\pgfsetfillcolor{textcolor}%
\pgftext[x=1.942468in,y=0.506790in,,top]{\color{textcolor}\rmfamily\fontsize{10.000000}{12.000000}\selectfont \(\displaystyle {20}\)}%
\end{pgfscope}%
\begin{pgfscope}%
\pgfpathrectangle{\pgfqpoint{1.748007in}{0.604012in}}{\pgfqpoint{4.667057in}{3.971316in}}%
\pgfusepath{clip}%
\pgfsetrectcap%
\pgfsetroundjoin%
\pgfsetlinewidth{0.803000pt}%
\definecolor{currentstroke}{rgb}{0.690196,0.690196,0.690196}%
\pgfsetstrokecolor{currentstroke}%
\pgfsetdash{}{0pt}%
\pgfpathmoveto{\pgfqpoint{2.720311in}{0.604012in}}%
\pgfpathlineto{\pgfqpoint{2.720311in}{4.575328in}}%
\pgfusepath{stroke}%
\end{pgfscope}%
\begin{pgfscope}%
\pgfsetbuttcap%
\pgfsetroundjoin%
\definecolor{currentfill}{rgb}{0.000000,0.000000,0.000000}%
\pgfsetfillcolor{currentfill}%
\pgfsetlinewidth{0.803000pt}%
\definecolor{currentstroke}{rgb}{0.000000,0.000000,0.000000}%
\pgfsetstrokecolor{currentstroke}%
\pgfsetdash{}{0pt}%
\pgfsys@defobject{currentmarker}{\pgfqpoint{0.000000in}{-0.048611in}}{\pgfqpoint{0.000000in}{0.000000in}}{%
\pgfpathmoveto{\pgfqpoint{0.000000in}{0.000000in}}%
\pgfpathlineto{\pgfqpoint{0.000000in}{-0.048611in}}%
\pgfusepath{stroke,fill}%
}%
\begin{pgfscope}%
\pgfsys@transformshift{2.720311in}{0.604012in}%
\pgfsys@useobject{currentmarker}{}%
\end{pgfscope}%
\end{pgfscope}%
\begin{pgfscope}%
\definecolor{textcolor}{rgb}{0.000000,0.000000,0.000000}%
\pgfsetstrokecolor{textcolor}%
\pgfsetfillcolor{textcolor}%
\pgftext[x=2.720311in,y=0.506790in,,top]{\color{textcolor}\rmfamily\fontsize{10.000000}{12.000000}\selectfont \(\displaystyle {100}\)}%
\end{pgfscope}%
\begin{pgfscope}%
\pgfpathrectangle{\pgfqpoint{1.748007in}{0.604012in}}{\pgfqpoint{4.667057in}{3.971316in}}%
\pgfusepath{clip}%
\pgfsetrectcap%
\pgfsetroundjoin%
\pgfsetlinewidth{0.803000pt}%
\definecolor{currentstroke}{rgb}{0.690196,0.690196,0.690196}%
\pgfsetstrokecolor{currentstroke}%
\pgfsetdash{}{0pt}%
\pgfpathmoveto{\pgfqpoint{3.692614in}{0.604012in}}%
\pgfpathlineto{\pgfqpoint{3.692614in}{4.575328in}}%
\pgfusepath{stroke}%
\end{pgfscope}%
\begin{pgfscope}%
\pgfsetbuttcap%
\pgfsetroundjoin%
\definecolor{currentfill}{rgb}{0.000000,0.000000,0.000000}%
\pgfsetfillcolor{currentfill}%
\pgfsetlinewidth{0.803000pt}%
\definecolor{currentstroke}{rgb}{0.000000,0.000000,0.000000}%
\pgfsetstrokecolor{currentstroke}%
\pgfsetdash{}{0pt}%
\pgfsys@defobject{currentmarker}{\pgfqpoint{0.000000in}{-0.048611in}}{\pgfqpoint{0.000000in}{0.000000in}}{%
\pgfpathmoveto{\pgfqpoint{0.000000in}{0.000000in}}%
\pgfpathlineto{\pgfqpoint{0.000000in}{-0.048611in}}%
\pgfusepath{stroke,fill}%
}%
\begin{pgfscope}%
\pgfsys@transformshift{3.692614in}{0.604012in}%
\pgfsys@useobject{currentmarker}{}%
\end{pgfscope}%
\end{pgfscope}%
\begin{pgfscope}%
\definecolor{textcolor}{rgb}{0.000000,0.000000,0.000000}%
\pgfsetstrokecolor{textcolor}%
\pgfsetfillcolor{textcolor}%
\pgftext[x=3.692614in,y=0.506790in,,top]{\color{textcolor}\rmfamily\fontsize{10.000000}{12.000000}\selectfont \(\displaystyle {200}\)}%
\end{pgfscope}%
\begin{pgfscope}%
\pgfpathrectangle{\pgfqpoint{1.748007in}{0.604012in}}{\pgfqpoint{4.667057in}{3.971316in}}%
\pgfusepath{clip}%
\pgfsetrectcap%
\pgfsetroundjoin%
\pgfsetlinewidth{0.803000pt}%
\definecolor{currentstroke}{rgb}{0.690196,0.690196,0.690196}%
\pgfsetstrokecolor{currentstroke}%
\pgfsetdash{}{0pt}%
\pgfpathmoveto{\pgfqpoint{4.664918in}{0.604012in}}%
\pgfpathlineto{\pgfqpoint{4.664918in}{4.575328in}}%
\pgfusepath{stroke}%
\end{pgfscope}%
\begin{pgfscope}%
\pgfsetbuttcap%
\pgfsetroundjoin%
\definecolor{currentfill}{rgb}{0.000000,0.000000,0.000000}%
\pgfsetfillcolor{currentfill}%
\pgfsetlinewidth{0.803000pt}%
\definecolor{currentstroke}{rgb}{0.000000,0.000000,0.000000}%
\pgfsetstrokecolor{currentstroke}%
\pgfsetdash{}{0pt}%
\pgfsys@defobject{currentmarker}{\pgfqpoint{0.000000in}{-0.048611in}}{\pgfqpoint{0.000000in}{0.000000in}}{%
\pgfpathmoveto{\pgfqpoint{0.000000in}{0.000000in}}%
\pgfpathlineto{\pgfqpoint{0.000000in}{-0.048611in}}%
\pgfusepath{stroke,fill}%
}%
\begin{pgfscope}%
\pgfsys@transformshift{4.664918in}{0.604012in}%
\pgfsys@useobject{currentmarker}{}%
\end{pgfscope}%
\end{pgfscope}%
\begin{pgfscope}%
\definecolor{textcolor}{rgb}{0.000000,0.000000,0.000000}%
\pgfsetstrokecolor{textcolor}%
\pgfsetfillcolor{textcolor}%
\pgftext[x=4.664918in,y=0.506790in,,top]{\color{textcolor}\rmfamily\fontsize{10.000000}{12.000000}\selectfont \(\displaystyle {300}\)}%
\end{pgfscope}%
\begin{pgfscope}%
\pgfpathrectangle{\pgfqpoint{1.748007in}{0.604012in}}{\pgfqpoint{4.667057in}{3.971316in}}%
\pgfusepath{clip}%
\pgfsetrectcap%
\pgfsetroundjoin%
\pgfsetlinewidth{0.803000pt}%
\definecolor{currentstroke}{rgb}{0.690196,0.690196,0.690196}%
\pgfsetstrokecolor{currentstroke}%
\pgfsetdash{}{0pt}%
\pgfpathmoveto{\pgfqpoint{5.637221in}{0.604012in}}%
\pgfpathlineto{\pgfqpoint{5.637221in}{4.575328in}}%
\pgfusepath{stroke}%
\end{pgfscope}%
\begin{pgfscope}%
\pgfsetbuttcap%
\pgfsetroundjoin%
\definecolor{currentfill}{rgb}{0.000000,0.000000,0.000000}%
\pgfsetfillcolor{currentfill}%
\pgfsetlinewidth{0.803000pt}%
\definecolor{currentstroke}{rgb}{0.000000,0.000000,0.000000}%
\pgfsetstrokecolor{currentstroke}%
\pgfsetdash{}{0pt}%
\pgfsys@defobject{currentmarker}{\pgfqpoint{0.000000in}{-0.048611in}}{\pgfqpoint{0.000000in}{0.000000in}}{%
\pgfpathmoveto{\pgfqpoint{0.000000in}{0.000000in}}%
\pgfpathlineto{\pgfqpoint{0.000000in}{-0.048611in}}%
\pgfusepath{stroke,fill}%
}%
\begin{pgfscope}%
\pgfsys@transformshift{5.637221in}{0.604012in}%
\pgfsys@useobject{currentmarker}{}%
\end{pgfscope}%
\end{pgfscope}%
\begin{pgfscope}%
\definecolor{textcolor}{rgb}{0.000000,0.000000,0.000000}%
\pgfsetstrokecolor{textcolor}%
\pgfsetfillcolor{textcolor}%
\pgftext[x=5.637221in,y=0.506790in,,top]{\color{textcolor}\rmfamily\fontsize{10.000000}{12.000000}\selectfont \(\displaystyle {400}\)}%
\end{pgfscope}%
\begin{pgfscope}%
\definecolor{textcolor}{rgb}{0.000000,0.000000,0.000000}%
\pgfsetstrokecolor{textcolor}%
\pgfsetfillcolor{textcolor}%
\pgftext[x=4.081536in,y=0.327777in,,top]{\color{textcolor}\rmfamily\fontsize{15.000000}{18.000000}\selectfont \(\displaystyle m\)}%
\end{pgfscope}%
\begin{pgfscope}%
\pgfpathrectangle{\pgfqpoint{1.748007in}{0.604012in}}{\pgfqpoint{4.667057in}{3.971316in}}%
\pgfusepath{clip}%
\pgfsetrectcap%
\pgfsetroundjoin%
\pgfsetlinewidth{0.803000pt}%
\definecolor{currentstroke}{rgb}{0.690196,0.690196,0.690196}%
\pgfsetstrokecolor{currentstroke}%
\pgfsetdash{}{0pt}%
\pgfpathmoveto{\pgfqpoint{1.748007in}{0.873547in}}%
\pgfpathlineto{\pgfqpoint{6.415064in}{0.873547in}}%
\pgfusepath{stroke}%
\end{pgfscope}%
\begin{pgfscope}%
\pgfsetbuttcap%
\pgfsetroundjoin%
\definecolor{currentfill}{rgb}{0.000000,0.000000,0.000000}%
\pgfsetfillcolor{currentfill}%
\pgfsetlinewidth{0.803000pt}%
\definecolor{currentstroke}{rgb}{0.000000,0.000000,0.000000}%
\pgfsetstrokecolor{currentstroke}%
\pgfsetdash{}{0pt}%
\pgfsys@defobject{currentmarker}{\pgfqpoint{-0.048611in}{0.000000in}}{\pgfqpoint{-0.000000in}{0.000000in}}{%
\pgfpathmoveto{\pgfqpoint{-0.000000in}{0.000000in}}%
\pgfpathlineto{\pgfqpoint{-0.048611in}{0.000000in}}%
\pgfusepath{stroke,fill}%
}%
\begin{pgfscope}%
\pgfsys@transformshift{1.748007in}{0.873547in}%
\pgfsys@useobject{currentmarker}{}%
\end{pgfscope}%
\end{pgfscope}%
\begin{pgfscope}%
\definecolor{textcolor}{rgb}{0.000000,0.000000,0.000000}%
\pgfsetstrokecolor{textcolor}%
\pgfsetfillcolor{textcolor}%
\pgftext[x=1.403871in, y=0.825322in, left, base]{\color{textcolor}\rmfamily\fontsize{10.000000}{12.000000}\selectfont \(\displaystyle {4.00}\)}%
\end{pgfscope}%
\begin{pgfscope}%
\pgfpathrectangle{\pgfqpoint{1.748007in}{0.604012in}}{\pgfqpoint{4.667057in}{3.971316in}}%
\pgfusepath{clip}%
\pgfsetrectcap%
\pgfsetroundjoin%
\pgfsetlinewidth{0.803000pt}%
\definecolor{currentstroke}{rgb}{0.690196,0.690196,0.690196}%
\pgfsetstrokecolor{currentstroke}%
\pgfsetdash{}{0pt}%
\pgfpathmoveto{\pgfqpoint{1.748007in}{1.252710in}}%
\pgfpathlineto{\pgfqpoint{6.415064in}{1.252710in}}%
\pgfusepath{stroke}%
\end{pgfscope}%
\begin{pgfscope}%
\pgfsetbuttcap%
\pgfsetroundjoin%
\definecolor{currentfill}{rgb}{0.000000,0.000000,0.000000}%
\pgfsetfillcolor{currentfill}%
\pgfsetlinewidth{0.803000pt}%
\definecolor{currentstroke}{rgb}{0.000000,0.000000,0.000000}%
\pgfsetstrokecolor{currentstroke}%
\pgfsetdash{}{0pt}%
\pgfsys@defobject{currentmarker}{\pgfqpoint{-0.048611in}{0.000000in}}{\pgfqpoint{-0.000000in}{0.000000in}}{%
\pgfpathmoveto{\pgfqpoint{-0.000000in}{0.000000in}}%
\pgfpathlineto{\pgfqpoint{-0.048611in}{0.000000in}}%
\pgfusepath{stroke,fill}%
}%
\begin{pgfscope}%
\pgfsys@transformshift{1.748007in}{1.252710in}%
\pgfsys@useobject{currentmarker}{}%
\end{pgfscope}%
\end{pgfscope}%
\begin{pgfscope}%
\definecolor{textcolor}{rgb}{0.000000,0.000000,0.000000}%
\pgfsetstrokecolor{textcolor}%
\pgfsetfillcolor{textcolor}%
\pgftext[x=1.403871in, y=1.204484in, left, base]{\color{textcolor}\rmfamily\fontsize{10.000000}{12.000000}\selectfont \(\displaystyle {5.15}\)}%
\end{pgfscope}%
\begin{pgfscope}%
\pgfpathrectangle{\pgfqpoint{1.748007in}{0.604012in}}{\pgfqpoint{4.667057in}{3.971316in}}%
\pgfusepath{clip}%
\pgfsetrectcap%
\pgfsetroundjoin%
\pgfsetlinewidth{0.803000pt}%
\definecolor{currentstroke}{rgb}{0.690196,0.690196,0.690196}%
\pgfsetstrokecolor{currentstroke}%
\pgfsetdash{}{0pt}%
\pgfpathmoveto{\pgfqpoint{1.748007in}{1.532960in}}%
\pgfpathlineto{\pgfqpoint{6.415064in}{1.532960in}}%
\pgfusepath{stroke}%
\end{pgfscope}%
\begin{pgfscope}%
\pgfsetbuttcap%
\pgfsetroundjoin%
\definecolor{currentfill}{rgb}{0.000000,0.000000,0.000000}%
\pgfsetfillcolor{currentfill}%
\pgfsetlinewidth{0.803000pt}%
\definecolor{currentstroke}{rgb}{0.000000,0.000000,0.000000}%
\pgfsetstrokecolor{currentstroke}%
\pgfsetdash{}{0pt}%
\pgfsys@defobject{currentmarker}{\pgfqpoint{-0.048611in}{0.000000in}}{\pgfqpoint{-0.000000in}{0.000000in}}{%
\pgfpathmoveto{\pgfqpoint{-0.000000in}{0.000000in}}%
\pgfpathlineto{\pgfqpoint{-0.048611in}{0.000000in}}%
\pgfusepath{stroke,fill}%
}%
\begin{pgfscope}%
\pgfsys@transformshift{1.748007in}{1.532960in}%
\pgfsys@useobject{currentmarker}{}%
\end{pgfscope}%
\end{pgfscope}%
\begin{pgfscope}%
\definecolor{textcolor}{rgb}{0.000000,0.000000,0.000000}%
\pgfsetstrokecolor{textcolor}%
\pgfsetfillcolor{textcolor}%
\pgftext[x=1.403871in, y=1.484735in, left, base]{\color{textcolor}\rmfamily\fontsize{10.000000}{12.000000}\selectfont \(\displaystyle {6.00}\)}%
\end{pgfscope}%
\begin{pgfscope}%
\pgfpathrectangle{\pgfqpoint{1.748007in}{0.604012in}}{\pgfqpoint{4.667057in}{3.971316in}}%
\pgfusepath{clip}%
\pgfsetrectcap%
\pgfsetroundjoin%
\pgfsetlinewidth{0.803000pt}%
\definecolor{currentstroke}{rgb}{0.690196,0.690196,0.690196}%
\pgfsetstrokecolor{currentstroke}%
\pgfsetdash{}{0pt}%
\pgfpathmoveto{\pgfqpoint{1.748007in}{2.192373in}}%
\pgfpathlineto{\pgfqpoint{6.415064in}{2.192373in}}%
\pgfusepath{stroke}%
\end{pgfscope}%
\begin{pgfscope}%
\pgfsetbuttcap%
\pgfsetroundjoin%
\definecolor{currentfill}{rgb}{0.000000,0.000000,0.000000}%
\pgfsetfillcolor{currentfill}%
\pgfsetlinewidth{0.803000pt}%
\definecolor{currentstroke}{rgb}{0.000000,0.000000,0.000000}%
\pgfsetstrokecolor{currentstroke}%
\pgfsetdash{}{0pt}%
\pgfsys@defobject{currentmarker}{\pgfqpoint{-0.048611in}{0.000000in}}{\pgfqpoint{-0.000000in}{0.000000in}}{%
\pgfpathmoveto{\pgfqpoint{-0.000000in}{0.000000in}}%
\pgfpathlineto{\pgfqpoint{-0.048611in}{0.000000in}}%
\pgfusepath{stroke,fill}%
}%
\begin{pgfscope}%
\pgfsys@transformshift{1.748007in}{2.192373in}%
\pgfsys@useobject{currentmarker}{}%
\end{pgfscope}%
\end{pgfscope}%
\begin{pgfscope}%
\definecolor{textcolor}{rgb}{0.000000,0.000000,0.000000}%
\pgfsetstrokecolor{textcolor}%
\pgfsetfillcolor{textcolor}%
\pgftext[x=1.403871in, y=2.144148in, left, base]{\color{textcolor}\rmfamily\fontsize{10.000000}{12.000000}\selectfont \(\displaystyle {8.00}\)}%
\end{pgfscope}%
\begin{pgfscope}%
\pgfpathrectangle{\pgfqpoint{1.748007in}{0.604012in}}{\pgfqpoint{4.667057in}{3.971316in}}%
\pgfusepath{clip}%
\pgfsetrectcap%
\pgfsetroundjoin%
\pgfsetlinewidth{0.803000pt}%
\definecolor{currentstroke}{rgb}{0.690196,0.690196,0.690196}%
\pgfsetstrokecolor{currentstroke}%
\pgfsetdash{}{0pt}%
\pgfpathmoveto{\pgfqpoint{1.748007in}{2.851787in}}%
\pgfpathlineto{\pgfqpoint{6.415064in}{2.851787in}}%
\pgfusepath{stroke}%
\end{pgfscope}%
\begin{pgfscope}%
\pgfsetbuttcap%
\pgfsetroundjoin%
\definecolor{currentfill}{rgb}{0.000000,0.000000,0.000000}%
\pgfsetfillcolor{currentfill}%
\pgfsetlinewidth{0.803000pt}%
\definecolor{currentstroke}{rgb}{0.000000,0.000000,0.000000}%
\pgfsetstrokecolor{currentstroke}%
\pgfsetdash{}{0pt}%
\pgfsys@defobject{currentmarker}{\pgfqpoint{-0.048611in}{0.000000in}}{\pgfqpoint{-0.000000in}{0.000000in}}{%
\pgfpathmoveto{\pgfqpoint{-0.000000in}{0.000000in}}%
\pgfpathlineto{\pgfqpoint{-0.048611in}{0.000000in}}%
\pgfusepath{stroke,fill}%
}%
\begin{pgfscope}%
\pgfsys@transformshift{1.748007in}{2.851787in}%
\pgfsys@useobject{currentmarker}{}%
\end{pgfscope}%
\end{pgfscope}%
\begin{pgfscope}%
\definecolor{textcolor}{rgb}{0.000000,0.000000,0.000000}%
\pgfsetstrokecolor{textcolor}%
\pgfsetfillcolor{textcolor}%
\pgftext[x=1.334426in, y=2.803561in, left, base]{\color{textcolor}\rmfamily\fontsize{10.000000}{12.000000}\selectfont \(\displaystyle {10.00}\)}%
\end{pgfscope}%
\begin{pgfscope}%
\pgfpathrectangle{\pgfqpoint{1.748007in}{0.604012in}}{\pgfqpoint{4.667057in}{3.971316in}}%
\pgfusepath{clip}%
\pgfsetrectcap%
\pgfsetroundjoin%
\pgfsetlinewidth{0.803000pt}%
\definecolor{currentstroke}{rgb}{0.690196,0.690196,0.690196}%
\pgfsetstrokecolor{currentstroke}%
\pgfsetdash{}{0pt}%
\pgfpathmoveto{\pgfqpoint{1.748007in}{3.511200in}}%
\pgfpathlineto{\pgfqpoint{6.415064in}{3.511200in}}%
\pgfusepath{stroke}%
\end{pgfscope}%
\begin{pgfscope}%
\pgfsetbuttcap%
\pgfsetroundjoin%
\definecolor{currentfill}{rgb}{0.000000,0.000000,0.000000}%
\pgfsetfillcolor{currentfill}%
\pgfsetlinewidth{0.803000pt}%
\definecolor{currentstroke}{rgb}{0.000000,0.000000,0.000000}%
\pgfsetstrokecolor{currentstroke}%
\pgfsetdash{}{0pt}%
\pgfsys@defobject{currentmarker}{\pgfqpoint{-0.048611in}{0.000000in}}{\pgfqpoint{-0.000000in}{0.000000in}}{%
\pgfpathmoveto{\pgfqpoint{-0.000000in}{0.000000in}}%
\pgfpathlineto{\pgfqpoint{-0.048611in}{0.000000in}}%
\pgfusepath{stroke,fill}%
}%
\begin{pgfscope}%
\pgfsys@transformshift{1.748007in}{3.511200in}%
\pgfsys@useobject{currentmarker}{}%
\end{pgfscope}%
\end{pgfscope}%
\begin{pgfscope}%
\definecolor{textcolor}{rgb}{0.000000,0.000000,0.000000}%
\pgfsetstrokecolor{textcolor}%
\pgfsetfillcolor{textcolor}%
\pgftext[x=1.334426in, y=3.462975in, left, base]{\color{textcolor}\rmfamily\fontsize{10.000000}{12.000000}\selectfont \(\displaystyle {12.00}\)}%
\end{pgfscope}%
\begin{pgfscope}%
\pgfpathrectangle{\pgfqpoint{1.748007in}{0.604012in}}{\pgfqpoint{4.667057in}{3.971316in}}%
\pgfusepath{clip}%
\pgfsetrectcap%
\pgfsetroundjoin%
\pgfsetlinewidth{0.803000pt}%
\definecolor{currentstroke}{rgb}{0.690196,0.690196,0.690196}%
\pgfsetstrokecolor{currentstroke}%
\pgfsetdash{}{0pt}%
\pgfpathmoveto{\pgfqpoint{1.748007in}{4.170613in}}%
\pgfpathlineto{\pgfqpoint{6.415064in}{4.170613in}}%
\pgfusepath{stroke}%
\end{pgfscope}%
\begin{pgfscope}%
\pgfsetbuttcap%
\pgfsetroundjoin%
\definecolor{currentfill}{rgb}{0.000000,0.000000,0.000000}%
\pgfsetfillcolor{currentfill}%
\pgfsetlinewidth{0.803000pt}%
\definecolor{currentstroke}{rgb}{0.000000,0.000000,0.000000}%
\pgfsetstrokecolor{currentstroke}%
\pgfsetdash{}{0pt}%
\pgfsys@defobject{currentmarker}{\pgfqpoint{-0.048611in}{0.000000in}}{\pgfqpoint{-0.000000in}{0.000000in}}{%
\pgfpathmoveto{\pgfqpoint{-0.000000in}{0.000000in}}%
\pgfpathlineto{\pgfqpoint{-0.048611in}{0.000000in}}%
\pgfusepath{stroke,fill}%
}%
\begin{pgfscope}%
\pgfsys@transformshift{1.748007in}{4.170613in}%
\pgfsys@useobject{currentmarker}{}%
\end{pgfscope}%
\end{pgfscope}%
\begin{pgfscope}%
\definecolor{textcolor}{rgb}{0.000000,0.000000,0.000000}%
\pgfsetstrokecolor{textcolor}%
\pgfsetfillcolor{textcolor}%
\pgftext[x=1.334426in, y=4.122388in, left, base]{\color{textcolor}\rmfamily\fontsize{10.000000}{12.000000}\selectfont \(\displaystyle {14.00}\)}%
\end{pgfscope}%
\begin{pgfscope}%
\definecolor{textcolor}{rgb}{0.000000,0.000000,0.000000}%
\pgfsetstrokecolor{textcolor}%
\pgfsetfillcolor{textcolor}%
\pgftext[x=0.709426in,y=2.589670in,,bottom]{\color{textcolor}\rmfamily\fontsize{15.000000}{18.000000}\selectfont CPU-time [s]}%
\end{pgfscope}%
\begin{pgfscope}%
\pgfpathrectangle{\pgfqpoint{1.748007in}{0.604012in}}{\pgfqpoint{4.667057in}{3.971316in}}%
\pgfusepath{clip}%
\pgfsetrectcap%
\pgfsetroundjoin%
\pgfsetlinewidth{1.505625pt}%
\definecolor{currentstroke}{rgb}{0.121569,0.466667,0.705882}%
\pgfsetstrokecolor{currentstroke}%
\pgfsetdash{}{0pt}%
\pgfpathmoveto{\pgfqpoint{1.942468in}{4.035433in}}%
\pgfpathlineto{\pgfqpoint{1.952191in}{4.394814in}}%
\pgfpathlineto{\pgfqpoint{1.961914in}{4.196990in}}%
\pgfpathlineto{\pgfqpoint{1.971637in}{4.140940in}}%
\pgfpathlineto{\pgfqpoint{1.981360in}{2.739686in}}%
\pgfpathlineto{\pgfqpoint{1.991083in}{3.013343in}}%
\pgfpathlineto{\pgfqpoint{2.000806in}{3.784856in}}%
\pgfpathlineto{\pgfqpoint{2.010529in}{3.336455in}}%
\pgfpathlineto{\pgfqpoint{2.020252in}{2.400089in}}%
\pgfpathlineto{\pgfqpoint{2.029975in}{1.803320in}}%
\pgfpathlineto{\pgfqpoint{2.039698in}{1.971470in}}%
\pgfpathlineto{\pgfqpoint{2.049421in}{1.734081in}}%
\pgfpathlineto{\pgfqpoint{2.059144in}{1.450534in}}%
\pgfpathlineto{\pgfqpoint{2.068868in}{1.269195in}}%
\pgfpathlineto{\pgfqpoint{2.078591in}{2.436356in}}%
\pgfpathlineto{\pgfqpoint{2.088314in}{1.186768in}}%
\pgfpathlineto{\pgfqpoint{2.098037in}{1.110936in}}%
\pgfpathlineto{\pgfqpoint{2.107760in}{1.414266in}}%
\pgfpathlineto{\pgfqpoint{2.117483in}{1.575822in}}%
\pgfpathlineto{\pgfqpoint{2.127206in}{1.196659in}}%
\pgfpathlineto{\pgfqpoint{2.136929in}{1.157095in}}%
\pgfpathlineto{\pgfqpoint{2.146652in}{1.312057in}}%
\pgfpathlineto{\pgfqpoint{2.156375in}{0.814200in}}%
\pgfpathlineto{\pgfqpoint{2.166098in}{0.939488in}}%
\pgfpathlineto{\pgfqpoint{2.175821in}{0.899923in}}%
\pgfpathlineto{\pgfqpoint{2.204990in}{1.035103in}}%
\pgfpathlineto{\pgfqpoint{2.214713in}{1.077965in}}%
\pgfpathlineto{\pgfqpoint{2.224436in}{1.028509in}}%
\pgfpathlineto{\pgfqpoint{2.234159in}{1.035103in}}%
\pgfpathlineto{\pgfqpoint{2.243882in}{1.081262in}}%
\pgfpathlineto{\pgfqpoint{2.253605in}{0.906518in}}%
\pgfpathlineto{\pgfqpoint{2.263328in}{1.282383in}}%
\pgfpathlineto{\pgfqpoint{2.273051in}{1.058183in}}%
\pgfpathlineto{\pgfqpoint{2.282774in}{1.219739in}}%
\pgfpathlineto{\pgfqpoint{2.292497in}{1.018618in}}%
\pgfpathlineto{\pgfqpoint{2.302220in}{1.229630in}}%
\pgfpathlineto{\pgfqpoint{2.311943in}{1.315354in}}%
\pgfpathlineto{\pgfqpoint{2.321666in}{1.186768in}}%
\pgfpathlineto{\pgfqpoint{2.331389in}{0.866953in}}%
\pgfpathlineto{\pgfqpoint{2.341112in}{0.942785in}}%
\pgfpathlineto{\pgfqpoint{2.350836in}{0.982350in}}%
\pgfpathlineto{\pgfqpoint{2.360559in}{0.784526in}}%
\pgfpathlineto{\pgfqpoint{2.370282in}{1.120827in}}%
\pgfpathlineto{\pgfqpoint{2.380005in}{1.114233in}}%
\pgfpathlineto{\pgfqpoint{2.389728in}{1.087856in}}%
\pgfpathlineto{\pgfqpoint{2.399451in}{0.893329in}}%
\pgfpathlineto{\pgfqpoint{2.409174in}{0.926300in}}%
\pgfpathlineto{\pgfqpoint{2.418897in}{0.899923in}}%
\pgfpathlineto{\pgfqpoint{2.428620in}{0.936191in}}%
\pgfpathlineto{\pgfqpoint{2.438343in}{0.903221in}}%
\pgfpathlineto{\pgfqpoint{2.448066in}{1.186768in}}%
\pgfpathlineto{\pgfqpoint{2.457789in}{1.236224in}}%
\pgfpathlineto{\pgfqpoint{2.467512in}{1.226333in}}%
\pgfpathlineto{\pgfqpoint{2.477235in}{1.223036in}}%
\pgfpathlineto{\pgfqpoint{2.486958in}{0.998835in}}%
\pgfpathlineto{\pgfqpoint{2.496681in}{0.860359in}}%
\pgfpathlineto{\pgfqpoint{2.506404in}{1.107639in}}%
\pgfpathlineto{\pgfqpoint{2.516127in}{1.064777in}}%
\pgfpathlineto{\pgfqpoint{2.525850in}{1.087856in}}%
\pgfpathlineto{\pgfqpoint{2.535573in}{0.979053in}}%
\pgfpathlineto{\pgfqpoint{2.545296in}{0.850468in}}%
\pgfpathlineto{\pgfqpoint{2.555019in}{0.932894in}}%
\pgfpathlineto{\pgfqpoint{2.564742in}{0.880141in}}%
\pgfpathlineto{\pgfqpoint{2.574465in}{1.074668in}}%
\pgfpathlineto{\pgfqpoint{2.584188in}{1.091153in}}%
\pgfpathlineto{\pgfqpoint{2.593911in}{1.035103in}}%
\pgfpathlineto{\pgfqpoint{2.603634in}{1.064777in}}%
\pgfpathlineto{\pgfqpoint{2.613357in}{1.008727in}}%
\pgfpathlineto{\pgfqpoint{2.623080in}{1.012024in}}%
\pgfpathlineto{\pgfqpoint{2.632804in}{1.041697in}}%
\pgfpathlineto{\pgfqpoint{2.642527in}{0.969162in}}%
\pgfpathlineto{\pgfqpoint{2.652250in}{0.913112in}}%
\pgfpathlineto{\pgfqpoint{2.661973in}{0.959271in}}%
\pgfpathlineto{\pgfqpoint{2.671696in}{0.955974in}}%
\pgfpathlineto{\pgfqpoint{2.681419in}{0.982350in}}%
\pgfpathlineto{\pgfqpoint{2.691142in}{0.962568in}}%
\pgfpathlineto{\pgfqpoint{2.700865in}{1.041697in}}%
\pgfpathlineto{\pgfqpoint{2.710588in}{0.975756in}}%
\pgfpathlineto{\pgfqpoint{2.720311in}{0.979053in}}%
\pgfpathlineto{\pgfqpoint{2.730034in}{0.909815in}}%
\pgfpathlineto{\pgfqpoint{2.739757in}{0.932894in}}%
\pgfpathlineto{\pgfqpoint{2.749480in}{0.942785in}}%
\pgfpathlineto{\pgfqpoint{2.759203in}{0.972459in}}%
\pgfpathlineto{\pgfqpoint{2.768926in}{0.886735in}}%
\pgfpathlineto{\pgfqpoint{2.778649in}{0.896626in}}%
\pgfpathlineto{\pgfqpoint{2.788372in}{0.913112in}}%
\pgfpathlineto{\pgfqpoint{2.798095in}{0.939488in}}%
\pgfpathlineto{\pgfqpoint{2.807818in}{0.962568in}}%
\pgfpathlineto{\pgfqpoint{2.817541in}{0.866953in}}%
\pgfpathlineto{\pgfqpoint{2.827264in}{0.873547in}}%
\pgfpathlineto{\pgfqpoint{2.836987in}{0.893329in}}%
\pgfpathlineto{\pgfqpoint{2.846710in}{0.923003in}}%
\pgfpathlineto{\pgfqpoint{2.856433in}{0.926300in}}%
\pgfpathlineto{\pgfqpoint{2.866156in}{0.952677in}}%
\pgfpathlineto{\pgfqpoint{2.875879in}{0.985647in}}%
\pgfpathlineto{\pgfqpoint{2.885602in}{0.873547in}}%
\pgfpathlineto{\pgfqpoint{2.895325in}{0.896626in}}%
\pgfpathlineto{\pgfqpoint{2.905048in}{0.909815in}}%
\pgfpathlineto{\pgfqpoint{2.914772in}{0.929597in}}%
\pgfpathlineto{\pgfqpoint{2.924495in}{0.932894in}}%
\pgfpathlineto{\pgfqpoint{2.934218in}{0.972459in}}%
\pgfpathlineto{\pgfqpoint{2.943941in}{0.975756in}}%
\pgfpathlineto{\pgfqpoint{2.953664in}{0.876844in}}%
\pgfpathlineto{\pgfqpoint{2.963387in}{0.880141in}}%
\pgfpathlineto{\pgfqpoint{2.973110in}{0.893329in}}%
\pgfpathlineto{\pgfqpoint{2.982833in}{0.923003in}}%
\pgfpathlineto{\pgfqpoint{2.992556in}{0.932894in}}%
\pgfpathlineto{\pgfqpoint{3.002279in}{0.939488in}}%
\pgfpathlineto{\pgfqpoint{3.012002in}{0.807606in}}%
\pgfpathlineto{\pgfqpoint{3.021725in}{0.824091in}}%
\pgfpathlineto{\pgfqpoint{3.031448in}{0.857062in}}%
\pgfpathlineto{\pgfqpoint{3.041171in}{0.866953in}}%
\pgfpathlineto{\pgfqpoint{3.050894in}{0.873547in}}%
\pgfpathlineto{\pgfqpoint{3.060617in}{0.909815in}}%
\pgfpathlineto{\pgfqpoint{3.070340in}{0.919706in}}%
\pgfpathlineto{\pgfqpoint{3.080063in}{0.952677in}}%
\pgfpathlineto{\pgfqpoint{3.089786in}{0.959271in}}%
\pgfpathlineto{\pgfqpoint{3.099509in}{0.969162in}}%
\pgfpathlineto{\pgfqpoint{3.109232in}{0.975756in}}%
\pgfpathlineto{\pgfqpoint{3.118955in}{0.985647in}}%
\pgfpathlineto{\pgfqpoint{3.128678in}{0.827388in}}%
\pgfpathlineto{\pgfqpoint{3.138401in}{0.850468in}}%
\pgfpathlineto{\pgfqpoint{3.148124in}{0.863656in}}%
\pgfpathlineto{\pgfqpoint{3.157847in}{0.880141in}}%
\pgfpathlineto{\pgfqpoint{3.167570in}{0.899923in}}%
\pgfpathlineto{\pgfqpoint{3.177293in}{0.909815in}}%
\pgfpathlineto{\pgfqpoint{3.187016in}{0.916409in}}%
\pgfpathlineto{\pgfqpoint{3.196740in}{0.962568in}}%
\pgfpathlineto{\pgfqpoint{3.206463in}{0.955974in}}%
\pgfpathlineto{\pgfqpoint{3.216186in}{0.972459in}}%
\pgfpathlineto{\pgfqpoint{3.225909in}{0.985647in}}%
\pgfpathlineto{\pgfqpoint{3.235632in}{1.005430in}}%
\pgfpathlineto{\pgfqpoint{3.245355in}{1.018618in}}%
\pgfpathlineto{\pgfqpoint{3.255078in}{0.863656in}}%
\pgfpathlineto{\pgfqpoint{3.264801in}{0.896626in}}%
\pgfpathlineto{\pgfqpoint{3.274524in}{0.903221in}}%
\pgfpathlineto{\pgfqpoint{3.284247in}{0.890032in}}%
\pgfpathlineto{\pgfqpoint{3.293970in}{0.919706in}}%
\pgfpathlineto{\pgfqpoint{3.303693in}{0.923003in}}%
\pgfpathlineto{\pgfqpoint{3.313416in}{0.939488in}}%
\pgfpathlineto{\pgfqpoint{3.323139in}{0.942785in}}%
\pgfpathlineto{\pgfqpoint{3.332862in}{0.959271in}}%
\pgfpathlineto{\pgfqpoint{3.342585in}{0.979053in}}%
\pgfpathlineto{\pgfqpoint{3.352308in}{0.992241in}}%
\pgfpathlineto{\pgfqpoint{3.362031in}{0.998835in}}%
\pgfpathlineto{\pgfqpoint{3.371754in}{1.021915in}}%
\pgfpathlineto{\pgfqpoint{3.381477in}{1.021915in}}%
\pgfpathlineto{\pgfqpoint{3.391200in}{1.048291in}}%
\pgfpathlineto{\pgfqpoint{3.400923in}{1.058183in}}%
\pgfpathlineto{\pgfqpoint{3.410646in}{1.064777in}}%
\pgfpathlineto{\pgfqpoint{3.420369in}{1.091153in}}%
\pgfpathlineto{\pgfqpoint{3.430092in}{1.101045in}}%
\pgfpathlineto{\pgfqpoint{3.439815in}{1.127421in}}%
\pgfpathlineto{\pgfqpoint{3.449538in}{0.863656in}}%
\pgfpathlineto{\pgfqpoint{3.459261in}{0.883438in}}%
\pgfpathlineto{\pgfqpoint{3.468984in}{0.899923in}}%
\pgfpathlineto{\pgfqpoint{3.478708in}{0.906518in}}%
\pgfpathlineto{\pgfqpoint{3.488431in}{0.923003in}}%
\pgfpathlineto{\pgfqpoint{3.498154in}{0.936191in}}%
\pgfpathlineto{\pgfqpoint{3.507877in}{0.952677in}}%
\pgfpathlineto{\pgfqpoint{3.517600in}{0.965865in}}%
\pgfpathlineto{\pgfqpoint{3.527323in}{0.982350in}}%
\pgfpathlineto{\pgfqpoint{3.537046in}{0.979053in}}%
\pgfpathlineto{\pgfqpoint{3.546769in}{1.012024in}}%
\pgfpathlineto{\pgfqpoint{3.556492in}{1.018618in}}%
\pgfpathlineto{\pgfqpoint{3.566215in}{1.038400in}}%
\pgfpathlineto{\pgfqpoint{3.575938in}{1.061480in}}%
\pgfpathlineto{\pgfqpoint{3.585661in}{1.064777in}}%
\pgfpathlineto{\pgfqpoint{3.614830in}{1.094450in}}%
\pgfpathlineto{\pgfqpoint{3.643999in}{1.143906in}}%
\pgfpathlineto{\pgfqpoint{3.653722in}{1.153798in}}%
\pgfpathlineto{\pgfqpoint{3.663445in}{1.180174in}}%
\pgfpathlineto{\pgfqpoint{3.673168in}{1.199957in}}%
\pgfpathlineto{\pgfqpoint{3.682891in}{0.906518in}}%
\pgfpathlineto{\pgfqpoint{3.692614in}{0.886735in}}%
\pgfpathlineto{\pgfqpoint{3.702337in}{0.899923in}}%
\pgfpathlineto{\pgfqpoint{3.721783in}{0.919706in}}%
\pgfpathlineto{\pgfqpoint{3.731506in}{0.939488in}}%
\pgfpathlineto{\pgfqpoint{3.741229in}{0.949379in}}%
\pgfpathlineto{\pgfqpoint{3.750953in}{0.975756in}}%
\pgfpathlineto{\pgfqpoint{3.760676in}{0.988944in}}%
\pgfpathlineto{\pgfqpoint{3.770399in}{0.998835in}}%
\pgfpathlineto{\pgfqpoint{3.780122in}{0.995538in}}%
\pgfpathlineto{\pgfqpoint{3.789845in}{1.005430in}}%
\pgfpathlineto{\pgfqpoint{3.799568in}{1.021915in}}%
\pgfpathlineto{\pgfqpoint{3.809291in}{1.028509in}}%
\pgfpathlineto{\pgfqpoint{3.819014in}{1.044994in}}%
\pgfpathlineto{\pgfqpoint{3.828737in}{1.054886in}}%
\pgfpathlineto{\pgfqpoint{3.838460in}{1.074668in}}%
\pgfpathlineto{\pgfqpoint{3.848183in}{1.087856in}}%
\pgfpathlineto{\pgfqpoint{3.857906in}{1.097747in}}%
\pgfpathlineto{\pgfqpoint{3.877352in}{1.130718in}}%
\pgfpathlineto{\pgfqpoint{3.887075in}{1.166986in}}%
\pgfpathlineto{\pgfqpoint{3.896798in}{1.173580in}}%
\pgfpathlineto{\pgfqpoint{3.906521in}{1.176877in}}%
\pgfpathlineto{\pgfqpoint{3.916244in}{1.216442in}}%
\pgfpathlineto{\pgfqpoint{3.925967in}{1.193362in}}%
\pgfpathlineto{\pgfqpoint{3.935690in}{1.209848in}}%
\pgfpathlineto{\pgfqpoint{3.945413in}{1.229630in}}%
\pgfpathlineto{\pgfqpoint{3.955136in}{1.269195in}}%
\pgfpathlineto{\pgfqpoint{3.964859in}{1.239521in}}%
\pgfpathlineto{\pgfqpoint{3.984305in}{1.292274in}}%
\pgfpathlineto{\pgfqpoint{3.994028in}{1.295571in}}%
\pgfpathlineto{\pgfqpoint{4.003751in}{1.308760in}}%
\pgfpathlineto{\pgfqpoint{4.013474in}{1.328542in}}%
\pgfpathlineto{\pgfqpoint{4.023197in}{1.345027in}}%
\pgfpathlineto{\pgfqpoint{4.032921in}{1.358216in}}%
\pgfpathlineto{\pgfqpoint{4.042644in}{1.361513in}}%
\pgfpathlineto{\pgfqpoint{4.052367in}{1.374701in}}%
\pgfpathlineto{\pgfqpoint{4.062090in}{1.384592in}}%
\pgfpathlineto{\pgfqpoint{4.071813in}{1.391186in}}%
\pgfpathlineto{\pgfqpoint{4.081536in}{1.404375in}}%
\pgfpathlineto{\pgfqpoint{4.091259in}{0.946082in}}%
\pgfpathlineto{\pgfqpoint{4.100982in}{0.975756in}}%
\pgfpathlineto{\pgfqpoint{4.110705in}{0.975756in}}%
\pgfpathlineto{\pgfqpoint{4.120428in}{0.985647in}}%
\pgfpathlineto{\pgfqpoint{4.130151in}{0.998835in}}%
\pgfpathlineto{\pgfqpoint{4.139874in}{1.005430in}}%
\pgfpathlineto{\pgfqpoint{4.149597in}{1.021915in}}%
\pgfpathlineto{\pgfqpoint{4.159320in}{1.031806in}}%
\pgfpathlineto{\pgfqpoint{4.169043in}{1.044994in}}%
\pgfpathlineto{\pgfqpoint{4.178766in}{1.051589in}}%
\pgfpathlineto{\pgfqpoint{4.188489in}{1.064777in}}%
\pgfpathlineto{\pgfqpoint{4.207935in}{1.084559in}}%
\pgfpathlineto{\pgfqpoint{4.217658in}{1.097747in}}%
\pgfpathlineto{\pgfqpoint{4.227381in}{1.104342in}}%
\pgfpathlineto{\pgfqpoint{4.237104in}{1.114233in}}%
\pgfpathlineto{\pgfqpoint{4.246827in}{1.120827in}}%
\pgfpathlineto{\pgfqpoint{4.256550in}{1.143906in}}%
\pgfpathlineto{\pgfqpoint{4.266273in}{1.170283in}}%
\pgfpathlineto{\pgfqpoint{4.275996in}{1.180174in}}%
\pgfpathlineto{\pgfqpoint{4.285719in}{1.183471in}}%
\pgfpathlineto{\pgfqpoint{4.295442in}{1.183471in}}%
\pgfpathlineto{\pgfqpoint{4.305165in}{1.193362in}}%
\pgfpathlineto{\pgfqpoint{4.314889in}{1.219739in}}%
\pgfpathlineto{\pgfqpoint{4.324612in}{1.223036in}}%
\pgfpathlineto{\pgfqpoint{4.334335in}{1.246115in}}%
\pgfpathlineto{\pgfqpoint{4.353781in}{1.265898in}}%
\pgfpathlineto{\pgfqpoint{4.363504in}{1.279086in}}%
\pgfpathlineto{\pgfqpoint{4.373227in}{1.282383in}}%
\pgfpathlineto{\pgfqpoint{4.382950in}{1.308760in}}%
\pgfpathlineto{\pgfqpoint{4.402396in}{1.315354in}}%
\pgfpathlineto{\pgfqpoint{4.412119in}{1.321948in}}%
\pgfpathlineto{\pgfqpoint{4.421842in}{1.354919in}}%
\pgfpathlineto{\pgfqpoint{4.431565in}{1.361513in}}%
\pgfpathlineto{\pgfqpoint{4.441288in}{1.387889in}}%
\pgfpathlineto{\pgfqpoint{4.451011in}{1.391186in}}%
\pgfpathlineto{\pgfqpoint{4.460734in}{1.407672in}}%
\pgfpathlineto{\pgfqpoint{4.470457in}{1.414266in}}%
\pgfpathlineto{\pgfqpoint{4.480180in}{1.417563in}}%
\pgfpathlineto{\pgfqpoint{4.489903in}{1.427454in}}%
\pgfpathlineto{\pgfqpoint{4.499626in}{1.453831in}}%
\pgfpathlineto{\pgfqpoint{4.509349in}{1.473613in}}%
\pgfpathlineto{\pgfqpoint{4.519072in}{1.476910in}}%
\pgfpathlineto{\pgfqpoint{4.528795in}{1.467019in}}%
\pgfpathlineto{\pgfqpoint{4.538518in}{1.476910in}}%
\pgfpathlineto{\pgfqpoint{4.548241in}{1.490098in}}%
\pgfpathlineto{\pgfqpoint{4.557964in}{1.499990in}}%
\pgfpathlineto{\pgfqpoint{4.567687in}{1.526366in}}%
\pgfpathlineto{\pgfqpoint{4.577410in}{1.542851in}}%
\pgfpathlineto{\pgfqpoint{4.587133in}{1.549446in}}%
\pgfpathlineto{\pgfqpoint{4.606580in}{1.569228in}}%
\pgfpathlineto{\pgfqpoint{4.616303in}{1.589010in}}%
\pgfpathlineto{\pgfqpoint{4.626026in}{1.615387in}}%
\pgfpathlineto{\pgfqpoint{4.635749in}{1.648358in}}%
\pgfpathlineto{\pgfqpoint{4.645472in}{1.664843in}}%
\pgfpathlineto{\pgfqpoint{4.664918in}{1.684625in}}%
\pgfpathlineto{\pgfqpoint{4.674641in}{1.674734in}}%
\pgfpathlineto{\pgfqpoint{4.684364in}{1.687922in}}%
\pgfpathlineto{\pgfqpoint{4.694087in}{1.704408in}}%
\pgfpathlineto{\pgfqpoint{4.713533in}{1.717596in}}%
\pgfpathlineto{\pgfqpoint{4.723256in}{1.740675in}}%
\pgfpathlineto{\pgfqpoint{4.732979in}{1.734081in}}%
\pgfpathlineto{\pgfqpoint{4.742702in}{1.747269in}}%
\pgfpathlineto{\pgfqpoint{4.752425in}{1.776943in}}%
\pgfpathlineto{\pgfqpoint{4.762148in}{1.773646in}}%
\pgfpathlineto{\pgfqpoint{4.771871in}{1.790131in}}%
\pgfpathlineto{\pgfqpoint{4.781594in}{1.800023in}}%
\pgfpathlineto{\pgfqpoint{4.791317in}{1.819805in}}%
\pgfpathlineto{\pgfqpoint{4.810763in}{1.839587in}}%
\pgfpathlineto{\pgfqpoint{4.820486in}{1.856073in}}%
\pgfpathlineto{\pgfqpoint{4.830209in}{1.889043in}}%
\pgfpathlineto{\pgfqpoint{4.839932in}{1.902232in}}%
\pgfpathlineto{\pgfqpoint{4.849655in}{1.895637in}}%
\pgfpathlineto{\pgfqpoint{4.907994in}{1.974767in}}%
\pgfpathlineto{\pgfqpoint{4.917717in}{1.176877in}}%
\pgfpathlineto{\pgfqpoint{4.927440in}{1.183471in}}%
\pgfpathlineto{\pgfqpoint{4.937163in}{1.196659in}}%
\pgfpathlineto{\pgfqpoint{4.946886in}{1.206551in}}%
\pgfpathlineto{\pgfqpoint{4.956609in}{1.219739in}}%
\pgfpathlineto{\pgfqpoint{4.966332in}{1.239521in}}%
\pgfpathlineto{\pgfqpoint{4.976055in}{1.262601in}}%
\pgfpathlineto{\pgfqpoint{4.985778in}{1.259304in}}%
\pgfpathlineto{\pgfqpoint{4.995501in}{1.265898in}}%
\pgfpathlineto{\pgfqpoint{5.005224in}{1.285680in}}%
\pgfpathlineto{\pgfqpoint{5.014947in}{1.285680in}}%
\pgfpathlineto{\pgfqpoint{5.024670in}{1.305463in}}%
\pgfpathlineto{\pgfqpoint{5.034393in}{1.312057in}}%
\pgfpathlineto{\pgfqpoint{5.044116in}{1.321948in}}%
\pgfpathlineto{\pgfqpoint{5.053839in}{1.328542in}}%
\pgfpathlineto{\pgfqpoint{5.063562in}{1.338433in}}%
\pgfpathlineto{\pgfqpoint{5.073285in}{1.335136in}}%
\pgfpathlineto{\pgfqpoint{5.083008in}{1.345027in}}%
\pgfpathlineto{\pgfqpoint{5.092731in}{1.351622in}}%
\pgfpathlineto{\pgfqpoint{5.112177in}{1.371404in}}%
\pgfpathlineto{\pgfqpoint{5.121900in}{1.377998in}}%
\pgfpathlineto{\pgfqpoint{5.131623in}{1.387889in}}%
\pgfpathlineto{\pgfqpoint{5.141346in}{1.401078in}}%
\pgfpathlineto{\pgfqpoint{5.151069in}{1.407672in}}%
\pgfpathlineto{\pgfqpoint{5.160793in}{1.420860in}}%
\pgfpathlineto{\pgfqpoint{5.180239in}{1.434048in}}%
\pgfpathlineto{\pgfqpoint{5.189962in}{1.453831in}}%
\pgfpathlineto{\pgfqpoint{5.199685in}{1.463722in}}%
\pgfpathlineto{\pgfqpoint{5.209408in}{1.470316in}}%
\pgfpathlineto{\pgfqpoint{5.219131in}{1.480207in}}%
\pgfpathlineto{\pgfqpoint{5.228854in}{1.486801in}}%
\pgfpathlineto{\pgfqpoint{5.238577in}{1.516475in}}%
\pgfpathlineto{\pgfqpoint{5.248300in}{1.529663in}}%
\pgfpathlineto{\pgfqpoint{5.258023in}{1.526366in}}%
\pgfpathlineto{\pgfqpoint{5.267746in}{1.529663in}}%
\pgfpathlineto{\pgfqpoint{5.277469in}{1.539554in}}%
\pgfpathlineto{\pgfqpoint{5.296915in}{1.598902in}}%
\pgfpathlineto{\pgfqpoint{5.306638in}{1.569228in}}%
\pgfpathlineto{\pgfqpoint{5.326084in}{1.621981in}}%
\pgfpathlineto{\pgfqpoint{5.335807in}{1.612090in}}%
\pgfpathlineto{\pgfqpoint{5.345530in}{1.615387in}}%
\pgfpathlineto{\pgfqpoint{5.355253in}{1.625278in}}%
\pgfpathlineto{\pgfqpoint{5.364976in}{1.641763in}}%
\pgfpathlineto{\pgfqpoint{5.374699in}{1.674734in}}%
\pgfpathlineto{\pgfqpoint{5.384422in}{1.691219in}}%
\pgfpathlineto{\pgfqpoint{5.394145in}{1.684625in}}%
\pgfpathlineto{\pgfqpoint{5.403868in}{1.704408in}}%
\pgfpathlineto{\pgfqpoint{5.413591in}{1.701111in}}%
\pgfpathlineto{\pgfqpoint{5.423314in}{1.727487in}}%
\pgfpathlineto{\pgfqpoint{5.433037in}{1.720893in}}%
\pgfpathlineto{\pgfqpoint{5.442761in}{1.730784in}}%
\pgfpathlineto{\pgfqpoint{5.462207in}{1.737378in}}%
\pgfpathlineto{\pgfqpoint{5.471930in}{1.753864in}}%
\pgfpathlineto{\pgfqpoint{5.481653in}{1.760458in}}%
\pgfpathlineto{\pgfqpoint{5.491376in}{1.783537in}}%
\pgfpathlineto{\pgfqpoint{5.501099in}{1.786834in}}%
\pgfpathlineto{\pgfqpoint{5.510822in}{1.793428in}}%
\pgfpathlineto{\pgfqpoint{5.520545in}{1.806617in}}%
\pgfpathlineto{\pgfqpoint{5.539991in}{1.826399in}}%
\pgfpathlineto{\pgfqpoint{5.549714in}{1.839587in}}%
\pgfpathlineto{\pgfqpoint{5.559437in}{1.849479in}}%
\pgfpathlineto{\pgfqpoint{5.569160in}{1.852776in}}%
\pgfpathlineto{\pgfqpoint{5.578883in}{1.872558in}}%
\pgfpathlineto{\pgfqpoint{5.588606in}{1.905529in}}%
\pgfpathlineto{\pgfqpoint{5.608052in}{1.905529in}}%
\pgfpathlineto{\pgfqpoint{5.617775in}{1.912123in}}%
\pgfpathlineto{\pgfqpoint{5.627498in}{1.922014in}}%
\pgfpathlineto{\pgfqpoint{5.637221in}{1.935202in}}%
\pgfpathlineto{\pgfqpoint{5.646944in}{1.945093in}}%
\pgfpathlineto{\pgfqpoint{5.656667in}{1.964876in}}%
\pgfpathlineto{\pgfqpoint{5.666390in}{1.961579in}}%
\pgfpathlineto{\pgfqpoint{5.676113in}{1.981361in}}%
\pgfpathlineto{\pgfqpoint{5.685836in}{1.991252in}}%
\pgfpathlineto{\pgfqpoint{5.695559in}{2.011035in}}%
\pgfpathlineto{\pgfqpoint{5.705282in}{2.063788in}}%
\pgfpathlineto{\pgfqpoint{5.715006in}{2.047303in}}%
\pgfpathlineto{\pgfqpoint{5.724729in}{2.063788in}}%
\pgfpathlineto{\pgfqpoint{5.744175in}{2.083570in}}%
\pgfpathlineto{\pgfqpoint{5.753898in}{2.129729in}}%
\pgfpathlineto{\pgfqpoint{5.763621in}{2.139620in}}%
\pgfpathlineto{\pgfqpoint{5.773344in}{2.096759in}}%
\pgfpathlineto{\pgfqpoint{5.792790in}{2.123135in}}%
\pgfpathlineto{\pgfqpoint{5.802513in}{2.133026in}}%
\pgfpathlineto{\pgfqpoint{5.812236in}{2.172591in}}%
\pgfpathlineto{\pgfqpoint{5.821959in}{2.169294in}}%
\pgfpathlineto{\pgfqpoint{5.831682in}{2.198968in}}%
\pgfpathlineto{\pgfqpoint{5.841405in}{2.179185in}}%
\pgfpathlineto{\pgfqpoint{5.851128in}{2.218750in}}%
\pgfpathlineto{\pgfqpoint{5.860851in}{2.231938in}}%
\pgfpathlineto{\pgfqpoint{5.870574in}{2.271503in}}%
\pgfpathlineto{\pgfqpoint{5.880297in}{2.261612in}}%
\pgfpathlineto{\pgfqpoint{5.890020in}{2.248424in}}%
\pgfpathlineto{\pgfqpoint{5.899743in}{2.258315in}}%
\pgfpathlineto{\pgfqpoint{5.909466in}{2.274800in}}%
\pgfpathlineto{\pgfqpoint{5.919189in}{2.281394in}}%
\pgfpathlineto{\pgfqpoint{5.938635in}{2.314365in}}%
\pgfpathlineto{\pgfqpoint{5.948358in}{2.340741in}}%
\pgfpathlineto{\pgfqpoint{5.958081in}{2.363821in}}%
\pgfpathlineto{\pgfqpoint{5.967804in}{2.367118in}}%
\pgfpathlineto{\pgfqpoint{5.977527in}{2.409980in}}%
\pgfpathlineto{\pgfqpoint{5.987250in}{2.396792in}}%
\pgfpathlineto{\pgfqpoint{5.996974in}{2.426465in}}%
\pgfpathlineto{\pgfqpoint{6.016420in}{2.406683in}}%
\pgfpathlineto{\pgfqpoint{6.026143in}{2.409980in}}%
\pgfpathlineto{\pgfqpoint{6.035866in}{2.429762in}}%
\pgfpathlineto{\pgfqpoint{6.045589in}{2.439653in}}%
\pgfpathlineto{\pgfqpoint{6.055312in}{2.462733in}}%
\pgfpathlineto{\pgfqpoint{6.065035in}{2.459436in}}%
\pgfpathlineto{\pgfqpoint{6.074758in}{2.485812in}}%
\pgfpathlineto{\pgfqpoint{6.084481in}{2.502298in}}%
\pgfpathlineto{\pgfqpoint{6.094204in}{2.508892in}}%
\pgfpathlineto{\pgfqpoint{6.103927in}{2.541862in}}%
\pgfpathlineto{\pgfqpoint{6.113650in}{2.522080in}}%
\pgfpathlineto{\pgfqpoint{6.123373in}{2.545160in}}%
\pgfpathlineto{\pgfqpoint{6.133096in}{2.601210in}}%
\pgfpathlineto{\pgfqpoint{6.142819in}{2.564942in}}%
\pgfpathlineto{\pgfqpoint{6.152542in}{2.571536in}}%
\pgfpathlineto{\pgfqpoint{6.162265in}{2.581427in}}%
\pgfpathlineto{\pgfqpoint{6.171988in}{2.611101in}}%
\pgfpathlineto{\pgfqpoint{6.181711in}{2.611101in}}%
\pgfpathlineto{\pgfqpoint{6.191434in}{2.663854in}}%
\pgfpathlineto{\pgfqpoint{6.201157in}{2.710013in}}%
\pgfpathlineto{\pgfqpoint{6.210880in}{2.680339in}}%
\pgfpathlineto{\pgfqpoint{6.220603in}{2.690230in}}%
\pgfpathlineto{\pgfqpoint{6.230326in}{2.696825in}}%
\pgfpathlineto{\pgfqpoint{6.240049in}{2.693527in}}%
\pgfpathlineto{\pgfqpoint{6.249772in}{2.706716in}}%
\pgfpathlineto{\pgfqpoint{6.259495in}{2.729795in}}%
\pgfpathlineto{\pgfqpoint{6.269218in}{2.742983in}}%
\pgfpathlineto{\pgfqpoint{6.278942in}{2.759469in}}%
\pgfpathlineto{\pgfqpoint{6.288665in}{2.822113in}}%
\pgfpathlineto{\pgfqpoint{6.298388in}{2.769360in}}%
\pgfpathlineto{\pgfqpoint{6.308111in}{2.782548in}}%
\pgfpathlineto{\pgfqpoint{6.317834in}{2.802331in}}%
\pgfpathlineto{\pgfqpoint{6.337280in}{2.808925in}}%
\pgfpathlineto{\pgfqpoint{6.347003in}{2.832004in}}%
\pgfpathlineto{\pgfqpoint{6.356726in}{2.874866in}}%
\pgfpathlineto{\pgfqpoint{6.366449in}{2.861678in}}%
\pgfpathlineto{\pgfqpoint{6.376172in}{2.861678in}}%
\pgfpathlineto{\pgfqpoint{6.385895in}{2.868272in}}%
\pgfpathlineto{\pgfqpoint{6.395618in}{2.891351in}}%
\pgfpathlineto{\pgfqpoint{6.395618in}{2.891351in}}%
\pgfusepath{stroke}%
\end{pgfscope}%
\begin{pgfscope}%
\pgfpathrectangle{\pgfqpoint{1.748007in}{0.604012in}}{\pgfqpoint{4.667057in}{3.971316in}}%
\pgfusepath{clip}%
\pgfsetbuttcap%
\pgfsetroundjoin%
\pgfsetlinewidth{0.501875pt}%
\definecolor{currentstroke}{rgb}{0.000000,0.000000,0.000000}%
\pgfsetstrokecolor{currentstroke}%
\pgfsetdash{{1.850000pt}{0.800000pt}}{0.000000pt}%
\pgfpathmoveto{\pgfqpoint{1.942468in}{0.604012in}}%
\pgfpathlineto{\pgfqpoint{1.942468in}{4.575328in}}%
\pgfusepath{stroke}%
\end{pgfscope}%
\begin{pgfscope}%
\pgfpathrectangle{\pgfqpoint{1.748007in}{0.604012in}}{\pgfqpoint{4.667057in}{3.971316in}}%
\pgfusepath{clip}%
\pgfsetrectcap%
\pgfsetroundjoin%
\pgfsetlinewidth{0.501875pt}%
\definecolor{currentstroke}{rgb}{1.000000,0.000000,0.000000}%
\pgfsetstrokecolor{currentstroke}%
\pgfsetdash{}{0pt}%
\pgfpathmoveto{\pgfqpoint{1.748007in}{1.252710in}}%
\pgfpathlineto{\pgfqpoint{6.415064in}{1.252710in}}%
\pgfusepath{stroke}%
\end{pgfscope}%
\begin{pgfscope}%
\pgfsetrectcap%
\pgfsetmiterjoin%
\pgfsetlinewidth{0.803000pt}%
\definecolor{currentstroke}{rgb}{0.000000,0.000000,0.000000}%
\pgfsetstrokecolor{currentstroke}%
\pgfsetdash{}{0pt}%
\pgfpathmoveto{\pgfqpoint{1.748007in}{0.604012in}}%
\pgfpathlineto{\pgfqpoint{1.748007in}{4.575328in}}%
\pgfusepath{stroke}%
\end{pgfscope}%
\begin{pgfscope}%
\pgfsetrectcap%
\pgfsetmiterjoin%
\pgfsetlinewidth{0.803000pt}%
\definecolor{currentstroke}{rgb}{0.000000,0.000000,0.000000}%
\pgfsetstrokecolor{currentstroke}%
\pgfsetdash{}{0pt}%
\pgfpathmoveto{\pgfqpoint{6.415064in}{0.604012in}}%
\pgfpathlineto{\pgfqpoint{6.415064in}{4.575328in}}%
\pgfusepath{stroke}%
\end{pgfscope}%
\begin{pgfscope}%
\pgfsetrectcap%
\pgfsetmiterjoin%
\pgfsetlinewidth{0.803000pt}%
\definecolor{currentstroke}{rgb}{0.000000,0.000000,0.000000}%
\pgfsetstrokecolor{currentstroke}%
\pgfsetdash{}{0pt}%
\pgfpathmoveto{\pgfqpoint{1.748007in}{0.604012in}}%
\pgfpathlineto{\pgfqpoint{6.415064in}{0.604012in}}%
\pgfusepath{stroke}%
\end{pgfscope}%
\begin{pgfscope}%
\pgfsetrectcap%
\pgfsetmiterjoin%
\pgfsetlinewidth{0.803000pt}%
\definecolor{currentstroke}{rgb}{0.000000,0.000000,0.000000}%
\pgfsetstrokecolor{currentstroke}%
\pgfsetdash{}{0pt}%
\pgfpathmoveto{\pgfqpoint{1.748007in}{4.575328in}}%
\pgfpathlineto{\pgfqpoint{6.415064in}{4.575328in}}%
\pgfusepath{stroke}%
\end{pgfscope}%
\end{pgfpicture}%
\makeatother%
\endgroup%
}
	\caption{CPU-timings of restarted GMRES for different restart parameters $m = {20...478} $ with global minimum at $m = 63$ with $3.73 s$, using clang compiler with no optimization (-O0 flag) + disabled Output (-DDISABLEIO) and without preconditioning}
	\label{fig::Timings}
\end{figure}
%
The globally best restart parameter was found to be $m=63$, requiring 25 iterations to converge in 3.73 seconds, which is a decrease by 27.6\% opposed to the full GMRES method. The red line in \refFig{fig::Timings} denotes the CPU-time of the full GMRES method ($m=479$). For all restart parameter m for which the blue line is below the CPU time for full GMRES (indicated by the red line), the restarted formulation of the GMRES method is faster than full GMRES. The paragraphs in the course of CPU time can be attributed to the reduction of the required loop iterations.
%
\begin{figure}[!htbp]
	\centering
	\hspace*{0.8cm}
	\leavevmode
	\resizebox{0.9\width}{!}{%% Creator: Matplotlib, PGF backend
%%
%% To include the figure in your LaTeX document, write
%%   \input{<filename>.pgf}
%%
%% Make sure the required packages are loaded in your preamble
%%   \usepackage{pgf}
%%
%% Also ensure that all the required font packages are loaded; for instance,
%% the lmodern package is sometimes necessary when using math font.
%%   \usepackage{lmodern}
%%
%% Figures using additional raster images can only be included by \input if
%% they are in the same directory as the main LaTeX file. For loading figures
%% from other directories you can use the `import` package
%%   \usepackage{import}
%%
%% and then include the figures with
%%   \import{<path to file>}{<filename>.pgf}
%%
%% Matplotlib used the following preamble
%%   
%%   \makeatletter\@ifpackageloaded{underscore}{}{\usepackage[strings]{underscore}}\makeatother
%%
\begingroup%
\makeatletter%
\begin{pgfpicture}%
\pgfpathrectangle{\pgfpointorigin}{\pgfqpoint{6.565064in}{4.725328in}}%
\pgfusepath{use as bounding box, clip}%
\begin{pgfscope}%
\pgfsetbuttcap%
\pgfsetmiterjoin%
\definecolor{currentfill}{rgb}{1.000000,1.000000,1.000000}%
\pgfsetfillcolor{currentfill}%
\pgfsetlinewidth{0.000000pt}%
\definecolor{currentstroke}{rgb}{1.000000,1.000000,1.000000}%
\pgfsetstrokecolor{currentstroke}%
\pgfsetdash{}{0pt}%
\pgfpathmoveto{\pgfqpoint{0.000000in}{0.000000in}}%
\pgfpathlineto{\pgfqpoint{6.565064in}{0.000000in}}%
\pgfpathlineto{\pgfqpoint{6.565064in}{4.725328in}}%
\pgfpathlineto{\pgfqpoint{0.000000in}{4.725328in}}%
\pgfpathlineto{\pgfqpoint{0.000000in}{0.000000in}}%
\pgfpathclose%
\pgfusepath{fill}%
\end{pgfscope}%
\begin{pgfscope}%
\pgfsetbuttcap%
\pgfsetmiterjoin%
\definecolor{currentfill}{rgb}{1.000000,1.000000,1.000000}%
\pgfsetfillcolor{currentfill}%
\pgfsetlinewidth{0.000000pt}%
\definecolor{currentstroke}{rgb}{0.000000,0.000000,0.000000}%
\pgfsetstrokecolor{currentstroke}%
\pgfsetstrokeopacity{0.000000}%
\pgfsetdash{}{0pt}%
\pgfpathmoveto{\pgfqpoint{1.075083in}{0.549691in}}%
\pgfpathlineto{\pgfqpoint{6.330793in}{0.549691in}}%
\pgfpathlineto{\pgfqpoint{6.330793in}{4.575328in}}%
\pgfpathlineto{\pgfqpoint{1.075083in}{4.575328in}}%
\pgfpathlineto{\pgfqpoint{1.075083in}{0.549691in}}%
\pgfpathclose%
\pgfusepath{fill}%
\end{pgfscope}%
\begin{pgfscope}%
\pgfpathrectangle{\pgfqpoint{1.075083in}{0.549691in}}{\pgfqpoint{5.255710in}{4.025637in}}%
\pgfusepath{clip}%
\pgfsetrectcap%
\pgfsetroundjoin%
\pgfsetlinewidth{0.803000pt}%
\definecolor{currentstroke}{rgb}{0.690196,0.690196,0.690196}%
\pgfsetstrokecolor{currentstroke}%
\pgfsetdash{}{0pt}%
\pgfpathmoveto{\pgfqpoint{1.075083in}{0.549691in}}%
\pgfpathlineto{\pgfqpoint{1.075083in}{4.575328in}}%
\pgfusepath{stroke}%
\end{pgfscope}%
\begin{pgfscope}%
\pgfsetbuttcap%
\pgfsetroundjoin%
\definecolor{currentfill}{rgb}{0.000000,0.000000,0.000000}%
\pgfsetfillcolor{currentfill}%
\pgfsetlinewidth{0.803000pt}%
\definecolor{currentstroke}{rgb}{0.000000,0.000000,0.000000}%
\pgfsetstrokecolor{currentstroke}%
\pgfsetdash{}{0pt}%
\pgfsys@defobject{currentmarker}{\pgfqpoint{0.000000in}{-0.048611in}}{\pgfqpoint{0.000000in}{0.000000in}}{%
\pgfpathmoveto{\pgfqpoint{0.000000in}{0.000000in}}%
\pgfpathlineto{\pgfqpoint{0.000000in}{-0.048611in}}%
\pgfusepath{stroke,fill}%
}%
\begin{pgfscope}%
\pgfsys@transformshift{1.075083in}{0.549691in}%
\pgfsys@useobject{currentmarker}{}%
\end{pgfscope}%
\end{pgfscope}%
\begin{pgfscope}%
\definecolor{textcolor}{rgb}{0.000000,0.000000,0.000000}%
\pgfsetstrokecolor{textcolor}%
\pgfsetfillcolor{textcolor}%
\pgftext[x=1.075083in,y=0.452469in,,top]{\color{textcolor}\rmfamily\fontsize{10.000000}{12.000000}\selectfont \(\displaystyle {0}\)}%
\end{pgfscope}%
\begin{pgfscope}%
\pgfpathrectangle{\pgfqpoint{1.075083in}{0.549691in}}{\pgfqpoint{5.255710in}{4.025637in}}%
\pgfusepath{clip}%
\pgfsetrectcap%
\pgfsetroundjoin%
\pgfsetlinewidth{0.803000pt}%
\definecolor{currentstroke}{rgb}{0.690196,0.690196,0.690196}%
\pgfsetstrokecolor{currentstroke}%
\pgfsetdash{}{0pt}%
\pgfpathmoveto{\pgfqpoint{2.122246in}{0.549691in}}%
\pgfpathlineto{\pgfqpoint{2.122246in}{4.575328in}}%
\pgfusepath{stroke}%
\end{pgfscope}%
\begin{pgfscope}%
\pgfsetbuttcap%
\pgfsetroundjoin%
\definecolor{currentfill}{rgb}{0.000000,0.000000,0.000000}%
\pgfsetfillcolor{currentfill}%
\pgfsetlinewidth{0.803000pt}%
\definecolor{currentstroke}{rgb}{0.000000,0.000000,0.000000}%
\pgfsetstrokecolor{currentstroke}%
\pgfsetdash{}{0pt}%
\pgfsys@defobject{currentmarker}{\pgfqpoint{0.000000in}{-0.048611in}}{\pgfqpoint{0.000000in}{0.000000in}}{%
\pgfpathmoveto{\pgfqpoint{0.000000in}{0.000000in}}%
\pgfpathlineto{\pgfqpoint{0.000000in}{-0.048611in}}%
\pgfusepath{stroke,fill}%
}%
\begin{pgfscope}%
\pgfsys@transformshift{2.122246in}{0.549691in}%
\pgfsys@useobject{currentmarker}{}%
\end{pgfscope}%
\end{pgfscope}%
\begin{pgfscope}%
\definecolor{textcolor}{rgb}{0.000000,0.000000,0.000000}%
\pgfsetstrokecolor{textcolor}%
\pgfsetfillcolor{textcolor}%
\pgftext[x=2.122246in,y=0.452469in,,top]{\color{textcolor}\rmfamily\fontsize{10.000000}{12.000000}\selectfont \(\displaystyle {100}\)}%
\end{pgfscope}%
\begin{pgfscope}%
\pgfpathrectangle{\pgfqpoint{1.075083in}{0.549691in}}{\pgfqpoint{5.255710in}{4.025637in}}%
\pgfusepath{clip}%
\pgfsetrectcap%
\pgfsetroundjoin%
\pgfsetlinewidth{0.803000pt}%
\definecolor{currentstroke}{rgb}{0.690196,0.690196,0.690196}%
\pgfsetstrokecolor{currentstroke}%
\pgfsetdash{}{0pt}%
\pgfpathmoveto{\pgfqpoint{3.169408in}{0.549691in}}%
\pgfpathlineto{\pgfqpoint{3.169408in}{4.575328in}}%
\pgfusepath{stroke}%
\end{pgfscope}%
\begin{pgfscope}%
\pgfsetbuttcap%
\pgfsetroundjoin%
\definecolor{currentfill}{rgb}{0.000000,0.000000,0.000000}%
\pgfsetfillcolor{currentfill}%
\pgfsetlinewidth{0.803000pt}%
\definecolor{currentstroke}{rgb}{0.000000,0.000000,0.000000}%
\pgfsetstrokecolor{currentstroke}%
\pgfsetdash{}{0pt}%
\pgfsys@defobject{currentmarker}{\pgfqpoint{0.000000in}{-0.048611in}}{\pgfqpoint{0.000000in}{0.000000in}}{%
\pgfpathmoveto{\pgfqpoint{0.000000in}{0.000000in}}%
\pgfpathlineto{\pgfqpoint{0.000000in}{-0.048611in}}%
\pgfusepath{stroke,fill}%
}%
\begin{pgfscope}%
\pgfsys@transformshift{3.169408in}{0.549691in}%
\pgfsys@useobject{currentmarker}{}%
\end{pgfscope}%
\end{pgfscope}%
\begin{pgfscope}%
\definecolor{textcolor}{rgb}{0.000000,0.000000,0.000000}%
\pgfsetstrokecolor{textcolor}%
\pgfsetfillcolor{textcolor}%
\pgftext[x=3.169408in,y=0.452469in,,top]{\color{textcolor}\rmfamily\fontsize{10.000000}{12.000000}\selectfont \(\displaystyle {200}\)}%
\end{pgfscope}%
\begin{pgfscope}%
\pgfpathrectangle{\pgfqpoint{1.075083in}{0.549691in}}{\pgfqpoint{5.255710in}{4.025637in}}%
\pgfusepath{clip}%
\pgfsetrectcap%
\pgfsetroundjoin%
\pgfsetlinewidth{0.803000pt}%
\definecolor{currentstroke}{rgb}{0.690196,0.690196,0.690196}%
\pgfsetstrokecolor{currentstroke}%
\pgfsetdash{}{0pt}%
\pgfpathmoveto{\pgfqpoint{4.216571in}{0.549691in}}%
\pgfpathlineto{\pgfqpoint{4.216571in}{4.575328in}}%
\pgfusepath{stroke}%
\end{pgfscope}%
\begin{pgfscope}%
\pgfsetbuttcap%
\pgfsetroundjoin%
\definecolor{currentfill}{rgb}{0.000000,0.000000,0.000000}%
\pgfsetfillcolor{currentfill}%
\pgfsetlinewidth{0.803000pt}%
\definecolor{currentstroke}{rgb}{0.000000,0.000000,0.000000}%
\pgfsetstrokecolor{currentstroke}%
\pgfsetdash{}{0pt}%
\pgfsys@defobject{currentmarker}{\pgfqpoint{0.000000in}{-0.048611in}}{\pgfqpoint{0.000000in}{0.000000in}}{%
\pgfpathmoveto{\pgfqpoint{0.000000in}{0.000000in}}%
\pgfpathlineto{\pgfqpoint{0.000000in}{-0.048611in}}%
\pgfusepath{stroke,fill}%
}%
\begin{pgfscope}%
\pgfsys@transformshift{4.216571in}{0.549691in}%
\pgfsys@useobject{currentmarker}{}%
\end{pgfscope}%
\end{pgfscope}%
\begin{pgfscope}%
\definecolor{textcolor}{rgb}{0.000000,0.000000,0.000000}%
\pgfsetstrokecolor{textcolor}%
\pgfsetfillcolor{textcolor}%
\pgftext[x=4.216571in,y=0.452469in,,top]{\color{textcolor}\rmfamily\fontsize{10.000000}{12.000000}\selectfont \(\displaystyle {300}\)}%
\end{pgfscope}%
\begin{pgfscope}%
\pgfpathrectangle{\pgfqpoint{1.075083in}{0.549691in}}{\pgfqpoint{5.255710in}{4.025637in}}%
\pgfusepath{clip}%
\pgfsetrectcap%
\pgfsetroundjoin%
\pgfsetlinewidth{0.803000pt}%
\definecolor{currentstroke}{rgb}{0.690196,0.690196,0.690196}%
\pgfsetstrokecolor{currentstroke}%
\pgfsetdash{}{0pt}%
\pgfpathmoveto{\pgfqpoint{5.263734in}{0.549691in}}%
\pgfpathlineto{\pgfqpoint{5.263734in}{4.575328in}}%
\pgfusepath{stroke}%
\end{pgfscope}%
\begin{pgfscope}%
\pgfsetbuttcap%
\pgfsetroundjoin%
\definecolor{currentfill}{rgb}{0.000000,0.000000,0.000000}%
\pgfsetfillcolor{currentfill}%
\pgfsetlinewidth{0.803000pt}%
\definecolor{currentstroke}{rgb}{0.000000,0.000000,0.000000}%
\pgfsetstrokecolor{currentstroke}%
\pgfsetdash{}{0pt}%
\pgfsys@defobject{currentmarker}{\pgfqpoint{0.000000in}{-0.048611in}}{\pgfqpoint{0.000000in}{0.000000in}}{%
\pgfpathmoveto{\pgfqpoint{0.000000in}{0.000000in}}%
\pgfpathlineto{\pgfqpoint{0.000000in}{-0.048611in}}%
\pgfusepath{stroke,fill}%
}%
\begin{pgfscope}%
\pgfsys@transformshift{5.263734in}{0.549691in}%
\pgfsys@useobject{currentmarker}{}%
\end{pgfscope}%
\end{pgfscope}%
\begin{pgfscope}%
\definecolor{textcolor}{rgb}{0.000000,0.000000,0.000000}%
\pgfsetstrokecolor{textcolor}%
\pgfsetfillcolor{textcolor}%
\pgftext[x=5.263734in,y=0.452469in,,top]{\color{textcolor}\rmfamily\fontsize{10.000000}{12.000000}\selectfont \(\displaystyle {400}\)}%
\end{pgfscope}%
\begin{pgfscope}%
\pgfpathrectangle{\pgfqpoint{1.075083in}{0.549691in}}{\pgfqpoint{5.255710in}{4.025637in}}%
\pgfusepath{clip}%
\pgfsetrectcap%
\pgfsetroundjoin%
\pgfsetlinewidth{0.803000pt}%
\definecolor{currentstroke}{rgb}{0.690196,0.690196,0.690196}%
\pgfsetstrokecolor{currentstroke}%
\pgfsetdash{}{0pt}%
\pgfpathmoveto{\pgfqpoint{6.310896in}{0.549691in}}%
\pgfpathlineto{\pgfqpoint{6.310896in}{4.575328in}}%
\pgfusepath{stroke}%
\end{pgfscope}%
\begin{pgfscope}%
\pgfsetbuttcap%
\pgfsetroundjoin%
\definecolor{currentfill}{rgb}{0.000000,0.000000,0.000000}%
\pgfsetfillcolor{currentfill}%
\pgfsetlinewidth{0.803000pt}%
\definecolor{currentstroke}{rgb}{0.000000,0.000000,0.000000}%
\pgfsetstrokecolor{currentstroke}%
\pgfsetdash{}{0pt}%
\pgfsys@defobject{currentmarker}{\pgfqpoint{0.000000in}{-0.048611in}}{\pgfqpoint{0.000000in}{0.000000in}}{%
\pgfpathmoveto{\pgfqpoint{0.000000in}{0.000000in}}%
\pgfpathlineto{\pgfqpoint{0.000000in}{-0.048611in}}%
\pgfusepath{stroke,fill}%
}%
\begin{pgfscope}%
\pgfsys@transformshift{6.310896in}{0.549691in}%
\pgfsys@useobject{currentmarker}{}%
\end{pgfscope}%
\end{pgfscope}%
\begin{pgfscope}%
\definecolor{textcolor}{rgb}{0.000000,0.000000,0.000000}%
\pgfsetstrokecolor{textcolor}%
\pgfsetfillcolor{textcolor}%
\pgftext[x=6.310896in,y=0.452469in,,top]{\color{textcolor}\rmfamily\fontsize{10.000000}{12.000000}\selectfont \(\displaystyle {500}\)}%
\end{pgfscope}%
\begin{pgfscope}%
\definecolor{textcolor}{rgb}{0.000000,0.000000,0.000000}%
\pgfsetstrokecolor{textcolor}%
\pgfsetfillcolor{textcolor}%
\pgftext[x=3.702938in,y=0.273457in,,top]{\color{textcolor}\rmfamily\fontsize{10.000000}{12.000000}\selectfont \(\displaystyle k\)}%
\end{pgfscope}%
\begin{pgfscope}%
\pgfpathrectangle{\pgfqpoint{1.075083in}{0.549691in}}{\pgfqpoint{5.255710in}{4.025637in}}%
\pgfusepath{clip}%
\pgfsetrectcap%
\pgfsetroundjoin%
\pgfsetlinewidth{0.803000pt}%
\definecolor{currentstroke}{rgb}{0.690196,0.690196,0.690196}%
\pgfsetstrokecolor{currentstroke}%
\pgfsetdash{}{0pt}%
\pgfpathmoveto{\pgfqpoint{1.075083in}{0.939076in}}%
\pgfpathlineto{\pgfqpoint{6.330793in}{0.939076in}}%
\pgfusepath{stroke}%
\end{pgfscope}%
\begin{pgfscope}%
\pgfsetbuttcap%
\pgfsetroundjoin%
\definecolor{currentfill}{rgb}{0.000000,0.000000,0.000000}%
\pgfsetfillcolor{currentfill}%
\pgfsetlinewidth{0.803000pt}%
\definecolor{currentstroke}{rgb}{0.000000,0.000000,0.000000}%
\pgfsetstrokecolor{currentstroke}%
\pgfsetdash{}{0pt}%
\pgfsys@defobject{currentmarker}{\pgfqpoint{-0.048611in}{0.000000in}}{\pgfqpoint{-0.000000in}{0.000000in}}{%
\pgfpathmoveto{\pgfqpoint{-0.000000in}{0.000000in}}%
\pgfpathlineto{\pgfqpoint{-0.048611in}{0.000000in}}%
\pgfusepath{stroke,fill}%
}%
\begin{pgfscope}%
\pgfsys@transformshift{1.075083in}{0.939076in}%
\pgfsys@useobject{currentmarker}{}%
\end{pgfscope}%
\end{pgfscope}%
\begin{pgfscope}%
\definecolor{textcolor}{rgb}{0.000000,0.000000,0.000000}%
\pgfsetstrokecolor{textcolor}%
\pgfsetfillcolor{textcolor}%
\pgftext[x=0.634495in, y=0.890851in, left, base]{\color{textcolor}\rmfamily\fontsize{10.000000}{12.000000}\selectfont \(\displaystyle {10^{-16}}\)}%
\end{pgfscope}%
\begin{pgfscope}%
\pgfpathrectangle{\pgfqpoint{1.075083in}{0.549691in}}{\pgfqpoint{5.255710in}{4.025637in}}%
\pgfusepath{clip}%
\pgfsetrectcap%
\pgfsetroundjoin%
\pgfsetlinewidth{0.803000pt}%
\definecolor{currentstroke}{rgb}{0.690196,0.690196,0.690196}%
\pgfsetstrokecolor{currentstroke}%
\pgfsetdash{}{0pt}%
\pgfpathmoveto{\pgfqpoint{1.075083in}{1.602181in}}%
\pgfpathlineto{\pgfqpoint{6.330793in}{1.602181in}}%
\pgfusepath{stroke}%
\end{pgfscope}%
\begin{pgfscope}%
\pgfsetbuttcap%
\pgfsetroundjoin%
\definecolor{currentfill}{rgb}{0.000000,0.000000,0.000000}%
\pgfsetfillcolor{currentfill}%
\pgfsetlinewidth{0.803000pt}%
\definecolor{currentstroke}{rgb}{0.000000,0.000000,0.000000}%
\pgfsetstrokecolor{currentstroke}%
\pgfsetdash{}{0pt}%
\pgfsys@defobject{currentmarker}{\pgfqpoint{-0.048611in}{0.000000in}}{\pgfqpoint{-0.000000in}{0.000000in}}{%
\pgfpathmoveto{\pgfqpoint{-0.000000in}{0.000000in}}%
\pgfpathlineto{\pgfqpoint{-0.048611in}{0.000000in}}%
\pgfusepath{stroke,fill}%
}%
\begin{pgfscope}%
\pgfsys@transformshift{1.075083in}{1.602181in}%
\pgfsys@useobject{currentmarker}{}%
\end{pgfscope}%
\end{pgfscope}%
\begin{pgfscope}%
\definecolor{textcolor}{rgb}{0.000000,0.000000,0.000000}%
\pgfsetstrokecolor{textcolor}%
\pgfsetfillcolor{textcolor}%
\pgftext[x=0.634495in, y=1.553956in, left, base]{\color{textcolor}\rmfamily\fontsize{10.000000}{12.000000}\selectfont \(\displaystyle {10^{-14}}\)}%
\end{pgfscope}%
\begin{pgfscope}%
\pgfpathrectangle{\pgfqpoint{1.075083in}{0.549691in}}{\pgfqpoint{5.255710in}{4.025637in}}%
\pgfusepath{clip}%
\pgfsetrectcap%
\pgfsetroundjoin%
\pgfsetlinewidth{0.803000pt}%
\definecolor{currentstroke}{rgb}{0.690196,0.690196,0.690196}%
\pgfsetstrokecolor{currentstroke}%
\pgfsetdash{}{0pt}%
\pgfpathmoveto{\pgfqpoint{1.075083in}{2.265287in}}%
\pgfpathlineto{\pgfqpoint{6.330793in}{2.265287in}}%
\pgfusepath{stroke}%
\end{pgfscope}%
\begin{pgfscope}%
\pgfsetbuttcap%
\pgfsetroundjoin%
\definecolor{currentfill}{rgb}{0.000000,0.000000,0.000000}%
\pgfsetfillcolor{currentfill}%
\pgfsetlinewidth{0.803000pt}%
\definecolor{currentstroke}{rgb}{0.000000,0.000000,0.000000}%
\pgfsetstrokecolor{currentstroke}%
\pgfsetdash{}{0pt}%
\pgfsys@defobject{currentmarker}{\pgfqpoint{-0.048611in}{0.000000in}}{\pgfqpoint{-0.000000in}{0.000000in}}{%
\pgfpathmoveto{\pgfqpoint{-0.000000in}{0.000000in}}%
\pgfpathlineto{\pgfqpoint{-0.048611in}{0.000000in}}%
\pgfusepath{stroke,fill}%
}%
\begin{pgfscope}%
\pgfsys@transformshift{1.075083in}{2.265287in}%
\pgfsys@useobject{currentmarker}{}%
\end{pgfscope}%
\end{pgfscope}%
\begin{pgfscope}%
\definecolor{textcolor}{rgb}{0.000000,0.000000,0.000000}%
\pgfsetstrokecolor{textcolor}%
\pgfsetfillcolor{textcolor}%
\pgftext[x=0.634495in, y=2.217061in, left, base]{\color{textcolor}\rmfamily\fontsize{10.000000}{12.000000}\selectfont \(\displaystyle {10^{-12}}\)}%
\end{pgfscope}%
\begin{pgfscope}%
\pgfpathrectangle{\pgfqpoint{1.075083in}{0.549691in}}{\pgfqpoint{5.255710in}{4.025637in}}%
\pgfusepath{clip}%
\pgfsetrectcap%
\pgfsetroundjoin%
\pgfsetlinewidth{0.803000pt}%
\definecolor{currentstroke}{rgb}{0.690196,0.690196,0.690196}%
\pgfsetstrokecolor{currentstroke}%
\pgfsetdash{}{0pt}%
\pgfpathmoveto{\pgfqpoint{1.075083in}{2.928392in}}%
\pgfpathlineto{\pgfqpoint{6.330793in}{2.928392in}}%
\pgfusepath{stroke}%
\end{pgfscope}%
\begin{pgfscope}%
\pgfsetbuttcap%
\pgfsetroundjoin%
\definecolor{currentfill}{rgb}{0.000000,0.000000,0.000000}%
\pgfsetfillcolor{currentfill}%
\pgfsetlinewidth{0.803000pt}%
\definecolor{currentstroke}{rgb}{0.000000,0.000000,0.000000}%
\pgfsetstrokecolor{currentstroke}%
\pgfsetdash{}{0pt}%
\pgfsys@defobject{currentmarker}{\pgfqpoint{-0.048611in}{0.000000in}}{\pgfqpoint{-0.000000in}{0.000000in}}{%
\pgfpathmoveto{\pgfqpoint{-0.000000in}{0.000000in}}%
\pgfpathlineto{\pgfqpoint{-0.048611in}{0.000000in}}%
\pgfusepath{stroke,fill}%
}%
\begin{pgfscope}%
\pgfsys@transformshift{1.075083in}{2.928392in}%
\pgfsys@useobject{currentmarker}{}%
\end{pgfscope}%
\end{pgfscope}%
\begin{pgfscope}%
\definecolor{textcolor}{rgb}{0.000000,0.000000,0.000000}%
\pgfsetstrokecolor{textcolor}%
\pgfsetfillcolor{textcolor}%
\pgftext[x=0.634495in, y=2.880167in, left, base]{\color{textcolor}\rmfamily\fontsize{10.000000}{12.000000}\selectfont \(\displaystyle {10^{-10}}\)}%
\end{pgfscope}%
\begin{pgfscope}%
\pgfpathrectangle{\pgfqpoint{1.075083in}{0.549691in}}{\pgfqpoint{5.255710in}{4.025637in}}%
\pgfusepath{clip}%
\pgfsetrectcap%
\pgfsetroundjoin%
\pgfsetlinewidth{0.803000pt}%
\definecolor{currentstroke}{rgb}{0.690196,0.690196,0.690196}%
\pgfsetstrokecolor{currentstroke}%
\pgfsetdash{}{0pt}%
\pgfpathmoveto{\pgfqpoint{1.075083in}{3.591497in}}%
\pgfpathlineto{\pgfqpoint{6.330793in}{3.591497in}}%
\pgfusepath{stroke}%
\end{pgfscope}%
\begin{pgfscope}%
\pgfsetbuttcap%
\pgfsetroundjoin%
\definecolor{currentfill}{rgb}{0.000000,0.000000,0.000000}%
\pgfsetfillcolor{currentfill}%
\pgfsetlinewidth{0.803000pt}%
\definecolor{currentstroke}{rgb}{0.000000,0.000000,0.000000}%
\pgfsetstrokecolor{currentstroke}%
\pgfsetdash{}{0pt}%
\pgfsys@defobject{currentmarker}{\pgfqpoint{-0.048611in}{0.000000in}}{\pgfqpoint{-0.000000in}{0.000000in}}{%
\pgfpathmoveto{\pgfqpoint{-0.000000in}{0.000000in}}%
\pgfpathlineto{\pgfqpoint{-0.048611in}{0.000000in}}%
\pgfusepath{stroke,fill}%
}%
\begin{pgfscope}%
\pgfsys@transformshift{1.075083in}{3.591497in}%
\pgfsys@useobject{currentmarker}{}%
\end{pgfscope}%
\end{pgfscope}%
\begin{pgfscope}%
\definecolor{textcolor}{rgb}{0.000000,0.000000,0.000000}%
\pgfsetstrokecolor{textcolor}%
\pgfsetfillcolor{textcolor}%
\pgftext[x=0.689858in, y=3.543272in, left, base]{\color{textcolor}\rmfamily\fontsize{10.000000}{12.000000}\selectfont \(\displaystyle {10^{-8}}\)}%
\end{pgfscope}%
\begin{pgfscope}%
\pgfpathrectangle{\pgfqpoint{1.075083in}{0.549691in}}{\pgfqpoint{5.255710in}{4.025637in}}%
\pgfusepath{clip}%
\pgfsetrectcap%
\pgfsetroundjoin%
\pgfsetlinewidth{0.803000pt}%
\definecolor{currentstroke}{rgb}{0.690196,0.690196,0.690196}%
\pgfsetstrokecolor{currentstroke}%
\pgfsetdash{}{0pt}%
\pgfpathmoveto{\pgfqpoint{1.075083in}{4.254602in}}%
\pgfpathlineto{\pgfqpoint{6.330793in}{4.254602in}}%
\pgfusepath{stroke}%
\end{pgfscope}%
\begin{pgfscope}%
\pgfsetbuttcap%
\pgfsetroundjoin%
\definecolor{currentfill}{rgb}{0.000000,0.000000,0.000000}%
\pgfsetfillcolor{currentfill}%
\pgfsetlinewidth{0.803000pt}%
\definecolor{currentstroke}{rgb}{0.000000,0.000000,0.000000}%
\pgfsetstrokecolor{currentstroke}%
\pgfsetdash{}{0pt}%
\pgfsys@defobject{currentmarker}{\pgfqpoint{-0.048611in}{0.000000in}}{\pgfqpoint{-0.000000in}{0.000000in}}{%
\pgfpathmoveto{\pgfqpoint{-0.000000in}{0.000000in}}%
\pgfpathlineto{\pgfqpoint{-0.048611in}{0.000000in}}%
\pgfusepath{stroke,fill}%
}%
\begin{pgfscope}%
\pgfsys@transformshift{1.075083in}{4.254602in}%
\pgfsys@useobject{currentmarker}{}%
\end{pgfscope}%
\end{pgfscope}%
\begin{pgfscope}%
\definecolor{textcolor}{rgb}{0.000000,0.000000,0.000000}%
\pgfsetstrokecolor{textcolor}%
\pgfsetfillcolor{textcolor}%
\pgftext[x=0.689858in, y=4.206377in, left, base]{\color{textcolor}\rmfamily\fontsize{10.000000}{12.000000}\selectfont \(\displaystyle {10^{-6}}\)}%
\end{pgfscope}%
\begin{pgfscope}%
\definecolor{textcolor}{rgb}{0.000000,0.000000,0.000000}%
\pgfsetstrokecolor{textcolor}%
\pgfsetfillcolor{textcolor}%
\pgftext[x=0.384495in,y=2.562510in,,bottom]{\color{textcolor}\rmfamily\fontsize{10.000000}{12.000000}\selectfont \(\displaystyle (\mathbf{v}_1, \mathbf{v}_k)\)}%
\end{pgfscope}%
\begin{pgfscope}%
\pgfpathrectangle{\pgfqpoint{1.075083in}{0.549691in}}{\pgfqpoint{5.255710in}{4.025637in}}%
\pgfusepath{clip}%
\pgfsetrectcap%
\pgfsetroundjoin%
\pgfsetlinewidth{1.505625pt}%
\definecolor{currentstroke}{rgb}{0.121569,0.466667,0.705882}%
\pgfsetstrokecolor{currentstroke}%
\pgfsetdash{}{0pt}%
\pgfpathmoveto{\pgfqpoint{1.075083in}{0.732675in}}%
\pgfpathlineto{\pgfqpoint{1.075089in}{0.539691in}}%
\pgfpathmoveto{\pgfqpoint{1.190253in}{0.539691in}}%
\pgfpathlineto{\pgfqpoint{1.190271in}{1.083980in}}%
\pgfpathlineto{\pgfqpoint{1.190288in}{0.539691in}}%
\pgfpathmoveto{\pgfqpoint{1.211189in}{0.539691in}}%
\pgfpathlineto{\pgfqpoint{1.211214in}{1.324125in}}%
\pgfpathlineto{\pgfqpoint{1.221686in}{1.356702in}}%
\pgfpathlineto{\pgfqpoint{1.232157in}{1.458849in}}%
\pgfpathlineto{\pgfqpoint{1.242629in}{1.542182in}}%
\pgfpathlineto{\pgfqpoint{1.253101in}{1.492138in}}%
\pgfpathlineto{\pgfqpoint{1.263572in}{1.488035in}}%
\pgfpathlineto{\pgfqpoint{1.274044in}{1.556247in}}%
\pgfpathlineto{\pgfqpoint{1.284515in}{1.534957in}}%
\pgfpathlineto{\pgfqpoint{1.294987in}{1.552447in}}%
\pgfpathlineto{\pgfqpoint{1.305459in}{1.551957in}}%
\pgfpathlineto{\pgfqpoint{1.315930in}{1.608982in}}%
\pgfpathlineto{\pgfqpoint{1.326402in}{1.566990in}}%
\pgfpathlineto{\pgfqpoint{1.336874in}{1.594151in}}%
\pgfpathlineto{\pgfqpoint{1.347345in}{1.628187in}}%
\pgfpathlineto{\pgfqpoint{1.357817in}{1.611249in}}%
\pgfpathlineto{\pgfqpoint{1.368288in}{1.692293in}}%
\pgfpathlineto{\pgfqpoint{1.378760in}{1.649491in}}%
\pgfpathlineto{\pgfqpoint{1.389232in}{1.653395in}}%
\pgfpathlineto{\pgfqpoint{1.399703in}{1.728004in}}%
\pgfpathlineto{\pgfqpoint{1.410175in}{1.710032in}}%
\pgfpathlineto{\pgfqpoint{1.420647in}{1.726486in}}%
\pgfpathlineto{\pgfqpoint{1.431118in}{1.844713in}}%
\pgfpathlineto{\pgfqpoint{1.441590in}{1.810263in}}%
\pgfpathlineto{\pgfqpoint{1.452061in}{1.814295in}}%
\pgfpathlineto{\pgfqpoint{1.462533in}{1.779827in}}%
\pgfpathlineto{\pgfqpoint{1.473005in}{1.758414in}}%
\pgfpathlineto{\pgfqpoint{1.483476in}{1.602082in}}%
\pgfpathlineto{\pgfqpoint{1.493948in}{1.639842in}}%
\pgfpathlineto{\pgfqpoint{1.504420in}{1.846159in}}%
\pgfpathlineto{\pgfqpoint{1.514891in}{2.024132in}}%
\pgfpathlineto{\pgfqpoint{1.525363in}{2.031153in}}%
\pgfpathlineto{\pgfqpoint{1.535834in}{1.974980in}}%
\pgfpathlineto{\pgfqpoint{1.546306in}{1.953202in}}%
\pgfpathlineto{\pgfqpoint{1.556778in}{1.908567in}}%
\pgfpathlineto{\pgfqpoint{1.567249in}{1.930951in}}%
\pgfpathlineto{\pgfqpoint{1.577721in}{1.958234in}}%
\pgfpathlineto{\pgfqpoint{1.588193in}{1.915929in}}%
\pgfpathlineto{\pgfqpoint{1.598664in}{1.945324in}}%
\pgfpathlineto{\pgfqpoint{1.609136in}{1.884542in}}%
\pgfpathlineto{\pgfqpoint{1.619607in}{1.897582in}}%
\pgfpathlineto{\pgfqpoint{1.630079in}{1.799986in}}%
\pgfpathlineto{\pgfqpoint{1.640551in}{1.929111in}}%
\pgfpathlineto{\pgfqpoint{1.651022in}{2.013248in}}%
\pgfpathlineto{\pgfqpoint{1.661494in}{1.998254in}}%
\pgfpathlineto{\pgfqpoint{1.671966in}{2.030197in}}%
\pgfpathlineto{\pgfqpoint{1.692909in}{2.069639in}}%
\pgfpathlineto{\pgfqpoint{1.703380in}{2.005247in}}%
\pgfpathlineto{\pgfqpoint{1.713852in}{2.005410in}}%
\pgfpathlineto{\pgfqpoint{1.724324in}{1.944327in}}%
\pgfpathlineto{\pgfqpoint{1.734795in}{1.975301in}}%
\pgfpathlineto{\pgfqpoint{1.745267in}{1.984150in}}%
\pgfpathlineto{\pgfqpoint{1.755739in}{1.919644in}}%
\pgfpathlineto{\pgfqpoint{1.766210in}{2.048853in}}%
\pgfpathlineto{\pgfqpoint{1.776682in}{2.001594in}}%
\pgfpathlineto{\pgfqpoint{1.787154in}{2.063233in}}%
\pgfpathlineto{\pgfqpoint{1.797625in}{2.022011in}}%
\pgfpathlineto{\pgfqpoint{1.808097in}{2.104659in}}%
\pgfpathlineto{\pgfqpoint{1.818568in}{2.154377in}}%
\pgfpathlineto{\pgfqpoint{1.829040in}{2.092192in}}%
\pgfpathlineto{\pgfqpoint{1.839512in}{2.044309in}}%
\pgfpathlineto{\pgfqpoint{1.849983in}{1.982591in}}%
\pgfpathlineto{\pgfqpoint{1.860455in}{2.000597in}}%
\pgfpathlineto{\pgfqpoint{1.870927in}{1.971714in}}%
\pgfpathlineto{\pgfqpoint{1.881398in}{2.094038in}}%
\pgfpathlineto{\pgfqpoint{1.891870in}{2.193578in}}%
\pgfpathlineto{\pgfqpoint{1.902341in}{2.118169in}}%
\pgfpathlineto{\pgfqpoint{1.912813in}{2.203798in}}%
\pgfpathlineto{\pgfqpoint{1.923285in}{2.129626in}}%
\pgfpathlineto{\pgfqpoint{1.933756in}{2.140659in}}%
\pgfpathlineto{\pgfqpoint{1.944228in}{2.103447in}}%
\pgfpathlineto{\pgfqpoint{1.954700in}{2.084075in}}%
\pgfpathlineto{\pgfqpoint{1.965171in}{2.082543in}}%
\pgfpathlineto{\pgfqpoint{1.975643in}{2.029602in}}%
\pgfpathlineto{\pgfqpoint{1.986114in}{2.093269in}}%
\pgfpathlineto{\pgfqpoint{1.996586in}{2.050380in}}%
\pgfpathlineto{\pgfqpoint{2.007058in}{2.088330in}}%
\pgfpathlineto{\pgfqpoint{2.017529in}{2.201925in}}%
\pgfpathlineto{\pgfqpoint{2.028001in}{2.241300in}}%
\pgfpathlineto{\pgfqpoint{2.038473in}{2.299834in}}%
\pgfpathlineto{\pgfqpoint{2.048944in}{2.234182in}}%
\pgfpathlineto{\pgfqpoint{2.059416in}{2.202401in}}%
\pgfpathlineto{\pgfqpoint{2.069887in}{2.112969in}}%
\pgfpathlineto{\pgfqpoint{2.080359in}{2.132515in}}%
\pgfpathlineto{\pgfqpoint{2.090831in}{2.242851in}}%
\pgfpathlineto{\pgfqpoint{2.101302in}{2.294869in}}%
\pgfpathlineto{\pgfqpoint{2.111774in}{2.159403in}}%
\pgfpathlineto{\pgfqpoint{2.122246in}{2.294912in}}%
\pgfpathlineto{\pgfqpoint{2.132717in}{2.300766in}}%
\pgfpathlineto{\pgfqpoint{2.143189in}{2.250320in}}%
\pgfpathlineto{\pgfqpoint{2.153660in}{2.184851in}}%
\pgfpathlineto{\pgfqpoint{2.164132in}{2.207572in}}%
\pgfpathlineto{\pgfqpoint{2.174604in}{2.194541in}}%
\pgfpathlineto{\pgfqpoint{2.185075in}{2.173529in}}%
\pgfpathlineto{\pgfqpoint{2.195547in}{2.202750in}}%
\pgfpathlineto{\pgfqpoint{2.206019in}{2.274652in}}%
\pgfpathlineto{\pgfqpoint{2.216490in}{2.362237in}}%
\pgfpathlineto{\pgfqpoint{2.226962in}{2.359780in}}%
\pgfpathlineto{\pgfqpoint{2.237433in}{2.332800in}}%
\pgfpathlineto{\pgfqpoint{2.247905in}{2.321815in}}%
\pgfpathlineto{\pgfqpoint{2.258377in}{2.186101in}}%
\pgfpathlineto{\pgfqpoint{2.268848in}{2.276899in}}%
\pgfpathlineto{\pgfqpoint{2.279320in}{2.310435in}}%
\pgfpathlineto{\pgfqpoint{2.289792in}{2.246901in}}%
\pgfpathlineto{\pgfqpoint{2.300263in}{2.304746in}}%
\pgfpathlineto{\pgfqpoint{2.310735in}{2.335699in}}%
\pgfpathlineto{\pgfqpoint{2.321206in}{2.386748in}}%
\pgfpathlineto{\pgfqpoint{2.331678in}{2.317435in}}%
\pgfpathlineto{\pgfqpoint{2.342150in}{2.323067in}}%
\pgfpathlineto{\pgfqpoint{2.352621in}{2.227344in}}%
\pgfpathlineto{\pgfqpoint{2.363093in}{2.033837in}}%
\pgfpathlineto{\pgfqpoint{2.373565in}{2.291789in}}%
\pgfpathlineto{\pgfqpoint{2.384036in}{2.315346in}}%
\pgfpathlineto{\pgfqpoint{2.394508in}{2.330527in}}%
\pgfpathlineto{\pgfqpoint{2.404980in}{2.378596in}}%
\pgfpathlineto{\pgfqpoint{2.415451in}{2.409791in}}%
\pgfpathlineto{\pgfqpoint{2.425923in}{2.348139in}}%
\pgfpathlineto{\pgfqpoint{2.436394in}{2.357357in}}%
\pgfpathlineto{\pgfqpoint{2.446866in}{2.325597in}}%
\pgfpathlineto{\pgfqpoint{2.457338in}{2.310943in}}%
\pgfpathlineto{\pgfqpoint{2.467809in}{2.215638in}}%
\pgfpathlineto{\pgfqpoint{2.478281in}{2.246691in}}%
\pgfpathlineto{\pgfqpoint{2.488753in}{2.282373in}}%
\pgfpathlineto{\pgfqpoint{2.499224in}{2.361062in}}%
\pgfpathlineto{\pgfqpoint{2.509696in}{2.457406in}}%
\pgfpathlineto{\pgfqpoint{2.520167in}{2.481848in}}%
\pgfpathlineto{\pgfqpoint{2.530639in}{2.316512in}}%
\pgfpathlineto{\pgfqpoint{2.541111in}{2.330508in}}%
\pgfpathlineto{\pgfqpoint{2.551582in}{2.370890in}}%
\pgfpathlineto{\pgfqpoint{2.562054in}{2.311010in}}%
\pgfpathlineto{\pgfqpoint{2.572526in}{2.327438in}}%
\pgfpathlineto{\pgfqpoint{2.582997in}{2.430756in}}%
\pgfpathlineto{\pgfqpoint{2.593469in}{2.422577in}}%
\pgfpathlineto{\pgfqpoint{2.603940in}{2.468787in}}%
\pgfpathlineto{\pgfqpoint{2.614412in}{2.481387in}}%
\pgfpathlineto{\pgfqpoint{2.624884in}{2.520925in}}%
\pgfpathlineto{\pgfqpoint{2.635355in}{2.460847in}}%
\pgfpathlineto{\pgfqpoint{2.645827in}{2.416240in}}%
\pgfpathlineto{\pgfqpoint{2.656299in}{2.294734in}}%
\pgfpathlineto{\pgfqpoint{2.666770in}{2.373306in}}%
\pgfpathlineto{\pgfqpoint{2.677242in}{2.357220in}}%
\pgfpathlineto{\pgfqpoint{2.687713in}{2.511581in}}%
\pgfpathlineto{\pgfqpoint{2.698185in}{2.542733in}}%
\pgfpathlineto{\pgfqpoint{2.708657in}{2.476240in}}%
\pgfpathlineto{\pgfqpoint{2.719128in}{2.461189in}}%
\pgfpathlineto{\pgfqpoint{2.729600in}{2.441082in}}%
\pgfpathlineto{\pgfqpoint{2.740072in}{2.458160in}}%
\pgfpathlineto{\pgfqpoint{2.750543in}{2.426546in}}%
\pgfpathlineto{\pgfqpoint{2.761015in}{2.420678in}}%
\pgfpathlineto{\pgfqpoint{2.781958in}{2.576697in}}%
\pgfpathlineto{\pgfqpoint{2.792430in}{2.535044in}}%
\pgfpathlineto{\pgfqpoint{2.802901in}{2.549128in}}%
\pgfpathlineto{\pgfqpoint{2.813373in}{2.521884in}}%
\pgfpathlineto{\pgfqpoint{2.823845in}{2.433506in}}%
\pgfpathlineto{\pgfqpoint{2.834316in}{2.386754in}}%
\pgfpathlineto{\pgfqpoint{2.844788in}{2.396182in}}%
\pgfpathlineto{\pgfqpoint{2.855259in}{2.514948in}}%
\pgfpathlineto{\pgfqpoint{2.865731in}{2.558926in}}%
\pgfpathlineto{\pgfqpoint{2.876203in}{2.554632in}}%
\pgfpathlineto{\pgfqpoint{2.886674in}{2.599259in}}%
\pgfpathlineto{\pgfqpoint{2.897146in}{2.590219in}}%
\pgfpathlineto{\pgfqpoint{2.907618in}{2.597392in}}%
\pgfpathlineto{\pgfqpoint{2.918089in}{2.539491in}}%
\pgfpathlineto{\pgfqpoint{2.928561in}{2.451412in}}%
\pgfpathlineto{\pgfqpoint{2.939033in}{2.408058in}}%
\pgfpathlineto{\pgfqpoint{2.949504in}{2.512845in}}%
\pgfpathlineto{\pgfqpoint{2.959976in}{2.581542in}}%
\pgfpathlineto{\pgfqpoint{2.970447in}{2.633906in}}%
\pgfpathlineto{\pgfqpoint{2.980919in}{2.582122in}}%
\pgfpathlineto{\pgfqpoint{2.991391in}{2.625803in}}%
\pgfpathlineto{\pgfqpoint{3.001862in}{2.578851in}}%
\pgfpathlineto{\pgfqpoint{3.012334in}{2.539912in}}%
\pgfpathlineto{\pgfqpoint{3.022806in}{2.567333in}}%
\pgfpathlineto{\pgfqpoint{3.033277in}{2.538014in}}%
\pgfpathlineto{\pgfqpoint{3.043749in}{2.544597in}}%
\pgfpathlineto{\pgfqpoint{3.054220in}{2.574651in}}%
\pgfpathlineto{\pgfqpoint{3.064692in}{2.620590in}}%
\pgfpathlineto{\pgfqpoint{3.075164in}{2.630648in}}%
\pgfpathlineto{\pgfqpoint{3.085635in}{2.670764in}}%
\pgfpathlineto{\pgfqpoint{3.096107in}{2.687434in}}%
\pgfpathlineto{\pgfqpoint{3.106579in}{2.463086in}}%
\pgfpathlineto{\pgfqpoint{3.117050in}{2.499367in}}%
\pgfpathlineto{\pgfqpoint{3.137993in}{2.742058in}}%
\pgfpathlineto{\pgfqpoint{3.148465in}{2.678904in}}%
\pgfpathlineto{\pgfqpoint{3.158937in}{2.733724in}}%
\pgfpathlineto{\pgfqpoint{3.169408in}{2.666188in}}%
\pgfpathlineto{\pgfqpoint{3.179880in}{2.735883in}}%
\pgfpathlineto{\pgfqpoint{3.190352in}{2.664108in}}%
\pgfpathlineto{\pgfqpoint{3.200823in}{2.577824in}}%
\pgfpathlineto{\pgfqpoint{3.211295in}{2.542848in}}%
\pgfpathlineto{\pgfqpoint{3.221766in}{2.610831in}}%
\pgfpathlineto{\pgfqpoint{3.242710in}{2.659991in}}%
\pgfpathlineto{\pgfqpoint{3.253181in}{2.738556in}}%
\pgfpathlineto{\pgfqpoint{3.263653in}{2.765276in}}%
\pgfpathlineto{\pgfqpoint{3.274125in}{2.691677in}}%
\pgfpathlineto{\pgfqpoint{3.284596in}{2.591030in}}%
\pgfpathlineto{\pgfqpoint{3.295068in}{2.592791in}}%
\pgfpathlineto{\pgfqpoint{3.305539in}{2.688412in}}%
\pgfpathlineto{\pgfqpoint{3.316011in}{2.624961in}}%
\pgfpathlineto{\pgfqpoint{3.326483in}{2.702031in}}%
\pgfpathlineto{\pgfqpoint{3.336954in}{2.810667in}}%
\pgfpathlineto{\pgfqpoint{3.347426in}{2.766496in}}%
\pgfpathlineto{\pgfqpoint{3.357898in}{2.700410in}}%
\pgfpathlineto{\pgfqpoint{3.368369in}{2.711013in}}%
\pgfpathlineto{\pgfqpoint{3.378841in}{2.595110in}}%
\pgfpathlineto{\pgfqpoint{3.389312in}{2.628220in}}%
\pgfpathlineto{\pgfqpoint{3.399784in}{2.692869in}}%
\pgfpathlineto{\pgfqpoint{3.410256in}{2.675910in}}%
\pgfpathlineto{\pgfqpoint{3.420727in}{2.717497in}}%
\pgfpathlineto{\pgfqpoint{3.431199in}{2.809986in}}%
\pgfpathlineto{\pgfqpoint{3.441671in}{2.758437in}}%
\pgfpathlineto{\pgfqpoint{3.452142in}{2.748733in}}%
\pgfpathlineto{\pgfqpoint{3.462614in}{2.717273in}}%
\pgfpathlineto{\pgfqpoint{3.473085in}{2.750030in}}%
\pgfpathlineto{\pgfqpoint{3.483557in}{2.630636in}}%
\pgfpathlineto{\pgfqpoint{3.494029in}{2.652030in}}%
\pgfpathlineto{\pgfqpoint{3.504500in}{2.767538in}}%
\pgfpathlineto{\pgfqpoint{3.514972in}{2.860781in}}%
\pgfpathlineto{\pgfqpoint{3.525444in}{2.808282in}}%
\pgfpathlineto{\pgfqpoint{3.535915in}{2.699779in}}%
\pgfpathlineto{\pgfqpoint{3.546387in}{2.731000in}}%
\pgfpathlineto{\pgfqpoint{3.556859in}{2.661709in}}%
\pgfpathlineto{\pgfqpoint{3.567330in}{2.709545in}}%
\pgfpathlineto{\pgfqpoint{3.577802in}{2.695208in}}%
\pgfpathlineto{\pgfqpoint{3.588273in}{2.742374in}}%
\pgfpathlineto{\pgfqpoint{3.598745in}{2.767455in}}%
\pgfpathlineto{\pgfqpoint{3.609217in}{2.821270in}}%
\pgfpathlineto{\pgfqpoint{3.619688in}{2.852092in}}%
\pgfpathlineto{\pgfqpoint{3.630160in}{2.716719in}}%
\pgfpathlineto{\pgfqpoint{3.640632in}{2.788883in}}%
\pgfpathlineto{\pgfqpoint{3.651103in}{2.731600in}}%
\pgfpathlineto{\pgfqpoint{3.661575in}{2.723617in}}%
\pgfpathlineto{\pgfqpoint{3.672046in}{2.817216in}}%
\pgfpathlineto{\pgfqpoint{3.682518in}{2.850066in}}%
\pgfpathlineto{\pgfqpoint{3.692990in}{2.903678in}}%
\pgfpathlineto{\pgfqpoint{3.703461in}{2.844780in}}%
\pgfpathlineto{\pgfqpoint{3.713933in}{2.824997in}}%
\pgfpathlineto{\pgfqpoint{3.724405in}{2.817052in}}%
\pgfpathlineto{\pgfqpoint{3.734876in}{2.798194in}}%
\pgfpathlineto{\pgfqpoint{3.745348in}{2.781792in}}%
\pgfpathlineto{\pgfqpoint{3.755819in}{2.834767in}}%
\pgfpathlineto{\pgfqpoint{3.766291in}{2.832016in}}%
\pgfpathlineto{\pgfqpoint{3.776763in}{2.956321in}}%
\pgfpathlineto{\pgfqpoint{3.787234in}{2.945859in}}%
\pgfpathlineto{\pgfqpoint{3.797706in}{2.854847in}}%
\pgfpathlineto{\pgfqpoint{3.808178in}{2.812124in}}%
\pgfpathlineto{\pgfqpoint{3.818649in}{2.855113in}}%
\pgfpathlineto{\pgfqpoint{3.829121in}{2.716490in}}%
\pgfpathlineto{\pgfqpoint{3.839592in}{2.771330in}}%
\pgfpathlineto{\pgfqpoint{3.850064in}{2.727654in}}%
\pgfpathlineto{\pgfqpoint{3.860536in}{2.891632in}}%
\pgfpathlineto{\pgfqpoint{3.871007in}{2.974220in}}%
\pgfpathlineto{\pgfqpoint{3.881479in}{2.857930in}}%
\pgfpathlineto{\pgfqpoint{3.891951in}{2.874353in}}%
\pgfpathlineto{\pgfqpoint{3.902422in}{2.887058in}}%
\pgfpathlineto{\pgfqpoint{3.912894in}{2.742082in}}%
\pgfpathlineto{\pgfqpoint{3.923365in}{2.752108in}}%
\pgfpathlineto{\pgfqpoint{3.933837in}{2.897224in}}%
\pgfpathlineto{\pgfqpoint{3.944309in}{2.975316in}}%
\pgfpathlineto{\pgfqpoint{3.954780in}{2.936950in}}%
\pgfpathlineto{\pgfqpoint{3.965252in}{2.882614in}}%
\pgfpathlineto{\pgfqpoint{3.975724in}{2.990457in}}%
\pgfpathlineto{\pgfqpoint{3.986195in}{2.880275in}}%
\pgfpathlineto{\pgfqpoint{3.996667in}{2.811767in}}%
\pgfpathlineto{\pgfqpoint{4.007138in}{2.956457in}}%
\pgfpathlineto{\pgfqpoint{4.017610in}{2.932997in}}%
\pgfpathlineto{\pgfqpoint{4.028082in}{3.023746in}}%
\pgfpathlineto{\pgfqpoint{4.038553in}{2.999306in}}%
\pgfpathlineto{\pgfqpoint{4.049025in}{2.982948in}}%
\pgfpathlineto{\pgfqpoint{4.059497in}{2.925100in}}%
\pgfpathlineto{\pgfqpoint{4.069968in}{2.934815in}}%
\pgfpathlineto{\pgfqpoint{4.080440in}{2.857976in}}%
\pgfpathlineto{\pgfqpoint{4.090912in}{2.842813in}}%
\pgfpathlineto{\pgfqpoint{4.101383in}{2.911370in}}%
\pgfpathlineto{\pgfqpoint{4.111855in}{3.035745in}}%
\pgfpathlineto{\pgfqpoint{4.122326in}{3.090009in}}%
\pgfpathlineto{\pgfqpoint{4.132798in}{2.964755in}}%
\pgfpathlineto{\pgfqpoint{4.143270in}{2.995603in}}%
\pgfpathlineto{\pgfqpoint{4.164213in}{2.871725in}}%
\pgfpathlineto{\pgfqpoint{4.174685in}{2.888137in}}%
\pgfpathlineto{\pgfqpoint{4.185156in}{3.049320in}}%
\pgfpathlineto{\pgfqpoint{4.195628in}{3.051601in}}%
\pgfpathlineto{\pgfqpoint{4.206099in}{3.020378in}}%
\pgfpathlineto{\pgfqpoint{4.216571in}{3.126283in}}%
\pgfpathlineto{\pgfqpoint{4.227043in}{3.052669in}}%
\pgfpathlineto{\pgfqpoint{4.237514in}{2.992288in}}%
\pgfpathlineto{\pgfqpoint{4.247986in}{3.022699in}}%
\pgfpathlineto{\pgfqpoint{4.258458in}{3.043600in}}%
\pgfpathlineto{\pgfqpoint{4.268929in}{3.124382in}}%
\pgfpathlineto{\pgfqpoint{4.279401in}{3.080385in}}%
\pgfpathlineto{\pgfqpoint{4.289872in}{3.064194in}}%
\pgfpathlineto{\pgfqpoint{4.300344in}{3.113232in}}%
\pgfpathlineto{\pgfqpoint{4.310816in}{2.861654in}}%
\pgfpathlineto{\pgfqpoint{4.321287in}{2.910376in}}%
\pgfpathlineto{\pgfqpoint{4.331759in}{3.025929in}}%
\pgfpathlineto{\pgfqpoint{4.342231in}{3.188485in}}%
\pgfpathlineto{\pgfqpoint{4.352702in}{3.242603in}}%
\pgfpathlineto{\pgfqpoint{4.363174in}{3.129246in}}%
\pgfpathlineto{\pgfqpoint{4.373645in}{3.049907in}}%
\pgfpathlineto{\pgfqpoint{4.384117in}{3.028914in}}%
\pgfpathlineto{\pgfqpoint{4.394589in}{3.150903in}}%
\pgfpathlineto{\pgfqpoint{4.405060in}{3.163750in}}%
\pgfpathlineto{\pgfqpoint{4.415532in}{3.168648in}}%
\pgfpathlineto{\pgfqpoint{4.426004in}{3.242675in}}%
\pgfpathlineto{\pgfqpoint{4.436475in}{3.264311in}}%
\pgfpathlineto{\pgfqpoint{4.446947in}{3.109127in}}%
\pgfpathlineto{\pgfqpoint{4.457418in}{3.144387in}}%
\pgfpathlineto{\pgfqpoint{4.467890in}{3.263402in}}%
\pgfpathlineto{\pgfqpoint{4.478362in}{3.320187in}}%
\pgfpathlineto{\pgfqpoint{4.488833in}{3.313777in}}%
\pgfpathlineto{\pgfqpoint{4.499305in}{3.270933in}}%
\pgfpathlineto{\pgfqpoint{4.509777in}{3.154906in}}%
\pgfpathlineto{\pgfqpoint{4.520248in}{3.178076in}}%
\pgfpathlineto{\pgfqpoint{4.530720in}{3.296274in}}%
\pgfpathlineto{\pgfqpoint{4.541191in}{3.342976in}}%
\pgfpathlineto{\pgfqpoint{4.551663in}{3.208082in}}%
\pgfpathlineto{\pgfqpoint{4.562135in}{3.277715in}}%
\pgfpathlineto{\pgfqpoint{4.572606in}{3.213325in}}%
\pgfpathlineto{\pgfqpoint{4.583078in}{3.233097in}}%
\pgfpathlineto{\pgfqpoint{4.593550in}{3.222737in}}%
\pgfpathlineto{\pgfqpoint{4.614493in}{3.413666in}}%
\pgfpathlineto{\pgfqpoint{4.624965in}{3.346816in}}%
\pgfpathlineto{\pgfqpoint{4.635436in}{3.150965in}}%
\pgfpathlineto{\pgfqpoint{4.645908in}{3.279439in}}%
\pgfpathlineto{\pgfqpoint{4.656379in}{3.349612in}}%
\pgfpathlineto{\pgfqpoint{4.666851in}{3.371339in}}%
\pgfpathlineto{\pgfqpoint{4.677323in}{3.382525in}}%
\pgfpathlineto{\pgfqpoint{4.687794in}{3.389904in}}%
\pgfpathlineto{\pgfqpoint{4.698266in}{3.270462in}}%
\pgfpathlineto{\pgfqpoint{4.708738in}{3.344628in}}%
\pgfpathlineto{\pgfqpoint{4.719209in}{3.487497in}}%
\pgfpathlineto{\pgfqpoint{4.729681in}{3.402314in}}%
\pgfpathlineto{\pgfqpoint{4.740152in}{3.419262in}}%
\pgfpathlineto{\pgfqpoint{4.750624in}{3.429299in}}%
\pgfpathlineto{\pgfqpoint{4.761096in}{3.356001in}}%
\pgfpathlineto{\pgfqpoint{4.771567in}{3.383547in}}%
\pgfpathlineto{\pgfqpoint{4.782039in}{3.426278in}}%
\pgfpathlineto{\pgfqpoint{4.792511in}{3.510273in}}%
\pgfpathlineto{\pgfqpoint{4.802982in}{3.365661in}}%
\pgfpathlineto{\pgfqpoint{4.813454in}{3.340131in}}%
\pgfpathlineto{\pgfqpoint{4.823925in}{3.221956in}}%
\pgfpathlineto{\pgfqpoint{4.834397in}{3.330155in}}%
\pgfpathlineto{\pgfqpoint{4.844869in}{3.392249in}}%
\pgfpathlineto{\pgfqpoint{4.855340in}{3.480357in}}%
\pgfpathlineto{\pgfqpoint{4.865812in}{3.506680in}}%
\pgfpathlineto{\pgfqpoint{4.876284in}{3.313454in}}%
\pgfpathlineto{\pgfqpoint{4.886755in}{3.286328in}}%
\pgfpathlineto{\pgfqpoint{4.897227in}{3.495991in}}%
\pgfpathlineto{\pgfqpoint{4.907698in}{3.535445in}}%
\pgfpathlineto{\pgfqpoint{4.918170in}{3.533915in}}%
\pgfpathlineto{\pgfqpoint{4.928642in}{3.511178in}}%
\pgfpathlineto{\pgfqpoint{4.939113in}{3.473540in}}%
\pgfpathlineto{\pgfqpoint{4.949585in}{3.402820in}}%
\pgfpathlineto{\pgfqpoint{4.960057in}{3.533569in}}%
\pgfpathlineto{\pgfqpoint{4.970528in}{3.561653in}}%
\pgfpathlineto{\pgfqpoint{4.981000in}{3.470495in}}%
\pgfpathlineto{\pgfqpoint{4.991471in}{3.516320in}}%
\pgfpathlineto{\pgfqpoint{5.001943in}{3.434713in}}%
\pgfpathlineto{\pgfqpoint{5.012415in}{3.412934in}}%
\pgfpathlineto{\pgfqpoint{5.022886in}{3.611331in}}%
\pgfpathlineto{\pgfqpoint{5.033358in}{3.614596in}}%
\pgfpathlineto{\pgfqpoint{5.043830in}{3.572455in}}%
\pgfpathlineto{\pgfqpoint{5.054301in}{3.545648in}}%
\pgfpathlineto{\pgfqpoint{5.064773in}{3.503923in}}%
\pgfpathlineto{\pgfqpoint{5.075244in}{3.473605in}}%
\pgfpathlineto{\pgfqpoint{5.085716in}{3.608440in}}%
\pgfpathlineto{\pgfqpoint{5.096188in}{3.666789in}}%
\pgfpathlineto{\pgfqpoint{5.106659in}{3.509302in}}%
\pgfpathlineto{\pgfqpoint{5.117131in}{3.583895in}}%
\pgfpathlineto{\pgfqpoint{5.127603in}{3.599402in}}%
\pgfpathlineto{\pgfqpoint{5.138074in}{3.573404in}}%
\pgfpathlineto{\pgfqpoint{5.148546in}{3.651428in}}%
\pgfpathlineto{\pgfqpoint{5.159017in}{3.715078in}}%
\pgfpathlineto{\pgfqpoint{5.169489in}{3.600846in}}%
\pgfpathlineto{\pgfqpoint{5.179961in}{3.551342in}}%
\pgfpathlineto{\pgfqpoint{5.190432in}{3.694132in}}%
\pgfpathlineto{\pgfqpoint{5.200904in}{3.722375in}}%
\pgfpathlineto{\pgfqpoint{5.211376in}{3.689269in}}%
\pgfpathlineto{\pgfqpoint{5.221847in}{3.690827in}}%
\pgfpathlineto{\pgfqpoint{5.232319in}{3.606234in}}%
\pgfpathlineto{\pgfqpoint{5.242791in}{3.542507in}}%
\pgfpathlineto{\pgfqpoint{5.253262in}{3.798342in}}%
\pgfpathlineto{\pgfqpoint{5.263734in}{3.772064in}}%
\pgfpathlineto{\pgfqpoint{5.274205in}{3.703473in}}%
\pgfpathlineto{\pgfqpoint{5.284677in}{3.712873in}}%
\pgfpathlineto{\pgfqpoint{5.295149in}{3.631900in}}%
\pgfpathlineto{\pgfqpoint{5.305620in}{3.813264in}}%
\pgfpathlineto{\pgfqpoint{5.316092in}{3.776426in}}%
\pgfpathlineto{\pgfqpoint{5.326564in}{3.668903in}}%
\pgfpathlineto{\pgfqpoint{5.347507in}{3.732113in}}%
\pgfpathlineto{\pgfqpoint{5.357978in}{3.859975in}}%
\pgfpathlineto{\pgfqpoint{5.368450in}{3.827814in}}%
\pgfpathlineto{\pgfqpoint{5.378922in}{3.695442in}}%
\pgfpathlineto{\pgfqpoint{5.389393in}{3.778720in}}%
\pgfpathlineto{\pgfqpoint{5.399865in}{3.840492in}}%
\pgfpathlineto{\pgfqpoint{5.410337in}{3.860365in}}%
\pgfpathlineto{\pgfqpoint{5.420808in}{3.932632in}}%
\pgfpathlineto{\pgfqpoint{5.431280in}{3.698040in}}%
\pgfpathlineto{\pgfqpoint{5.441751in}{3.771016in}}%
\pgfpathlineto{\pgfqpoint{5.452223in}{3.906900in}}%
\pgfpathlineto{\pgfqpoint{5.462695in}{3.924435in}}%
\pgfpathlineto{\pgfqpoint{5.473166in}{3.980005in}}%
\pgfpathlineto{\pgfqpoint{5.483638in}{3.768096in}}%
\pgfpathlineto{\pgfqpoint{5.504581in}{3.996101in}}%
\pgfpathlineto{\pgfqpoint{5.515053in}{3.881678in}}%
\pgfpathlineto{\pgfqpoint{5.525524in}{3.848890in}}%
\pgfpathlineto{\pgfqpoint{5.535996in}{3.849934in}}%
\pgfpathlineto{\pgfqpoint{5.546468in}{3.853138in}}%
\pgfpathlineto{\pgfqpoint{5.556939in}{4.058610in}}%
\pgfpathlineto{\pgfqpoint{5.567411in}{3.872285in}}%
\pgfpathlineto{\pgfqpoint{5.577883in}{3.894719in}}%
\pgfpathlineto{\pgfqpoint{5.588354in}{3.932879in}}%
\pgfpathlineto{\pgfqpoint{5.598826in}{3.949085in}}%
\pgfpathlineto{\pgfqpoint{5.609297in}{4.054576in}}%
\pgfpathlineto{\pgfqpoint{5.619769in}{3.927981in}}%
\pgfpathlineto{\pgfqpoint{5.630241in}{3.885859in}}%
\pgfpathlineto{\pgfqpoint{5.640712in}{4.051805in}}%
\pgfpathlineto{\pgfqpoint{5.651184in}{4.034580in}}%
\pgfpathlineto{\pgfqpoint{5.661656in}{4.108656in}}%
\pgfpathlineto{\pgfqpoint{5.672127in}{3.999045in}}%
\pgfpathlineto{\pgfqpoint{5.682599in}{3.969369in}}%
\pgfpathlineto{\pgfqpoint{5.693070in}{4.146729in}}%
\pgfpathlineto{\pgfqpoint{5.703542in}{4.057968in}}%
\pgfpathlineto{\pgfqpoint{5.714014in}{4.048667in}}%
\pgfpathlineto{\pgfqpoint{5.724485in}{3.980667in}}%
\pgfpathlineto{\pgfqpoint{5.734957in}{3.988781in}}%
\pgfpathlineto{\pgfqpoint{5.745429in}{4.138859in}}%
\pgfpathlineto{\pgfqpoint{5.755900in}{4.190389in}}%
\pgfpathlineto{\pgfqpoint{5.766372in}{4.037053in}}%
\pgfpathlineto{\pgfqpoint{5.776844in}{4.071975in}}%
\pgfpathlineto{\pgfqpoint{5.787315in}{4.051667in}}%
\pgfpathlineto{\pgfqpoint{5.797787in}{4.185752in}}%
\pgfpathlineto{\pgfqpoint{5.808258in}{4.183889in}}%
\pgfpathlineto{\pgfqpoint{5.818730in}{4.022517in}}%
\pgfpathlineto{\pgfqpoint{5.829202in}{4.089885in}}%
\pgfpathlineto{\pgfqpoint{5.839673in}{4.079506in}}%
\pgfpathlineto{\pgfqpoint{5.850145in}{4.229096in}}%
\pgfpathlineto{\pgfqpoint{5.860617in}{4.256176in}}%
\pgfpathlineto{\pgfqpoint{5.871088in}{4.108749in}}%
\pgfpathlineto{\pgfqpoint{5.892031in}{4.254143in}}%
\pgfpathlineto{\pgfqpoint{5.902503in}{4.233148in}}%
\pgfpathlineto{\pgfqpoint{5.912975in}{4.249752in}}%
\pgfpathlineto{\pgfqpoint{5.923446in}{4.176074in}}%
\pgfpathlineto{\pgfqpoint{5.933918in}{4.267196in}}%
\pgfpathlineto{\pgfqpoint{5.944390in}{4.319252in}}%
\pgfpathlineto{\pgfqpoint{5.954861in}{4.249070in}}%
\pgfpathlineto{\pgfqpoint{5.965333in}{4.208296in}}%
\pgfpathlineto{\pgfqpoint{5.975804in}{4.150846in}}%
\pgfpathlineto{\pgfqpoint{5.986276in}{4.262677in}}%
\pgfpathlineto{\pgfqpoint{5.996748in}{4.356567in}}%
\pgfpathlineto{\pgfqpoint{6.007219in}{4.260181in}}%
\pgfpathlineto{\pgfqpoint{6.017691in}{4.244025in}}%
\pgfpathlineto{\pgfqpoint{6.028163in}{4.222702in}}%
\pgfpathlineto{\pgfqpoint{6.038634in}{4.373287in}}%
\pgfpathlineto{\pgfqpoint{6.049106in}{4.392345in}}%
\pgfpathlineto{\pgfqpoint{6.059577in}{4.258429in}}%
\pgfpathlineto{\pgfqpoint{6.070049in}{4.330463in}}%
\pgfpathlineto{\pgfqpoint{6.080521in}{4.336770in}}%
\pgfpathlineto{\pgfqpoint{6.080521in}{4.336770in}}%
\pgfusepath{stroke}%
\end{pgfscope}%
\begin{pgfscope}%
\pgfsetrectcap%
\pgfsetmiterjoin%
\pgfsetlinewidth{0.803000pt}%
\definecolor{currentstroke}{rgb}{0.000000,0.000000,0.000000}%
\pgfsetstrokecolor{currentstroke}%
\pgfsetdash{}{0pt}%
\pgfpathmoveto{\pgfqpoint{1.075083in}{0.549691in}}%
\pgfpathlineto{\pgfqpoint{1.075083in}{4.575328in}}%
\pgfusepath{stroke}%
\end{pgfscope}%
\begin{pgfscope}%
\pgfsetrectcap%
\pgfsetmiterjoin%
\pgfsetlinewidth{0.803000pt}%
\definecolor{currentstroke}{rgb}{0.000000,0.000000,0.000000}%
\pgfsetstrokecolor{currentstroke}%
\pgfsetdash{}{0pt}%
\pgfpathmoveto{\pgfqpoint{6.330793in}{0.549691in}}%
\pgfpathlineto{\pgfqpoint{6.330793in}{4.575328in}}%
\pgfusepath{stroke}%
\end{pgfscope}%
\begin{pgfscope}%
\pgfsetrectcap%
\pgfsetmiterjoin%
\pgfsetlinewidth{0.803000pt}%
\definecolor{currentstroke}{rgb}{0.000000,0.000000,0.000000}%
\pgfsetstrokecolor{currentstroke}%
\pgfsetdash{}{0pt}%
\pgfpathmoveto{\pgfqpoint{1.075083in}{0.549691in}}%
\pgfpathlineto{\pgfqpoint{6.330793in}{0.549691in}}%
\pgfusepath{stroke}%
\end{pgfscope}%
\begin{pgfscope}%
\pgfsetrectcap%
\pgfsetmiterjoin%
\pgfsetlinewidth{0.803000pt}%
\definecolor{currentstroke}{rgb}{0.000000,0.000000,0.000000}%
\pgfsetstrokecolor{currentstroke}%
\pgfsetdash{}{0pt}%
\pgfpathmoveto{\pgfqpoint{1.075083in}{4.575328in}}%
\pgfpathlineto{\pgfqpoint{6.330793in}{4.575328in}}%
\pgfusepath{stroke}%
\end{pgfscope}%
\end{pgfpicture}%
\makeatother%
\endgroup%
}
	\caption{Orthogonality of the Krylov vectors. Plot of the dot product $(\mathbf{v}_1, \mathbf{v}_k)$ against the iteration index $k$ of the top loop in the GMRES method}
	\label{fig::DotP}
\end{figure}



\section{Conjugate Gradient Method - CG}
\label{chapter:CG}

\begin{figure}[!htbp]
	\centering
	\hspace*{0.8cm}
	\leavevmode
	\resizebox{0.9\width}{!}{%% Creator: Matplotlib, PGF backend
%%
%% To include the figure in your LaTeX document, write
%%   \input{<filename>.pgf}
%%
%% Make sure the required packages are loaded in your preamble
%%   \usepackage{pgf}
%%
%% Also ensure that all the required font packages are loaded; for instance,
%% the lmodern package is sometimes necessary when using math font.
%%   \usepackage{lmodern}
%%
%% Figures using additional raster images can only be included by \input if
%% they are in the same directory as the main LaTeX file. For loading figures
%% from other directories you can use the `import` package
%%   \usepackage{import}
%%
%% and then include the figures with
%%   \import{<path to file>}{<filename>.pgf}
%%
%% Matplotlib used the following preamble
%%   
%%   \makeatletter\@ifpackageloaded{underscore}{}{\usepackage[strings]{underscore}}\makeatother
%%
\begingroup%
\makeatletter%
\begin{pgfpicture}%
\pgfpathrectangle{\pgfpointorigin}{\pgfqpoint{6.565064in}{4.725328in}}%
\pgfusepath{use as bounding box, clip}%
\begin{pgfscope}%
\pgfsetbuttcap%
\pgfsetmiterjoin%
\definecolor{currentfill}{rgb}{1.000000,1.000000,1.000000}%
\pgfsetfillcolor{currentfill}%
\pgfsetlinewidth{0.000000pt}%
\definecolor{currentstroke}{rgb}{1.000000,1.000000,1.000000}%
\pgfsetstrokecolor{currentstroke}%
\pgfsetdash{}{0pt}%
\pgfpathmoveto{\pgfqpoint{0.000000in}{0.000000in}}%
\pgfpathlineto{\pgfqpoint{6.565064in}{0.000000in}}%
\pgfpathlineto{\pgfqpoint{6.565064in}{4.725328in}}%
\pgfpathlineto{\pgfqpoint{0.000000in}{4.725328in}}%
\pgfpathlineto{\pgfqpoint{0.000000in}{0.000000in}}%
\pgfpathclose%
\pgfusepath{fill}%
\end{pgfscope}%
\begin{pgfscope}%
\pgfsetbuttcap%
\pgfsetmiterjoin%
\definecolor{currentfill}{rgb}{1.000000,1.000000,1.000000}%
\pgfsetfillcolor{currentfill}%
\pgfsetlinewidth{0.000000pt}%
\definecolor{currentstroke}{rgb}{0.000000,0.000000,0.000000}%
\pgfsetstrokecolor{currentstroke}%
\pgfsetstrokeopacity{0.000000}%
\pgfsetdash{}{0pt}%
\pgfpathmoveto{\pgfqpoint{1.116292in}{0.549691in}}%
\pgfpathlineto{\pgfqpoint{6.415064in}{0.549691in}}%
\pgfpathlineto{\pgfqpoint{6.415064in}{4.556718in}}%
\pgfpathlineto{\pgfqpoint{1.116292in}{4.556718in}}%
\pgfpathlineto{\pgfqpoint{1.116292in}{0.549691in}}%
\pgfpathclose%
\pgfusepath{fill}%
\end{pgfscope}%
\begin{pgfscope}%
\pgfpathrectangle{\pgfqpoint{1.116292in}{0.549691in}}{\pgfqpoint{5.298772in}{4.007027in}}%
\pgfusepath{clip}%
\pgfsetrectcap%
\pgfsetroundjoin%
\pgfsetlinewidth{0.803000pt}%
\definecolor{currentstroke}{rgb}{0.690196,0.690196,0.690196}%
\pgfsetstrokecolor{currentstroke}%
\pgfsetdash{}{0pt}%
\pgfpathmoveto{\pgfqpoint{1.116292in}{0.549691in}}%
\pgfpathlineto{\pgfqpoint{1.116292in}{4.556718in}}%
\pgfusepath{stroke}%
\end{pgfscope}%
\begin{pgfscope}%
\pgfsetbuttcap%
\pgfsetroundjoin%
\definecolor{currentfill}{rgb}{0.000000,0.000000,0.000000}%
\pgfsetfillcolor{currentfill}%
\pgfsetlinewidth{0.803000pt}%
\definecolor{currentstroke}{rgb}{0.000000,0.000000,0.000000}%
\pgfsetstrokecolor{currentstroke}%
\pgfsetdash{}{0pt}%
\pgfsys@defobject{currentmarker}{\pgfqpoint{0.000000in}{-0.048611in}}{\pgfqpoint{0.000000in}{0.000000in}}{%
\pgfpathmoveto{\pgfqpoint{0.000000in}{0.000000in}}%
\pgfpathlineto{\pgfqpoint{0.000000in}{-0.048611in}}%
\pgfusepath{stroke,fill}%
}%
\begin{pgfscope}%
\pgfsys@transformshift{1.116292in}{0.549691in}%
\pgfsys@useobject{currentmarker}{}%
\end{pgfscope}%
\end{pgfscope}%
\begin{pgfscope}%
\definecolor{textcolor}{rgb}{0.000000,0.000000,0.000000}%
\pgfsetstrokecolor{textcolor}%
\pgfsetfillcolor{textcolor}%
\pgftext[x=1.116292in,y=0.452469in,,top]{\color{textcolor}\rmfamily\fontsize{10.000000}{12.000000}\selectfont \(\displaystyle {0}\)}%
\end{pgfscope}%
\begin{pgfscope}%
\pgfpathrectangle{\pgfqpoint{1.116292in}{0.549691in}}{\pgfqpoint{5.298772in}{4.007027in}}%
\pgfusepath{clip}%
\pgfsetrectcap%
\pgfsetroundjoin%
\pgfsetlinewidth{0.803000pt}%
\definecolor{currentstroke}{rgb}{0.690196,0.690196,0.690196}%
\pgfsetstrokecolor{currentstroke}%
\pgfsetdash{}{0pt}%
\pgfpathmoveto{\pgfqpoint{2.285588in}{0.549691in}}%
\pgfpathlineto{\pgfqpoint{2.285588in}{4.556718in}}%
\pgfusepath{stroke}%
\end{pgfscope}%
\begin{pgfscope}%
\pgfsetbuttcap%
\pgfsetroundjoin%
\definecolor{currentfill}{rgb}{0.000000,0.000000,0.000000}%
\pgfsetfillcolor{currentfill}%
\pgfsetlinewidth{0.803000pt}%
\definecolor{currentstroke}{rgb}{0.000000,0.000000,0.000000}%
\pgfsetstrokecolor{currentstroke}%
\pgfsetdash{}{0pt}%
\pgfsys@defobject{currentmarker}{\pgfqpoint{0.000000in}{-0.048611in}}{\pgfqpoint{0.000000in}{0.000000in}}{%
\pgfpathmoveto{\pgfqpoint{0.000000in}{0.000000in}}%
\pgfpathlineto{\pgfqpoint{0.000000in}{-0.048611in}}%
\pgfusepath{stroke,fill}%
}%
\begin{pgfscope}%
\pgfsys@transformshift{2.285588in}{0.549691in}%
\pgfsys@useobject{currentmarker}{}%
\end{pgfscope}%
\end{pgfscope}%
\begin{pgfscope}%
\definecolor{textcolor}{rgb}{0.000000,0.000000,0.000000}%
\pgfsetstrokecolor{textcolor}%
\pgfsetfillcolor{textcolor}%
\pgftext[x=2.285588in,y=0.452469in,,top]{\color{textcolor}\rmfamily\fontsize{10.000000}{12.000000}\selectfont \(\displaystyle {5000}\)}%
\end{pgfscope}%
\begin{pgfscope}%
\pgfpathrectangle{\pgfqpoint{1.116292in}{0.549691in}}{\pgfqpoint{5.298772in}{4.007027in}}%
\pgfusepath{clip}%
\pgfsetrectcap%
\pgfsetroundjoin%
\pgfsetlinewidth{0.803000pt}%
\definecolor{currentstroke}{rgb}{0.690196,0.690196,0.690196}%
\pgfsetstrokecolor{currentstroke}%
\pgfsetdash{}{0pt}%
\pgfpathmoveto{\pgfqpoint{3.454885in}{0.549691in}}%
\pgfpathlineto{\pgfqpoint{3.454885in}{4.556718in}}%
\pgfusepath{stroke}%
\end{pgfscope}%
\begin{pgfscope}%
\pgfsetbuttcap%
\pgfsetroundjoin%
\definecolor{currentfill}{rgb}{0.000000,0.000000,0.000000}%
\pgfsetfillcolor{currentfill}%
\pgfsetlinewidth{0.803000pt}%
\definecolor{currentstroke}{rgb}{0.000000,0.000000,0.000000}%
\pgfsetstrokecolor{currentstroke}%
\pgfsetdash{}{0pt}%
\pgfsys@defobject{currentmarker}{\pgfqpoint{0.000000in}{-0.048611in}}{\pgfqpoint{0.000000in}{0.000000in}}{%
\pgfpathmoveto{\pgfqpoint{0.000000in}{0.000000in}}%
\pgfpathlineto{\pgfqpoint{0.000000in}{-0.048611in}}%
\pgfusepath{stroke,fill}%
}%
\begin{pgfscope}%
\pgfsys@transformshift{3.454885in}{0.549691in}%
\pgfsys@useobject{currentmarker}{}%
\end{pgfscope}%
\end{pgfscope}%
\begin{pgfscope}%
\definecolor{textcolor}{rgb}{0.000000,0.000000,0.000000}%
\pgfsetstrokecolor{textcolor}%
\pgfsetfillcolor{textcolor}%
\pgftext[x=3.454885in,y=0.452469in,,top]{\color{textcolor}\rmfamily\fontsize{10.000000}{12.000000}\selectfont \(\displaystyle {10000}\)}%
\end{pgfscope}%
\begin{pgfscope}%
\pgfpathrectangle{\pgfqpoint{1.116292in}{0.549691in}}{\pgfqpoint{5.298772in}{4.007027in}}%
\pgfusepath{clip}%
\pgfsetrectcap%
\pgfsetroundjoin%
\pgfsetlinewidth{0.803000pt}%
\definecolor{currentstroke}{rgb}{0.690196,0.690196,0.690196}%
\pgfsetstrokecolor{currentstroke}%
\pgfsetdash{}{0pt}%
\pgfpathmoveto{\pgfqpoint{4.624181in}{0.549691in}}%
\pgfpathlineto{\pgfqpoint{4.624181in}{4.556718in}}%
\pgfusepath{stroke}%
\end{pgfscope}%
\begin{pgfscope}%
\pgfsetbuttcap%
\pgfsetroundjoin%
\definecolor{currentfill}{rgb}{0.000000,0.000000,0.000000}%
\pgfsetfillcolor{currentfill}%
\pgfsetlinewidth{0.803000pt}%
\definecolor{currentstroke}{rgb}{0.000000,0.000000,0.000000}%
\pgfsetstrokecolor{currentstroke}%
\pgfsetdash{}{0pt}%
\pgfsys@defobject{currentmarker}{\pgfqpoint{0.000000in}{-0.048611in}}{\pgfqpoint{0.000000in}{0.000000in}}{%
\pgfpathmoveto{\pgfqpoint{0.000000in}{0.000000in}}%
\pgfpathlineto{\pgfqpoint{0.000000in}{-0.048611in}}%
\pgfusepath{stroke,fill}%
}%
\begin{pgfscope}%
\pgfsys@transformshift{4.624181in}{0.549691in}%
\pgfsys@useobject{currentmarker}{}%
\end{pgfscope}%
\end{pgfscope}%
\begin{pgfscope}%
\definecolor{textcolor}{rgb}{0.000000,0.000000,0.000000}%
\pgfsetstrokecolor{textcolor}%
\pgfsetfillcolor{textcolor}%
\pgftext[x=4.624181in,y=0.452469in,,top]{\color{textcolor}\rmfamily\fontsize{10.000000}{12.000000}\selectfont \(\displaystyle {15000}\)}%
\end{pgfscope}%
\begin{pgfscope}%
\pgfpathrectangle{\pgfqpoint{1.116292in}{0.549691in}}{\pgfqpoint{5.298772in}{4.007027in}}%
\pgfusepath{clip}%
\pgfsetrectcap%
\pgfsetroundjoin%
\pgfsetlinewidth{0.803000pt}%
\definecolor{currentstroke}{rgb}{0.690196,0.690196,0.690196}%
\pgfsetstrokecolor{currentstroke}%
\pgfsetdash{}{0pt}%
\pgfpathmoveto{\pgfqpoint{5.793478in}{0.549691in}}%
\pgfpathlineto{\pgfqpoint{5.793478in}{4.556718in}}%
\pgfusepath{stroke}%
\end{pgfscope}%
\begin{pgfscope}%
\pgfsetbuttcap%
\pgfsetroundjoin%
\definecolor{currentfill}{rgb}{0.000000,0.000000,0.000000}%
\pgfsetfillcolor{currentfill}%
\pgfsetlinewidth{0.803000pt}%
\definecolor{currentstroke}{rgb}{0.000000,0.000000,0.000000}%
\pgfsetstrokecolor{currentstroke}%
\pgfsetdash{}{0pt}%
\pgfsys@defobject{currentmarker}{\pgfqpoint{0.000000in}{-0.048611in}}{\pgfqpoint{0.000000in}{0.000000in}}{%
\pgfpathmoveto{\pgfqpoint{0.000000in}{0.000000in}}%
\pgfpathlineto{\pgfqpoint{0.000000in}{-0.048611in}}%
\pgfusepath{stroke,fill}%
}%
\begin{pgfscope}%
\pgfsys@transformshift{5.793478in}{0.549691in}%
\pgfsys@useobject{currentmarker}{}%
\end{pgfscope}%
\end{pgfscope}%
\begin{pgfscope}%
\definecolor{textcolor}{rgb}{0.000000,0.000000,0.000000}%
\pgfsetstrokecolor{textcolor}%
\pgfsetfillcolor{textcolor}%
\pgftext[x=5.793478in,y=0.452469in,,top]{\color{textcolor}\rmfamily\fontsize{10.000000}{12.000000}\selectfont \(\displaystyle {20000}\)}%
\end{pgfscope}%
\begin{pgfscope}%
\definecolor{textcolor}{rgb}{0.000000,0.000000,0.000000}%
\pgfsetstrokecolor{textcolor}%
\pgfsetfillcolor{textcolor}%
\pgftext[x=3.765678in,y=0.273457in,,top]{\color{textcolor}\rmfamily\fontsize{10.000000}{12.000000}\selectfont \(\displaystyle k\)}%
\end{pgfscope}%
\begin{pgfscope}%
\pgfpathrectangle{\pgfqpoint{1.116292in}{0.549691in}}{\pgfqpoint{5.298772in}{4.007027in}}%
\pgfusepath{clip}%
\pgfsetrectcap%
\pgfsetroundjoin%
\pgfsetlinewidth{0.803000pt}%
\definecolor{currentstroke}{rgb}{0.690196,0.690196,0.690196}%
\pgfsetstrokecolor{currentstroke}%
\pgfsetdash{}{0pt}%
\pgfpathmoveto{\pgfqpoint{1.116292in}{0.754689in}}%
\pgfpathlineto{\pgfqpoint{6.415064in}{0.754689in}}%
\pgfusepath{stroke}%
\end{pgfscope}%
\begin{pgfscope}%
\pgfsetbuttcap%
\pgfsetroundjoin%
\definecolor{currentfill}{rgb}{0.000000,0.000000,0.000000}%
\pgfsetfillcolor{currentfill}%
\pgfsetlinewidth{0.803000pt}%
\definecolor{currentstroke}{rgb}{0.000000,0.000000,0.000000}%
\pgfsetstrokecolor{currentstroke}%
\pgfsetdash{}{0pt}%
\pgfsys@defobject{currentmarker}{\pgfqpoint{-0.048611in}{0.000000in}}{\pgfqpoint{-0.000000in}{0.000000in}}{%
\pgfpathmoveto{\pgfqpoint{-0.000000in}{0.000000in}}%
\pgfpathlineto{\pgfqpoint{-0.048611in}{0.000000in}}%
\pgfusepath{stroke,fill}%
}%
\begin{pgfscope}%
\pgfsys@transformshift{1.116292in}{0.754689in}%
\pgfsys@useobject{currentmarker}{}%
\end{pgfscope}%
\end{pgfscope}%
\begin{pgfscope}%
\definecolor{textcolor}{rgb}{0.000000,0.000000,0.000000}%
\pgfsetstrokecolor{textcolor}%
\pgfsetfillcolor{textcolor}%
\pgftext[x=0.731067in, y=0.706464in, left, base]{\color{textcolor}\rmfamily\fontsize{10.000000}{12.000000}\selectfont \(\displaystyle {10^{-3}}\)}%
\end{pgfscope}%
\begin{pgfscope}%
\pgfpathrectangle{\pgfqpoint{1.116292in}{0.549691in}}{\pgfqpoint{5.298772in}{4.007027in}}%
\pgfusepath{clip}%
\pgfsetrectcap%
\pgfsetroundjoin%
\pgfsetlinewidth{0.803000pt}%
\definecolor{currentstroke}{rgb}{0.690196,0.690196,0.690196}%
\pgfsetstrokecolor{currentstroke}%
\pgfsetdash{}{0pt}%
\pgfpathmoveto{\pgfqpoint{1.116292in}{1.293605in}}%
\pgfpathlineto{\pgfqpoint{6.415064in}{1.293605in}}%
\pgfusepath{stroke}%
\end{pgfscope}%
\begin{pgfscope}%
\pgfsetbuttcap%
\pgfsetroundjoin%
\definecolor{currentfill}{rgb}{0.000000,0.000000,0.000000}%
\pgfsetfillcolor{currentfill}%
\pgfsetlinewidth{0.803000pt}%
\definecolor{currentstroke}{rgb}{0.000000,0.000000,0.000000}%
\pgfsetstrokecolor{currentstroke}%
\pgfsetdash{}{0pt}%
\pgfsys@defobject{currentmarker}{\pgfqpoint{-0.048611in}{0.000000in}}{\pgfqpoint{-0.000000in}{0.000000in}}{%
\pgfpathmoveto{\pgfqpoint{-0.000000in}{0.000000in}}%
\pgfpathlineto{\pgfqpoint{-0.048611in}{0.000000in}}%
\pgfusepath{stroke,fill}%
}%
\begin{pgfscope}%
\pgfsys@transformshift{1.116292in}{1.293605in}%
\pgfsys@useobject{currentmarker}{}%
\end{pgfscope}%
\end{pgfscope}%
\begin{pgfscope}%
\definecolor{textcolor}{rgb}{0.000000,0.000000,0.000000}%
\pgfsetstrokecolor{textcolor}%
\pgfsetfillcolor{textcolor}%
\pgftext[x=0.731067in, y=1.245380in, left, base]{\color{textcolor}\rmfamily\fontsize{10.000000}{12.000000}\selectfont \(\displaystyle {10^{-2}}\)}%
\end{pgfscope}%
\begin{pgfscope}%
\pgfpathrectangle{\pgfqpoint{1.116292in}{0.549691in}}{\pgfqpoint{5.298772in}{4.007027in}}%
\pgfusepath{clip}%
\pgfsetrectcap%
\pgfsetroundjoin%
\pgfsetlinewidth{0.803000pt}%
\definecolor{currentstroke}{rgb}{0.690196,0.690196,0.690196}%
\pgfsetstrokecolor{currentstroke}%
\pgfsetdash{}{0pt}%
\pgfpathmoveto{\pgfqpoint{1.116292in}{1.832522in}}%
\pgfpathlineto{\pgfqpoint{6.415064in}{1.832522in}}%
\pgfusepath{stroke}%
\end{pgfscope}%
\begin{pgfscope}%
\pgfsetbuttcap%
\pgfsetroundjoin%
\definecolor{currentfill}{rgb}{0.000000,0.000000,0.000000}%
\pgfsetfillcolor{currentfill}%
\pgfsetlinewidth{0.803000pt}%
\definecolor{currentstroke}{rgb}{0.000000,0.000000,0.000000}%
\pgfsetstrokecolor{currentstroke}%
\pgfsetdash{}{0pt}%
\pgfsys@defobject{currentmarker}{\pgfqpoint{-0.048611in}{0.000000in}}{\pgfqpoint{-0.000000in}{0.000000in}}{%
\pgfpathmoveto{\pgfqpoint{-0.000000in}{0.000000in}}%
\pgfpathlineto{\pgfqpoint{-0.048611in}{0.000000in}}%
\pgfusepath{stroke,fill}%
}%
\begin{pgfscope}%
\pgfsys@transformshift{1.116292in}{1.832522in}%
\pgfsys@useobject{currentmarker}{}%
\end{pgfscope}%
\end{pgfscope}%
\begin{pgfscope}%
\definecolor{textcolor}{rgb}{0.000000,0.000000,0.000000}%
\pgfsetstrokecolor{textcolor}%
\pgfsetfillcolor{textcolor}%
\pgftext[x=0.731067in, y=1.784296in, left, base]{\color{textcolor}\rmfamily\fontsize{10.000000}{12.000000}\selectfont \(\displaystyle {10^{-1}}\)}%
\end{pgfscope}%
\begin{pgfscope}%
\pgfpathrectangle{\pgfqpoint{1.116292in}{0.549691in}}{\pgfqpoint{5.298772in}{4.007027in}}%
\pgfusepath{clip}%
\pgfsetrectcap%
\pgfsetroundjoin%
\pgfsetlinewidth{0.803000pt}%
\definecolor{currentstroke}{rgb}{0.690196,0.690196,0.690196}%
\pgfsetstrokecolor{currentstroke}%
\pgfsetdash{}{0pt}%
\pgfpathmoveto{\pgfqpoint{1.116292in}{2.371438in}}%
\pgfpathlineto{\pgfqpoint{6.415064in}{2.371438in}}%
\pgfusepath{stroke}%
\end{pgfscope}%
\begin{pgfscope}%
\pgfsetbuttcap%
\pgfsetroundjoin%
\definecolor{currentfill}{rgb}{0.000000,0.000000,0.000000}%
\pgfsetfillcolor{currentfill}%
\pgfsetlinewidth{0.803000pt}%
\definecolor{currentstroke}{rgb}{0.000000,0.000000,0.000000}%
\pgfsetstrokecolor{currentstroke}%
\pgfsetdash{}{0pt}%
\pgfsys@defobject{currentmarker}{\pgfqpoint{-0.048611in}{0.000000in}}{\pgfqpoint{-0.000000in}{0.000000in}}{%
\pgfpathmoveto{\pgfqpoint{-0.000000in}{0.000000in}}%
\pgfpathlineto{\pgfqpoint{-0.048611in}{0.000000in}}%
\pgfusepath{stroke,fill}%
}%
\begin{pgfscope}%
\pgfsys@transformshift{1.116292in}{2.371438in}%
\pgfsys@useobject{currentmarker}{}%
\end{pgfscope}%
\end{pgfscope}%
\begin{pgfscope}%
\definecolor{textcolor}{rgb}{0.000000,0.000000,0.000000}%
\pgfsetstrokecolor{textcolor}%
\pgfsetfillcolor{textcolor}%
\pgftext[x=0.817873in, y=2.323213in, left, base]{\color{textcolor}\rmfamily\fontsize{10.000000}{12.000000}\selectfont \(\displaystyle {10^{0}}\)}%
\end{pgfscope}%
\begin{pgfscope}%
\pgfpathrectangle{\pgfqpoint{1.116292in}{0.549691in}}{\pgfqpoint{5.298772in}{4.007027in}}%
\pgfusepath{clip}%
\pgfsetrectcap%
\pgfsetroundjoin%
\pgfsetlinewidth{0.803000pt}%
\definecolor{currentstroke}{rgb}{0.690196,0.690196,0.690196}%
\pgfsetstrokecolor{currentstroke}%
\pgfsetdash{}{0pt}%
\pgfpathmoveto{\pgfqpoint{1.116292in}{2.910354in}}%
\pgfpathlineto{\pgfqpoint{6.415064in}{2.910354in}}%
\pgfusepath{stroke}%
\end{pgfscope}%
\begin{pgfscope}%
\pgfsetbuttcap%
\pgfsetroundjoin%
\definecolor{currentfill}{rgb}{0.000000,0.000000,0.000000}%
\pgfsetfillcolor{currentfill}%
\pgfsetlinewidth{0.803000pt}%
\definecolor{currentstroke}{rgb}{0.000000,0.000000,0.000000}%
\pgfsetstrokecolor{currentstroke}%
\pgfsetdash{}{0pt}%
\pgfsys@defobject{currentmarker}{\pgfqpoint{-0.048611in}{0.000000in}}{\pgfqpoint{-0.000000in}{0.000000in}}{%
\pgfpathmoveto{\pgfqpoint{-0.000000in}{0.000000in}}%
\pgfpathlineto{\pgfqpoint{-0.048611in}{0.000000in}}%
\pgfusepath{stroke,fill}%
}%
\begin{pgfscope}%
\pgfsys@transformshift{1.116292in}{2.910354in}%
\pgfsys@useobject{currentmarker}{}%
\end{pgfscope}%
\end{pgfscope}%
\begin{pgfscope}%
\definecolor{textcolor}{rgb}{0.000000,0.000000,0.000000}%
\pgfsetstrokecolor{textcolor}%
\pgfsetfillcolor{textcolor}%
\pgftext[x=0.817873in, y=2.862129in, left, base]{\color{textcolor}\rmfamily\fontsize{10.000000}{12.000000}\selectfont \(\displaystyle {10^{1}}\)}%
\end{pgfscope}%
\begin{pgfscope}%
\pgfpathrectangle{\pgfqpoint{1.116292in}{0.549691in}}{\pgfqpoint{5.298772in}{4.007027in}}%
\pgfusepath{clip}%
\pgfsetrectcap%
\pgfsetroundjoin%
\pgfsetlinewidth{0.803000pt}%
\definecolor{currentstroke}{rgb}{0.690196,0.690196,0.690196}%
\pgfsetstrokecolor{currentstroke}%
\pgfsetdash{}{0pt}%
\pgfpathmoveto{\pgfqpoint{1.116292in}{3.449271in}}%
\pgfpathlineto{\pgfqpoint{6.415064in}{3.449271in}}%
\pgfusepath{stroke}%
\end{pgfscope}%
\begin{pgfscope}%
\pgfsetbuttcap%
\pgfsetroundjoin%
\definecolor{currentfill}{rgb}{0.000000,0.000000,0.000000}%
\pgfsetfillcolor{currentfill}%
\pgfsetlinewidth{0.803000pt}%
\definecolor{currentstroke}{rgb}{0.000000,0.000000,0.000000}%
\pgfsetstrokecolor{currentstroke}%
\pgfsetdash{}{0pt}%
\pgfsys@defobject{currentmarker}{\pgfqpoint{-0.048611in}{0.000000in}}{\pgfqpoint{-0.000000in}{0.000000in}}{%
\pgfpathmoveto{\pgfqpoint{-0.000000in}{0.000000in}}%
\pgfpathlineto{\pgfqpoint{-0.048611in}{0.000000in}}%
\pgfusepath{stroke,fill}%
}%
\begin{pgfscope}%
\pgfsys@transformshift{1.116292in}{3.449271in}%
\pgfsys@useobject{currentmarker}{}%
\end{pgfscope}%
\end{pgfscope}%
\begin{pgfscope}%
\definecolor{textcolor}{rgb}{0.000000,0.000000,0.000000}%
\pgfsetstrokecolor{textcolor}%
\pgfsetfillcolor{textcolor}%
\pgftext[x=0.817873in, y=3.401045in, left, base]{\color{textcolor}\rmfamily\fontsize{10.000000}{12.000000}\selectfont \(\displaystyle {10^{2}}\)}%
\end{pgfscope}%
\begin{pgfscope}%
\pgfpathrectangle{\pgfqpoint{1.116292in}{0.549691in}}{\pgfqpoint{5.298772in}{4.007027in}}%
\pgfusepath{clip}%
\pgfsetrectcap%
\pgfsetroundjoin%
\pgfsetlinewidth{0.803000pt}%
\definecolor{currentstroke}{rgb}{0.690196,0.690196,0.690196}%
\pgfsetstrokecolor{currentstroke}%
\pgfsetdash{}{0pt}%
\pgfpathmoveto{\pgfqpoint{1.116292in}{3.988187in}}%
\pgfpathlineto{\pgfqpoint{6.415064in}{3.988187in}}%
\pgfusepath{stroke}%
\end{pgfscope}%
\begin{pgfscope}%
\pgfsetbuttcap%
\pgfsetroundjoin%
\definecolor{currentfill}{rgb}{0.000000,0.000000,0.000000}%
\pgfsetfillcolor{currentfill}%
\pgfsetlinewidth{0.803000pt}%
\definecolor{currentstroke}{rgb}{0.000000,0.000000,0.000000}%
\pgfsetstrokecolor{currentstroke}%
\pgfsetdash{}{0pt}%
\pgfsys@defobject{currentmarker}{\pgfqpoint{-0.048611in}{0.000000in}}{\pgfqpoint{-0.000000in}{0.000000in}}{%
\pgfpathmoveto{\pgfqpoint{-0.000000in}{0.000000in}}%
\pgfpathlineto{\pgfqpoint{-0.048611in}{0.000000in}}%
\pgfusepath{stroke,fill}%
}%
\begin{pgfscope}%
\pgfsys@transformshift{1.116292in}{3.988187in}%
\pgfsys@useobject{currentmarker}{}%
\end{pgfscope}%
\end{pgfscope}%
\begin{pgfscope}%
\definecolor{textcolor}{rgb}{0.000000,0.000000,0.000000}%
\pgfsetstrokecolor{textcolor}%
\pgfsetfillcolor{textcolor}%
\pgftext[x=0.817873in, y=3.939962in, left, base]{\color{textcolor}\rmfamily\fontsize{10.000000}{12.000000}\selectfont \(\displaystyle {10^{3}}\)}%
\end{pgfscope}%
\begin{pgfscope}%
\pgfpathrectangle{\pgfqpoint{1.116292in}{0.549691in}}{\pgfqpoint{5.298772in}{4.007027in}}%
\pgfusepath{clip}%
\pgfsetrectcap%
\pgfsetroundjoin%
\pgfsetlinewidth{0.803000pt}%
\definecolor{currentstroke}{rgb}{0.690196,0.690196,0.690196}%
\pgfsetstrokecolor{currentstroke}%
\pgfsetdash{}{0pt}%
\pgfpathmoveto{\pgfqpoint{1.116292in}{4.527103in}}%
\pgfpathlineto{\pgfqpoint{6.415064in}{4.527103in}}%
\pgfusepath{stroke}%
\end{pgfscope}%
\begin{pgfscope}%
\pgfsetbuttcap%
\pgfsetroundjoin%
\definecolor{currentfill}{rgb}{0.000000,0.000000,0.000000}%
\pgfsetfillcolor{currentfill}%
\pgfsetlinewidth{0.803000pt}%
\definecolor{currentstroke}{rgb}{0.000000,0.000000,0.000000}%
\pgfsetstrokecolor{currentstroke}%
\pgfsetdash{}{0pt}%
\pgfsys@defobject{currentmarker}{\pgfqpoint{-0.048611in}{0.000000in}}{\pgfqpoint{-0.000000in}{0.000000in}}{%
\pgfpathmoveto{\pgfqpoint{-0.000000in}{0.000000in}}%
\pgfpathlineto{\pgfqpoint{-0.048611in}{0.000000in}}%
\pgfusepath{stroke,fill}%
}%
\begin{pgfscope}%
\pgfsys@transformshift{1.116292in}{4.527103in}%
\pgfsys@useobject{currentmarker}{}%
\end{pgfscope}%
\end{pgfscope}%
\begin{pgfscope}%
\definecolor{textcolor}{rgb}{0.000000,0.000000,0.000000}%
\pgfsetstrokecolor{textcolor}%
\pgfsetfillcolor{textcolor}%
\pgftext[x=0.817873in, y=4.478878in, left, base]{\color{textcolor}\rmfamily\fontsize{10.000000}{12.000000}\selectfont \(\displaystyle {10^{4}}\)}%
\end{pgfscope}%
\begin{pgfscope}%
\pgfsetbuttcap%
\pgfsetroundjoin%
\definecolor{currentfill}{rgb}{0.000000,0.000000,0.000000}%
\pgfsetfillcolor{currentfill}%
\pgfsetlinewidth{0.602250pt}%
\definecolor{currentstroke}{rgb}{0.000000,0.000000,0.000000}%
\pgfsetstrokecolor{currentstroke}%
\pgfsetdash{}{0pt}%
\pgfsys@defobject{currentmarker}{\pgfqpoint{-0.027778in}{0.000000in}}{\pgfqpoint{-0.000000in}{0.000000in}}{%
\pgfpathmoveto{\pgfqpoint{-0.000000in}{0.000000in}}%
\pgfpathlineto{\pgfqpoint{-0.027778in}{0.000000in}}%
\pgfusepath{stroke,fill}%
}%
\begin{pgfscope}%
\pgfsys@transformshift{1.116292in}{0.592459in}%
\pgfsys@useobject{currentmarker}{}%
\end{pgfscope}%
\end{pgfscope}%
\begin{pgfscope}%
\pgfsetbuttcap%
\pgfsetroundjoin%
\definecolor{currentfill}{rgb}{0.000000,0.000000,0.000000}%
\pgfsetfillcolor{currentfill}%
\pgfsetlinewidth{0.602250pt}%
\definecolor{currentstroke}{rgb}{0.000000,0.000000,0.000000}%
\pgfsetstrokecolor{currentstroke}%
\pgfsetdash{}{0pt}%
\pgfsys@defobject{currentmarker}{\pgfqpoint{-0.027778in}{0.000000in}}{\pgfqpoint{-0.000000in}{0.000000in}}{%
\pgfpathmoveto{\pgfqpoint{-0.000000in}{0.000000in}}%
\pgfpathlineto{\pgfqpoint{-0.027778in}{0.000000in}}%
\pgfusepath{stroke,fill}%
}%
\begin{pgfscope}%
\pgfsys@transformshift{1.116292in}{0.635131in}%
\pgfsys@useobject{currentmarker}{}%
\end{pgfscope}%
\end{pgfscope}%
\begin{pgfscope}%
\pgfsetbuttcap%
\pgfsetroundjoin%
\definecolor{currentfill}{rgb}{0.000000,0.000000,0.000000}%
\pgfsetfillcolor{currentfill}%
\pgfsetlinewidth{0.602250pt}%
\definecolor{currentstroke}{rgb}{0.000000,0.000000,0.000000}%
\pgfsetstrokecolor{currentstroke}%
\pgfsetdash{}{0pt}%
\pgfsys@defobject{currentmarker}{\pgfqpoint{-0.027778in}{0.000000in}}{\pgfqpoint{-0.000000in}{0.000000in}}{%
\pgfpathmoveto{\pgfqpoint{-0.000000in}{0.000000in}}%
\pgfpathlineto{\pgfqpoint{-0.027778in}{0.000000in}}%
\pgfusepath{stroke,fill}%
}%
\begin{pgfscope}%
\pgfsys@transformshift{1.116292in}{0.671210in}%
\pgfsys@useobject{currentmarker}{}%
\end{pgfscope}%
\end{pgfscope}%
\begin{pgfscope}%
\pgfsetbuttcap%
\pgfsetroundjoin%
\definecolor{currentfill}{rgb}{0.000000,0.000000,0.000000}%
\pgfsetfillcolor{currentfill}%
\pgfsetlinewidth{0.602250pt}%
\definecolor{currentstroke}{rgb}{0.000000,0.000000,0.000000}%
\pgfsetstrokecolor{currentstroke}%
\pgfsetdash{}{0pt}%
\pgfsys@defobject{currentmarker}{\pgfqpoint{-0.027778in}{0.000000in}}{\pgfqpoint{-0.000000in}{0.000000in}}{%
\pgfpathmoveto{\pgfqpoint{-0.000000in}{0.000000in}}%
\pgfpathlineto{\pgfqpoint{-0.027778in}{0.000000in}}%
\pgfusepath{stroke,fill}%
}%
\begin{pgfscope}%
\pgfsys@transformshift{1.116292in}{0.702463in}%
\pgfsys@useobject{currentmarker}{}%
\end{pgfscope}%
\end{pgfscope}%
\begin{pgfscope}%
\pgfsetbuttcap%
\pgfsetroundjoin%
\definecolor{currentfill}{rgb}{0.000000,0.000000,0.000000}%
\pgfsetfillcolor{currentfill}%
\pgfsetlinewidth{0.602250pt}%
\definecolor{currentstroke}{rgb}{0.000000,0.000000,0.000000}%
\pgfsetstrokecolor{currentstroke}%
\pgfsetdash{}{0pt}%
\pgfsys@defobject{currentmarker}{\pgfqpoint{-0.027778in}{0.000000in}}{\pgfqpoint{-0.000000in}{0.000000in}}{%
\pgfpathmoveto{\pgfqpoint{-0.000000in}{0.000000in}}%
\pgfpathlineto{\pgfqpoint{-0.027778in}{0.000000in}}%
\pgfusepath{stroke,fill}%
}%
\begin{pgfscope}%
\pgfsys@transformshift{1.116292in}{0.730030in}%
\pgfsys@useobject{currentmarker}{}%
\end{pgfscope}%
\end{pgfscope}%
\begin{pgfscope}%
\pgfsetbuttcap%
\pgfsetroundjoin%
\definecolor{currentfill}{rgb}{0.000000,0.000000,0.000000}%
\pgfsetfillcolor{currentfill}%
\pgfsetlinewidth{0.602250pt}%
\definecolor{currentstroke}{rgb}{0.000000,0.000000,0.000000}%
\pgfsetstrokecolor{currentstroke}%
\pgfsetdash{}{0pt}%
\pgfsys@defobject{currentmarker}{\pgfqpoint{-0.027778in}{0.000000in}}{\pgfqpoint{-0.000000in}{0.000000in}}{%
\pgfpathmoveto{\pgfqpoint{-0.000000in}{0.000000in}}%
\pgfpathlineto{\pgfqpoint{-0.027778in}{0.000000in}}%
\pgfusepath{stroke,fill}%
}%
\begin{pgfscope}%
\pgfsys@transformshift{1.116292in}{0.916919in}%
\pgfsys@useobject{currentmarker}{}%
\end{pgfscope}%
\end{pgfscope}%
\begin{pgfscope}%
\pgfsetbuttcap%
\pgfsetroundjoin%
\definecolor{currentfill}{rgb}{0.000000,0.000000,0.000000}%
\pgfsetfillcolor{currentfill}%
\pgfsetlinewidth{0.602250pt}%
\definecolor{currentstroke}{rgb}{0.000000,0.000000,0.000000}%
\pgfsetstrokecolor{currentstroke}%
\pgfsetdash{}{0pt}%
\pgfsys@defobject{currentmarker}{\pgfqpoint{-0.027778in}{0.000000in}}{\pgfqpoint{-0.000000in}{0.000000in}}{%
\pgfpathmoveto{\pgfqpoint{-0.000000in}{0.000000in}}%
\pgfpathlineto{\pgfqpoint{-0.027778in}{0.000000in}}%
\pgfusepath{stroke,fill}%
}%
\begin{pgfscope}%
\pgfsys@transformshift{1.116292in}{1.011817in}%
\pgfsys@useobject{currentmarker}{}%
\end{pgfscope}%
\end{pgfscope}%
\begin{pgfscope}%
\pgfsetbuttcap%
\pgfsetroundjoin%
\definecolor{currentfill}{rgb}{0.000000,0.000000,0.000000}%
\pgfsetfillcolor{currentfill}%
\pgfsetlinewidth{0.602250pt}%
\definecolor{currentstroke}{rgb}{0.000000,0.000000,0.000000}%
\pgfsetstrokecolor{currentstroke}%
\pgfsetdash{}{0pt}%
\pgfsys@defobject{currentmarker}{\pgfqpoint{-0.027778in}{0.000000in}}{\pgfqpoint{-0.000000in}{0.000000in}}{%
\pgfpathmoveto{\pgfqpoint{-0.000000in}{0.000000in}}%
\pgfpathlineto{\pgfqpoint{-0.027778in}{0.000000in}}%
\pgfusepath{stroke,fill}%
}%
\begin{pgfscope}%
\pgfsys@transformshift{1.116292in}{1.079149in}%
\pgfsys@useobject{currentmarker}{}%
\end{pgfscope}%
\end{pgfscope}%
\begin{pgfscope}%
\pgfsetbuttcap%
\pgfsetroundjoin%
\definecolor{currentfill}{rgb}{0.000000,0.000000,0.000000}%
\pgfsetfillcolor{currentfill}%
\pgfsetlinewidth{0.602250pt}%
\definecolor{currentstroke}{rgb}{0.000000,0.000000,0.000000}%
\pgfsetstrokecolor{currentstroke}%
\pgfsetdash{}{0pt}%
\pgfsys@defobject{currentmarker}{\pgfqpoint{-0.027778in}{0.000000in}}{\pgfqpoint{-0.000000in}{0.000000in}}{%
\pgfpathmoveto{\pgfqpoint{-0.000000in}{0.000000in}}%
\pgfpathlineto{\pgfqpoint{-0.027778in}{0.000000in}}%
\pgfusepath{stroke,fill}%
}%
\begin{pgfscope}%
\pgfsys@transformshift{1.116292in}{1.131375in}%
\pgfsys@useobject{currentmarker}{}%
\end{pgfscope}%
\end{pgfscope}%
\begin{pgfscope}%
\pgfsetbuttcap%
\pgfsetroundjoin%
\definecolor{currentfill}{rgb}{0.000000,0.000000,0.000000}%
\pgfsetfillcolor{currentfill}%
\pgfsetlinewidth{0.602250pt}%
\definecolor{currentstroke}{rgb}{0.000000,0.000000,0.000000}%
\pgfsetstrokecolor{currentstroke}%
\pgfsetdash{}{0pt}%
\pgfsys@defobject{currentmarker}{\pgfqpoint{-0.027778in}{0.000000in}}{\pgfqpoint{-0.000000in}{0.000000in}}{%
\pgfpathmoveto{\pgfqpoint{-0.000000in}{0.000000in}}%
\pgfpathlineto{\pgfqpoint{-0.027778in}{0.000000in}}%
\pgfusepath{stroke,fill}%
}%
\begin{pgfscope}%
\pgfsys@transformshift{1.116292in}{1.174047in}%
\pgfsys@useobject{currentmarker}{}%
\end{pgfscope}%
\end{pgfscope}%
\begin{pgfscope}%
\pgfsetbuttcap%
\pgfsetroundjoin%
\definecolor{currentfill}{rgb}{0.000000,0.000000,0.000000}%
\pgfsetfillcolor{currentfill}%
\pgfsetlinewidth{0.602250pt}%
\definecolor{currentstroke}{rgb}{0.000000,0.000000,0.000000}%
\pgfsetstrokecolor{currentstroke}%
\pgfsetdash{}{0pt}%
\pgfsys@defobject{currentmarker}{\pgfqpoint{-0.027778in}{0.000000in}}{\pgfqpoint{-0.000000in}{0.000000in}}{%
\pgfpathmoveto{\pgfqpoint{-0.000000in}{0.000000in}}%
\pgfpathlineto{\pgfqpoint{-0.027778in}{0.000000in}}%
\pgfusepath{stroke,fill}%
}%
\begin{pgfscope}%
\pgfsys@transformshift{1.116292in}{1.210126in}%
\pgfsys@useobject{currentmarker}{}%
\end{pgfscope}%
\end{pgfscope}%
\begin{pgfscope}%
\pgfsetbuttcap%
\pgfsetroundjoin%
\definecolor{currentfill}{rgb}{0.000000,0.000000,0.000000}%
\pgfsetfillcolor{currentfill}%
\pgfsetlinewidth{0.602250pt}%
\definecolor{currentstroke}{rgb}{0.000000,0.000000,0.000000}%
\pgfsetstrokecolor{currentstroke}%
\pgfsetdash{}{0pt}%
\pgfsys@defobject{currentmarker}{\pgfqpoint{-0.027778in}{0.000000in}}{\pgfqpoint{-0.000000in}{0.000000in}}{%
\pgfpathmoveto{\pgfqpoint{-0.000000in}{0.000000in}}%
\pgfpathlineto{\pgfqpoint{-0.027778in}{0.000000in}}%
\pgfusepath{stroke,fill}%
}%
\begin{pgfscope}%
\pgfsys@transformshift{1.116292in}{1.241379in}%
\pgfsys@useobject{currentmarker}{}%
\end{pgfscope}%
\end{pgfscope}%
\begin{pgfscope}%
\pgfsetbuttcap%
\pgfsetroundjoin%
\definecolor{currentfill}{rgb}{0.000000,0.000000,0.000000}%
\pgfsetfillcolor{currentfill}%
\pgfsetlinewidth{0.602250pt}%
\definecolor{currentstroke}{rgb}{0.000000,0.000000,0.000000}%
\pgfsetstrokecolor{currentstroke}%
\pgfsetdash{}{0pt}%
\pgfsys@defobject{currentmarker}{\pgfqpoint{-0.027778in}{0.000000in}}{\pgfqpoint{-0.000000in}{0.000000in}}{%
\pgfpathmoveto{\pgfqpoint{-0.000000in}{0.000000in}}%
\pgfpathlineto{\pgfqpoint{-0.027778in}{0.000000in}}%
\pgfusepath{stroke,fill}%
}%
\begin{pgfscope}%
\pgfsys@transformshift{1.116292in}{1.268946in}%
\pgfsys@useobject{currentmarker}{}%
\end{pgfscope}%
\end{pgfscope}%
\begin{pgfscope}%
\pgfsetbuttcap%
\pgfsetroundjoin%
\definecolor{currentfill}{rgb}{0.000000,0.000000,0.000000}%
\pgfsetfillcolor{currentfill}%
\pgfsetlinewidth{0.602250pt}%
\definecolor{currentstroke}{rgb}{0.000000,0.000000,0.000000}%
\pgfsetstrokecolor{currentstroke}%
\pgfsetdash{}{0pt}%
\pgfsys@defobject{currentmarker}{\pgfqpoint{-0.027778in}{0.000000in}}{\pgfqpoint{-0.000000in}{0.000000in}}{%
\pgfpathmoveto{\pgfqpoint{-0.000000in}{0.000000in}}%
\pgfpathlineto{\pgfqpoint{-0.027778in}{0.000000in}}%
\pgfusepath{stroke,fill}%
}%
\begin{pgfscope}%
\pgfsys@transformshift{1.116292in}{1.455835in}%
\pgfsys@useobject{currentmarker}{}%
\end{pgfscope}%
\end{pgfscope}%
\begin{pgfscope}%
\pgfsetbuttcap%
\pgfsetroundjoin%
\definecolor{currentfill}{rgb}{0.000000,0.000000,0.000000}%
\pgfsetfillcolor{currentfill}%
\pgfsetlinewidth{0.602250pt}%
\definecolor{currentstroke}{rgb}{0.000000,0.000000,0.000000}%
\pgfsetstrokecolor{currentstroke}%
\pgfsetdash{}{0pt}%
\pgfsys@defobject{currentmarker}{\pgfqpoint{-0.027778in}{0.000000in}}{\pgfqpoint{-0.000000in}{0.000000in}}{%
\pgfpathmoveto{\pgfqpoint{-0.000000in}{0.000000in}}%
\pgfpathlineto{\pgfqpoint{-0.027778in}{0.000000in}}%
\pgfusepath{stroke,fill}%
}%
\begin{pgfscope}%
\pgfsys@transformshift{1.116292in}{1.550734in}%
\pgfsys@useobject{currentmarker}{}%
\end{pgfscope}%
\end{pgfscope}%
\begin{pgfscope}%
\pgfsetbuttcap%
\pgfsetroundjoin%
\definecolor{currentfill}{rgb}{0.000000,0.000000,0.000000}%
\pgfsetfillcolor{currentfill}%
\pgfsetlinewidth{0.602250pt}%
\definecolor{currentstroke}{rgb}{0.000000,0.000000,0.000000}%
\pgfsetstrokecolor{currentstroke}%
\pgfsetdash{}{0pt}%
\pgfsys@defobject{currentmarker}{\pgfqpoint{-0.027778in}{0.000000in}}{\pgfqpoint{-0.000000in}{0.000000in}}{%
\pgfpathmoveto{\pgfqpoint{-0.000000in}{0.000000in}}%
\pgfpathlineto{\pgfqpoint{-0.027778in}{0.000000in}}%
\pgfusepath{stroke,fill}%
}%
\begin{pgfscope}%
\pgfsys@transformshift{1.116292in}{1.618065in}%
\pgfsys@useobject{currentmarker}{}%
\end{pgfscope}%
\end{pgfscope}%
\begin{pgfscope}%
\pgfsetbuttcap%
\pgfsetroundjoin%
\definecolor{currentfill}{rgb}{0.000000,0.000000,0.000000}%
\pgfsetfillcolor{currentfill}%
\pgfsetlinewidth{0.602250pt}%
\definecolor{currentstroke}{rgb}{0.000000,0.000000,0.000000}%
\pgfsetstrokecolor{currentstroke}%
\pgfsetdash{}{0pt}%
\pgfsys@defobject{currentmarker}{\pgfqpoint{-0.027778in}{0.000000in}}{\pgfqpoint{-0.000000in}{0.000000in}}{%
\pgfpathmoveto{\pgfqpoint{-0.000000in}{0.000000in}}%
\pgfpathlineto{\pgfqpoint{-0.027778in}{0.000000in}}%
\pgfusepath{stroke,fill}%
}%
\begin{pgfscope}%
\pgfsys@transformshift{1.116292in}{1.670292in}%
\pgfsys@useobject{currentmarker}{}%
\end{pgfscope}%
\end{pgfscope}%
\begin{pgfscope}%
\pgfsetbuttcap%
\pgfsetroundjoin%
\definecolor{currentfill}{rgb}{0.000000,0.000000,0.000000}%
\pgfsetfillcolor{currentfill}%
\pgfsetlinewidth{0.602250pt}%
\definecolor{currentstroke}{rgb}{0.000000,0.000000,0.000000}%
\pgfsetstrokecolor{currentstroke}%
\pgfsetdash{}{0pt}%
\pgfsys@defobject{currentmarker}{\pgfqpoint{-0.027778in}{0.000000in}}{\pgfqpoint{-0.000000in}{0.000000in}}{%
\pgfpathmoveto{\pgfqpoint{-0.000000in}{0.000000in}}%
\pgfpathlineto{\pgfqpoint{-0.027778in}{0.000000in}}%
\pgfusepath{stroke,fill}%
}%
\begin{pgfscope}%
\pgfsys@transformshift{1.116292in}{1.712964in}%
\pgfsys@useobject{currentmarker}{}%
\end{pgfscope}%
\end{pgfscope}%
\begin{pgfscope}%
\pgfsetbuttcap%
\pgfsetroundjoin%
\definecolor{currentfill}{rgb}{0.000000,0.000000,0.000000}%
\pgfsetfillcolor{currentfill}%
\pgfsetlinewidth{0.602250pt}%
\definecolor{currentstroke}{rgb}{0.000000,0.000000,0.000000}%
\pgfsetstrokecolor{currentstroke}%
\pgfsetdash{}{0pt}%
\pgfsys@defobject{currentmarker}{\pgfqpoint{-0.027778in}{0.000000in}}{\pgfqpoint{-0.000000in}{0.000000in}}{%
\pgfpathmoveto{\pgfqpoint{-0.000000in}{0.000000in}}%
\pgfpathlineto{\pgfqpoint{-0.027778in}{0.000000in}}%
\pgfusepath{stroke,fill}%
}%
\begin{pgfscope}%
\pgfsys@transformshift{1.116292in}{1.749042in}%
\pgfsys@useobject{currentmarker}{}%
\end{pgfscope}%
\end{pgfscope}%
\begin{pgfscope}%
\pgfsetbuttcap%
\pgfsetroundjoin%
\definecolor{currentfill}{rgb}{0.000000,0.000000,0.000000}%
\pgfsetfillcolor{currentfill}%
\pgfsetlinewidth{0.602250pt}%
\definecolor{currentstroke}{rgb}{0.000000,0.000000,0.000000}%
\pgfsetstrokecolor{currentstroke}%
\pgfsetdash{}{0pt}%
\pgfsys@defobject{currentmarker}{\pgfqpoint{-0.027778in}{0.000000in}}{\pgfqpoint{-0.000000in}{0.000000in}}{%
\pgfpathmoveto{\pgfqpoint{-0.000000in}{0.000000in}}%
\pgfpathlineto{\pgfqpoint{-0.027778in}{0.000000in}}%
\pgfusepath{stroke,fill}%
}%
\begin{pgfscope}%
\pgfsys@transformshift{1.116292in}{1.780295in}%
\pgfsys@useobject{currentmarker}{}%
\end{pgfscope}%
\end{pgfscope}%
\begin{pgfscope}%
\pgfsetbuttcap%
\pgfsetroundjoin%
\definecolor{currentfill}{rgb}{0.000000,0.000000,0.000000}%
\pgfsetfillcolor{currentfill}%
\pgfsetlinewidth{0.602250pt}%
\definecolor{currentstroke}{rgb}{0.000000,0.000000,0.000000}%
\pgfsetstrokecolor{currentstroke}%
\pgfsetdash{}{0pt}%
\pgfsys@defobject{currentmarker}{\pgfqpoint{-0.027778in}{0.000000in}}{\pgfqpoint{-0.000000in}{0.000000in}}{%
\pgfpathmoveto{\pgfqpoint{-0.000000in}{0.000000in}}%
\pgfpathlineto{\pgfqpoint{-0.027778in}{0.000000in}}%
\pgfusepath{stroke,fill}%
}%
\begin{pgfscope}%
\pgfsys@transformshift{1.116292in}{1.807862in}%
\pgfsys@useobject{currentmarker}{}%
\end{pgfscope}%
\end{pgfscope}%
\begin{pgfscope}%
\pgfsetbuttcap%
\pgfsetroundjoin%
\definecolor{currentfill}{rgb}{0.000000,0.000000,0.000000}%
\pgfsetfillcolor{currentfill}%
\pgfsetlinewidth{0.602250pt}%
\definecolor{currentstroke}{rgb}{0.000000,0.000000,0.000000}%
\pgfsetstrokecolor{currentstroke}%
\pgfsetdash{}{0pt}%
\pgfsys@defobject{currentmarker}{\pgfqpoint{-0.027778in}{0.000000in}}{\pgfqpoint{-0.000000in}{0.000000in}}{%
\pgfpathmoveto{\pgfqpoint{-0.000000in}{0.000000in}}%
\pgfpathlineto{\pgfqpoint{-0.027778in}{0.000000in}}%
\pgfusepath{stroke,fill}%
}%
\begin{pgfscope}%
\pgfsys@transformshift{1.116292in}{1.994752in}%
\pgfsys@useobject{currentmarker}{}%
\end{pgfscope}%
\end{pgfscope}%
\begin{pgfscope}%
\pgfsetbuttcap%
\pgfsetroundjoin%
\definecolor{currentfill}{rgb}{0.000000,0.000000,0.000000}%
\pgfsetfillcolor{currentfill}%
\pgfsetlinewidth{0.602250pt}%
\definecolor{currentstroke}{rgb}{0.000000,0.000000,0.000000}%
\pgfsetstrokecolor{currentstroke}%
\pgfsetdash{}{0pt}%
\pgfsys@defobject{currentmarker}{\pgfqpoint{-0.027778in}{0.000000in}}{\pgfqpoint{-0.000000in}{0.000000in}}{%
\pgfpathmoveto{\pgfqpoint{-0.000000in}{0.000000in}}%
\pgfpathlineto{\pgfqpoint{-0.027778in}{0.000000in}}%
\pgfusepath{stroke,fill}%
}%
\begin{pgfscope}%
\pgfsys@transformshift{1.116292in}{2.089650in}%
\pgfsys@useobject{currentmarker}{}%
\end{pgfscope}%
\end{pgfscope}%
\begin{pgfscope}%
\pgfsetbuttcap%
\pgfsetroundjoin%
\definecolor{currentfill}{rgb}{0.000000,0.000000,0.000000}%
\pgfsetfillcolor{currentfill}%
\pgfsetlinewidth{0.602250pt}%
\definecolor{currentstroke}{rgb}{0.000000,0.000000,0.000000}%
\pgfsetstrokecolor{currentstroke}%
\pgfsetdash{}{0pt}%
\pgfsys@defobject{currentmarker}{\pgfqpoint{-0.027778in}{0.000000in}}{\pgfqpoint{-0.000000in}{0.000000in}}{%
\pgfpathmoveto{\pgfqpoint{-0.000000in}{0.000000in}}%
\pgfpathlineto{\pgfqpoint{-0.027778in}{0.000000in}}%
\pgfusepath{stroke,fill}%
}%
\begin{pgfscope}%
\pgfsys@transformshift{1.116292in}{2.156982in}%
\pgfsys@useobject{currentmarker}{}%
\end{pgfscope}%
\end{pgfscope}%
\begin{pgfscope}%
\pgfsetbuttcap%
\pgfsetroundjoin%
\definecolor{currentfill}{rgb}{0.000000,0.000000,0.000000}%
\pgfsetfillcolor{currentfill}%
\pgfsetlinewidth{0.602250pt}%
\definecolor{currentstroke}{rgb}{0.000000,0.000000,0.000000}%
\pgfsetstrokecolor{currentstroke}%
\pgfsetdash{}{0pt}%
\pgfsys@defobject{currentmarker}{\pgfqpoint{-0.027778in}{0.000000in}}{\pgfqpoint{-0.000000in}{0.000000in}}{%
\pgfpathmoveto{\pgfqpoint{-0.000000in}{0.000000in}}%
\pgfpathlineto{\pgfqpoint{-0.027778in}{0.000000in}}%
\pgfusepath{stroke,fill}%
}%
\begin{pgfscope}%
\pgfsys@transformshift{1.116292in}{2.209208in}%
\pgfsys@useobject{currentmarker}{}%
\end{pgfscope}%
\end{pgfscope}%
\begin{pgfscope}%
\pgfsetbuttcap%
\pgfsetroundjoin%
\definecolor{currentfill}{rgb}{0.000000,0.000000,0.000000}%
\pgfsetfillcolor{currentfill}%
\pgfsetlinewidth{0.602250pt}%
\definecolor{currentstroke}{rgb}{0.000000,0.000000,0.000000}%
\pgfsetstrokecolor{currentstroke}%
\pgfsetdash{}{0pt}%
\pgfsys@defobject{currentmarker}{\pgfqpoint{-0.027778in}{0.000000in}}{\pgfqpoint{-0.000000in}{0.000000in}}{%
\pgfpathmoveto{\pgfqpoint{-0.000000in}{0.000000in}}%
\pgfpathlineto{\pgfqpoint{-0.027778in}{0.000000in}}%
\pgfusepath{stroke,fill}%
}%
\begin{pgfscope}%
\pgfsys@transformshift{1.116292in}{2.251880in}%
\pgfsys@useobject{currentmarker}{}%
\end{pgfscope}%
\end{pgfscope}%
\begin{pgfscope}%
\pgfsetbuttcap%
\pgfsetroundjoin%
\definecolor{currentfill}{rgb}{0.000000,0.000000,0.000000}%
\pgfsetfillcolor{currentfill}%
\pgfsetlinewidth{0.602250pt}%
\definecolor{currentstroke}{rgb}{0.000000,0.000000,0.000000}%
\pgfsetstrokecolor{currentstroke}%
\pgfsetdash{}{0pt}%
\pgfsys@defobject{currentmarker}{\pgfqpoint{-0.027778in}{0.000000in}}{\pgfqpoint{-0.000000in}{0.000000in}}{%
\pgfpathmoveto{\pgfqpoint{-0.000000in}{0.000000in}}%
\pgfpathlineto{\pgfqpoint{-0.027778in}{0.000000in}}%
\pgfusepath{stroke,fill}%
}%
\begin{pgfscope}%
\pgfsys@transformshift{1.116292in}{2.287959in}%
\pgfsys@useobject{currentmarker}{}%
\end{pgfscope}%
\end{pgfscope}%
\begin{pgfscope}%
\pgfsetbuttcap%
\pgfsetroundjoin%
\definecolor{currentfill}{rgb}{0.000000,0.000000,0.000000}%
\pgfsetfillcolor{currentfill}%
\pgfsetlinewidth{0.602250pt}%
\definecolor{currentstroke}{rgb}{0.000000,0.000000,0.000000}%
\pgfsetstrokecolor{currentstroke}%
\pgfsetdash{}{0pt}%
\pgfsys@defobject{currentmarker}{\pgfqpoint{-0.027778in}{0.000000in}}{\pgfqpoint{-0.000000in}{0.000000in}}{%
\pgfpathmoveto{\pgfqpoint{-0.000000in}{0.000000in}}%
\pgfpathlineto{\pgfqpoint{-0.027778in}{0.000000in}}%
\pgfusepath{stroke,fill}%
}%
\begin{pgfscope}%
\pgfsys@transformshift{1.116292in}{2.319212in}%
\pgfsys@useobject{currentmarker}{}%
\end{pgfscope}%
\end{pgfscope}%
\begin{pgfscope}%
\pgfsetbuttcap%
\pgfsetroundjoin%
\definecolor{currentfill}{rgb}{0.000000,0.000000,0.000000}%
\pgfsetfillcolor{currentfill}%
\pgfsetlinewidth{0.602250pt}%
\definecolor{currentstroke}{rgb}{0.000000,0.000000,0.000000}%
\pgfsetstrokecolor{currentstroke}%
\pgfsetdash{}{0pt}%
\pgfsys@defobject{currentmarker}{\pgfqpoint{-0.027778in}{0.000000in}}{\pgfqpoint{-0.000000in}{0.000000in}}{%
\pgfpathmoveto{\pgfqpoint{-0.000000in}{0.000000in}}%
\pgfpathlineto{\pgfqpoint{-0.027778in}{0.000000in}}%
\pgfusepath{stroke,fill}%
}%
\begin{pgfscope}%
\pgfsys@transformshift{1.116292in}{2.346779in}%
\pgfsys@useobject{currentmarker}{}%
\end{pgfscope}%
\end{pgfscope}%
\begin{pgfscope}%
\pgfsetbuttcap%
\pgfsetroundjoin%
\definecolor{currentfill}{rgb}{0.000000,0.000000,0.000000}%
\pgfsetfillcolor{currentfill}%
\pgfsetlinewidth{0.602250pt}%
\definecolor{currentstroke}{rgb}{0.000000,0.000000,0.000000}%
\pgfsetstrokecolor{currentstroke}%
\pgfsetdash{}{0pt}%
\pgfsys@defobject{currentmarker}{\pgfqpoint{-0.027778in}{0.000000in}}{\pgfqpoint{-0.000000in}{0.000000in}}{%
\pgfpathmoveto{\pgfqpoint{-0.000000in}{0.000000in}}%
\pgfpathlineto{\pgfqpoint{-0.027778in}{0.000000in}}%
\pgfusepath{stroke,fill}%
}%
\begin{pgfscope}%
\pgfsys@transformshift{1.116292in}{2.533668in}%
\pgfsys@useobject{currentmarker}{}%
\end{pgfscope}%
\end{pgfscope}%
\begin{pgfscope}%
\pgfsetbuttcap%
\pgfsetroundjoin%
\definecolor{currentfill}{rgb}{0.000000,0.000000,0.000000}%
\pgfsetfillcolor{currentfill}%
\pgfsetlinewidth{0.602250pt}%
\definecolor{currentstroke}{rgb}{0.000000,0.000000,0.000000}%
\pgfsetstrokecolor{currentstroke}%
\pgfsetdash{}{0pt}%
\pgfsys@defobject{currentmarker}{\pgfqpoint{-0.027778in}{0.000000in}}{\pgfqpoint{-0.000000in}{0.000000in}}{%
\pgfpathmoveto{\pgfqpoint{-0.000000in}{0.000000in}}%
\pgfpathlineto{\pgfqpoint{-0.027778in}{0.000000in}}%
\pgfusepath{stroke,fill}%
}%
\begin{pgfscope}%
\pgfsys@transformshift{1.116292in}{2.628566in}%
\pgfsys@useobject{currentmarker}{}%
\end{pgfscope}%
\end{pgfscope}%
\begin{pgfscope}%
\pgfsetbuttcap%
\pgfsetroundjoin%
\definecolor{currentfill}{rgb}{0.000000,0.000000,0.000000}%
\pgfsetfillcolor{currentfill}%
\pgfsetlinewidth{0.602250pt}%
\definecolor{currentstroke}{rgb}{0.000000,0.000000,0.000000}%
\pgfsetstrokecolor{currentstroke}%
\pgfsetdash{}{0pt}%
\pgfsys@defobject{currentmarker}{\pgfqpoint{-0.027778in}{0.000000in}}{\pgfqpoint{-0.000000in}{0.000000in}}{%
\pgfpathmoveto{\pgfqpoint{-0.000000in}{0.000000in}}%
\pgfpathlineto{\pgfqpoint{-0.027778in}{0.000000in}}%
\pgfusepath{stroke,fill}%
}%
\begin{pgfscope}%
\pgfsys@transformshift{1.116292in}{2.695898in}%
\pgfsys@useobject{currentmarker}{}%
\end{pgfscope}%
\end{pgfscope}%
\begin{pgfscope}%
\pgfsetbuttcap%
\pgfsetroundjoin%
\definecolor{currentfill}{rgb}{0.000000,0.000000,0.000000}%
\pgfsetfillcolor{currentfill}%
\pgfsetlinewidth{0.602250pt}%
\definecolor{currentstroke}{rgb}{0.000000,0.000000,0.000000}%
\pgfsetstrokecolor{currentstroke}%
\pgfsetdash{}{0pt}%
\pgfsys@defobject{currentmarker}{\pgfqpoint{-0.027778in}{0.000000in}}{\pgfqpoint{-0.000000in}{0.000000in}}{%
\pgfpathmoveto{\pgfqpoint{-0.000000in}{0.000000in}}%
\pgfpathlineto{\pgfqpoint{-0.027778in}{0.000000in}}%
\pgfusepath{stroke,fill}%
}%
\begin{pgfscope}%
\pgfsys@transformshift{1.116292in}{2.748124in}%
\pgfsys@useobject{currentmarker}{}%
\end{pgfscope}%
\end{pgfscope}%
\begin{pgfscope}%
\pgfsetbuttcap%
\pgfsetroundjoin%
\definecolor{currentfill}{rgb}{0.000000,0.000000,0.000000}%
\pgfsetfillcolor{currentfill}%
\pgfsetlinewidth{0.602250pt}%
\definecolor{currentstroke}{rgb}{0.000000,0.000000,0.000000}%
\pgfsetstrokecolor{currentstroke}%
\pgfsetdash{}{0pt}%
\pgfsys@defobject{currentmarker}{\pgfqpoint{-0.027778in}{0.000000in}}{\pgfqpoint{-0.000000in}{0.000000in}}{%
\pgfpathmoveto{\pgfqpoint{-0.000000in}{0.000000in}}%
\pgfpathlineto{\pgfqpoint{-0.027778in}{0.000000in}}%
\pgfusepath{stroke,fill}%
}%
\begin{pgfscope}%
\pgfsys@transformshift{1.116292in}{2.790796in}%
\pgfsys@useobject{currentmarker}{}%
\end{pgfscope}%
\end{pgfscope}%
\begin{pgfscope}%
\pgfsetbuttcap%
\pgfsetroundjoin%
\definecolor{currentfill}{rgb}{0.000000,0.000000,0.000000}%
\pgfsetfillcolor{currentfill}%
\pgfsetlinewidth{0.602250pt}%
\definecolor{currentstroke}{rgb}{0.000000,0.000000,0.000000}%
\pgfsetstrokecolor{currentstroke}%
\pgfsetdash{}{0pt}%
\pgfsys@defobject{currentmarker}{\pgfqpoint{-0.027778in}{0.000000in}}{\pgfqpoint{-0.000000in}{0.000000in}}{%
\pgfpathmoveto{\pgfqpoint{-0.000000in}{0.000000in}}%
\pgfpathlineto{\pgfqpoint{-0.027778in}{0.000000in}}%
\pgfusepath{stroke,fill}%
}%
\begin{pgfscope}%
\pgfsys@transformshift{1.116292in}{2.826875in}%
\pgfsys@useobject{currentmarker}{}%
\end{pgfscope}%
\end{pgfscope}%
\begin{pgfscope}%
\pgfsetbuttcap%
\pgfsetroundjoin%
\definecolor{currentfill}{rgb}{0.000000,0.000000,0.000000}%
\pgfsetfillcolor{currentfill}%
\pgfsetlinewidth{0.602250pt}%
\definecolor{currentstroke}{rgb}{0.000000,0.000000,0.000000}%
\pgfsetstrokecolor{currentstroke}%
\pgfsetdash{}{0pt}%
\pgfsys@defobject{currentmarker}{\pgfqpoint{-0.027778in}{0.000000in}}{\pgfqpoint{-0.000000in}{0.000000in}}{%
\pgfpathmoveto{\pgfqpoint{-0.000000in}{0.000000in}}%
\pgfpathlineto{\pgfqpoint{-0.027778in}{0.000000in}}%
\pgfusepath{stroke,fill}%
}%
\begin{pgfscope}%
\pgfsys@transformshift{1.116292in}{2.858128in}%
\pgfsys@useobject{currentmarker}{}%
\end{pgfscope}%
\end{pgfscope}%
\begin{pgfscope}%
\pgfsetbuttcap%
\pgfsetroundjoin%
\definecolor{currentfill}{rgb}{0.000000,0.000000,0.000000}%
\pgfsetfillcolor{currentfill}%
\pgfsetlinewidth{0.602250pt}%
\definecolor{currentstroke}{rgb}{0.000000,0.000000,0.000000}%
\pgfsetstrokecolor{currentstroke}%
\pgfsetdash{}{0pt}%
\pgfsys@defobject{currentmarker}{\pgfqpoint{-0.027778in}{0.000000in}}{\pgfqpoint{-0.000000in}{0.000000in}}{%
\pgfpathmoveto{\pgfqpoint{-0.000000in}{0.000000in}}%
\pgfpathlineto{\pgfqpoint{-0.027778in}{0.000000in}}%
\pgfusepath{stroke,fill}%
}%
\begin{pgfscope}%
\pgfsys@transformshift{1.116292in}{2.885695in}%
\pgfsys@useobject{currentmarker}{}%
\end{pgfscope}%
\end{pgfscope}%
\begin{pgfscope}%
\pgfsetbuttcap%
\pgfsetroundjoin%
\definecolor{currentfill}{rgb}{0.000000,0.000000,0.000000}%
\pgfsetfillcolor{currentfill}%
\pgfsetlinewidth{0.602250pt}%
\definecolor{currentstroke}{rgb}{0.000000,0.000000,0.000000}%
\pgfsetstrokecolor{currentstroke}%
\pgfsetdash{}{0pt}%
\pgfsys@defobject{currentmarker}{\pgfqpoint{-0.027778in}{0.000000in}}{\pgfqpoint{-0.000000in}{0.000000in}}{%
\pgfpathmoveto{\pgfqpoint{-0.000000in}{0.000000in}}%
\pgfpathlineto{\pgfqpoint{-0.027778in}{0.000000in}}%
\pgfusepath{stroke,fill}%
}%
\begin{pgfscope}%
\pgfsys@transformshift{1.116292in}{3.072584in}%
\pgfsys@useobject{currentmarker}{}%
\end{pgfscope}%
\end{pgfscope}%
\begin{pgfscope}%
\pgfsetbuttcap%
\pgfsetroundjoin%
\definecolor{currentfill}{rgb}{0.000000,0.000000,0.000000}%
\pgfsetfillcolor{currentfill}%
\pgfsetlinewidth{0.602250pt}%
\definecolor{currentstroke}{rgb}{0.000000,0.000000,0.000000}%
\pgfsetstrokecolor{currentstroke}%
\pgfsetdash{}{0pt}%
\pgfsys@defobject{currentmarker}{\pgfqpoint{-0.027778in}{0.000000in}}{\pgfqpoint{-0.000000in}{0.000000in}}{%
\pgfpathmoveto{\pgfqpoint{-0.000000in}{0.000000in}}%
\pgfpathlineto{\pgfqpoint{-0.027778in}{0.000000in}}%
\pgfusepath{stroke,fill}%
}%
\begin{pgfscope}%
\pgfsys@transformshift{1.116292in}{3.167483in}%
\pgfsys@useobject{currentmarker}{}%
\end{pgfscope}%
\end{pgfscope}%
\begin{pgfscope}%
\pgfsetbuttcap%
\pgfsetroundjoin%
\definecolor{currentfill}{rgb}{0.000000,0.000000,0.000000}%
\pgfsetfillcolor{currentfill}%
\pgfsetlinewidth{0.602250pt}%
\definecolor{currentstroke}{rgb}{0.000000,0.000000,0.000000}%
\pgfsetstrokecolor{currentstroke}%
\pgfsetdash{}{0pt}%
\pgfsys@defobject{currentmarker}{\pgfqpoint{-0.027778in}{0.000000in}}{\pgfqpoint{-0.000000in}{0.000000in}}{%
\pgfpathmoveto{\pgfqpoint{-0.000000in}{0.000000in}}%
\pgfpathlineto{\pgfqpoint{-0.027778in}{0.000000in}}%
\pgfusepath{stroke,fill}%
}%
\begin{pgfscope}%
\pgfsys@transformshift{1.116292in}{3.234814in}%
\pgfsys@useobject{currentmarker}{}%
\end{pgfscope}%
\end{pgfscope}%
\begin{pgfscope}%
\pgfsetbuttcap%
\pgfsetroundjoin%
\definecolor{currentfill}{rgb}{0.000000,0.000000,0.000000}%
\pgfsetfillcolor{currentfill}%
\pgfsetlinewidth{0.602250pt}%
\definecolor{currentstroke}{rgb}{0.000000,0.000000,0.000000}%
\pgfsetstrokecolor{currentstroke}%
\pgfsetdash{}{0pt}%
\pgfsys@defobject{currentmarker}{\pgfqpoint{-0.027778in}{0.000000in}}{\pgfqpoint{-0.000000in}{0.000000in}}{%
\pgfpathmoveto{\pgfqpoint{-0.000000in}{0.000000in}}%
\pgfpathlineto{\pgfqpoint{-0.027778in}{0.000000in}}%
\pgfusepath{stroke,fill}%
}%
\begin{pgfscope}%
\pgfsys@transformshift{1.116292in}{3.287041in}%
\pgfsys@useobject{currentmarker}{}%
\end{pgfscope}%
\end{pgfscope}%
\begin{pgfscope}%
\pgfsetbuttcap%
\pgfsetroundjoin%
\definecolor{currentfill}{rgb}{0.000000,0.000000,0.000000}%
\pgfsetfillcolor{currentfill}%
\pgfsetlinewidth{0.602250pt}%
\definecolor{currentstroke}{rgb}{0.000000,0.000000,0.000000}%
\pgfsetstrokecolor{currentstroke}%
\pgfsetdash{}{0pt}%
\pgfsys@defobject{currentmarker}{\pgfqpoint{-0.027778in}{0.000000in}}{\pgfqpoint{-0.000000in}{0.000000in}}{%
\pgfpathmoveto{\pgfqpoint{-0.000000in}{0.000000in}}%
\pgfpathlineto{\pgfqpoint{-0.027778in}{0.000000in}}%
\pgfusepath{stroke,fill}%
}%
\begin{pgfscope}%
\pgfsys@transformshift{1.116292in}{3.329713in}%
\pgfsys@useobject{currentmarker}{}%
\end{pgfscope}%
\end{pgfscope}%
\begin{pgfscope}%
\pgfsetbuttcap%
\pgfsetroundjoin%
\definecolor{currentfill}{rgb}{0.000000,0.000000,0.000000}%
\pgfsetfillcolor{currentfill}%
\pgfsetlinewidth{0.602250pt}%
\definecolor{currentstroke}{rgb}{0.000000,0.000000,0.000000}%
\pgfsetstrokecolor{currentstroke}%
\pgfsetdash{}{0pt}%
\pgfsys@defobject{currentmarker}{\pgfqpoint{-0.027778in}{0.000000in}}{\pgfqpoint{-0.000000in}{0.000000in}}{%
\pgfpathmoveto{\pgfqpoint{-0.000000in}{0.000000in}}%
\pgfpathlineto{\pgfqpoint{-0.027778in}{0.000000in}}%
\pgfusepath{stroke,fill}%
}%
\begin{pgfscope}%
\pgfsys@transformshift{1.116292in}{3.365791in}%
\pgfsys@useobject{currentmarker}{}%
\end{pgfscope}%
\end{pgfscope}%
\begin{pgfscope}%
\pgfsetbuttcap%
\pgfsetroundjoin%
\definecolor{currentfill}{rgb}{0.000000,0.000000,0.000000}%
\pgfsetfillcolor{currentfill}%
\pgfsetlinewidth{0.602250pt}%
\definecolor{currentstroke}{rgb}{0.000000,0.000000,0.000000}%
\pgfsetstrokecolor{currentstroke}%
\pgfsetdash{}{0pt}%
\pgfsys@defobject{currentmarker}{\pgfqpoint{-0.027778in}{0.000000in}}{\pgfqpoint{-0.000000in}{0.000000in}}{%
\pgfpathmoveto{\pgfqpoint{-0.000000in}{0.000000in}}%
\pgfpathlineto{\pgfqpoint{-0.027778in}{0.000000in}}%
\pgfusepath{stroke,fill}%
}%
\begin{pgfscope}%
\pgfsys@transformshift{1.116292in}{3.397044in}%
\pgfsys@useobject{currentmarker}{}%
\end{pgfscope}%
\end{pgfscope}%
\begin{pgfscope}%
\pgfsetbuttcap%
\pgfsetroundjoin%
\definecolor{currentfill}{rgb}{0.000000,0.000000,0.000000}%
\pgfsetfillcolor{currentfill}%
\pgfsetlinewidth{0.602250pt}%
\definecolor{currentstroke}{rgb}{0.000000,0.000000,0.000000}%
\pgfsetstrokecolor{currentstroke}%
\pgfsetdash{}{0pt}%
\pgfsys@defobject{currentmarker}{\pgfqpoint{-0.027778in}{0.000000in}}{\pgfqpoint{-0.000000in}{0.000000in}}{%
\pgfpathmoveto{\pgfqpoint{-0.000000in}{0.000000in}}%
\pgfpathlineto{\pgfqpoint{-0.027778in}{0.000000in}}%
\pgfusepath{stroke,fill}%
}%
\begin{pgfscope}%
\pgfsys@transformshift{1.116292in}{3.424611in}%
\pgfsys@useobject{currentmarker}{}%
\end{pgfscope}%
\end{pgfscope}%
\begin{pgfscope}%
\pgfsetbuttcap%
\pgfsetroundjoin%
\definecolor{currentfill}{rgb}{0.000000,0.000000,0.000000}%
\pgfsetfillcolor{currentfill}%
\pgfsetlinewidth{0.602250pt}%
\definecolor{currentstroke}{rgb}{0.000000,0.000000,0.000000}%
\pgfsetstrokecolor{currentstroke}%
\pgfsetdash{}{0pt}%
\pgfsys@defobject{currentmarker}{\pgfqpoint{-0.027778in}{0.000000in}}{\pgfqpoint{-0.000000in}{0.000000in}}{%
\pgfpathmoveto{\pgfqpoint{-0.000000in}{0.000000in}}%
\pgfpathlineto{\pgfqpoint{-0.027778in}{0.000000in}}%
\pgfusepath{stroke,fill}%
}%
\begin{pgfscope}%
\pgfsys@transformshift{1.116292in}{3.611501in}%
\pgfsys@useobject{currentmarker}{}%
\end{pgfscope}%
\end{pgfscope}%
\begin{pgfscope}%
\pgfsetbuttcap%
\pgfsetroundjoin%
\definecolor{currentfill}{rgb}{0.000000,0.000000,0.000000}%
\pgfsetfillcolor{currentfill}%
\pgfsetlinewidth{0.602250pt}%
\definecolor{currentstroke}{rgb}{0.000000,0.000000,0.000000}%
\pgfsetstrokecolor{currentstroke}%
\pgfsetdash{}{0pt}%
\pgfsys@defobject{currentmarker}{\pgfqpoint{-0.027778in}{0.000000in}}{\pgfqpoint{-0.000000in}{0.000000in}}{%
\pgfpathmoveto{\pgfqpoint{-0.000000in}{0.000000in}}%
\pgfpathlineto{\pgfqpoint{-0.027778in}{0.000000in}}%
\pgfusepath{stroke,fill}%
}%
\begin{pgfscope}%
\pgfsys@transformshift{1.116292in}{3.706399in}%
\pgfsys@useobject{currentmarker}{}%
\end{pgfscope}%
\end{pgfscope}%
\begin{pgfscope}%
\pgfsetbuttcap%
\pgfsetroundjoin%
\definecolor{currentfill}{rgb}{0.000000,0.000000,0.000000}%
\pgfsetfillcolor{currentfill}%
\pgfsetlinewidth{0.602250pt}%
\definecolor{currentstroke}{rgb}{0.000000,0.000000,0.000000}%
\pgfsetstrokecolor{currentstroke}%
\pgfsetdash{}{0pt}%
\pgfsys@defobject{currentmarker}{\pgfqpoint{-0.027778in}{0.000000in}}{\pgfqpoint{-0.000000in}{0.000000in}}{%
\pgfpathmoveto{\pgfqpoint{-0.000000in}{0.000000in}}%
\pgfpathlineto{\pgfqpoint{-0.027778in}{0.000000in}}%
\pgfusepath{stroke,fill}%
}%
\begin{pgfscope}%
\pgfsys@transformshift{1.116292in}{3.773731in}%
\pgfsys@useobject{currentmarker}{}%
\end{pgfscope}%
\end{pgfscope}%
\begin{pgfscope}%
\pgfsetbuttcap%
\pgfsetroundjoin%
\definecolor{currentfill}{rgb}{0.000000,0.000000,0.000000}%
\pgfsetfillcolor{currentfill}%
\pgfsetlinewidth{0.602250pt}%
\definecolor{currentstroke}{rgb}{0.000000,0.000000,0.000000}%
\pgfsetstrokecolor{currentstroke}%
\pgfsetdash{}{0pt}%
\pgfsys@defobject{currentmarker}{\pgfqpoint{-0.027778in}{0.000000in}}{\pgfqpoint{-0.000000in}{0.000000in}}{%
\pgfpathmoveto{\pgfqpoint{-0.000000in}{0.000000in}}%
\pgfpathlineto{\pgfqpoint{-0.027778in}{0.000000in}}%
\pgfusepath{stroke,fill}%
}%
\begin{pgfscope}%
\pgfsys@transformshift{1.116292in}{3.825957in}%
\pgfsys@useobject{currentmarker}{}%
\end{pgfscope}%
\end{pgfscope}%
\begin{pgfscope}%
\pgfsetbuttcap%
\pgfsetroundjoin%
\definecolor{currentfill}{rgb}{0.000000,0.000000,0.000000}%
\pgfsetfillcolor{currentfill}%
\pgfsetlinewidth{0.602250pt}%
\definecolor{currentstroke}{rgb}{0.000000,0.000000,0.000000}%
\pgfsetstrokecolor{currentstroke}%
\pgfsetdash{}{0pt}%
\pgfsys@defobject{currentmarker}{\pgfqpoint{-0.027778in}{0.000000in}}{\pgfqpoint{-0.000000in}{0.000000in}}{%
\pgfpathmoveto{\pgfqpoint{-0.000000in}{0.000000in}}%
\pgfpathlineto{\pgfqpoint{-0.027778in}{0.000000in}}%
\pgfusepath{stroke,fill}%
}%
\begin{pgfscope}%
\pgfsys@transformshift{1.116292in}{3.868629in}%
\pgfsys@useobject{currentmarker}{}%
\end{pgfscope}%
\end{pgfscope}%
\begin{pgfscope}%
\pgfsetbuttcap%
\pgfsetroundjoin%
\definecolor{currentfill}{rgb}{0.000000,0.000000,0.000000}%
\pgfsetfillcolor{currentfill}%
\pgfsetlinewidth{0.602250pt}%
\definecolor{currentstroke}{rgb}{0.000000,0.000000,0.000000}%
\pgfsetstrokecolor{currentstroke}%
\pgfsetdash{}{0pt}%
\pgfsys@defobject{currentmarker}{\pgfqpoint{-0.027778in}{0.000000in}}{\pgfqpoint{-0.000000in}{0.000000in}}{%
\pgfpathmoveto{\pgfqpoint{-0.000000in}{0.000000in}}%
\pgfpathlineto{\pgfqpoint{-0.027778in}{0.000000in}}%
\pgfusepath{stroke,fill}%
}%
\begin{pgfscope}%
\pgfsys@transformshift{1.116292in}{3.904708in}%
\pgfsys@useobject{currentmarker}{}%
\end{pgfscope}%
\end{pgfscope}%
\begin{pgfscope}%
\pgfsetbuttcap%
\pgfsetroundjoin%
\definecolor{currentfill}{rgb}{0.000000,0.000000,0.000000}%
\pgfsetfillcolor{currentfill}%
\pgfsetlinewidth{0.602250pt}%
\definecolor{currentstroke}{rgb}{0.000000,0.000000,0.000000}%
\pgfsetstrokecolor{currentstroke}%
\pgfsetdash{}{0pt}%
\pgfsys@defobject{currentmarker}{\pgfqpoint{-0.027778in}{0.000000in}}{\pgfqpoint{-0.000000in}{0.000000in}}{%
\pgfpathmoveto{\pgfqpoint{-0.000000in}{0.000000in}}%
\pgfpathlineto{\pgfqpoint{-0.027778in}{0.000000in}}%
\pgfusepath{stroke,fill}%
}%
\begin{pgfscope}%
\pgfsys@transformshift{1.116292in}{3.935961in}%
\pgfsys@useobject{currentmarker}{}%
\end{pgfscope}%
\end{pgfscope}%
\begin{pgfscope}%
\pgfsetbuttcap%
\pgfsetroundjoin%
\definecolor{currentfill}{rgb}{0.000000,0.000000,0.000000}%
\pgfsetfillcolor{currentfill}%
\pgfsetlinewidth{0.602250pt}%
\definecolor{currentstroke}{rgb}{0.000000,0.000000,0.000000}%
\pgfsetstrokecolor{currentstroke}%
\pgfsetdash{}{0pt}%
\pgfsys@defobject{currentmarker}{\pgfqpoint{-0.027778in}{0.000000in}}{\pgfqpoint{-0.000000in}{0.000000in}}{%
\pgfpathmoveto{\pgfqpoint{-0.000000in}{0.000000in}}%
\pgfpathlineto{\pgfqpoint{-0.027778in}{0.000000in}}%
\pgfusepath{stroke,fill}%
}%
\begin{pgfscope}%
\pgfsys@transformshift{1.116292in}{3.963528in}%
\pgfsys@useobject{currentmarker}{}%
\end{pgfscope}%
\end{pgfscope}%
\begin{pgfscope}%
\pgfsetbuttcap%
\pgfsetroundjoin%
\definecolor{currentfill}{rgb}{0.000000,0.000000,0.000000}%
\pgfsetfillcolor{currentfill}%
\pgfsetlinewidth{0.602250pt}%
\definecolor{currentstroke}{rgb}{0.000000,0.000000,0.000000}%
\pgfsetstrokecolor{currentstroke}%
\pgfsetdash{}{0pt}%
\pgfsys@defobject{currentmarker}{\pgfqpoint{-0.027778in}{0.000000in}}{\pgfqpoint{-0.000000in}{0.000000in}}{%
\pgfpathmoveto{\pgfqpoint{-0.000000in}{0.000000in}}%
\pgfpathlineto{\pgfqpoint{-0.027778in}{0.000000in}}%
\pgfusepath{stroke,fill}%
}%
\begin{pgfscope}%
\pgfsys@transformshift{1.116292in}{4.150417in}%
\pgfsys@useobject{currentmarker}{}%
\end{pgfscope}%
\end{pgfscope}%
\begin{pgfscope}%
\pgfsetbuttcap%
\pgfsetroundjoin%
\definecolor{currentfill}{rgb}{0.000000,0.000000,0.000000}%
\pgfsetfillcolor{currentfill}%
\pgfsetlinewidth{0.602250pt}%
\definecolor{currentstroke}{rgb}{0.000000,0.000000,0.000000}%
\pgfsetstrokecolor{currentstroke}%
\pgfsetdash{}{0pt}%
\pgfsys@defobject{currentmarker}{\pgfqpoint{-0.027778in}{0.000000in}}{\pgfqpoint{-0.000000in}{0.000000in}}{%
\pgfpathmoveto{\pgfqpoint{-0.000000in}{0.000000in}}%
\pgfpathlineto{\pgfqpoint{-0.027778in}{0.000000in}}%
\pgfusepath{stroke,fill}%
}%
\begin{pgfscope}%
\pgfsys@transformshift{1.116292in}{4.245315in}%
\pgfsys@useobject{currentmarker}{}%
\end{pgfscope}%
\end{pgfscope}%
\begin{pgfscope}%
\pgfsetbuttcap%
\pgfsetroundjoin%
\definecolor{currentfill}{rgb}{0.000000,0.000000,0.000000}%
\pgfsetfillcolor{currentfill}%
\pgfsetlinewidth{0.602250pt}%
\definecolor{currentstroke}{rgb}{0.000000,0.000000,0.000000}%
\pgfsetstrokecolor{currentstroke}%
\pgfsetdash{}{0pt}%
\pgfsys@defobject{currentmarker}{\pgfqpoint{-0.027778in}{0.000000in}}{\pgfqpoint{-0.000000in}{0.000000in}}{%
\pgfpathmoveto{\pgfqpoint{-0.000000in}{0.000000in}}%
\pgfpathlineto{\pgfqpoint{-0.027778in}{0.000000in}}%
\pgfusepath{stroke,fill}%
}%
\begin{pgfscope}%
\pgfsys@transformshift{1.116292in}{4.312647in}%
\pgfsys@useobject{currentmarker}{}%
\end{pgfscope}%
\end{pgfscope}%
\begin{pgfscope}%
\pgfsetbuttcap%
\pgfsetroundjoin%
\definecolor{currentfill}{rgb}{0.000000,0.000000,0.000000}%
\pgfsetfillcolor{currentfill}%
\pgfsetlinewidth{0.602250pt}%
\definecolor{currentstroke}{rgb}{0.000000,0.000000,0.000000}%
\pgfsetstrokecolor{currentstroke}%
\pgfsetdash{}{0pt}%
\pgfsys@defobject{currentmarker}{\pgfqpoint{-0.027778in}{0.000000in}}{\pgfqpoint{-0.000000in}{0.000000in}}{%
\pgfpathmoveto{\pgfqpoint{-0.000000in}{0.000000in}}%
\pgfpathlineto{\pgfqpoint{-0.027778in}{0.000000in}}%
\pgfusepath{stroke,fill}%
}%
\begin{pgfscope}%
\pgfsys@transformshift{1.116292in}{4.364873in}%
\pgfsys@useobject{currentmarker}{}%
\end{pgfscope}%
\end{pgfscope}%
\begin{pgfscope}%
\pgfsetbuttcap%
\pgfsetroundjoin%
\definecolor{currentfill}{rgb}{0.000000,0.000000,0.000000}%
\pgfsetfillcolor{currentfill}%
\pgfsetlinewidth{0.602250pt}%
\definecolor{currentstroke}{rgb}{0.000000,0.000000,0.000000}%
\pgfsetstrokecolor{currentstroke}%
\pgfsetdash{}{0pt}%
\pgfsys@defobject{currentmarker}{\pgfqpoint{-0.027778in}{0.000000in}}{\pgfqpoint{-0.000000in}{0.000000in}}{%
\pgfpathmoveto{\pgfqpoint{-0.000000in}{0.000000in}}%
\pgfpathlineto{\pgfqpoint{-0.027778in}{0.000000in}}%
\pgfusepath{stroke,fill}%
}%
\begin{pgfscope}%
\pgfsys@transformshift{1.116292in}{4.407545in}%
\pgfsys@useobject{currentmarker}{}%
\end{pgfscope}%
\end{pgfscope}%
\begin{pgfscope}%
\pgfsetbuttcap%
\pgfsetroundjoin%
\definecolor{currentfill}{rgb}{0.000000,0.000000,0.000000}%
\pgfsetfillcolor{currentfill}%
\pgfsetlinewidth{0.602250pt}%
\definecolor{currentstroke}{rgb}{0.000000,0.000000,0.000000}%
\pgfsetstrokecolor{currentstroke}%
\pgfsetdash{}{0pt}%
\pgfsys@defobject{currentmarker}{\pgfqpoint{-0.027778in}{0.000000in}}{\pgfqpoint{-0.000000in}{0.000000in}}{%
\pgfpathmoveto{\pgfqpoint{-0.000000in}{0.000000in}}%
\pgfpathlineto{\pgfqpoint{-0.027778in}{0.000000in}}%
\pgfusepath{stroke,fill}%
}%
\begin{pgfscope}%
\pgfsys@transformshift{1.116292in}{4.443624in}%
\pgfsys@useobject{currentmarker}{}%
\end{pgfscope}%
\end{pgfscope}%
\begin{pgfscope}%
\pgfsetbuttcap%
\pgfsetroundjoin%
\definecolor{currentfill}{rgb}{0.000000,0.000000,0.000000}%
\pgfsetfillcolor{currentfill}%
\pgfsetlinewidth{0.602250pt}%
\definecolor{currentstroke}{rgb}{0.000000,0.000000,0.000000}%
\pgfsetstrokecolor{currentstroke}%
\pgfsetdash{}{0pt}%
\pgfsys@defobject{currentmarker}{\pgfqpoint{-0.027778in}{0.000000in}}{\pgfqpoint{-0.000000in}{0.000000in}}{%
\pgfpathmoveto{\pgfqpoint{-0.000000in}{0.000000in}}%
\pgfpathlineto{\pgfqpoint{-0.027778in}{0.000000in}}%
\pgfusepath{stroke,fill}%
}%
\begin{pgfscope}%
\pgfsys@transformshift{1.116292in}{4.474877in}%
\pgfsys@useobject{currentmarker}{}%
\end{pgfscope}%
\end{pgfscope}%
\begin{pgfscope}%
\pgfsetbuttcap%
\pgfsetroundjoin%
\definecolor{currentfill}{rgb}{0.000000,0.000000,0.000000}%
\pgfsetfillcolor{currentfill}%
\pgfsetlinewidth{0.602250pt}%
\definecolor{currentstroke}{rgb}{0.000000,0.000000,0.000000}%
\pgfsetstrokecolor{currentstroke}%
\pgfsetdash{}{0pt}%
\pgfsys@defobject{currentmarker}{\pgfqpoint{-0.027778in}{0.000000in}}{\pgfqpoint{-0.000000in}{0.000000in}}{%
\pgfpathmoveto{\pgfqpoint{-0.000000in}{0.000000in}}%
\pgfpathlineto{\pgfqpoint{-0.027778in}{0.000000in}}%
\pgfusepath{stroke,fill}%
}%
\begin{pgfscope}%
\pgfsys@transformshift{1.116292in}{4.502444in}%
\pgfsys@useobject{currentmarker}{}%
\end{pgfscope}%
\end{pgfscope}%
\begin{pgfscope}%
\definecolor{textcolor}{rgb}{0.000000,0.000000,0.000000}%
\pgfsetstrokecolor{textcolor}%
\pgfsetfillcolor{textcolor}%
\pgftext[x=0.481067in,y=2.553204in,,bottom]{\color{textcolor}\rmfamily\fontsize{10.000000}{12.000000}\selectfont \(\displaystyle ||\mathbf{e}||_A, ||\mathbf{r}||_2\)}%
\end{pgfscope}%
\begin{pgfscope}%
\pgfpathrectangle{\pgfqpoint{1.116292in}{0.549691in}}{\pgfqpoint{5.298772in}{4.007027in}}%
\pgfusepath{clip}%
\pgfsetrectcap%
\pgfsetroundjoin%
\pgfsetlinewidth{1.505625pt}%
\definecolor{currentstroke}{rgb}{0.121569,0.466667,0.705882}%
\pgfsetstrokecolor{currentstroke}%
\pgfsetdash{}{0pt}%
\pgfpathmoveto{\pgfqpoint{1.116292in}{4.374580in}}%
\pgfpathlineto{\pgfqpoint{1.118397in}{4.072627in}}%
\pgfpathlineto{\pgfqpoint{1.123308in}{3.761372in}}%
\pgfpathlineto{\pgfqpoint{1.123541in}{3.766227in}}%
\pgfpathlineto{\pgfqpoint{1.124711in}{3.848921in}}%
\pgfpathlineto{\pgfqpoint{1.125178in}{3.827743in}}%
\pgfpathlineto{\pgfqpoint{1.125412in}{3.823760in}}%
\pgfpathlineto{\pgfqpoint{1.126114in}{3.769628in}}%
\pgfpathlineto{\pgfqpoint{1.126815in}{3.790069in}}%
\pgfpathlineto{\pgfqpoint{1.127049in}{3.806618in}}%
\pgfpathlineto{\pgfqpoint{1.127985in}{3.799110in}}%
\pgfpathlineto{\pgfqpoint{1.129388in}{3.858513in}}%
\pgfpathlineto{\pgfqpoint{1.129856in}{3.822767in}}%
\pgfpathlineto{\pgfqpoint{1.131727in}{3.681875in}}%
\pgfpathlineto{\pgfqpoint{1.132194in}{3.682923in}}%
\pgfpathlineto{\pgfqpoint{1.132428in}{3.690127in}}%
\pgfpathlineto{\pgfqpoint{1.132662in}{3.674805in}}%
\pgfpathlineto{\pgfqpoint{1.135001in}{3.497034in}}%
\pgfpathlineto{\pgfqpoint{1.135468in}{3.486239in}}%
\pgfpathlineto{\pgfqpoint{1.135936in}{3.498484in}}%
\pgfpathlineto{\pgfqpoint{1.136404in}{3.496243in}}%
\pgfpathlineto{\pgfqpoint{1.138275in}{3.604112in}}%
\pgfpathlineto{\pgfqpoint{1.138508in}{3.605204in}}%
\pgfpathlineto{\pgfqpoint{1.138742in}{3.605100in}}%
\pgfpathlineto{\pgfqpoint{1.139444in}{3.538176in}}%
\pgfpathlineto{\pgfqpoint{1.143186in}{3.330443in}}%
\pgfpathlineto{\pgfqpoint{1.143419in}{3.329271in}}%
\pgfpathlineto{\pgfqpoint{1.145524in}{3.427824in}}%
\pgfpathlineto{\pgfqpoint{1.145758in}{3.425140in}}%
\pgfpathlineto{\pgfqpoint{1.146460in}{3.448829in}}%
\pgfpathlineto{\pgfqpoint{1.146927in}{3.447947in}}%
\pgfpathlineto{\pgfqpoint{1.149500in}{3.347456in}}%
\pgfpathlineto{\pgfqpoint{1.150669in}{3.363713in}}%
\pgfpathlineto{\pgfqpoint{1.151605in}{3.354351in}}%
\pgfpathlineto{\pgfqpoint{1.152774in}{3.387233in}}%
\pgfpathlineto{\pgfqpoint{1.153008in}{3.384705in}}%
\pgfpathlineto{\pgfqpoint{1.153475in}{3.362956in}}%
\pgfpathlineto{\pgfqpoint{1.153943in}{3.383737in}}%
\pgfpathlineto{\pgfqpoint{1.154879in}{3.399315in}}%
\pgfpathlineto{\pgfqpoint{1.154411in}{3.381854in}}%
\pgfpathlineto{\pgfqpoint{1.155112in}{3.390326in}}%
\pgfpathlineto{\pgfqpoint{1.156048in}{3.361370in}}%
\pgfpathlineto{\pgfqpoint{1.156516in}{3.371832in}}%
\pgfpathlineto{\pgfqpoint{1.156749in}{3.387824in}}%
\pgfpathlineto{\pgfqpoint{1.157217in}{3.365737in}}%
\pgfpathlineto{\pgfqpoint{1.158620in}{3.272028in}}%
\pgfpathlineto{\pgfqpoint{1.159088in}{3.299921in}}%
\pgfpathlineto{\pgfqpoint{1.160023in}{3.342378in}}%
\pgfpathlineto{\pgfqpoint{1.160491in}{3.339046in}}%
\pgfpathlineto{\pgfqpoint{1.161661in}{3.237610in}}%
\pgfpathlineto{\pgfqpoint{1.162128in}{3.211517in}}%
\pgfpathlineto{\pgfqpoint{1.162596in}{3.238653in}}%
\pgfpathlineto{\pgfqpoint{1.162830in}{3.236535in}}%
\pgfpathlineto{\pgfqpoint{1.163531in}{3.283669in}}%
\pgfpathlineto{\pgfqpoint{1.163999in}{3.265310in}}%
\pgfpathlineto{\pgfqpoint{1.164467in}{3.241039in}}%
\pgfpathlineto{\pgfqpoint{1.164935in}{3.261844in}}%
\pgfpathlineto{\pgfqpoint{1.165402in}{3.314034in}}%
\pgfpathlineto{\pgfqpoint{1.166338in}{3.300756in}}%
\pgfpathlineto{\pgfqpoint{1.167273in}{3.323909in}}%
\pgfpathlineto{\pgfqpoint{1.167507in}{3.306742in}}%
\pgfpathlineto{\pgfqpoint{1.168910in}{3.248633in}}%
\pgfpathlineto{\pgfqpoint{1.169144in}{3.262252in}}%
\pgfpathlineto{\pgfqpoint{1.169612in}{3.247019in}}%
\pgfpathlineto{\pgfqpoint{1.171950in}{3.119227in}}%
\pgfpathlineto{\pgfqpoint{1.172652in}{3.085554in}}%
\pgfpathlineto{\pgfqpoint{1.173120in}{3.102583in}}%
\pgfpathlineto{\pgfqpoint{1.174523in}{3.171500in}}%
\pgfpathlineto{\pgfqpoint{1.174757in}{3.153843in}}%
\pgfpathlineto{\pgfqpoint{1.175692in}{3.124125in}}%
\pgfpathlineto{\pgfqpoint{1.175926in}{3.137661in}}%
\pgfpathlineto{\pgfqpoint{1.177095in}{3.095115in}}%
\pgfpathlineto{\pgfqpoint{1.178031in}{2.962869in}}%
\pgfpathlineto{\pgfqpoint{1.178732in}{2.986895in}}%
\pgfpathlineto{\pgfqpoint{1.181071in}{3.167831in}}%
\pgfpathlineto{\pgfqpoint{1.181539in}{3.165987in}}%
\pgfpathlineto{\pgfqpoint{1.182006in}{3.153710in}}%
\pgfpathlineto{\pgfqpoint{1.185748in}{2.939025in}}%
\pgfpathlineto{\pgfqpoint{1.186216in}{2.967070in}}%
\pgfpathlineto{\pgfqpoint{1.188320in}{3.133728in}}%
\pgfpathlineto{\pgfqpoint{1.188554in}{3.134299in}}%
\pgfpathlineto{\pgfqpoint{1.190893in}{3.004510in}}%
\pgfpathlineto{\pgfqpoint{1.191361in}{3.041552in}}%
\pgfpathlineto{\pgfqpoint{1.191828in}{3.078396in}}%
\pgfpathlineto{\pgfqpoint{1.192530in}{3.058501in}}%
\pgfpathlineto{\pgfqpoint{1.192998in}{3.052551in}}%
\pgfpathlineto{\pgfqpoint{1.195336in}{2.882645in}}%
\pgfpathlineto{\pgfqpoint{1.195804in}{2.884322in}}%
\pgfpathlineto{\pgfqpoint{1.196739in}{2.777806in}}%
\pgfpathlineto{\pgfqpoint{1.197441in}{2.825783in}}%
\pgfpathlineto{\pgfqpoint{1.199312in}{2.989190in}}%
\pgfpathlineto{\pgfqpoint{1.199546in}{2.971057in}}%
\pgfpathlineto{\pgfqpoint{1.200481in}{2.942733in}}%
\pgfpathlineto{\pgfqpoint{1.200715in}{2.961106in}}%
\pgfpathlineto{\pgfqpoint{1.202118in}{3.020818in}}%
\pgfpathlineto{\pgfqpoint{1.202586in}{2.998257in}}%
\pgfpathlineto{\pgfqpoint{1.204924in}{2.872723in}}%
\pgfpathlineto{\pgfqpoint{1.205158in}{2.870609in}}%
\pgfpathlineto{\pgfqpoint{1.205392in}{2.871787in}}%
\pgfpathlineto{\pgfqpoint{1.207029in}{2.941255in}}%
\pgfpathlineto{\pgfqpoint{1.209134in}{2.825089in}}%
\pgfpathlineto{\pgfqpoint{1.209836in}{2.795662in}}%
\pgfpathlineto{\pgfqpoint{1.210303in}{2.820741in}}%
\pgfpathlineto{\pgfqpoint{1.210771in}{2.810837in}}%
\pgfpathlineto{\pgfqpoint{1.211940in}{2.848452in}}%
\pgfpathlineto{\pgfqpoint{1.212174in}{2.841006in}}%
\pgfpathlineto{\pgfqpoint{1.212642in}{2.860530in}}%
\pgfpathlineto{\pgfqpoint{1.212876in}{2.859665in}}%
\pgfpathlineto{\pgfqpoint{1.213110in}{2.861839in}}%
\pgfpathlineto{\pgfqpoint{1.215448in}{2.981800in}}%
\pgfpathlineto{\pgfqpoint{1.215916in}{2.951894in}}%
\pgfpathlineto{\pgfqpoint{1.219658in}{2.719773in}}%
\pgfpathlineto{\pgfqpoint{1.220125in}{2.755281in}}%
\pgfpathlineto{\pgfqpoint{1.222464in}{2.912349in}}%
\pgfpathlineto{\pgfqpoint{1.222932in}{2.918239in}}%
\pgfpathlineto{\pgfqpoint{1.223166in}{2.933452in}}%
\pgfpathlineto{\pgfqpoint{1.223633in}{2.898643in}}%
\pgfpathlineto{\pgfqpoint{1.224335in}{2.908512in}}%
\pgfpathlineto{\pgfqpoint{1.225504in}{2.843833in}}%
\pgfpathlineto{\pgfqpoint{1.226907in}{2.773864in}}%
\pgfpathlineto{\pgfqpoint{1.228310in}{2.887651in}}%
\pgfpathlineto{\pgfqpoint{1.229480in}{2.872227in}}%
\pgfpathlineto{\pgfqpoint{1.232988in}{2.702502in}}%
\pgfpathlineto{\pgfqpoint{1.230415in}{2.873244in}}%
\pgfpathlineto{\pgfqpoint{1.233689in}{2.706573in}}%
\pgfpathlineto{\pgfqpoint{1.235326in}{2.798092in}}%
\pgfpathlineto{\pgfqpoint{1.235794in}{2.776741in}}%
\pgfpathlineto{\pgfqpoint{1.236028in}{2.761746in}}%
\pgfpathlineto{\pgfqpoint{1.236729in}{2.781210in}}%
\pgfpathlineto{\pgfqpoint{1.239536in}{2.899425in}}%
\pgfpathlineto{\pgfqpoint{1.241874in}{2.674683in}}%
\pgfpathlineto{\pgfqpoint{1.242108in}{2.680214in}}%
\pgfpathlineto{\pgfqpoint{1.244213in}{2.872992in}}%
\pgfpathlineto{\pgfqpoint{1.244914in}{2.857730in}}%
\pgfpathlineto{\pgfqpoint{1.245148in}{2.864794in}}%
\pgfpathlineto{\pgfqpoint{1.245616in}{2.854509in}}%
\pgfpathlineto{\pgfqpoint{1.246785in}{2.830021in}}%
\pgfpathlineto{\pgfqpoint{1.247019in}{2.841641in}}%
\pgfpathlineto{\pgfqpoint{1.247721in}{2.826142in}}%
\pgfpathlineto{\pgfqpoint{1.247955in}{2.838719in}}%
\pgfpathlineto{\pgfqpoint{1.251229in}{2.698234in}}%
\pgfpathlineto{\pgfqpoint{1.251462in}{2.698836in}}%
\pgfpathlineto{\pgfqpoint{1.254035in}{2.899041in}}%
\pgfpathlineto{\pgfqpoint{1.254737in}{2.870362in}}%
\pgfpathlineto{\pgfqpoint{1.258946in}{2.669537in}}%
\pgfpathlineto{\pgfqpoint{1.259180in}{2.675599in}}%
\pgfpathlineto{\pgfqpoint{1.261518in}{2.782976in}}%
\pgfpathlineto{\pgfqpoint{1.263623in}{2.652739in}}%
\pgfpathlineto{\pgfqpoint{1.264559in}{2.610288in}}%
\pgfpathlineto{\pgfqpoint{1.264792in}{2.616088in}}%
\pgfpathlineto{\pgfqpoint{1.265728in}{2.647314in}}%
\pgfpathlineto{\pgfqpoint{1.265962in}{2.634606in}}%
\pgfpathlineto{\pgfqpoint{1.266196in}{2.635407in}}%
\pgfpathlineto{\pgfqpoint{1.268066in}{2.780093in}}%
\pgfpathlineto{\pgfqpoint{1.268534in}{2.763973in}}%
\pgfpathlineto{\pgfqpoint{1.272978in}{2.558462in}}%
\pgfpathlineto{\pgfqpoint{1.273445in}{2.574176in}}%
\pgfpathlineto{\pgfqpoint{1.275784in}{2.724645in}}%
\pgfpathlineto{\pgfqpoint{1.276953in}{2.761508in}}%
\pgfpathlineto{\pgfqpoint{1.277187in}{2.755068in}}%
\pgfpathlineto{\pgfqpoint{1.280929in}{2.533898in}}%
\pgfpathlineto{\pgfqpoint{1.281163in}{2.556290in}}%
\pgfpathlineto{\pgfqpoint{1.281864in}{2.537120in}}%
\pgfpathlineto{\pgfqpoint{1.282098in}{2.536375in}}%
\pgfpathlineto{\pgfqpoint{1.282332in}{2.519476in}}%
\pgfpathlineto{\pgfqpoint{1.283033in}{2.542408in}}%
\pgfpathlineto{\pgfqpoint{1.285138in}{2.743003in}}%
\pgfpathlineto{\pgfqpoint{1.285606in}{2.699811in}}%
\pgfpathlineto{\pgfqpoint{1.287009in}{2.502876in}}%
\pgfpathlineto{\pgfqpoint{1.288646in}{2.416325in}}%
\pgfpathlineto{\pgfqpoint{1.289815in}{2.459804in}}%
\pgfpathlineto{\pgfqpoint{1.291452in}{2.653882in}}%
\pgfpathlineto{\pgfqpoint{1.291920in}{2.641423in}}%
\pgfpathlineto{\pgfqpoint{1.293791in}{2.508714in}}%
\pgfpathlineto{\pgfqpoint{1.295896in}{2.319282in}}%
\pgfpathlineto{\pgfqpoint{1.296130in}{2.319409in}}%
\pgfpathlineto{\pgfqpoint{1.299170in}{2.626373in}}%
\pgfpathlineto{\pgfqpoint{1.299404in}{2.606364in}}%
\pgfpathlineto{\pgfqpoint{1.301976in}{2.315624in}}%
\pgfpathlineto{\pgfqpoint{1.302444in}{2.357155in}}%
\pgfpathlineto{\pgfqpoint{1.303145in}{2.465837in}}%
\pgfpathlineto{\pgfqpoint{1.304315in}{2.626967in}}%
\pgfpathlineto{\pgfqpoint{1.305016in}{2.610396in}}%
\pgfpathlineto{\pgfqpoint{1.307589in}{2.288727in}}%
\pgfpathlineto{\pgfqpoint{1.307823in}{2.296011in}}%
\pgfpathlineto{\pgfqpoint{1.308290in}{2.294209in}}%
\pgfpathlineto{\pgfqpoint{1.310863in}{2.592955in}}%
\pgfpathlineto{\pgfqpoint{1.311798in}{2.546639in}}%
\pgfpathlineto{\pgfqpoint{1.313669in}{2.269858in}}%
\pgfpathlineto{\pgfqpoint{1.314371in}{2.287569in}}%
\pgfpathlineto{\pgfqpoint{1.316008in}{2.404157in}}%
\pgfpathlineto{\pgfqpoint{1.317411in}{2.626799in}}%
\pgfpathlineto{\pgfqpoint{1.317879in}{2.609469in}}%
\pgfpathlineto{\pgfqpoint{1.319983in}{2.254816in}}%
\pgfpathlineto{\pgfqpoint{1.320685in}{2.317587in}}%
\pgfpathlineto{\pgfqpoint{1.322088in}{2.620070in}}%
\pgfpathlineto{\pgfqpoint{1.322790in}{2.587094in}}%
\pgfpathlineto{\pgfqpoint{1.325362in}{2.215223in}}%
\pgfpathlineto{\pgfqpoint{1.327934in}{2.550581in}}%
\pgfpathlineto{\pgfqpoint{1.328870in}{2.500736in}}%
\pgfpathlineto{\pgfqpoint{1.330975in}{2.178824in}}%
\pgfpathlineto{\pgfqpoint{1.331442in}{2.235175in}}%
\pgfpathlineto{\pgfqpoint{1.333547in}{2.578179in}}%
\pgfpathlineto{\pgfqpoint{1.333781in}{2.561672in}}%
\pgfpathlineto{\pgfqpoint{1.335652in}{2.164914in}}%
\pgfpathlineto{\pgfqpoint{1.336120in}{2.217887in}}%
\pgfpathlineto{\pgfqpoint{1.337055in}{2.415611in}}%
\pgfpathlineto{\pgfqpoint{1.337990in}{2.588800in}}%
\pgfpathlineto{\pgfqpoint{1.338458in}{2.533053in}}%
\pgfpathlineto{\pgfqpoint{1.340329in}{2.180567in}}%
\pgfpathlineto{\pgfqpoint{1.341031in}{2.222587in}}%
\pgfpathlineto{\pgfqpoint{1.342668in}{2.561085in}}%
\pgfpathlineto{\pgfqpoint{1.343369in}{2.470214in}}%
\pgfpathlineto{\pgfqpoint{1.345240in}{2.269401in}}%
\pgfpathlineto{\pgfqpoint{1.345474in}{2.275724in}}%
\pgfpathlineto{\pgfqpoint{1.346643in}{2.435488in}}%
\pgfpathlineto{\pgfqpoint{1.347111in}{2.416308in}}%
\pgfpathlineto{\pgfqpoint{1.348982in}{2.226167in}}%
\pgfpathlineto{\pgfqpoint{1.349450in}{2.280880in}}%
\pgfpathlineto{\pgfqpoint{1.351087in}{2.486855in}}%
\pgfpathlineto{\pgfqpoint{1.352490in}{2.385296in}}%
\pgfpathlineto{\pgfqpoint{1.352724in}{2.404184in}}%
\pgfpathlineto{\pgfqpoint{1.353659in}{2.476308in}}%
\pgfpathlineto{\pgfqpoint{1.353893in}{2.456598in}}%
\pgfpathlineto{\pgfqpoint{1.355530in}{2.230505in}}%
\pgfpathlineto{\pgfqpoint{1.355998in}{2.264664in}}%
\pgfpathlineto{\pgfqpoint{1.357401in}{2.383091in}}%
\pgfpathlineto{\pgfqpoint{1.357635in}{2.357579in}}%
\pgfpathlineto{\pgfqpoint{1.358570in}{2.299454in}}%
\pgfpathlineto{\pgfqpoint{1.358804in}{2.335884in}}%
\pgfpathlineto{\pgfqpoint{1.360207in}{2.516475in}}%
\pgfpathlineto{\pgfqpoint{1.360675in}{2.461352in}}%
\pgfpathlineto{\pgfqpoint{1.361610in}{2.305531in}}%
\pgfpathlineto{\pgfqpoint{1.362078in}{2.362929in}}%
\pgfpathlineto{\pgfqpoint{1.363013in}{2.467737in}}%
\pgfpathlineto{\pgfqpoint{1.363481in}{2.458757in}}%
\pgfpathlineto{\pgfqpoint{1.364650in}{2.289168in}}%
\pgfpathlineto{\pgfqpoint{1.365352in}{2.346511in}}%
\pgfpathlineto{\pgfqpoint{1.365820in}{2.384018in}}%
\pgfpathlineto{\pgfqpoint{1.366287in}{2.365329in}}%
\pgfpathlineto{\pgfqpoint{1.367223in}{2.239733in}}%
\pgfpathlineto{\pgfqpoint{1.367691in}{2.297759in}}%
\pgfpathlineto{\pgfqpoint{1.368860in}{2.528497in}}%
\pgfpathlineto{\pgfqpoint{1.369328in}{2.481643in}}%
\pgfpathlineto{\pgfqpoint{1.370029in}{2.391060in}}%
\pgfpathlineto{\pgfqpoint{1.370731in}{2.430610in}}%
\pgfpathlineto{\pgfqpoint{1.371432in}{2.540136in}}%
\pgfpathlineto{\pgfqpoint{1.371900in}{2.496966in}}%
\pgfpathlineto{\pgfqpoint{1.373303in}{2.238538in}}%
\pgfpathlineto{\pgfqpoint{1.374005in}{2.274209in}}%
\pgfpathlineto{\pgfqpoint{1.374472in}{2.331950in}}%
\pgfpathlineto{\pgfqpoint{1.375174in}{2.427448in}}%
\pgfpathlineto{\pgfqpoint{1.375642in}{2.374195in}}%
\pgfpathlineto{\pgfqpoint{1.376577in}{2.309058in}}%
\pgfpathlineto{\pgfqpoint{1.377045in}{2.328910in}}%
\pgfpathlineto{\pgfqpoint{1.378682in}{2.576154in}}%
\pgfpathlineto{\pgfqpoint{1.379150in}{2.489925in}}%
\pgfpathlineto{\pgfqpoint{1.380553in}{2.157771in}}%
\pgfpathlineto{\pgfqpoint{1.381021in}{2.230575in}}%
\pgfpathlineto{\pgfqpoint{1.381722in}{2.373968in}}%
\pgfpathlineto{\pgfqpoint{1.382658in}{2.332623in}}%
\pgfpathlineto{\pgfqpoint{1.383125in}{2.330229in}}%
\pgfpathlineto{\pgfqpoint{1.383359in}{2.377550in}}%
\pgfpathlineto{\pgfqpoint{1.384061in}{2.516551in}}%
\pgfpathlineto{\pgfqpoint{1.384528in}{2.439988in}}%
\pgfpathlineto{\pgfqpoint{1.385698in}{2.319439in}}%
\pgfpathlineto{\pgfqpoint{1.386165in}{2.359997in}}%
\pgfpathlineto{\pgfqpoint{1.386399in}{2.366267in}}%
\pgfpathlineto{\pgfqpoint{1.386633in}{2.346506in}}%
\pgfpathlineto{\pgfqpoint{1.388270in}{2.239708in}}%
\pgfpathlineto{\pgfqpoint{1.388738in}{2.287245in}}%
\pgfpathlineto{\pgfqpoint{1.389439in}{2.296247in}}%
\pgfpathlineto{\pgfqpoint{1.390375in}{2.183282in}}%
\pgfpathlineto{\pgfqpoint{1.390843in}{2.224231in}}%
\pgfpathlineto{\pgfqpoint{1.392012in}{2.381541in}}%
\pgfpathlineto{\pgfqpoint{1.392246in}{2.346332in}}%
\pgfpathlineto{\pgfqpoint{1.393415in}{2.252135in}}%
\pgfpathlineto{\pgfqpoint{1.393649in}{2.296822in}}%
\pgfpathlineto{\pgfqpoint{1.394584in}{2.423604in}}%
\pgfpathlineto{\pgfqpoint{1.394818in}{2.364052in}}%
\pgfpathlineto{\pgfqpoint{1.395520in}{2.311920in}}%
\pgfpathlineto{\pgfqpoint{1.396221in}{2.342508in}}%
\pgfpathlineto{\pgfqpoint{1.396455in}{2.343706in}}%
\pgfpathlineto{\pgfqpoint{1.396923in}{2.345064in}}%
\pgfpathlineto{\pgfqpoint{1.397625in}{2.278906in}}%
\pgfpathlineto{\pgfqpoint{1.398092in}{2.219444in}}%
\pgfpathlineto{\pgfqpoint{1.398560in}{2.269665in}}%
\pgfpathlineto{\pgfqpoint{1.398794in}{2.276919in}}%
\pgfpathlineto{\pgfqpoint{1.399262in}{2.265020in}}%
\pgfpathlineto{\pgfqpoint{1.400431in}{2.114184in}}%
\pgfpathlineto{\pgfqpoint{1.400899in}{2.199080in}}%
\pgfpathlineto{\pgfqpoint{1.401834in}{2.304477in}}%
\pgfpathlineto{\pgfqpoint{1.402302in}{2.277297in}}%
\pgfpathlineto{\pgfqpoint{1.403471in}{2.363455in}}%
\pgfpathlineto{\pgfqpoint{1.403939in}{2.427636in}}%
\pgfpathlineto{\pgfqpoint{1.404406in}{2.420112in}}%
\pgfpathlineto{\pgfqpoint{1.405342in}{2.257459in}}%
\pgfpathlineto{\pgfqpoint{1.405810in}{2.281731in}}%
\pgfpathlineto{\pgfqpoint{1.406511in}{2.345038in}}%
\pgfpathlineto{\pgfqpoint{1.406745in}{2.288343in}}%
\pgfpathlineto{\pgfqpoint{1.408148in}{2.119740in}}%
\pgfpathlineto{\pgfqpoint{1.408382in}{2.122027in}}%
\pgfpathlineto{\pgfqpoint{1.409084in}{2.216103in}}%
\pgfpathlineto{\pgfqpoint{1.409551in}{2.177945in}}%
\pgfpathlineto{\pgfqpoint{1.409785in}{2.175113in}}%
\pgfpathlineto{\pgfqpoint{1.411188in}{2.389918in}}%
\pgfpathlineto{\pgfqpoint{1.411890in}{2.349652in}}%
\pgfpathlineto{\pgfqpoint{1.412124in}{2.323633in}}%
\pgfpathlineto{\pgfqpoint{1.412592in}{2.355538in}}%
\pgfpathlineto{\pgfqpoint{1.413059in}{2.414622in}}%
\pgfpathlineto{\pgfqpoint{1.413527in}{2.377976in}}%
\pgfpathlineto{\pgfqpoint{1.414696in}{2.163076in}}%
\pgfpathlineto{\pgfqpoint{1.414930in}{2.192340in}}%
\pgfpathlineto{\pgfqpoint{1.415398in}{2.265994in}}%
\pgfpathlineto{\pgfqpoint{1.416099in}{2.230755in}}%
\pgfpathlineto{\pgfqpoint{1.416801in}{2.172218in}}%
\pgfpathlineto{\pgfqpoint{1.417269in}{2.210396in}}%
\pgfpathlineto{\pgfqpoint{1.418204in}{2.265673in}}%
\pgfpathlineto{\pgfqpoint{1.418438in}{2.245713in}}%
\pgfpathlineto{\pgfqpoint{1.418672in}{2.205450in}}%
\pgfpathlineto{\pgfqpoint{1.419140in}{2.263200in}}%
\pgfpathlineto{\pgfqpoint{1.420075in}{2.445425in}}%
\pgfpathlineto{\pgfqpoint{1.420777in}{2.380550in}}%
\pgfpathlineto{\pgfqpoint{1.423115in}{2.178961in}}%
\pgfpathlineto{\pgfqpoint{1.423349in}{2.163622in}}%
\pgfpathlineto{\pgfqpoint{1.423583in}{2.184886in}}%
\pgfpathlineto{\pgfqpoint{1.424518in}{2.287042in}}%
\pgfpathlineto{\pgfqpoint{1.424986in}{2.241839in}}%
\pgfpathlineto{\pgfqpoint{1.425454in}{2.201713in}}%
\pgfpathlineto{\pgfqpoint{1.425922in}{2.244587in}}%
\pgfpathlineto{\pgfqpoint{1.426857in}{2.384667in}}%
\pgfpathlineto{\pgfqpoint{1.427091in}{2.356278in}}%
\pgfpathlineto{\pgfqpoint{1.427792in}{2.234095in}}%
\pgfpathlineto{\pgfqpoint{1.428260in}{2.281668in}}%
\pgfpathlineto{\pgfqpoint{1.428962in}{2.332210in}}%
\pgfpathlineto{\pgfqpoint{1.429196in}{2.321031in}}%
\pgfpathlineto{\pgfqpoint{1.429897in}{2.245230in}}%
\pgfpathlineto{\pgfqpoint{1.430365in}{2.269738in}}%
\pgfpathlineto{\pgfqpoint{1.431534in}{2.379858in}}%
\pgfpathlineto{\pgfqpoint{1.432002in}{2.331990in}}%
\pgfpathlineto{\pgfqpoint{1.433171in}{2.444589in}}%
\pgfpathlineto{\pgfqpoint{1.433639in}{2.419967in}}%
\pgfpathlineto{\pgfqpoint{1.434808in}{2.250073in}}%
\pgfpathlineto{\pgfqpoint{1.435042in}{2.262658in}}%
\pgfpathlineto{\pgfqpoint{1.435510in}{2.297524in}}%
\pgfpathlineto{\pgfqpoint{1.435744in}{2.243979in}}%
\pgfpathlineto{\pgfqpoint{1.436445in}{2.143138in}}%
\pgfpathlineto{\pgfqpoint{1.436913in}{2.191379in}}%
\pgfpathlineto{\pgfqpoint{1.437848in}{2.365990in}}%
\pgfpathlineto{\pgfqpoint{1.438316in}{2.332922in}}%
\pgfpathlineto{\pgfqpoint{1.438550in}{2.325469in}}%
\pgfpathlineto{\pgfqpoint{1.439485in}{2.467550in}}%
\pgfpathlineto{\pgfqpoint{1.439953in}{2.435176in}}%
\pgfpathlineto{\pgfqpoint{1.441356in}{2.338954in}}%
\pgfpathlineto{\pgfqpoint{1.441590in}{2.363483in}}%
\pgfpathlineto{\pgfqpoint{1.442058in}{2.329664in}}%
\pgfpathlineto{\pgfqpoint{1.442993in}{2.208395in}}%
\pgfpathlineto{\pgfqpoint{1.443227in}{2.242472in}}%
\pgfpathlineto{\pgfqpoint{1.443695in}{2.295462in}}%
\pgfpathlineto{\pgfqpoint{1.444163in}{2.264412in}}%
\pgfpathlineto{\pgfqpoint{1.444864in}{2.182684in}}%
\pgfpathlineto{\pgfqpoint{1.445332in}{2.239182in}}%
\pgfpathlineto{\pgfqpoint{1.447670in}{2.438690in}}%
\pgfpathlineto{\pgfqpoint{1.448138in}{2.421798in}}%
\pgfpathlineto{\pgfqpoint{1.448606in}{2.381476in}}%
\pgfpathlineto{\pgfqpoint{1.449307in}{2.413909in}}%
\pgfpathlineto{\pgfqpoint{1.450711in}{2.358029in}}%
\pgfpathlineto{\pgfqpoint{1.450009in}{2.417085in}}%
\pgfpathlineto{\pgfqpoint{1.450944in}{2.361490in}}%
\pgfpathlineto{\pgfqpoint{1.451646in}{2.392379in}}%
\pgfpathlineto{\pgfqpoint{1.453049in}{2.141165in}}%
\pgfpathlineto{\pgfqpoint{1.453517in}{2.210470in}}%
\pgfpathlineto{\pgfqpoint{1.454686in}{2.395870in}}%
\pgfpathlineto{\pgfqpoint{1.454920in}{2.386179in}}%
\pgfpathlineto{\pgfqpoint{1.455388in}{2.321679in}}%
\pgfpathlineto{\pgfqpoint{1.455856in}{2.376820in}}%
\pgfpathlineto{\pgfqpoint{1.456557in}{2.447473in}}%
\pgfpathlineto{\pgfqpoint{1.456791in}{2.400362in}}%
\pgfpathlineto{\pgfqpoint{1.457259in}{2.325785in}}%
\pgfpathlineto{\pgfqpoint{1.457960in}{2.374623in}}%
\pgfpathlineto{\pgfqpoint{1.458194in}{2.372927in}}%
\pgfpathlineto{\pgfqpoint{1.459130in}{2.283413in}}%
\pgfpathlineto{\pgfqpoint{1.459363in}{2.310127in}}%
\pgfpathlineto{\pgfqpoint{1.459831in}{2.402601in}}%
\pgfpathlineto{\pgfqpoint{1.460533in}{2.322080in}}%
\pgfpathlineto{\pgfqpoint{1.461000in}{2.234485in}}%
\pgfpathlineto{\pgfqpoint{1.461702in}{2.301333in}}%
\pgfpathlineto{\pgfqpoint{1.463807in}{2.490752in}}%
\pgfpathlineto{\pgfqpoint{1.464041in}{2.460544in}}%
\pgfpathlineto{\pgfqpoint{1.466613in}{2.253791in}}%
\pgfpathlineto{\pgfqpoint{1.468016in}{2.456807in}}%
\pgfpathlineto{\pgfqpoint{1.469185in}{2.442094in}}%
\pgfpathlineto{\pgfqpoint{1.469653in}{2.465010in}}%
\pgfpathlineto{\pgfqpoint{1.470121in}{2.436009in}}%
\pgfpathlineto{\pgfqpoint{1.470822in}{2.471443in}}%
\pgfpathlineto{\pgfqpoint{1.471290in}{2.515674in}}%
\pgfpathlineto{\pgfqpoint{1.471758in}{2.450180in}}%
\pgfpathlineto{\pgfqpoint{1.473161in}{2.328197in}}%
\pgfpathlineto{\pgfqpoint{1.473629in}{2.276444in}}%
\pgfpathlineto{\pgfqpoint{1.473863in}{2.245744in}}%
\pgfpathlineto{\pgfqpoint{1.474330in}{2.268033in}}%
\pgfpathlineto{\pgfqpoint{1.474798in}{2.363683in}}%
\pgfpathlineto{\pgfqpoint{1.475500in}{2.321837in}}%
\pgfpathlineto{\pgfqpoint{1.475967in}{2.400983in}}%
\pgfpathlineto{\pgfqpoint{1.477137in}{2.524492in}}%
\pgfpathlineto{\pgfqpoint{1.477371in}{2.488980in}}%
\pgfpathlineto{\pgfqpoint{1.478072in}{2.528368in}}%
\pgfpathlineto{\pgfqpoint{1.478540in}{2.495394in}}%
\pgfpathlineto{\pgfqpoint{1.478774in}{2.485989in}}%
\pgfpathlineto{\pgfqpoint{1.479475in}{2.363305in}}%
\pgfpathlineto{\pgfqpoint{1.480177in}{2.387208in}}%
\pgfpathlineto{\pgfqpoint{1.480411in}{2.385416in}}%
\pgfpathlineto{\pgfqpoint{1.481112in}{2.303797in}}%
\pgfpathlineto{\pgfqpoint{1.481580in}{2.340938in}}%
\pgfpathlineto{\pgfqpoint{1.481814in}{2.383343in}}%
\pgfpathlineto{\pgfqpoint{1.482515in}{2.343272in}}%
\pgfpathlineto{\pgfqpoint{1.483919in}{2.384919in}}%
\pgfpathlineto{\pgfqpoint{1.482983in}{2.335673in}}%
\pgfpathlineto{\pgfqpoint{1.484386in}{2.374710in}}%
\pgfpathlineto{\pgfqpoint{1.484620in}{2.367986in}}%
\pgfpathlineto{\pgfqpoint{1.485556in}{2.596118in}}%
\pgfpathlineto{\pgfqpoint{1.486257in}{2.487044in}}%
\pgfpathlineto{\pgfqpoint{1.486491in}{2.487452in}}%
\pgfpathlineto{\pgfqpoint{1.486725in}{2.479632in}}%
\pgfpathlineto{\pgfqpoint{1.486959in}{2.480376in}}%
\pgfpathlineto{\pgfqpoint{1.487660in}{2.522011in}}%
\pgfpathlineto{\pgfqpoint{1.487894in}{2.452723in}}%
\pgfpathlineto{\pgfqpoint{1.489999in}{2.241946in}}%
\pgfpathlineto{\pgfqpoint{1.492805in}{2.581692in}}%
\pgfpathlineto{\pgfqpoint{1.493507in}{2.472815in}}%
\pgfpathlineto{\pgfqpoint{1.495612in}{2.225490in}}%
\pgfpathlineto{\pgfqpoint{1.496079in}{2.299932in}}%
\pgfpathlineto{\pgfqpoint{1.496547in}{2.378247in}}%
\pgfpathlineto{\pgfqpoint{1.497015in}{2.356919in}}%
\pgfpathlineto{\pgfqpoint{1.497249in}{2.311156in}}%
\pgfpathlineto{\pgfqpoint{1.497716in}{2.374671in}}%
\pgfpathlineto{\pgfqpoint{1.498184in}{2.475160in}}%
\pgfpathlineto{\pgfqpoint{1.498886in}{2.404919in}}%
\pgfpathlineto{\pgfqpoint{1.499587in}{2.465521in}}%
\pgfpathlineto{\pgfqpoint{1.500055in}{2.416025in}}%
\pgfpathlineto{\pgfqpoint{1.501458in}{2.263871in}}%
\pgfpathlineto{\pgfqpoint{1.501692in}{2.270218in}}%
\pgfpathlineto{\pgfqpoint{1.502160in}{2.255577in}}%
\pgfpathlineto{\pgfqpoint{1.504498in}{2.525672in}}%
\pgfpathlineto{\pgfqpoint{1.504966in}{2.514253in}}%
\pgfpathlineto{\pgfqpoint{1.505901in}{2.318872in}}%
\pgfpathlineto{\pgfqpoint{1.506369in}{2.347264in}}%
\pgfpathlineto{\pgfqpoint{1.507305in}{2.207132in}}%
\pgfpathlineto{\pgfqpoint{1.507772in}{2.217739in}}%
\pgfpathlineto{\pgfqpoint{1.509877in}{2.499346in}}%
\pgfpathlineto{\pgfqpoint{1.510111in}{2.537692in}}%
\pgfpathlineto{\pgfqpoint{1.510579in}{2.469979in}}%
\pgfpathlineto{\pgfqpoint{1.512917in}{2.207903in}}%
\pgfpathlineto{\pgfqpoint{1.513385in}{2.225729in}}%
\pgfpathlineto{\pgfqpoint{1.513619in}{2.208737in}}%
\pgfpathlineto{\pgfqpoint{1.513853in}{2.197369in}}%
\pgfpathlineto{\pgfqpoint{1.514086in}{2.200883in}}%
\pgfpathlineto{\pgfqpoint{1.516191in}{2.494261in}}%
\pgfpathlineto{\pgfqpoint{1.518998in}{2.189227in}}%
\pgfpathlineto{\pgfqpoint{1.519231in}{2.193557in}}%
\pgfpathlineto{\pgfqpoint{1.520401in}{2.390939in}}%
\pgfpathlineto{\pgfqpoint{1.520868in}{2.364283in}}%
\pgfpathlineto{\pgfqpoint{1.521570in}{2.433452in}}%
\pgfpathlineto{\pgfqpoint{1.521804in}{2.367348in}}%
\pgfpathlineto{\pgfqpoint{1.523909in}{2.147609in}}%
\pgfpathlineto{\pgfqpoint{1.524142in}{2.174899in}}%
\pgfpathlineto{\pgfqpoint{1.525546in}{2.293210in}}%
\pgfpathlineto{\pgfqpoint{1.526481in}{2.470162in}}%
\pgfpathlineto{\pgfqpoint{1.526949in}{2.388915in}}%
\pgfpathlineto{\pgfqpoint{1.529053in}{2.157148in}}%
\pgfpathlineto{\pgfqpoint{1.529521in}{2.201252in}}%
\pgfpathlineto{\pgfqpoint{1.529755in}{2.210647in}}%
\pgfpathlineto{\pgfqpoint{1.530223in}{2.135510in}}%
\pgfpathlineto{\pgfqpoint{1.530690in}{2.212219in}}%
\pgfpathlineto{\pgfqpoint{1.532795in}{2.476636in}}%
\pgfpathlineto{\pgfqpoint{1.533263in}{2.403725in}}%
\pgfpathlineto{\pgfqpoint{1.535368in}{2.135667in}}%
\pgfpathlineto{\pgfqpoint{1.535602in}{2.135280in}}%
\pgfpathlineto{\pgfqpoint{1.535835in}{2.137276in}}%
\pgfpathlineto{\pgfqpoint{1.537940in}{2.437884in}}%
\pgfpathlineto{\pgfqpoint{1.540279in}{2.078714in}}%
\pgfpathlineto{\pgfqpoint{1.540746in}{2.153276in}}%
\pgfpathlineto{\pgfqpoint{1.542617in}{2.435118in}}%
\pgfpathlineto{\pgfqpoint{1.544488in}{2.282136in}}%
\pgfpathlineto{\pgfqpoint{1.545424in}{2.061828in}}%
\pgfpathlineto{\pgfqpoint{1.545891in}{2.121676in}}%
\pgfpathlineto{\pgfqpoint{1.546125in}{2.117928in}}%
\pgfpathlineto{\pgfqpoint{1.546359in}{2.120312in}}%
\pgfpathlineto{\pgfqpoint{1.547061in}{2.210862in}}%
\pgfpathlineto{\pgfqpoint{1.548230in}{2.422600in}}%
\pgfpathlineto{\pgfqpoint{1.548464in}{2.391238in}}%
\pgfpathlineto{\pgfqpoint{1.549399in}{2.307096in}}%
\pgfpathlineto{\pgfqpoint{1.550335in}{2.094367in}}%
\pgfpathlineto{\pgfqpoint{1.550802in}{2.141370in}}%
\pgfpathlineto{\pgfqpoint{1.552439in}{2.326725in}}%
\pgfpathlineto{\pgfqpoint{1.552673in}{2.302281in}}%
\pgfpathlineto{\pgfqpoint{1.552907in}{2.292597in}}%
\pgfpathlineto{\pgfqpoint{1.553609in}{2.195616in}}%
\pgfpathlineto{\pgfqpoint{1.554076in}{2.216044in}}%
\pgfpathlineto{\pgfqpoint{1.554310in}{2.216767in}}%
\pgfpathlineto{\pgfqpoint{1.554544in}{2.184417in}}%
\pgfpathlineto{\pgfqpoint{1.555012in}{2.203426in}}%
\pgfpathlineto{\pgfqpoint{1.555480in}{2.339099in}}%
\pgfpathlineto{\pgfqpoint{1.555947in}{2.266839in}}%
\pgfpathlineto{\pgfqpoint{1.556415in}{2.198456in}}%
\pgfpathlineto{\pgfqpoint{1.556883in}{2.242184in}}%
\pgfpathlineto{\pgfqpoint{1.558052in}{2.351706in}}%
\pgfpathlineto{\pgfqpoint{1.558286in}{2.422373in}}%
\pgfpathlineto{\pgfqpoint{1.558987in}{2.319560in}}%
\pgfpathlineto{\pgfqpoint{1.560858in}{2.056544in}}%
\pgfpathlineto{\pgfqpoint{1.561092in}{2.089109in}}%
\pgfpathlineto{\pgfqpoint{1.562963in}{2.417004in}}%
\pgfpathlineto{\pgfqpoint{1.563898in}{2.319359in}}%
\pgfpathlineto{\pgfqpoint{1.564132in}{2.318024in}}%
\pgfpathlineto{\pgfqpoint{1.565302in}{2.157022in}}%
\pgfpathlineto{\pgfqpoint{1.565769in}{2.239805in}}%
\pgfpathlineto{\pgfqpoint{1.566003in}{2.248311in}}%
\pgfpathlineto{\pgfqpoint{1.566237in}{2.236706in}}%
\pgfpathlineto{\pgfqpoint{1.567874in}{2.085262in}}%
\pgfpathlineto{\pgfqpoint{1.569979in}{2.253680in}}%
\pgfpathlineto{\pgfqpoint{1.570213in}{2.245459in}}%
\pgfpathlineto{\pgfqpoint{1.570447in}{2.236307in}}%
\pgfpathlineto{\pgfqpoint{1.570680in}{2.262497in}}%
\pgfpathlineto{\pgfqpoint{1.571382in}{2.298509in}}%
\pgfpathlineto{\pgfqpoint{1.571616in}{2.281458in}}%
\pgfpathlineto{\pgfqpoint{1.572317in}{2.230905in}}%
\pgfpathlineto{\pgfqpoint{1.572551in}{2.279939in}}%
\pgfpathlineto{\pgfqpoint{1.573954in}{2.343287in}}%
\pgfpathlineto{\pgfqpoint{1.574890in}{2.168679in}}%
\pgfpathlineto{\pgfqpoint{1.575358in}{2.277154in}}%
\pgfpathlineto{\pgfqpoint{1.575825in}{2.309356in}}%
\pgfpathlineto{\pgfqpoint{1.576293in}{2.283671in}}%
\pgfpathlineto{\pgfqpoint{1.577696in}{2.160959in}}%
\pgfpathlineto{\pgfqpoint{1.579099in}{2.054448in}}%
\pgfpathlineto{\pgfqpoint{1.580736in}{2.374308in}}%
\pgfpathlineto{\pgfqpoint{1.581672in}{2.315296in}}%
\pgfpathlineto{\pgfqpoint{1.582373in}{2.214369in}}%
\pgfpathlineto{\pgfqpoint{1.582841in}{2.261222in}}%
\pgfpathlineto{\pgfqpoint{1.584010in}{2.319922in}}%
\pgfpathlineto{\pgfqpoint{1.584244in}{2.329884in}}%
\pgfpathlineto{\pgfqpoint{1.584478in}{2.326658in}}%
\pgfpathlineto{\pgfqpoint{1.585647in}{2.071019in}}%
\pgfpathlineto{\pgfqpoint{1.586583in}{2.132341in}}%
\pgfpathlineto{\pgfqpoint{1.587051in}{2.201663in}}%
\pgfpathlineto{\pgfqpoint{1.587752in}{2.359880in}}%
\pgfpathlineto{\pgfqpoint{1.588220in}{2.312365in}}%
\pgfpathlineto{\pgfqpoint{1.588688in}{2.342975in}}%
\pgfpathlineto{\pgfqpoint{1.589623in}{2.260073in}}%
\pgfpathlineto{\pgfqpoint{1.590558in}{2.359748in}}%
\pgfpathlineto{\pgfqpoint{1.591260in}{2.338793in}}%
\pgfpathlineto{\pgfqpoint{1.591728in}{2.285523in}}%
\pgfpathlineto{\pgfqpoint{1.592429in}{2.091973in}}%
\pgfpathlineto{\pgfqpoint{1.593599in}{2.118980in}}%
\pgfpathlineto{\pgfqpoint{1.595703in}{2.435744in}}%
\pgfpathlineto{\pgfqpoint{1.596405in}{2.247375in}}%
\pgfpathlineto{\pgfqpoint{1.597340in}{2.282435in}}%
\pgfpathlineto{\pgfqpoint{1.597574in}{2.256973in}}%
\pgfpathlineto{\pgfqpoint{1.598042in}{2.317928in}}%
\pgfpathlineto{\pgfqpoint{1.598276in}{2.316527in}}%
\pgfpathlineto{\pgfqpoint{1.600381in}{2.097866in}}%
\pgfpathlineto{\pgfqpoint{1.601550in}{2.382500in}}%
\pgfpathlineto{\pgfqpoint{1.602251in}{2.302507in}}%
\pgfpathlineto{\pgfqpoint{1.602485in}{2.301579in}}%
\pgfpathlineto{\pgfqpoint{1.603187in}{2.166959in}}%
\pgfpathlineto{\pgfqpoint{1.603655in}{2.217057in}}%
\pgfpathlineto{\pgfqpoint{1.605058in}{2.364492in}}%
\pgfpathlineto{\pgfqpoint{1.605993in}{2.173157in}}%
\pgfpathlineto{\pgfqpoint{1.606695in}{2.277651in}}%
\pgfpathlineto{\pgfqpoint{1.606929in}{2.303400in}}%
\pgfpathlineto{\pgfqpoint{1.607630in}{2.268407in}}%
\pgfpathlineto{\pgfqpoint{1.608098in}{2.211295in}}%
\pgfpathlineto{\pgfqpoint{1.608566in}{2.092619in}}%
\pgfpathlineto{\pgfqpoint{1.609267in}{2.175911in}}%
\pgfpathlineto{\pgfqpoint{1.609735in}{2.173645in}}%
\pgfpathlineto{\pgfqpoint{1.611840in}{2.336749in}}%
\pgfpathlineto{\pgfqpoint{1.612541in}{2.316531in}}%
\pgfpathlineto{\pgfqpoint{1.612775in}{2.306642in}}%
\pgfpathlineto{\pgfqpoint{1.613243in}{2.412886in}}%
\pgfpathlineto{\pgfqpoint{1.613944in}{2.344077in}}%
\pgfpathlineto{\pgfqpoint{1.615114in}{2.144348in}}%
\pgfpathlineto{\pgfqpoint{1.615581in}{2.184182in}}%
\pgfpathlineto{\pgfqpoint{1.616049in}{2.209360in}}%
\pgfpathlineto{\pgfqpoint{1.616517in}{2.167606in}}%
\pgfpathlineto{\pgfqpoint{1.616985in}{2.203559in}}%
\pgfpathlineto{\pgfqpoint{1.617218in}{2.242405in}}%
\pgfpathlineto{\pgfqpoint{1.617686in}{2.183796in}}%
\pgfpathlineto{\pgfqpoint{1.618154in}{2.134360in}}%
\pgfpathlineto{\pgfqpoint{1.618855in}{2.141306in}}%
\pgfpathlineto{\pgfqpoint{1.619323in}{2.156493in}}%
\pgfpathlineto{\pgfqpoint{1.620960in}{2.401747in}}%
\pgfpathlineto{\pgfqpoint{1.624234in}{2.108174in}}%
\pgfpathlineto{\pgfqpoint{1.624468in}{2.122816in}}%
\pgfpathlineto{\pgfqpoint{1.624936in}{2.203169in}}%
\pgfpathlineto{\pgfqpoint{1.625637in}{2.149362in}}%
\pgfpathlineto{\pgfqpoint{1.625871in}{2.139410in}}%
\pgfpathlineto{\pgfqpoint{1.626105in}{2.151050in}}%
\pgfpathlineto{\pgfqpoint{1.628678in}{2.324143in}}%
\pgfpathlineto{\pgfqpoint{1.626807in}{2.127259in}}%
\pgfpathlineto{\pgfqpoint{1.628911in}{2.311283in}}%
\pgfpathlineto{\pgfqpoint{1.629379in}{2.281397in}}%
\pgfpathlineto{\pgfqpoint{1.629613in}{2.294772in}}%
\pgfpathlineto{\pgfqpoint{1.630081in}{2.379241in}}%
\pgfpathlineto{\pgfqpoint{1.630782in}{2.338637in}}%
\pgfpathlineto{\pgfqpoint{1.631016in}{2.341682in}}%
\pgfpathlineto{\pgfqpoint{1.631250in}{2.339972in}}%
\pgfpathlineto{\pgfqpoint{1.632419in}{2.140437in}}%
\pgfpathlineto{\pgfqpoint{1.633121in}{2.162183in}}%
\pgfpathlineto{\pgfqpoint{1.633355in}{2.152163in}}%
\pgfpathlineto{\pgfqpoint{1.633822in}{2.162913in}}%
\pgfpathlineto{\pgfqpoint{1.634056in}{2.194519in}}%
\pgfpathlineto{\pgfqpoint{1.634524in}{2.164470in}}%
\pgfpathlineto{\pgfqpoint{1.634758in}{2.138166in}}%
\pgfpathlineto{\pgfqpoint{1.635226in}{2.182365in}}%
\pgfpathlineto{\pgfqpoint{1.635927in}{2.286199in}}%
\pgfpathlineto{\pgfqpoint{1.636629in}{2.283121in}}%
\pgfpathlineto{\pgfqpoint{1.637096in}{2.238333in}}%
\pgfpathlineto{\pgfqpoint{1.637564in}{2.274385in}}%
\pgfpathlineto{\pgfqpoint{1.638032in}{2.389665in}}%
\pgfpathlineto{\pgfqpoint{1.638967in}{2.387809in}}%
\pgfpathlineto{\pgfqpoint{1.639201in}{2.384519in}}%
\pgfpathlineto{\pgfqpoint{1.640838in}{2.139269in}}%
\pgfpathlineto{\pgfqpoint{1.641072in}{2.165097in}}%
\pgfpathlineto{\pgfqpoint{1.641306in}{2.169307in}}%
\pgfpathlineto{\pgfqpoint{1.641774in}{2.236349in}}%
\pgfpathlineto{\pgfqpoint{1.642007in}{2.178337in}}%
\pgfpathlineto{\pgfqpoint{1.642709in}{2.067235in}}%
\pgfpathlineto{\pgfqpoint{1.643177in}{2.107829in}}%
\pgfpathlineto{\pgfqpoint{1.643878in}{2.187368in}}%
\pgfpathlineto{\pgfqpoint{1.644346in}{2.292164in}}%
\pgfpathlineto{\pgfqpoint{1.645048in}{2.242489in}}%
\pgfpathlineto{\pgfqpoint{1.646919in}{2.392568in}}%
\pgfpathlineto{\pgfqpoint{1.647152in}{2.380201in}}%
\pgfpathlineto{\pgfqpoint{1.648789in}{2.103076in}}%
\pgfpathlineto{\pgfqpoint{1.649023in}{2.128206in}}%
\pgfpathlineto{\pgfqpoint{1.650894in}{2.248024in}}%
\pgfpathlineto{\pgfqpoint{1.651362in}{2.213546in}}%
\pgfpathlineto{\pgfqpoint{1.651596in}{2.218916in}}%
\pgfpathlineto{\pgfqpoint{1.652063in}{2.304176in}}%
\pgfpathlineto{\pgfqpoint{1.652531in}{2.235365in}}%
\pgfpathlineto{\pgfqpoint{1.653700in}{2.180466in}}%
\pgfpathlineto{\pgfqpoint{1.653934in}{2.183122in}}%
\pgfpathlineto{\pgfqpoint{1.655571in}{2.302858in}}%
\pgfpathlineto{\pgfqpoint{1.655805in}{2.286392in}}%
\pgfpathlineto{\pgfqpoint{1.657208in}{2.230109in}}%
\pgfpathlineto{\pgfqpoint{1.658144in}{2.312211in}}%
\pgfpathlineto{\pgfqpoint{1.658378in}{2.282651in}}%
\pgfpathlineto{\pgfqpoint{1.658845in}{2.230185in}}%
\pgfpathlineto{\pgfqpoint{1.659547in}{2.261370in}}%
\pgfpathlineto{\pgfqpoint{1.660015in}{2.308956in}}%
\pgfpathlineto{\pgfqpoint{1.660482in}{2.350160in}}%
\pgfpathlineto{\pgfqpoint{1.660716in}{2.306880in}}%
\pgfpathlineto{\pgfqpoint{1.662587in}{2.122124in}}%
\pgfpathlineto{\pgfqpoint{1.662821in}{2.169001in}}%
\pgfpathlineto{\pgfqpoint{1.663523in}{2.168105in}}%
\pgfpathlineto{\pgfqpoint{1.665393in}{2.337020in}}%
\pgfpathlineto{\pgfqpoint{1.666095in}{2.245938in}}%
\pgfpathlineto{\pgfqpoint{1.666329in}{2.258829in}}%
\pgfpathlineto{\pgfqpoint{1.667732in}{2.372933in}}%
\pgfpathlineto{\pgfqpoint{1.670772in}{2.097543in}}%
\pgfpathlineto{\pgfqpoint{1.671006in}{2.102265in}}%
\pgfpathlineto{\pgfqpoint{1.673579in}{2.391223in}}%
\pgfpathlineto{\pgfqpoint{1.673812in}{2.399972in}}%
\pgfpathlineto{\pgfqpoint{1.674046in}{2.390689in}}%
\pgfpathlineto{\pgfqpoint{1.674982in}{2.398911in}}%
\pgfpathlineto{\pgfqpoint{1.675917in}{2.232289in}}%
\pgfpathlineto{\pgfqpoint{1.676151in}{2.242292in}}%
\pgfpathlineto{\pgfqpoint{1.676619in}{2.217355in}}%
\pgfpathlineto{\pgfqpoint{1.676853in}{2.240324in}}%
\pgfpathlineto{\pgfqpoint{1.677086in}{2.230998in}}%
\pgfpathlineto{\pgfqpoint{1.677320in}{2.245709in}}%
\pgfpathlineto{\pgfqpoint{1.677554in}{2.268185in}}%
\pgfpathlineto{\pgfqpoint{1.678256in}{2.239295in}}%
\pgfpathlineto{\pgfqpoint{1.678490in}{2.246147in}}%
\pgfpathlineto{\pgfqpoint{1.679191in}{2.108613in}}%
\pgfpathlineto{\pgfqpoint{1.679659in}{2.180239in}}%
\pgfpathlineto{\pgfqpoint{1.682231in}{2.449893in}}%
\pgfpathlineto{\pgfqpoint{1.682465in}{2.464950in}}%
\pgfpathlineto{\pgfqpoint{1.682699in}{2.434115in}}%
\pgfpathlineto{\pgfqpoint{1.682933in}{2.444525in}}%
\pgfpathlineto{\pgfqpoint{1.683167in}{2.437500in}}%
\pgfpathlineto{\pgfqpoint{1.685271in}{2.175212in}}%
\pgfpathlineto{\pgfqpoint{1.685505in}{2.200769in}}%
\pgfpathlineto{\pgfqpoint{1.686207in}{2.160389in}}%
\pgfpathlineto{\pgfqpoint{1.687376in}{2.299534in}}%
\pgfpathlineto{\pgfqpoint{1.689247in}{2.414702in}}%
\pgfpathlineto{\pgfqpoint{1.689481in}{2.433579in}}%
\pgfpathlineto{\pgfqpoint{1.690183in}{2.411144in}}%
\pgfpathlineto{\pgfqpoint{1.691118in}{2.275594in}}%
\pgfpathlineto{\pgfqpoint{1.691586in}{2.342590in}}%
\pgfpathlineto{\pgfqpoint{1.691820in}{2.348081in}}%
\pgfpathlineto{\pgfqpoint{1.692053in}{2.328350in}}%
\pgfpathlineto{\pgfqpoint{1.692287in}{2.332143in}}%
\pgfpathlineto{\pgfqpoint{1.692521in}{2.330681in}}%
\pgfpathlineto{\pgfqpoint{1.694392in}{2.207656in}}%
\pgfpathlineto{\pgfqpoint{1.694626in}{2.200809in}}%
\pgfpathlineto{\pgfqpoint{1.696731in}{2.362871in}}%
\pgfpathlineto{\pgfqpoint{1.696964in}{2.365257in}}%
\pgfpathlineto{\pgfqpoint{1.698835in}{2.471040in}}%
\pgfpathlineto{\pgfqpoint{1.700238in}{2.221943in}}%
\pgfpathlineto{\pgfqpoint{1.700940in}{2.101743in}}%
\pgfpathlineto{\pgfqpoint{1.701408in}{2.130560in}}%
\pgfpathlineto{\pgfqpoint{1.703279in}{2.403278in}}%
\pgfpathlineto{\pgfqpoint{1.703512in}{2.402033in}}%
\pgfpathlineto{\pgfqpoint{1.704448in}{2.499012in}}%
\pgfpathlineto{\pgfqpoint{1.704682in}{2.468361in}}%
\pgfpathlineto{\pgfqpoint{1.706319in}{2.147257in}}%
\pgfpathlineto{\pgfqpoint{1.706553in}{2.176207in}}%
\pgfpathlineto{\pgfqpoint{1.707020in}{2.193298in}}%
\pgfpathlineto{\pgfqpoint{1.707722in}{2.119692in}}%
\pgfpathlineto{\pgfqpoint{1.708190in}{2.175800in}}%
\pgfpathlineto{\pgfqpoint{1.709359in}{2.455921in}}%
\pgfpathlineto{\pgfqpoint{1.710528in}{2.381747in}}%
\pgfpathlineto{\pgfqpoint{1.713335in}{2.082708in}}%
\pgfpathlineto{\pgfqpoint{1.713568in}{2.086984in}}%
\pgfpathlineto{\pgfqpoint{1.716141in}{2.371595in}}%
\pgfpathlineto{\pgfqpoint{1.717310in}{2.403791in}}%
\pgfpathlineto{\pgfqpoint{1.720350in}{2.014262in}}%
\pgfpathlineto{\pgfqpoint{1.721987in}{2.416308in}}%
\pgfpathlineto{\pgfqpoint{1.722923in}{2.290126in}}%
\pgfpathlineto{\pgfqpoint{1.723391in}{2.287834in}}%
\pgfpathlineto{\pgfqpoint{1.723624in}{2.296992in}}%
\pgfpathlineto{\pgfqpoint{1.724326in}{2.249133in}}%
\pgfpathlineto{\pgfqpoint{1.725028in}{2.114518in}}%
\pgfpathlineto{\pgfqpoint{1.725729in}{2.155168in}}%
\pgfpathlineto{\pgfqpoint{1.726665in}{2.338854in}}%
\pgfpathlineto{\pgfqpoint{1.727600in}{2.301523in}}%
\pgfpathlineto{\pgfqpoint{1.727834in}{2.302902in}}%
\pgfpathlineto{\pgfqpoint{1.728068in}{2.301196in}}%
\pgfpathlineto{\pgfqpoint{1.728535in}{2.349147in}}%
\pgfpathlineto{\pgfqpoint{1.729003in}{2.295934in}}%
\pgfpathlineto{\pgfqpoint{1.730172in}{2.078227in}}%
\pgfpathlineto{\pgfqpoint{1.730640in}{2.090510in}}%
\pgfpathlineto{\pgfqpoint{1.733213in}{2.432878in}}%
\pgfpathlineto{\pgfqpoint{1.733446in}{2.428314in}}%
\pgfpathlineto{\pgfqpoint{1.733680in}{2.443997in}}%
\pgfpathlineto{\pgfqpoint{1.735551in}{2.153376in}}%
\pgfpathlineto{\pgfqpoint{1.736253in}{2.047582in}}%
\pgfpathlineto{\pgfqpoint{1.736954in}{2.072271in}}%
\pgfpathlineto{\pgfqpoint{1.737656in}{2.247101in}}%
\pgfpathlineto{\pgfqpoint{1.739059in}{2.395773in}}%
\pgfpathlineto{\pgfqpoint{1.741865in}{2.008626in}}%
\pgfpathlineto{\pgfqpoint{1.742567in}{2.094989in}}%
\pgfpathlineto{\pgfqpoint{1.745607in}{2.331940in}}%
\pgfpathlineto{\pgfqpoint{1.748413in}{2.079556in}}%
\pgfpathlineto{\pgfqpoint{1.749115in}{2.159694in}}%
\pgfpathlineto{\pgfqpoint{1.749349in}{2.116177in}}%
\pgfpathlineto{\pgfqpoint{1.749817in}{2.043931in}}%
\pgfpathlineto{\pgfqpoint{1.750284in}{2.115618in}}%
\pgfpathlineto{\pgfqpoint{1.750752in}{2.151721in}}%
\pgfpathlineto{\pgfqpoint{1.752155in}{2.214394in}}%
\pgfpathlineto{\pgfqpoint{1.752389in}{2.210796in}}%
\pgfpathlineto{\pgfqpoint{1.753325in}{2.288444in}}%
\pgfpathlineto{\pgfqpoint{1.753558in}{2.246866in}}%
\pgfpathlineto{\pgfqpoint{1.754260in}{2.088534in}}%
\pgfpathlineto{\pgfqpoint{1.754962in}{2.143080in}}%
\pgfpathlineto{\pgfqpoint{1.757768in}{2.290441in}}%
\pgfpathlineto{\pgfqpoint{1.758703in}{2.064995in}}%
\pgfpathlineto{\pgfqpoint{1.759639in}{2.079674in}}%
\pgfpathlineto{\pgfqpoint{1.759873in}{2.081067in}}%
\pgfpathlineto{\pgfqpoint{1.760574in}{2.062231in}}%
\pgfpathlineto{\pgfqpoint{1.761276in}{2.243285in}}%
\pgfpathlineto{\pgfqpoint{1.762211in}{2.197972in}}%
\pgfpathlineto{\pgfqpoint{1.763848in}{2.083685in}}%
\pgfpathlineto{\pgfqpoint{1.764550in}{1.980083in}}%
\pgfpathlineto{\pgfqpoint{1.765017in}{2.054281in}}%
\pgfpathlineto{\pgfqpoint{1.766421in}{2.229337in}}%
\pgfpathlineto{\pgfqpoint{1.766888in}{2.176679in}}%
\pgfpathlineto{\pgfqpoint{1.767122in}{2.167329in}}%
\pgfpathlineto{\pgfqpoint{1.767356in}{2.197141in}}%
\pgfpathlineto{\pgfqpoint{1.768058in}{2.254923in}}%
\pgfpathlineto{\pgfqpoint{1.768525in}{2.233956in}}%
\pgfpathlineto{\pgfqpoint{1.769227in}{2.074199in}}%
\pgfpathlineto{\pgfqpoint{1.769929in}{2.107464in}}%
\pgfpathlineto{\pgfqpoint{1.770396in}{2.116037in}}%
\pgfpathlineto{\pgfqpoint{1.771098in}{2.075541in}}%
\pgfpathlineto{\pgfqpoint{1.771566in}{2.077255in}}%
\pgfpathlineto{\pgfqpoint{1.772033in}{2.130457in}}%
\pgfpathlineto{\pgfqpoint{1.772735in}{2.126706in}}%
\pgfpathlineto{\pgfqpoint{1.772969in}{2.114706in}}%
\pgfpathlineto{\pgfqpoint{1.773436in}{2.138685in}}%
\pgfpathlineto{\pgfqpoint{1.773904in}{2.182702in}}%
\pgfpathlineto{\pgfqpoint{1.774138in}{2.134235in}}%
\pgfpathlineto{\pgfqpoint{1.775307in}{2.004316in}}%
\pgfpathlineto{\pgfqpoint{1.775541in}{2.027765in}}%
\pgfpathlineto{\pgfqpoint{1.777178in}{2.296971in}}%
\pgfpathlineto{\pgfqpoint{1.777412in}{2.279215in}}%
\pgfpathlineto{\pgfqpoint{1.779517in}{2.026172in}}%
\pgfpathlineto{\pgfqpoint{1.779751in}{2.063834in}}%
\pgfpathlineto{\pgfqpoint{1.780218in}{2.161284in}}%
\pgfpathlineto{\pgfqpoint{1.780686in}{2.050105in}}%
\pgfpathlineto{\pgfqpoint{1.780920in}{2.065780in}}%
\pgfpathlineto{\pgfqpoint{1.781154in}{2.048175in}}%
\pgfpathlineto{\pgfqpoint{1.782557in}{1.927067in}}%
\pgfpathlineto{\pgfqpoint{1.782791in}{1.948437in}}%
\pgfpathlineto{\pgfqpoint{1.784428in}{2.297499in}}%
\pgfpathlineto{\pgfqpoint{1.784662in}{2.248009in}}%
\pgfpathlineto{\pgfqpoint{1.786766in}{1.978087in}}%
\pgfpathlineto{\pgfqpoint{1.787234in}{1.943436in}}%
\pgfpathlineto{\pgfqpoint{1.787468in}{1.978350in}}%
\pgfpathlineto{\pgfqpoint{1.788403in}{2.112921in}}%
\pgfpathlineto{\pgfqpoint{1.788871in}{2.066237in}}%
\pgfpathlineto{\pgfqpoint{1.789105in}{1.994167in}}%
\pgfpathlineto{\pgfqpoint{1.790040in}{2.035953in}}%
\pgfpathlineto{\pgfqpoint{1.790274in}{2.035577in}}%
\pgfpathlineto{\pgfqpoint{1.791210in}{2.034257in}}%
\pgfpathlineto{\pgfqpoint{1.792145in}{2.156734in}}%
\pgfpathlineto{\pgfqpoint{1.793548in}{2.050123in}}%
\pgfpathlineto{\pgfqpoint{1.793782in}{2.060659in}}%
\pgfpathlineto{\pgfqpoint{1.794484in}{1.992024in}}%
\pgfpathlineto{\pgfqpoint{1.794951in}{2.056051in}}%
\pgfpathlineto{\pgfqpoint{1.795887in}{2.135460in}}%
\pgfpathlineto{\pgfqpoint{1.796121in}{2.082896in}}%
\pgfpathlineto{\pgfqpoint{1.797056in}{1.990573in}}%
\pgfpathlineto{\pgfqpoint{1.797524in}{2.055437in}}%
\pgfpathlineto{\pgfqpoint{1.799161in}{2.174918in}}%
\pgfpathlineto{\pgfqpoint{1.800798in}{1.826330in}}%
\pgfpathlineto{\pgfqpoint{1.801733in}{1.955253in}}%
\pgfpathlineto{\pgfqpoint{1.803137in}{2.243422in}}%
\pgfpathlineto{\pgfqpoint{1.803604in}{2.207259in}}%
\pgfpathlineto{\pgfqpoint{1.805709in}{1.982658in}}%
\pgfpathlineto{\pgfqpoint{1.805943in}{1.977699in}}%
\pgfpathlineto{\pgfqpoint{1.806177in}{1.994283in}}%
\pgfpathlineto{\pgfqpoint{1.806411in}{2.018033in}}%
\pgfpathlineto{\pgfqpoint{1.807112in}{1.981759in}}%
\pgfpathlineto{\pgfqpoint{1.808281in}{1.863605in}}%
\pgfpathlineto{\pgfqpoint{1.808749in}{1.881619in}}%
\pgfpathlineto{\pgfqpoint{1.810620in}{2.158970in}}%
\pgfpathlineto{\pgfqpoint{1.811322in}{2.135718in}}%
\pgfpathlineto{\pgfqpoint{1.812257in}{1.967293in}}%
\pgfpathlineto{\pgfqpoint{1.812959in}{2.020462in}}%
\pgfpathlineto{\pgfqpoint{1.813426in}{2.095731in}}%
\pgfpathlineto{\pgfqpoint{1.813660in}{2.076524in}}%
\pgfpathlineto{\pgfqpoint{1.814830in}{1.824700in}}%
\pgfpathlineto{\pgfqpoint{1.815297in}{1.860797in}}%
\pgfpathlineto{\pgfqpoint{1.816934in}{1.985562in}}%
\pgfpathlineto{\pgfqpoint{1.817402in}{2.078113in}}%
\pgfpathlineto{\pgfqpoint{1.817870in}{2.009494in}}%
\pgfpathlineto{\pgfqpoint{1.818337in}{1.952491in}}%
\pgfpathlineto{\pgfqpoint{1.819039in}{1.978995in}}%
\pgfpathlineto{\pgfqpoint{1.820910in}{2.082985in}}%
\pgfpathlineto{\pgfqpoint{1.822547in}{1.882403in}}%
\pgfpathlineto{\pgfqpoint{1.823248in}{1.893789in}}%
\pgfpathlineto{\pgfqpoint{1.824652in}{2.017862in}}%
\pgfpathlineto{\pgfqpoint{1.825119in}{1.984721in}}%
\pgfpathlineto{\pgfqpoint{1.826289in}{1.800598in}}%
\pgfpathlineto{\pgfqpoint{1.826522in}{1.830582in}}%
\pgfpathlineto{\pgfqpoint{1.827926in}{2.016310in}}%
\pgfpathlineto{\pgfqpoint{1.828159in}{2.012814in}}%
\pgfpathlineto{\pgfqpoint{1.828393in}{2.009663in}}%
\pgfpathlineto{\pgfqpoint{1.828861in}{2.063762in}}%
\pgfpathlineto{\pgfqpoint{1.829095in}{2.056852in}}%
\pgfpathlineto{\pgfqpoint{1.829797in}{1.939363in}}%
\pgfpathlineto{\pgfqpoint{1.830264in}{2.020737in}}%
\pgfpathlineto{\pgfqpoint{1.830732in}{2.111012in}}%
\pgfpathlineto{\pgfqpoint{1.831200in}{2.094119in}}%
\pgfpathlineto{\pgfqpoint{1.832603in}{1.794936in}}%
\pgfpathlineto{\pgfqpoint{1.832837in}{1.797090in}}%
\pgfpathlineto{\pgfqpoint{1.833071in}{1.801906in}}%
\pgfpathlineto{\pgfqpoint{1.833538in}{1.750134in}}%
\pgfpathlineto{\pgfqpoint{1.833772in}{1.822094in}}%
\pgfpathlineto{\pgfqpoint{1.834941in}{1.957834in}}%
\pgfpathlineto{\pgfqpoint{1.835877in}{1.947378in}}%
\pgfpathlineto{\pgfqpoint{1.836345in}{1.922681in}}%
\pgfpathlineto{\pgfqpoint{1.836578in}{1.929228in}}%
\pgfpathlineto{\pgfqpoint{1.837982in}{2.093746in}}%
\pgfpathlineto{\pgfqpoint{1.839852in}{1.769828in}}%
\pgfpathlineto{\pgfqpoint{1.840086in}{1.777506in}}%
\pgfpathlineto{\pgfqpoint{1.841022in}{1.846142in}}%
\pgfpathlineto{\pgfqpoint{1.841489in}{1.804076in}}%
\pgfpathlineto{\pgfqpoint{1.841957in}{1.850963in}}%
\pgfpathlineto{\pgfqpoint{1.842425in}{1.812020in}}%
\pgfpathlineto{\pgfqpoint{1.842659in}{1.795291in}}%
\pgfpathlineto{\pgfqpoint{1.842893in}{1.842016in}}%
\pgfpathlineto{\pgfqpoint{1.844062in}{2.135062in}}%
\pgfpathlineto{\pgfqpoint{1.844764in}{2.021028in}}%
\pgfpathlineto{\pgfqpoint{1.846868in}{1.763333in}}%
\pgfpathlineto{\pgfqpoint{1.847336in}{1.793760in}}%
\pgfpathlineto{\pgfqpoint{1.848038in}{1.868000in}}%
\pgfpathlineto{\pgfqpoint{1.848505in}{1.831017in}}%
\pgfpathlineto{\pgfqpoint{1.848739in}{1.815544in}}%
\pgfpathlineto{\pgfqpoint{1.848973in}{1.864225in}}%
\pgfpathlineto{\pgfqpoint{1.849207in}{1.847333in}}%
\pgfpathlineto{\pgfqpoint{1.849675in}{1.861511in}}%
\pgfpathlineto{\pgfqpoint{1.851779in}{1.983295in}}%
\pgfpathlineto{\pgfqpoint{1.850376in}{1.855152in}}%
\pgfpathlineto{\pgfqpoint{1.852013in}{1.982394in}}%
\pgfpathlineto{\pgfqpoint{1.852481in}{1.954937in}}%
\pgfpathlineto{\pgfqpoint{1.852715in}{2.009230in}}%
\pgfpathlineto{\pgfqpoint{1.852949in}{2.026253in}}%
\pgfpathlineto{\pgfqpoint{1.853182in}{2.001974in}}%
\pgfpathlineto{\pgfqpoint{1.855053in}{1.768434in}}%
\pgfpathlineto{\pgfqpoint{1.855287in}{1.743176in}}%
\pgfpathlineto{\pgfqpoint{1.855755in}{1.782485in}}%
\pgfpathlineto{\pgfqpoint{1.856690in}{1.894154in}}%
\pgfpathlineto{\pgfqpoint{1.856924in}{1.870340in}}%
\pgfpathlineto{\pgfqpoint{1.857392in}{1.803755in}}%
\pgfpathlineto{\pgfqpoint{1.857626in}{1.835803in}}%
\pgfpathlineto{\pgfqpoint{1.858561in}{1.978886in}}%
\pgfpathlineto{\pgfqpoint{1.858795in}{1.961033in}}%
\pgfpathlineto{\pgfqpoint{1.859497in}{1.859270in}}%
\pgfpathlineto{\pgfqpoint{1.860432in}{1.876563in}}%
\pgfpathlineto{\pgfqpoint{1.860666in}{1.912558in}}%
\pgfpathlineto{\pgfqpoint{1.861368in}{1.862825in}}%
\pgfpathlineto{\pgfqpoint{1.861601in}{1.905862in}}%
\pgfpathlineto{\pgfqpoint{1.863238in}{1.751414in}}%
\pgfpathlineto{\pgfqpoint{1.863472in}{1.828799in}}%
\pgfpathlineto{\pgfqpoint{1.863706in}{1.879526in}}%
\pgfpathlineto{\pgfqpoint{1.864408in}{1.822915in}}%
\pgfpathlineto{\pgfqpoint{1.864642in}{1.864438in}}%
\pgfpathlineto{\pgfqpoint{1.864875in}{1.831408in}}%
\pgfpathlineto{\pgfqpoint{1.865343in}{1.905156in}}%
\pgfpathlineto{\pgfqpoint{1.866279in}{1.993622in}}%
\pgfpathlineto{\pgfqpoint{1.866512in}{1.978046in}}%
\pgfpathlineto{\pgfqpoint{1.867916in}{1.746592in}}%
\pgfpathlineto{\pgfqpoint{1.868383in}{1.824606in}}%
\pgfpathlineto{\pgfqpoint{1.868617in}{1.825536in}}%
\pgfpathlineto{\pgfqpoint{1.869085in}{1.880781in}}%
\pgfpathlineto{\pgfqpoint{1.869786in}{1.860936in}}%
\pgfpathlineto{\pgfqpoint{1.870020in}{1.875467in}}%
\pgfpathlineto{\pgfqpoint{1.870488in}{1.852270in}}%
\pgfpathlineto{\pgfqpoint{1.871190in}{1.761847in}}%
\pgfpathlineto{\pgfqpoint{1.871657in}{1.817546in}}%
\pgfpathlineto{\pgfqpoint{1.871891in}{1.861106in}}%
\pgfpathlineto{\pgfqpoint{1.872593in}{1.846100in}}%
\pgfpathlineto{\pgfqpoint{1.872827in}{1.797280in}}%
\pgfpathlineto{\pgfqpoint{1.873060in}{1.846351in}}%
\pgfpathlineto{\pgfqpoint{1.873762in}{1.824736in}}%
\pgfpathlineto{\pgfqpoint{1.875165in}{1.930247in}}%
\pgfpathlineto{\pgfqpoint{1.875399in}{1.900323in}}%
\pgfpathlineto{\pgfqpoint{1.876568in}{1.728503in}}%
\pgfpathlineto{\pgfqpoint{1.876802in}{1.813141in}}%
\pgfpathlineto{\pgfqpoint{1.877036in}{1.838723in}}%
\pgfpathlineto{\pgfqpoint{1.877270in}{1.809866in}}%
\pgfpathlineto{\pgfqpoint{1.877504in}{1.759593in}}%
\pgfpathlineto{\pgfqpoint{1.877972in}{1.888417in}}%
\pgfpathlineto{\pgfqpoint{1.878907in}{2.061976in}}%
\pgfpathlineto{\pgfqpoint{1.879141in}{1.983422in}}%
\pgfpathlineto{\pgfqpoint{1.880310in}{1.712652in}}%
\pgfpathlineto{\pgfqpoint{1.880778in}{1.753752in}}%
\pgfpathlineto{\pgfqpoint{1.881012in}{1.784351in}}%
\pgfpathlineto{\pgfqpoint{1.881246in}{1.733302in}}%
\pgfpathlineto{\pgfqpoint{1.881947in}{1.775646in}}%
\pgfpathlineto{\pgfqpoint{1.882415in}{1.782706in}}%
\pgfpathlineto{\pgfqpoint{1.883116in}{1.713762in}}%
\pgfpathlineto{\pgfqpoint{1.884520in}{1.906675in}}%
\pgfpathlineto{\pgfqpoint{1.884987in}{1.875761in}}%
\pgfpathlineto{\pgfqpoint{1.885455in}{1.905765in}}%
\pgfpathlineto{\pgfqpoint{1.885689in}{1.859169in}}%
\pgfpathlineto{\pgfqpoint{1.886390in}{1.922824in}}%
\pgfpathlineto{\pgfqpoint{1.886858in}{1.906963in}}%
\pgfpathlineto{\pgfqpoint{1.887092in}{1.957260in}}%
\pgfpathlineto{\pgfqpoint{1.887560in}{1.925080in}}%
\pgfpathlineto{\pgfqpoint{1.888729in}{1.808393in}}%
\pgfpathlineto{\pgfqpoint{1.888963in}{1.832113in}}%
\pgfpathlineto{\pgfqpoint{1.889197in}{1.829056in}}%
\pgfpathlineto{\pgfqpoint{1.891302in}{1.681905in}}%
\pgfpathlineto{\pgfqpoint{1.891535in}{1.693958in}}%
\pgfpathlineto{\pgfqpoint{1.893874in}{2.035908in}}%
\pgfpathlineto{\pgfqpoint{1.894108in}{2.022065in}}%
\pgfpathlineto{\pgfqpoint{1.895277in}{1.661643in}}%
\pgfpathlineto{\pgfqpoint{1.895979in}{1.687044in}}%
\pgfpathlineto{\pgfqpoint{1.897616in}{1.829857in}}%
\pgfpathlineto{\pgfqpoint{1.898317in}{1.798942in}}%
\pgfpathlineto{\pgfqpoint{1.898551in}{1.762984in}}%
\pgfpathlineto{\pgfqpoint{1.899019in}{1.830050in}}%
\pgfpathlineto{\pgfqpoint{1.900188in}{1.900725in}}%
\pgfpathlineto{\pgfqpoint{1.900422in}{1.896945in}}%
\pgfpathlineto{\pgfqpoint{1.900656in}{1.899681in}}%
\pgfpathlineto{\pgfqpoint{1.901591in}{1.878363in}}%
\pgfpathlineto{\pgfqpoint{1.902059in}{1.940331in}}%
\pgfpathlineto{\pgfqpoint{1.902527in}{1.916596in}}%
\pgfpathlineto{\pgfqpoint{1.902761in}{1.955425in}}%
\pgfpathlineto{\pgfqpoint{1.903228in}{1.870833in}}%
\pgfpathlineto{\pgfqpoint{1.904164in}{1.613397in}}%
\pgfpathlineto{\pgfqpoint{1.904631in}{1.675422in}}%
\pgfpathlineto{\pgfqpoint{1.908373in}{2.023167in}}%
\pgfpathlineto{\pgfqpoint{1.909075in}{1.852732in}}%
\pgfpathlineto{\pgfqpoint{1.909776in}{1.855388in}}%
\pgfpathlineto{\pgfqpoint{1.910010in}{1.865080in}}%
\pgfpathlineto{\pgfqpoint{1.911647in}{1.658030in}}%
\pgfpathlineto{\pgfqpoint{1.911881in}{1.677635in}}%
\pgfpathlineto{\pgfqpoint{1.912115in}{1.668969in}}%
\pgfpathlineto{\pgfqpoint{1.912349in}{1.690179in}}%
\pgfpathlineto{\pgfqpoint{1.912583in}{1.678511in}}%
\pgfpathlineto{\pgfqpoint{1.913752in}{1.933767in}}%
\pgfpathlineto{\pgfqpoint{1.914220in}{1.888951in}}%
\pgfpathlineto{\pgfqpoint{1.915155in}{2.011897in}}%
\pgfpathlineto{\pgfqpoint{1.915623in}{1.958562in}}%
\pgfpathlineto{\pgfqpoint{1.917026in}{1.698650in}}%
\pgfpathlineto{\pgfqpoint{1.917260in}{1.700398in}}%
\pgfpathlineto{\pgfqpoint{1.917961in}{1.643687in}}%
\pgfpathlineto{\pgfqpoint{1.918195in}{1.670994in}}%
\pgfpathlineto{\pgfqpoint{1.918897in}{1.666099in}}%
\pgfpathlineto{\pgfqpoint{1.921235in}{1.970444in}}%
\pgfpathlineto{\pgfqpoint{1.922405in}{1.836299in}}%
\pgfpathlineto{\pgfqpoint{1.924510in}{1.601426in}}%
\pgfpathlineto{\pgfqpoint{1.924743in}{1.642363in}}%
\pgfpathlineto{\pgfqpoint{1.926147in}{1.810977in}}%
\pgfpathlineto{\pgfqpoint{1.926614in}{1.969827in}}%
\pgfpathlineto{\pgfqpoint{1.927316in}{1.896191in}}%
\pgfpathlineto{\pgfqpoint{1.927784in}{1.870948in}}%
\pgfpathlineto{\pgfqpoint{1.928017in}{1.886698in}}%
\pgfpathlineto{\pgfqpoint{1.928251in}{1.926303in}}%
\pgfpathlineto{\pgfqpoint{1.928719in}{1.832800in}}%
\pgfpathlineto{\pgfqpoint{1.930590in}{1.646774in}}%
\pgfpathlineto{\pgfqpoint{1.931291in}{1.579843in}}%
\pgfpathlineto{\pgfqpoint{1.931525in}{1.615334in}}%
\pgfpathlineto{\pgfqpoint{1.933630in}{1.998063in}}%
\pgfpathlineto{\pgfqpoint{1.934332in}{1.801220in}}%
\pgfpathlineto{\pgfqpoint{1.935267in}{1.834045in}}%
\pgfpathlineto{\pgfqpoint{1.935735in}{1.759348in}}%
\pgfpathlineto{\pgfqpoint{1.936436in}{1.599217in}}%
\pgfpathlineto{\pgfqpoint{1.937138in}{1.666800in}}%
\pgfpathlineto{\pgfqpoint{1.937606in}{1.666147in}}%
\pgfpathlineto{\pgfqpoint{1.938073in}{1.731023in}}%
\pgfpathlineto{\pgfqpoint{1.938775in}{1.804504in}}%
\pgfpathlineto{\pgfqpoint{1.939243in}{1.779666in}}%
\pgfpathlineto{\pgfqpoint{1.939477in}{1.761187in}}%
\pgfpathlineto{\pgfqpoint{1.939710in}{1.798191in}}%
\pgfpathlineto{\pgfqpoint{1.941114in}{1.911765in}}%
\pgfpathlineto{\pgfqpoint{1.944855in}{1.580065in}}%
\pgfpathlineto{\pgfqpoint{1.941815in}{1.925195in}}%
\pgfpathlineto{\pgfqpoint{1.945557in}{1.660811in}}%
\pgfpathlineto{\pgfqpoint{1.947662in}{1.903724in}}%
\pgfpathlineto{\pgfqpoint{1.947895in}{1.963638in}}%
\pgfpathlineto{\pgfqpoint{1.948597in}{1.863158in}}%
\pgfpathlineto{\pgfqpoint{1.948831in}{1.867912in}}%
\pgfpathlineto{\pgfqpoint{1.950234in}{1.557210in}}%
\pgfpathlineto{\pgfqpoint{1.950936in}{1.676744in}}%
\pgfpathlineto{\pgfqpoint{1.952339in}{1.777567in}}%
\pgfpathlineto{\pgfqpoint{1.952573in}{1.744021in}}%
\pgfpathlineto{\pgfqpoint{1.953274in}{1.783612in}}%
\pgfpathlineto{\pgfqpoint{1.953508in}{1.779815in}}%
\pgfpathlineto{\pgfqpoint{1.953742in}{1.791810in}}%
\pgfpathlineto{\pgfqpoint{1.953976in}{1.807794in}}%
\pgfpathlineto{\pgfqpoint{1.954210in}{1.760901in}}%
\pgfpathlineto{\pgfqpoint{1.954911in}{1.683791in}}%
\pgfpathlineto{\pgfqpoint{1.955379in}{1.714927in}}%
\pgfpathlineto{\pgfqpoint{1.956548in}{1.794535in}}%
\pgfpathlineto{\pgfqpoint{1.956782in}{1.778038in}}%
\pgfpathlineto{\pgfqpoint{1.958185in}{1.878172in}}%
\pgfpathlineto{\pgfqpoint{1.960290in}{1.646355in}}%
\pgfpathlineto{\pgfqpoint{1.960524in}{1.690433in}}%
\pgfpathlineto{\pgfqpoint{1.961459in}{1.671529in}}%
\pgfpathlineto{\pgfqpoint{1.961693in}{1.640387in}}%
\pgfpathlineto{\pgfqpoint{1.962161in}{1.681185in}}%
\pgfpathlineto{\pgfqpoint{1.963330in}{1.814599in}}%
\pgfpathlineto{\pgfqpoint{1.963564in}{1.787414in}}%
\pgfpathlineto{\pgfqpoint{1.964499in}{1.897189in}}%
\pgfpathlineto{\pgfqpoint{1.965201in}{1.880199in}}%
\pgfpathlineto{\pgfqpoint{1.967072in}{1.615502in}}%
\pgfpathlineto{\pgfqpoint{1.967306in}{1.636540in}}%
\pgfpathlineto{\pgfqpoint{1.968709in}{1.798096in}}%
\pgfpathlineto{\pgfqpoint{1.968943in}{1.796932in}}%
\pgfpathlineto{\pgfqpoint{1.969177in}{1.821386in}}%
\pgfpathlineto{\pgfqpoint{1.969411in}{1.766384in}}%
\pgfpathlineto{\pgfqpoint{1.970346in}{1.630234in}}%
\pgfpathlineto{\pgfqpoint{1.971048in}{1.682443in}}%
\pgfpathlineto{\pgfqpoint{1.971515in}{1.683637in}}%
\pgfpathlineto{\pgfqpoint{1.971983in}{1.650230in}}%
\pgfpathlineto{\pgfqpoint{1.974088in}{1.972151in}}%
\pgfpathlineto{\pgfqpoint{1.976192in}{1.659285in}}%
\pgfpathlineto{\pgfqpoint{1.976426in}{1.665206in}}%
\pgfpathlineto{\pgfqpoint{1.976660in}{1.666855in}}%
\pgfpathlineto{\pgfqpoint{1.977596in}{1.598198in}}%
\pgfpathlineto{\pgfqpoint{1.978999in}{1.835667in}}%
\pgfpathlineto{\pgfqpoint{1.979233in}{1.817482in}}%
\pgfpathlineto{\pgfqpoint{1.979466in}{1.783901in}}%
\pgfpathlineto{\pgfqpoint{1.979934in}{1.846735in}}%
\pgfpathlineto{\pgfqpoint{1.980168in}{1.829985in}}%
\pgfpathlineto{\pgfqpoint{1.981103in}{1.829478in}}%
\pgfpathlineto{\pgfqpoint{1.981571in}{1.885675in}}%
\pgfpathlineto{\pgfqpoint{1.981805in}{1.858028in}}%
\pgfpathlineto{\pgfqpoint{1.982273in}{1.890891in}}%
\pgfpathlineto{\pgfqpoint{1.982507in}{1.869941in}}%
\pgfpathlineto{\pgfqpoint{1.982974in}{1.947674in}}%
\pgfpathlineto{\pgfqpoint{1.983442in}{1.849709in}}%
\pgfpathlineto{\pgfqpoint{1.984611in}{1.645336in}}%
\pgfpathlineto{\pgfqpoint{1.985781in}{1.673159in}}%
\pgfpathlineto{\pgfqpoint{1.987885in}{1.949233in}}%
\pgfpathlineto{\pgfqpoint{1.989756in}{1.801358in}}%
\pgfpathlineto{\pgfqpoint{1.989990in}{1.828790in}}%
\pgfpathlineto{\pgfqpoint{1.990458in}{1.788629in}}%
\pgfpathlineto{\pgfqpoint{1.990926in}{1.808789in}}%
\pgfpathlineto{\pgfqpoint{1.992095in}{1.713770in}}%
\pgfpathlineto{\pgfqpoint{1.992329in}{1.772206in}}%
\pgfpathlineto{\pgfqpoint{1.993264in}{1.760468in}}%
\pgfpathlineto{\pgfqpoint{1.994200in}{1.690499in}}%
\pgfpathlineto{\pgfqpoint{1.994433in}{1.724099in}}%
\pgfpathlineto{\pgfqpoint{1.996304in}{1.883789in}}%
\pgfpathlineto{\pgfqpoint{1.996772in}{1.827396in}}%
\pgfpathlineto{\pgfqpoint{1.997474in}{1.725382in}}%
\pgfpathlineto{\pgfqpoint{1.997941in}{1.813536in}}%
\pgfpathlineto{\pgfqpoint{1.998643in}{1.862323in}}%
\pgfpathlineto{\pgfqpoint{1.998877in}{1.828950in}}%
\pgfpathlineto{\pgfqpoint{1.999111in}{1.793379in}}%
\pgfpathlineto{\pgfqpoint{1.999578in}{1.864019in}}%
\pgfpathlineto{\pgfqpoint{1.999812in}{1.811276in}}%
\pgfpathlineto{\pgfqpoint{2.000280in}{1.842660in}}%
\pgfpathlineto{\pgfqpoint{2.000748in}{1.797440in}}%
\pgfpathlineto{\pgfqpoint{2.000982in}{1.827984in}}%
\pgfpathlineto{\pgfqpoint{2.001449in}{1.697719in}}%
\pgfpathlineto{\pgfqpoint{2.001917in}{1.789596in}}%
\pgfpathlineto{\pgfqpoint{2.002852in}{1.990276in}}%
\pgfpathlineto{\pgfqpoint{2.003320in}{1.948349in}}%
\pgfpathlineto{\pgfqpoint{2.005191in}{1.737518in}}%
\pgfpathlineto{\pgfqpoint{2.006360in}{1.677078in}}%
\pgfpathlineto{\pgfqpoint{2.007530in}{1.985870in}}%
\pgfpathlineto{\pgfqpoint{2.008231in}{1.964746in}}%
\pgfpathlineto{\pgfqpoint{2.010804in}{1.733185in}}%
\pgfpathlineto{\pgfqpoint{2.008933in}{1.980361in}}%
\pgfpathlineto{\pgfqpoint{2.011505in}{1.782152in}}%
\pgfpathlineto{\pgfqpoint{2.011973in}{1.808373in}}%
\pgfpathlineto{\pgfqpoint{2.012207in}{1.759262in}}%
\pgfpathlineto{\pgfqpoint{2.012674in}{1.667558in}}%
\pgfpathlineto{\pgfqpoint{2.013376in}{1.735050in}}%
\pgfpathlineto{\pgfqpoint{2.016182in}{2.034829in}}%
\pgfpathlineto{\pgfqpoint{2.016650in}{1.944262in}}%
\pgfpathlineto{\pgfqpoint{2.018053in}{1.771041in}}%
\pgfpathlineto{\pgfqpoint{2.018287in}{1.773626in}}%
\pgfpathlineto{\pgfqpoint{2.018989in}{1.942386in}}%
\pgfpathlineto{\pgfqpoint{2.019456in}{1.827939in}}%
\pgfpathlineto{\pgfqpoint{2.019924in}{1.703489in}}%
\pgfpathlineto{\pgfqpoint{2.020860in}{1.722476in}}%
\pgfpathlineto{\pgfqpoint{2.022029in}{1.952536in}}%
\pgfpathlineto{\pgfqpoint{2.022263in}{1.943681in}}%
\pgfpathlineto{\pgfqpoint{2.022730in}{1.978473in}}%
\pgfpathlineto{\pgfqpoint{2.024601in}{1.813057in}}%
\pgfpathlineto{\pgfqpoint{2.024835in}{1.819671in}}%
\pgfpathlineto{\pgfqpoint{2.025771in}{1.950196in}}%
\pgfpathlineto{\pgfqpoint{2.026004in}{1.940263in}}%
\pgfpathlineto{\pgfqpoint{2.027174in}{1.742611in}}%
\pgfpathlineto{\pgfqpoint{2.027641in}{1.759231in}}%
\pgfpathlineto{\pgfqpoint{2.029278in}{2.012591in}}%
\pgfpathlineto{\pgfqpoint{2.029512in}{1.987647in}}%
\pgfpathlineto{\pgfqpoint{2.029746in}{1.976818in}}%
\pgfpathlineto{\pgfqpoint{2.030916in}{1.776742in}}%
\pgfpathlineto{\pgfqpoint{2.031149in}{1.798153in}}%
\pgfpathlineto{\pgfqpoint{2.031851in}{1.854441in}}%
\pgfpathlineto{\pgfqpoint{2.032553in}{1.847531in}}%
\pgfpathlineto{\pgfqpoint{2.033020in}{1.746539in}}%
\pgfpathlineto{\pgfqpoint{2.033722in}{1.773654in}}%
\pgfpathlineto{\pgfqpoint{2.034657in}{1.853609in}}%
\pgfpathlineto{\pgfqpoint{2.035359in}{2.036480in}}%
\pgfpathlineto{\pgfqpoint{2.035827in}{2.028587in}}%
\pgfpathlineto{\pgfqpoint{2.037230in}{1.776949in}}%
\pgfpathlineto{\pgfqpoint{2.037464in}{1.807387in}}%
\pgfpathlineto{\pgfqpoint{2.037697in}{1.821482in}}%
\pgfpathlineto{\pgfqpoint{2.038399in}{1.976292in}}%
\pgfpathlineto{\pgfqpoint{2.038867in}{1.873773in}}%
\pgfpathlineto{\pgfqpoint{2.039101in}{1.813826in}}%
\pgfpathlineto{\pgfqpoint{2.039802in}{1.868163in}}%
\pgfpathlineto{\pgfqpoint{2.040036in}{1.878751in}}%
\pgfpathlineto{\pgfqpoint{2.040504in}{1.775344in}}%
\pgfpathlineto{\pgfqpoint{2.041205in}{1.801313in}}%
\pgfpathlineto{\pgfqpoint{2.042375in}{1.901690in}}%
\pgfpathlineto{\pgfqpoint{2.042608in}{1.891168in}}%
\pgfpathlineto{\pgfqpoint{2.043544in}{1.812562in}}%
\pgfpathlineto{\pgfqpoint{2.043778in}{1.850556in}}%
\pgfpathlineto{\pgfqpoint{2.044245in}{1.846025in}}%
\pgfpathlineto{\pgfqpoint{2.044713in}{1.930762in}}%
\pgfpathlineto{\pgfqpoint{2.046116in}{2.068487in}}%
\pgfpathlineto{\pgfqpoint{2.046350in}{2.060260in}}%
\pgfpathlineto{\pgfqpoint{2.048221in}{1.712378in}}%
\pgfpathlineto{\pgfqpoint{2.048923in}{1.744096in}}%
\pgfpathlineto{\pgfqpoint{2.049858in}{1.978253in}}%
\pgfpathlineto{\pgfqpoint{2.050560in}{1.944882in}}%
\pgfpathlineto{\pgfqpoint{2.050794in}{1.946181in}}%
\pgfpathlineto{\pgfqpoint{2.051729in}{1.883496in}}%
\pgfpathlineto{\pgfqpoint{2.052197in}{1.988567in}}%
\pgfpathlineto{\pgfqpoint{2.052664in}{1.958473in}}%
\pgfpathlineto{\pgfqpoint{2.053366in}{1.778883in}}%
\pgfpathlineto{\pgfqpoint{2.054068in}{1.818029in}}%
\pgfpathlineto{\pgfqpoint{2.055003in}{1.879932in}}%
\pgfpathlineto{\pgfqpoint{2.056406in}{2.034795in}}%
\pgfpathlineto{\pgfqpoint{2.056640in}{2.038273in}}%
\pgfpathlineto{\pgfqpoint{2.057809in}{1.802405in}}%
\pgfpathlineto{\pgfqpoint{2.058043in}{1.854982in}}%
\pgfpathlineto{\pgfqpoint{2.058979in}{1.917084in}}%
\pgfpathlineto{\pgfqpoint{2.059212in}{1.868549in}}%
\pgfpathlineto{\pgfqpoint{2.060148in}{1.744454in}}%
\pgfpathlineto{\pgfqpoint{2.060616in}{1.752784in}}%
\pgfpathlineto{\pgfqpoint{2.062487in}{2.068209in}}%
\pgfpathlineto{\pgfqpoint{2.062720in}{2.027880in}}%
\pgfpathlineto{\pgfqpoint{2.065293in}{1.798285in}}%
\pgfpathlineto{\pgfqpoint{2.066228in}{1.927735in}}%
\pgfpathlineto{\pgfqpoint{2.066462in}{1.897375in}}%
\pgfpathlineto{\pgfqpoint{2.067398in}{1.732434in}}%
\pgfpathlineto{\pgfqpoint{2.067865in}{1.760520in}}%
\pgfpathlineto{\pgfqpoint{2.070672in}{2.050717in}}%
\pgfpathlineto{\pgfqpoint{2.071139in}{2.012974in}}%
\pgfpathlineto{\pgfqpoint{2.073010in}{1.763145in}}%
\pgfpathlineto{\pgfqpoint{2.073244in}{1.772865in}}%
\pgfpathlineto{\pgfqpoint{2.073478in}{1.807550in}}%
\pgfpathlineto{\pgfqpoint{2.073946in}{1.765548in}}%
\pgfpathlineto{\pgfqpoint{2.074179in}{1.737758in}}%
\pgfpathlineto{\pgfqpoint{2.074413in}{1.779799in}}%
\pgfpathlineto{\pgfqpoint{2.074647in}{1.773601in}}%
\pgfpathlineto{\pgfqpoint{2.075583in}{1.849738in}}%
\pgfpathlineto{\pgfqpoint{2.076050in}{1.848945in}}%
\pgfpathlineto{\pgfqpoint{2.076284in}{1.822246in}}%
\pgfpathlineto{\pgfqpoint{2.076518in}{1.865371in}}%
\pgfpathlineto{\pgfqpoint{2.076986in}{1.959777in}}%
\pgfpathlineto{\pgfqpoint{2.077687in}{1.900541in}}%
\pgfpathlineto{\pgfqpoint{2.077921in}{1.843959in}}%
\pgfpathlineto{\pgfqpoint{2.078389in}{1.933300in}}%
\pgfpathlineto{\pgfqpoint{2.079792in}{2.003918in}}%
\pgfpathlineto{\pgfqpoint{2.080728in}{1.912275in}}%
\pgfpathlineto{\pgfqpoint{2.080961in}{1.978359in}}%
\pgfpathlineto{\pgfqpoint{2.081195in}{1.987936in}}%
\pgfpathlineto{\pgfqpoint{2.081429in}{1.970478in}}%
\pgfpathlineto{\pgfqpoint{2.082832in}{1.746721in}}%
\pgfpathlineto{\pgfqpoint{2.083066in}{1.754605in}}%
\pgfpathlineto{\pgfqpoint{2.085171in}{1.973990in}}%
\pgfpathlineto{\pgfqpoint{2.085639in}{1.932481in}}%
\pgfpathlineto{\pgfqpoint{2.086106in}{1.857644in}}%
\pgfpathlineto{\pgfqpoint{2.086340in}{1.940780in}}%
\pgfpathlineto{\pgfqpoint{2.086574in}{1.926325in}}%
\pgfpathlineto{\pgfqpoint{2.087042in}{1.945886in}}%
\pgfpathlineto{\pgfqpoint{2.087276in}{1.919427in}}%
\pgfpathlineto{\pgfqpoint{2.088913in}{1.808689in}}%
\pgfpathlineto{\pgfqpoint{2.089848in}{1.915468in}}%
\pgfpathlineto{\pgfqpoint{2.090082in}{1.891393in}}%
\pgfpathlineto{\pgfqpoint{2.090783in}{1.968169in}}%
\pgfpathlineto{\pgfqpoint{2.091017in}{1.916554in}}%
\pgfpathlineto{\pgfqpoint{2.091485in}{1.833241in}}%
\pgfpathlineto{\pgfqpoint{2.092187in}{1.866673in}}%
\pgfpathlineto{\pgfqpoint{2.092888in}{1.913185in}}%
\pgfpathlineto{\pgfqpoint{2.093356in}{1.898744in}}%
\pgfpathlineto{\pgfqpoint{2.094291in}{1.834739in}}%
\pgfpathlineto{\pgfqpoint{2.094525in}{1.850675in}}%
\pgfpathlineto{\pgfqpoint{2.095695in}{1.888474in}}%
\pgfpathlineto{\pgfqpoint{2.096864in}{1.750602in}}%
\pgfpathlineto{\pgfqpoint{2.097098in}{1.759338in}}%
\pgfpathlineto{\pgfqpoint{2.098969in}{2.105634in}}%
\pgfpathlineto{\pgfqpoint{2.100606in}{1.849278in}}%
\pgfpathlineto{\pgfqpoint{2.102009in}{1.722300in}}%
\pgfpathlineto{\pgfqpoint{2.102243in}{1.710308in}}%
\pgfpathlineto{\pgfqpoint{2.103178in}{1.924185in}}%
\pgfpathlineto{\pgfqpoint{2.103880in}{1.909244in}}%
\pgfpathlineto{\pgfqpoint{2.104113in}{1.870288in}}%
\pgfpathlineto{\pgfqpoint{2.104581in}{1.934641in}}%
\pgfpathlineto{\pgfqpoint{2.105517in}{2.130091in}}%
\pgfpathlineto{\pgfqpoint{2.105984in}{2.087385in}}%
\pgfpathlineto{\pgfqpoint{2.107387in}{1.834827in}}%
\pgfpathlineto{\pgfqpoint{2.108089in}{1.903040in}}%
\pgfpathlineto{\pgfqpoint{2.108557in}{1.851902in}}%
\pgfpathlineto{\pgfqpoint{2.109258in}{1.890279in}}%
\pgfpathlineto{\pgfqpoint{2.109025in}{1.841100in}}%
\pgfpathlineto{\pgfqpoint{2.109492in}{1.876975in}}%
\pgfpathlineto{\pgfqpoint{2.110662in}{1.746442in}}%
\pgfpathlineto{\pgfqpoint{2.111129in}{1.783427in}}%
\pgfpathlineto{\pgfqpoint{2.111597in}{1.894055in}}%
\pgfpathlineto{\pgfqpoint{2.113468in}{2.097867in}}%
\pgfpathlineto{\pgfqpoint{2.116508in}{1.673149in}}%
\pgfpathlineto{\pgfqpoint{2.116976in}{1.688806in}}%
\pgfpathlineto{\pgfqpoint{2.118847in}{2.061041in}}%
\pgfpathlineto{\pgfqpoint{2.119314in}{2.052532in}}%
\pgfpathlineto{\pgfqpoint{2.120016in}{2.007606in}}%
\pgfpathlineto{\pgfqpoint{2.120717in}{1.873302in}}%
\pgfpathlineto{\pgfqpoint{2.121185in}{1.921749in}}%
\pgfpathlineto{\pgfqpoint{2.121653in}{1.969596in}}%
\pgfpathlineto{\pgfqpoint{2.121887in}{1.898319in}}%
\pgfpathlineto{\pgfqpoint{2.122588in}{1.727047in}}%
\pgfpathlineto{\pgfqpoint{2.123290in}{1.761300in}}%
\pgfpathlineto{\pgfqpoint{2.123524in}{1.738290in}}%
\pgfpathlineto{\pgfqpoint{2.123758in}{1.800765in}}%
\pgfpathlineto{\pgfqpoint{2.126798in}{2.055361in}}%
\pgfpathlineto{\pgfqpoint{2.127266in}{2.034733in}}%
\pgfpathlineto{\pgfqpoint{2.130072in}{1.744858in}}%
\pgfpathlineto{\pgfqpoint{2.131475in}{1.918868in}}%
\pgfpathlineto{\pgfqpoint{2.131943in}{1.885358in}}%
\pgfpathlineto{\pgfqpoint{2.132177in}{1.882161in}}%
\pgfpathlineto{\pgfqpoint{2.132644in}{1.931755in}}%
\pgfpathlineto{\pgfqpoint{2.133580in}{2.100510in}}%
\pgfpathlineto{\pgfqpoint{2.133814in}{2.020853in}}%
\pgfpathlineto{\pgfqpoint{2.134749in}{1.765238in}}%
\pgfpathlineto{\pgfqpoint{2.134983in}{1.722612in}}%
\pgfpathlineto{\pgfqpoint{2.135451in}{1.770856in}}%
\pgfpathlineto{\pgfqpoint{2.135918in}{1.727951in}}%
\pgfpathlineto{\pgfqpoint{2.136152in}{1.731753in}}%
\pgfpathlineto{\pgfqpoint{2.139426in}{2.056343in}}%
\pgfpathlineto{\pgfqpoint{2.140362in}{1.942991in}}%
\pgfpathlineto{\pgfqpoint{2.140596in}{2.001002in}}%
\pgfpathlineto{\pgfqpoint{2.142233in}{1.750616in}}%
\pgfpathlineto{\pgfqpoint{2.142466in}{1.717757in}}%
\pgfpathlineto{\pgfqpoint{2.142700in}{1.789466in}}%
\pgfpathlineto{\pgfqpoint{2.142934in}{1.760121in}}%
\pgfpathlineto{\pgfqpoint{2.143168in}{1.805049in}}%
\pgfpathlineto{\pgfqpoint{2.143636in}{1.741669in}}%
\pgfpathlineto{\pgfqpoint{2.143870in}{1.759505in}}%
\pgfpathlineto{\pgfqpoint{2.144337in}{1.778389in}}%
\pgfpathlineto{\pgfqpoint{2.145974in}{2.046927in}}%
\pgfpathlineto{\pgfqpoint{2.146208in}{2.010545in}}%
\pgfpathlineto{\pgfqpoint{2.149014in}{1.807460in}}%
\pgfpathlineto{\pgfqpoint{2.146676in}{2.023065in}}%
\pgfpathlineto{\pgfqpoint{2.149248in}{1.815686in}}%
\pgfpathlineto{\pgfqpoint{2.149716in}{1.684461in}}%
\pgfpathlineto{\pgfqpoint{2.150184in}{1.744784in}}%
\pgfpathlineto{\pgfqpoint{2.151821in}{1.965110in}}%
\pgfpathlineto{\pgfqpoint{2.152288in}{1.928570in}}%
\pgfpathlineto{\pgfqpoint{2.152522in}{1.947513in}}%
\pgfpathlineto{\pgfqpoint{2.152756in}{2.031713in}}%
\pgfpathlineto{\pgfqpoint{2.153458in}{1.951920in}}%
\pgfpathlineto{\pgfqpoint{2.154393in}{1.844032in}}%
\pgfpathlineto{\pgfqpoint{2.154861in}{1.846292in}}%
\pgfpathlineto{\pgfqpoint{2.156498in}{2.047583in}}%
\pgfpathlineto{\pgfqpoint{2.156966in}{1.995163in}}%
\pgfpathlineto{\pgfqpoint{2.159070in}{1.759101in}}%
\pgfpathlineto{\pgfqpoint{2.159304in}{1.766751in}}%
\pgfpathlineto{\pgfqpoint{2.159538in}{1.767273in}}%
\pgfpathlineto{\pgfqpoint{2.160240in}{1.882400in}}%
\pgfpathlineto{\pgfqpoint{2.161643in}{2.068619in}}%
\pgfpathlineto{\pgfqpoint{2.161877in}{2.033335in}}%
\pgfpathlineto{\pgfqpoint{2.163981in}{1.799964in}}%
\pgfpathlineto{\pgfqpoint{2.164449in}{1.799131in}}%
\pgfpathlineto{\pgfqpoint{2.165852in}{1.895784in}}%
\pgfpathlineto{\pgfqpoint{2.166320in}{1.877152in}}%
\pgfpathlineto{\pgfqpoint{2.166554in}{1.891991in}}%
\pgfpathlineto{\pgfqpoint{2.166788in}{1.915517in}}%
\pgfpathlineto{\pgfqpoint{2.167255in}{1.854684in}}%
\pgfpathlineto{\pgfqpoint{2.168659in}{1.726315in}}%
\pgfpathlineto{\pgfqpoint{2.167723in}{1.861468in}}%
\pgfpathlineto{\pgfqpoint{2.168892in}{1.757134in}}%
\pgfpathlineto{\pgfqpoint{2.169126in}{1.757709in}}%
\pgfpathlineto{\pgfqpoint{2.171699in}{2.035650in}}%
\pgfpathlineto{\pgfqpoint{2.173336in}{1.814168in}}%
\pgfpathlineto{\pgfqpoint{2.174739in}{1.676862in}}%
\pgfpathlineto{\pgfqpoint{2.174973in}{1.711938in}}%
\pgfpathlineto{\pgfqpoint{2.175441in}{1.769983in}}%
\pgfpathlineto{\pgfqpoint{2.176610in}{1.933574in}}%
\pgfpathlineto{\pgfqpoint{2.176844in}{1.933027in}}%
\pgfpathlineto{\pgfqpoint{2.177545in}{1.844606in}}%
\pgfpathlineto{\pgfqpoint{2.177779in}{1.911326in}}%
\pgfpathlineto{\pgfqpoint{2.178247in}{1.993342in}}%
\pgfpathlineto{\pgfqpoint{2.178715in}{1.941534in}}%
\pgfpathlineto{\pgfqpoint{2.179884in}{1.817788in}}%
\pgfpathlineto{\pgfqpoint{2.180118in}{1.891055in}}%
\pgfpathlineto{\pgfqpoint{2.180352in}{1.898952in}}%
\pgfpathlineto{\pgfqpoint{2.180585in}{1.887470in}}%
\pgfpathlineto{\pgfqpoint{2.182456in}{1.689915in}}%
\pgfpathlineto{\pgfqpoint{2.182924in}{1.698328in}}%
\pgfpathlineto{\pgfqpoint{2.183158in}{1.682390in}}%
\pgfpathlineto{\pgfqpoint{2.183392in}{1.728814in}}%
\pgfpathlineto{\pgfqpoint{2.185496in}{1.932851in}}%
\pgfpathlineto{\pgfqpoint{2.185730in}{1.898324in}}%
\pgfpathlineto{\pgfqpoint{2.186198in}{1.842110in}}%
\pgfpathlineto{\pgfqpoint{2.186900in}{1.867184in}}%
\pgfpathlineto{\pgfqpoint{2.187835in}{1.869158in}}%
\pgfpathlineto{\pgfqpoint{2.188537in}{1.797308in}}%
\pgfpathlineto{\pgfqpoint{2.189238in}{1.746867in}}%
\pgfpathlineto{\pgfqpoint{2.189706in}{1.761017in}}%
\pgfpathlineto{\pgfqpoint{2.189940in}{1.769124in}}%
\pgfpathlineto{\pgfqpoint{2.190174in}{1.751734in}}%
\pgfpathlineto{\pgfqpoint{2.190408in}{1.703163in}}%
\pgfpathlineto{\pgfqpoint{2.191343in}{1.717592in}}%
\pgfpathlineto{\pgfqpoint{2.191577in}{1.722403in}}%
\pgfpathlineto{\pgfqpoint{2.193214in}{1.890204in}}%
\pgfpathlineto{\pgfqpoint{2.193915in}{1.854629in}}%
\pgfpathlineto{\pgfqpoint{2.194149in}{1.893627in}}%
\pgfpathlineto{\pgfqpoint{2.194383in}{1.909771in}}%
\pgfpathlineto{\pgfqpoint{2.194617in}{1.864856in}}%
\pgfpathlineto{\pgfqpoint{2.197189in}{1.659687in}}%
\pgfpathlineto{\pgfqpoint{2.197657in}{1.686396in}}%
\pgfpathlineto{\pgfqpoint{2.200463in}{1.946511in}}%
\pgfpathlineto{\pgfqpoint{2.200697in}{1.888614in}}%
\pgfpathlineto{\pgfqpoint{2.202334in}{1.739587in}}%
\pgfpathlineto{\pgfqpoint{2.202568in}{1.763197in}}%
\pgfpathlineto{\pgfqpoint{2.202802in}{1.715043in}}%
\pgfpathlineto{\pgfqpoint{2.203504in}{1.603255in}}%
\pgfpathlineto{\pgfqpoint{2.203971in}{1.648430in}}%
\pgfpathlineto{\pgfqpoint{2.205141in}{1.794414in}}%
\pgfpathlineto{\pgfqpoint{2.205375in}{1.720341in}}%
\pgfpathlineto{\pgfqpoint{2.205842in}{1.698512in}}%
\pgfpathlineto{\pgfqpoint{2.206076in}{1.722077in}}%
\pgfpathlineto{\pgfqpoint{2.206544in}{1.738770in}}%
\pgfpathlineto{\pgfqpoint{2.206778in}{1.716165in}}%
\pgfpathlineto{\pgfqpoint{2.207012in}{1.680484in}}%
\pgfpathlineto{\pgfqpoint{2.207479in}{1.776954in}}%
\pgfpathlineto{\pgfqpoint{2.209116in}{1.882677in}}%
\pgfpathlineto{\pgfqpoint{2.209584in}{1.884268in}}%
\pgfpathlineto{\pgfqpoint{2.210286in}{1.769894in}}%
\pgfpathlineto{\pgfqpoint{2.211923in}{1.608166in}}%
\pgfpathlineto{\pgfqpoint{2.212858in}{1.687303in}}%
\pgfpathlineto{\pgfqpoint{2.213326in}{1.673699in}}%
\pgfpathlineto{\pgfqpoint{2.214027in}{1.574843in}}%
\pgfpathlineto{\pgfqpoint{2.214495in}{1.625517in}}%
\pgfpathlineto{\pgfqpoint{2.215664in}{1.927191in}}%
\pgfpathlineto{\pgfqpoint{2.216132in}{1.871201in}}%
\pgfpathlineto{\pgfqpoint{2.217769in}{1.680723in}}%
\pgfpathlineto{\pgfqpoint{2.218003in}{1.684921in}}%
\pgfpathlineto{\pgfqpoint{2.219172in}{1.582440in}}%
\pgfpathlineto{\pgfqpoint{2.219640in}{1.582836in}}%
\pgfpathlineto{\pgfqpoint{2.220809in}{1.754583in}}%
\pgfpathlineto{\pgfqpoint{2.221277in}{1.706700in}}%
\pgfpathlineto{\pgfqpoint{2.221745in}{1.665926in}}%
\pgfpathlineto{\pgfqpoint{2.221979in}{1.695850in}}%
\pgfpathlineto{\pgfqpoint{2.223616in}{1.882434in}}%
\pgfpathlineto{\pgfqpoint{2.223849in}{1.883661in}}%
\pgfpathlineto{\pgfqpoint{2.226188in}{1.666758in}}%
\pgfpathlineto{\pgfqpoint{2.226422in}{1.664257in}}%
\pgfpathlineto{\pgfqpoint{2.227591in}{1.593162in}}%
\pgfpathlineto{\pgfqpoint{2.227825in}{1.609177in}}%
\pgfpathlineto{\pgfqpoint{2.230631in}{1.849054in}}%
\pgfpathlineto{\pgfqpoint{2.230865in}{1.827540in}}%
\pgfpathlineto{\pgfqpoint{2.232034in}{1.750026in}}%
\pgfpathlineto{\pgfqpoint{2.233204in}{1.890012in}}%
\pgfpathlineto{\pgfqpoint{2.233672in}{1.834173in}}%
\pgfpathlineto{\pgfqpoint{2.234841in}{1.603691in}}%
\pgfpathlineto{\pgfqpoint{2.235309in}{1.657705in}}%
\pgfpathlineto{\pgfqpoint{2.235542in}{1.680829in}}%
\pgfpathlineto{\pgfqpoint{2.236010in}{1.647563in}}%
\pgfpathlineto{\pgfqpoint{2.236244in}{1.673253in}}%
\pgfpathlineto{\pgfqpoint{2.237179in}{1.628616in}}%
\pgfpathlineto{\pgfqpoint{2.237413in}{1.661388in}}%
\pgfpathlineto{\pgfqpoint{2.237881in}{1.726551in}}%
\pgfpathlineto{\pgfqpoint{2.238583in}{1.690366in}}%
\pgfpathlineto{\pgfqpoint{2.239284in}{1.722964in}}%
\pgfpathlineto{\pgfqpoint{2.239518in}{1.719114in}}%
\pgfpathlineto{\pgfqpoint{2.240453in}{1.910393in}}%
\pgfpathlineto{\pgfqpoint{2.240921in}{1.828248in}}%
\pgfpathlineto{\pgfqpoint{2.242090in}{1.641586in}}%
\pgfpathlineto{\pgfqpoint{2.242558in}{1.659014in}}%
\pgfpathlineto{\pgfqpoint{2.243961in}{1.773141in}}%
\pgfpathlineto{\pgfqpoint{2.244429in}{1.719528in}}%
\pgfpathlineto{\pgfqpoint{2.244663in}{1.688222in}}%
\pgfpathlineto{\pgfqpoint{2.245131in}{1.731204in}}%
\pgfpathlineto{\pgfqpoint{2.245598in}{1.696321in}}%
\pgfpathlineto{\pgfqpoint{2.245832in}{1.698392in}}%
\pgfpathlineto{\pgfqpoint{2.246534in}{1.593682in}}%
\pgfpathlineto{\pgfqpoint{2.247001in}{1.629050in}}%
\pgfpathlineto{\pgfqpoint{2.248639in}{1.799047in}}%
\pgfpathlineto{\pgfqpoint{2.248872in}{1.790959in}}%
\pgfpathlineto{\pgfqpoint{2.249106in}{1.802541in}}%
\pgfpathlineto{\pgfqpoint{2.249340in}{1.825591in}}%
\pgfpathlineto{\pgfqpoint{2.249808in}{1.794259in}}%
\pgfpathlineto{\pgfqpoint{2.250276in}{1.745657in}}%
\pgfpathlineto{\pgfqpoint{2.250743in}{1.787631in}}%
\pgfpathlineto{\pgfqpoint{2.250977in}{1.808479in}}%
\pgfpathlineto{\pgfqpoint{2.251445in}{1.779462in}}%
\pgfpathlineto{\pgfqpoint{2.254017in}{1.519866in}}%
\pgfpathlineto{\pgfqpoint{2.254485in}{1.546539in}}%
\pgfpathlineto{\pgfqpoint{2.256356in}{1.921989in}}%
\pgfpathlineto{\pgfqpoint{2.256590in}{1.920664in}}%
\pgfpathlineto{\pgfqpoint{2.258227in}{1.474702in}}%
\pgfpathlineto{\pgfqpoint{2.258694in}{1.582140in}}%
\pgfpathlineto{\pgfqpoint{2.260098in}{1.877217in}}%
\pgfpathlineto{\pgfqpoint{2.260565in}{1.814753in}}%
\pgfpathlineto{\pgfqpoint{2.261735in}{1.723083in}}%
\pgfpathlineto{\pgfqpoint{2.262202in}{1.749523in}}%
\pgfpathlineto{\pgfqpoint{2.262670in}{1.720155in}}%
\pgfpathlineto{\pgfqpoint{2.263138in}{1.657434in}}%
\pgfpathlineto{\pgfqpoint{2.263606in}{1.702308in}}%
\pgfpathlineto{\pgfqpoint{2.263839in}{1.705897in}}%
\pgfpathlineto{\pgfqpoint{2.265243in}{1.551344in}}%
\pgfpathlineto{\pgfqpoint{2.267113in}{1.822643in}}%
\pgfpathlineto{\pgfqpoint{2.267347in}{1.819681in}}%
\pgfpathlineto{\pgfqpoint{2.269218in}{1.670362in}}%
\pgfpathlineto{\pgfqpoint{2.269920in}{1.606833in}}%
\pgfpathlineto{\pgfqpoint{2.270387in}{1.635503in}}%
\pgfpathlineto{\pgfqpoint{2.270621in}{1.705659in}}%
\pgfpathlineto{\pgfqpoint{2.271323in}{1.666692in}}%
\pgfpathlineto{\pgfqpoint{2.271557in}{1.621313in}}%
\pgfpathlineto{\pgfqpoint{2.272258in}{1.668211in}}%
\pgfpathlineto{\pgfqpoint{2.273895in}{1.840883in}}%
\pgfpathlineto{\pgfqpoint{2.274129in}{1.822322in}}%
\pgfpathlineto{\pgfqpoint{2.275766in}{1.599281in}}%
\pgfpathlineto{\pgfqpoint{2.276000in}{1.608110in}}%
\pgfpathlineto{\pgfqpoint{2.277637in}{1.698602in}}%
\pgfpathlineto{\pgfqpoint{2.278572in}{1.495856in}}%
\pgfpathlineto{\pgfqpoint{2.279040in}{1.587707in}}%
\pgfpathlineto{\pgfqpoint{2.280210in}{1.899297in}}%
\pgfpathlineto{\pgfqpoint{2.280677in}{1.820529in}}%
\pgfpathlineto{\pgfqpoint{2.282548in}{1.520320in}}%
\pgfpathlineto{\pgfqpoint{2.282782in}{1.546340in}}%
\pgfpathlineto{\pgfqpoint{2.283717in}{1.771461in}}%
\pgfpathlineto{\pgfqpoint{2.284419in}{1.728994in}}%
\pgfpathlineto{\pgfqpoint{2.285121in}{1.622909in}}%
\pgfpathlineto{\pgfqpoint{2.285588in}{1.678712in}}%
\pgfpathlineto{\pgfqpoint{2.285822in}{1.677119in}}%
\pgfpathlineto{\pgfqpoint{2.286290in}{1.637602in}}%
\pgfpathlineto{\pgfqpoint{2.286524in}{1.640804in}}%
\pgfpathlineto{\pgfqpoint{2.287225in}{1.764564in}}%
\pgfpathlineto{\pgfqpoint{2.287693in}{1.715891in}}%
\pgfpathlineto{\pgfqpoint{2.288395in}{1.618552in}}%
\pgfpathlineto{\pgfqpoint{2.288862in}{1.690973in}}%
\pgfpathlineto{\pgfqpoint{2.289798in}{1.769240in}}%
\pgfpathlineto{\pgfqpoint{2.290032in}{1.752350in}}%
\pgfpathlineto{\pgfqpoint{2.290499in}{1.670221in}}%
\pgfpathlineto{\pgfqpoint{2.291201in}{1.689726in}}%
\pgfpathlineto{\pgfqpoint{2.292604in}{1.827500in}}%
\pgfpathlineto{\pgfqpoint{2.292838in}{1.809643in}}%
\pgfpathlineto{\pgfqpoint{2.294241in}{1.596575in}}%
\pgfpathlineto{\pgfqpoint{2.294475in}{1.608324in}}%
\pgfpathlineto{\pgfqpoint{2.294709in}{1.632202in}}%
\pgfpathlineto{\pgfqpoint{2.295177in}{1.577338in}}%
\pgfpathlineto{\pgfqpoint{2.295410in}{1.528373in}}%
\pgfpathlineto{\pgfqpoint{2.295878in}{1.582416in}}%
\pgfpathlineto{\pgfqpoint{2.297281in}{1.746762in}}%
\pgfpathlineto{\pgfqpoint{2.297515in}{1.717808in}}%
\pgfpathlineto{\pgfqpoint{2.297983in}{1.750434in}}%
\pgfpathlineto{\pgfqpoint{2.298217in}{1.785372in}}%
\pgfpathlineto{\pgfqpoint{2.298684in}{1.737208in}}%
\pgfpathlineto{\pgfqpoint{2.298918in}{1.671936in}}%
\pgfpathlineto{\pgfqpoint{2.299854in}{1.698966in}}%
\pgfpathlineto{\pgfqpoint{2.300555in}{1.800137in}}%
\pgfpathlineto{\pgfqpoint{2.301257in}{1.756841in}}%
\pgfpathlineto{\pgfqpoint{2.302192in}{1.587277in}}%
\pgfpathlineto{\pgfqpoint{2.303128in}{1.615306in}}%
\pgfpathlineto{\pgfqpoint{2.304063in}{1.543540in}}%
\pgfpathlineto{\pgfqpoint{2.304531in}{1.559139in}}%
\pgfpathlineto{\pgfqpoint{2.307337in}{1.801920in}}%
\pgfpathlineto{\pgfqpoint{2.307805in}{1.752313in}}%
\pgfpathlineto{\pgfqpoint{2.309442in}{1.545936in}}%
\pgfpathlineto{\pgfqpoint{2.311313in}{1.810398in}}%
\pgfpathlineto{\pgfqpoint{2.312014in}{1.778760in}}%
\pgfpathlineto{\pgfqpoint{2.313184in}{1.689491in}}%
\pgfpathlineto{\pgfqpoint{2.313651in}{1.717077in}}%
\pgfpathlineto{\pgfqpoint{2.313885in}{1.756691in}}%
\pgfpathlineto{\pgfqpoint{2.314587in}{1.697461in}}%
\pgfpathlineto{\pgfqpoint{2.315756in}{1.546918in}}%
\pgfpathlineto{\pgfqpoint{2.315990in}{1.571629in}}%
\pgfpathlineto{\pgfqpoint{2.319498in}{1.855613in}}%
\pgfpathlineto{\pgfqpoint{2.319732in}{1.828423in}}%
\pgfpathlineto{\pgfqpoint{2.321836in}{1.570814in}}%
\pgfpathlineto{\pgfqpoint{2.322304in}{1.644853in}}%
\pgfpathlineto{\pgfqpoint{2.322772in}{1.710407in}}%
\pgfpathlineto{\pgfqpoint{2.323473in}{1.696889in}}%
\pgfpathlineto{\pgfqpoint{2.323707in}{1.692598in}}%
\pgfpathlineto{\pgfqpoint{2.323941in}{1.647081in}}%
\pgfpathlineto{\pgfqpoint{2.324409in}{1.679694in}}%
\pgfpathlineto{\pgfqpoint{2.325344in}{1.826720in}}%
\pgfpathlineto{\pgfqpoint{2.325812in}{1.818813in}}%
\pgfpathlineto{\pgfqpoint{2.326981in}{1.675757in}}%
\pgfpathlineto{\pgfqpoint{2.328385in}{1.712875in}}%
\pgfpathlineto{\pgfqpoint{2.328618in}{1.714242in}}%
\pgfpathlineto{\pgfqpoint{2.329086in}{1.748477in}}%
\pgfpathlineto{\pgfqpoint{2.329554in}{1.723580in}}%
\pgfpathlineto{\pgfqpoint{2.330255in}{1.601964in}}%
\pgfpathlineto{\pgfqpoint{2.330723in}{1.643411in}}%
\pgfpathlineto{\pgfqpoint{2.331892in}{1.855490in}}%
\pgfpathlineto{\pgfqpoint{2.332360in}{1.796596in}}%
\pgfpathlineto{\pgfqpoint{2.334231in}{1.559964in}}%
\pgfpathlineto{\pgfqpoint{2.334699in}{1.588120in}}%
\pgfpathlineto{\pgfqpoint{2.337037in}{1.830186in}}%
\pgfpathlineto{\pgfqpoint{2.337505in}{1.759831in}}%
\pgfpathlineto{\pgfqpoint{2.337973in}{1.797222in}}%
\pgfpathlineto{\pgfqpoint{2.338207in}{1.853448in}}%
\pgfpathlineto{\pgfqpoint{2.338908in}{1.811073in}}%
\pgfpathlineto{\pgfqpoint{2.341013in}{1.461415in}}%
\pgfpathlineto{\pgfqpoint{2.341715in}{1.582042in}}%
\pgfpathlineto{\pgfqpoint{2.343819in}{1.835378in}}%
\pgfpathlineto{\pgfqpoint{2.344287in}{1.883684in}}%
\pgfpathlineto{\pgfqpoint{2.344755in}{1.857946in}}%
\pgfpathlineto{\pgfqpoint{2.347561in}{1.499436in}}%
\pgfpathlineto{\pgfqpoint{2.347795in}{1.518864in}}%
\pgfpathlineto{\pgfqpoint{2.349432in}{1.776972in}}%
\pgfpathlineto{\pgfqpoint{2.349900in}{1.776268in}}%
\pgfpathlineto{\pgfqpoint{2.350133in}{1.780807in}}%
\pgfpathlineto{\pgfqpoint{2.350367in}{1.774694in}}%
\pgfpathlineto{\pgfqpoint{2.350835in}{1.759623in}}%
\pgfpathlineto{\pgfqpoint{2.351303in}{1.819380in}}%
\pgfpathlineto{\pgfqpoint{2.352004in}{1.777448in}}%
\pgfpathlineto{\pgfqpoint{2.353875in}{1.550404in}}%
\pgfpathlineto{\pgfqpoint{2.354109in}{1.552336in}}%
\pgfpathlineto{\pgfqpoint{2.354343in}{1.551004in}}%
\pgfpathlineto{\pgfqpoint{2.354577in}{1.531849in}}%
\pgfpathlineto{\pgfqpoint{2.355044in}{1.546052in}}%
\pgfpathlineto{\pgfqpoint{2.355980in}{1.779554in}}%
\pgfpathlineto{\pgfqpoint{2.356915in}{1.752202in}}%
\pgfpathlineto{\pgfqpoint{2.357617in}{1.641086in}}%
\pgfpathlineto{\pgfqpoint{2.358319in}{1.649341in}}%
\pgfpathlineto{\pgfqpoint{2.358552in}{1.630913in}}%
\pgfpathlineto{\pgfqpoint{2.358786in}{1.669174in}}%
\pgfpathlineto{\pgfqpoint{2.359254in}{1.709059in}}%
\pgfpathlineto{\pgfqpoint{2.359722in}{1.655960in}}%
\pgfpathlineto{\pgfqpoint{2.359956in}{1.647308in}}%
\pgfpathlineto{\pgfqpoint{2.361593in}{1.766374in}}%
\pgfpathlineto{\pgfqpoint{2.362762in}{1.586499in}}%
\pgfpathlineto{\pgfqpoint{2.364165in}{1.444980in}}%
\pgfpathlineto{\pgfqpoint{2.365334in}{1.781220in}}%
\pgfpathlineto{\pgfqpoint{2.366504in}{1.748164in}}%
\pgfpathlineto{\pgfqpoint{2.367673in}{1.643300in}}%
\pgfpathlineto{\pgfqpoint{2.368842in}{1.557836in}}%
\pgfpathlineto{\pgfqpoint{2.369310in}{1.656012in}}%
\pgfpathlineto{\pgfqpoint{2.370245in}{1.653129in}}%
\pgfpathlineto{\pgfqpoint{2.370479in}{1.654029in}}%
\pgfpathlineto{\pgfqpoint{2.370713in}{1.651213in}}%
\pgfpathlineto{\pgfqpoint{2.370947in}{1.645646in}}%
\pgfpathlineto{\pgfqpoint{2.371648in}{1.735757in}}%
\pgfpathlineto{\pgfqpoint{2.371882in}{1.671277in}}%
\pgfpathlineto{\pgfqpoint{2.373052in}{1.527181in}}%
\pgfpathlineto{\pgfqpoint{2.373286in}{1.552347in}}%
\pgfpathlineto{\pgfqpoint{2.376092in}{1.778883in}}%
\pgfpathlineto{\pgfqpoint{2.376326in}{1.765082in}}%
\pgfpathlineto{\pgfqpoint{2.376560in}{1.781537in}}%
\pgfpathlineto{\pgfqpoint{2.377027in}{1.743494in}}%
\pgfpathlineto{\pgfqpoint{2.378664in}{1.583570in}}%
\pgfpathlineto{\pgfqpoint{2.378898in}{1.566432in}}%
\pgfpathlineto{\pgfqpoint{2.379366in}{1.612362in}}%
\pgfpathlineto{\pgfqpoint{2.379600in}{1.614244in}}%
\pgfpathlineto{\pgfqpoint{2.379834in}{1.651255in}}%
\pgfpathlineto{\pgfqpoint{2.380301in}{1.609080in}}%
\pgfpathlineto{\pgfqpoint{2.380769in}{1.570636in}}%
\pgfpathlineto{\pgfqpoint{2.381237in}{1.618154in}}%
\pgfpathlineto{\pgfqpoint{2.381471in}{1.626257in}}%
\pgfpathlineto{\pgfqpoint{2.381704in}{1.620358in}}%
\pgfpathlineto{\pgfqpoint{2.381938in}{1.601904in}}%
\pgfpathlineto{\pgfqpoint{2.383341in}{1.836915in}}%
\pgfpathlineto{\pgfqpoint{2.383575in}{1.836161in}}%
\pgfpathlineto{\pgfqpoint{2.384745in}{1.758454in}}%
\pgfpathlineto{\pgfqpoint{2.386382in}{1.531700in}}%
\pgfpathlineto{\pgfqpoint{2.386849in}{1.573422in}}%
\pgfpathlineto{\pgfqpoint{2.387083in}{1.600730in}}%
\pgfpathlineto{\pgfqpoint{2.387551in}{1.529909in}}%
\pgfpathlineto{\pgfqpoint{2.388253in}{1.604290in}}%
\pgfpathlineto{\pgfqpoint{2.389890in}{1.786290in}}%
\pgfpathlineto{\pgfqpoint{2.390591in}{1.692883in}}%
\pgfpathlineto{\pgfqpoint{2.391059in}{1.733557in}}%
\pgfpathlineto{\pgfqpoint{2.391527in}{1.817346in}}%
\pgfpathlineto{\pgfqpoint{2.392228in}{1.765095in}}%
\pgfpathlineto{\pgfqpoint{2.392696in}{1.771766in}}%
\pgfpathlineto{\pgfqpoint{2.392930in}{1.766449in}}%
\pgfpathlineto{\pgfqpoint{2.393164in}{1.800340in}}%
\pgfpathlineto{\pgfqpoint{2.393631in}{1.715871in}}%
\pgfpathlineto{\pgfqpoint{2.395502in}{1.501657in}}%
\pgfpathlineto{\pgfqpoint{2.397607in}{1.911060in}}%
\pgfpathlineto{\pgfqpoint{2.398075in}{1.852419in}}%
\pgfpathlineto{\pgfqpoint{2.400647in}{1.524034in}}%
\pgfpathlineto{\pgfqpoint{2.401582in}{1.598640in}}%
\pgfpathlineto{\pgfqpoint{2.402050in}{1.661518in}}%
\pgfpathlineto{\pgfqpoint{2.402752in}{1.628452in}}%
\pgfpathlineto{\pgfqpoint{2.402986in}{1.618318in}}%
\pgfpathlineto{\pgfqpoint{2.403220in}{1.641026in}}%
\pgfpathlineto{\pgfqpoint{2.403453in}{1.648232in}}%
\pgfpathlineto{\pgfqpoint{2.405324in}{1.871236in}}%
\pgfpathlineto{\pgfqpoint{2.407429in}{1.686434in}}%
\pgfpathlineto{\pgfqpoint{2.408832in}{1.570749in}}%
\pgfpathlineto{\pgfqpoint{2.409066in}{1.584345in}}%
\pgfpathlineto{\pgfqpoint{2.409300in}{1.572489in}}%
\pgfpathlineto{\pgfqpoint{2.409768in}{1.602522in}}%
\pgfpathlineto{\pgfqpoint{2.413509in}{1.852981in}}%
\pgfpathlineto{\pgfqpoint{2.414211in}{1.819972in}}%
\pgfpathlineto{\pgfqpoint{2.415146in}{1.695005in}}%
\pgfpathlineto{\pgfqpoint{2.415614in}{1.746048in}}%
\pgfpathlineto{\pgfqpoint{2.416082in}{1.772532in}}%
\pgfpathlineto{\pgfqpoint{2.416316in}{1.713235in}}%
\pgfpathlineto{\pgfqpoint{2.417485in}{1.584412in}}%
\pgfpathlineto{\pgfqpoint{2.417719in}{1.590741in}}%
\pgfpathlineto{\pgfqpoint{2.419356in}{1.679231in}}%
\pgfpathlineto{\pgfqpoint{2.420291in}{1.718639in}}%
\pgfpathlineto{\pgfqpoint{2.421694in}{1.860477in}}%
\pgfpathlineto{\pgfqpoint{2.422162in}{1.829979in}}%
\pgfpathlineto{\pgfqpoint{2.422864in}{1.691049in}}%
\pgfpathlineto{\pgfqpoint{2.423799in}{1.717924in}}%
\pgfpathlineto{\pgfqpoint{2.424501in}{1.702940in}}%
\pgfpathlineto{\pgfqpoint{2.424968in}{1.749120in}}%
\pgfpathlineto{\pgfqpoint{2.426138in}{1.614599in}}%
\pgfpathlineto{\pgfqpoint{2.426372in}{1.619642in}}%
\pgfpathlineto{\pgfqpoint{2.426605in}{1.641915in}}%
\pgfpathlineto{\pgfqpoint{2.426839in}{1.596397in}}%
\pgfpathlineto{\pgfqpoint{2.427307in}{1.539123in}}%
\pgfpathlineto{\pgfqpoint{2.427775in}{1.546591in}}%
\pgfpathlineto{\pgfqpoint{2.429879in}{1.848805in}}%
\pgfpathlineto{\pgfqpoint{2.430113in}{1.831364in}}%
\pgfpathlineto{\pgfqpoint{2.430347in}{1.818722in}}%
\pgfpathlineto{\pgfqpoint{2.430581in}{1.845885in}}%
\pgfpathlineto{\pgfqpoint{2.431049in}{1.873330in}}%
\pgfpathlineto{\pgfqpoint{2.431283in}{1.830858in}}%
\pgfpathlineto{\pgfqpoint{2.432686in}{1.656382in}}%
\pgfpathlineto{\pgfqpoint{2.433387in}{1.670884in}}%
\pgfpathlineto{\pgfqpoint{2.435024in}{1.566079in}}%
\pgfpathlineto{\pgfqpoint{2.433855in}{1.680946in}}%
\pgfpathlineto{\pgfqpoint{2.435258in}{1.591518in}}%
\pgfpathlineto{\pgfqpoint{2.438298in}{1.861079in}}%
\pgfpathlineto{\pgfqpoint{2.438532in}{1.838668in}}%
\pgfpathlineto{\pgfqpoint{2.438766in}{1.783015in}}%
\pgfpathlineto{\pgfqpoint{2.439702in}{1.800152in}}%
\pgfpathlineto{\pgfqpoint{2.441806in}{1.524852in}}%
\pgfpathlineto{\pgfqpoint{2.442508in}{1.627333in}}%
\pgfpathlineto{\pgfqpoint{2.443911in}{1.786510in}}%
\pgfpathlineto{\pgfqpoint{2.445080in}{1.603622in}}%
\pgfpathlineto{\pgfqpoint{2.445548in}{1.668568in}}%
\pgfpathlineto{\pgfqpoint{2.446250in}{1.764171in}}%
\pgfpathlineto{\pgfqpoint{2.446717in}{1.736255in}}%
\pgfpathlineto{\pgfqpoint{2.446951in}{1.734447in}}%
\pgfpathlineto{\pgfqpoint{2.447887in}{1.822542in}}%
\pgfpathlineto{\pgfqpoint{2.448120in}{1.799421in}}%
\pgfpathlineto{\pgfqpoint{2.449290in}{1.674904in}}%
\pgfpathlineto{\pgfqpoint{2.449524in}{1.732943in}}%
\pgfpathlineto{\pgfqpoint{2.450459in}{1.769665in}}%
\pgfpathlineto{\pgfqpoint{2.450693in}{1.746867in}}%
\pgfpathlineto{\pgfqpoint{2.452798in}{1.499857in}}%
\pgfpathlineto{\pgfqpoint{2.453032in}{1.541587in}}%
\pgfpathlineto{\pgfqpoint{2.454201in}{1.878849in}}%
\pgfpathlineto{\pgfqpoint{2.454902in}{1.808074in}}%
\pgfpathlineto{\pgfqpoint{2.455370in}{1.798494in}}%
\pgfpathlineto{\pgfqpoint{2.455604in}{1.821187in}}%
\pgfpathlineto{\pgfqpoint{2.455838in}{1.764828in}}%
\pgfpathlineto{\pgfqpoint{2.457475in}{1.569184in}}%
\pgfpathlineto{\pgfqpoint{2.458878in}{1.662210in}}%
\pgfpathlineto{\pgfqpoint{2.459112in}{1.632468in}}%
\pgfpathlineto{\pgfqpoint{2.459813in}{1.681225in}}%
\pgfpathlineto{\pgfqpoint{2.460281in}{1.669829in}}%
\pgfpathlineto{\pgfqpoint{2.460515in}{1.705421in}}%
\pgfpathlineto{\pgfqpoint{2.461217in}{1.680233in}}%
\pgfpathlineto{\pgfqpoint{2.461450in}{1.660904in}}%
\pgfpathlineto{\pgfqpoint{2.461684in}{1.714721in}}%
\pgfpathlineto{\pgfqpoint{2.463087in}{1.819183in}}%
\pgfpathlineto{\pgfqpoint{2.463555in}{1.796749in}}%
\pgfpathlineto{\pgfqpoint{2.463789in}{1.827923in}}%
\pgfpathlineto{\pgfqpoint{2.464023in}{1.862665in}}%
\pgfpathlineto{\pgfqpoint{2.464491in}{1.797514in}}%
\pgfpathlineto{\pgfqpoint{2.466595in}{1.576345in}}%
\pgfpathlineto{\pgfqpoint{2.468700in}{1.857587in}}%
\pgfpathlineto{\pgfqpoint{2.467531in}{1.568445in}}%
\pgfpathlineto{\pgfqpoint{2.469402in}{1.768571in}}%
\pgfpathlineto{\pgfqpoint{2.469636in}{1.764495in}}%
\pgfpathlineto{\pgfqpoint{2.470337in}{1.840272in}}%
\pgfpathlineto{\pgfqpoint{2.470571in}{1.799654in}}%
\pgfpathlineto{\pgfqpoint{2.470805in}{1.755982in}}%
\pgfpathlineto{\pgfqpoint{2.471740in}{1.769499in}}%
\pgfpathlineto{\pgfqpoint{2.472208in}{1.795526in}}%
\pgfpathlineto{\pgfqpoint{2.472442in}{1.771255in}}%
\pgfpathlineto{\pgfqpoint{2.474313in}{1.561168in}}%
\pgfpathlineto{\pgfqpoint{2.474547in}{1.561630in}}%
\pgfpathlineto{\pgfqpoint{2.475014in}{1.585512in}}%
\pgfpathlineto{\pgfqpoint{2.476885in}{1.908805in}}%
\pgfpathlineto{\pgfqpoint{2.477353in}{1.854622in}}%
\pgfpathlineto{\pgfqpoint{2.479925in}{1.540017in}}%
\pgfpathlineto{\pgfqpoint{2.480393in}{1.591386in}}%
\pgfpathlineto{\pgfqpoint{2.481796in}{1.795410in}}%
\pgfpathlineto{\pgfqpoint{2.482030in}{1.789872in}}%
\pgfpathlineto{\pgfqpoint{2.482966in}{1.748686in}}%
\pgfpathlineto{\pgfqpoint{2.483199in}{1.748943in}}%
\pgfpathlineto{\pgfqpoint{2.483667in}{1.778542in}}%
\pgfpathlineto{\pgfqpoint{2.484135in}{1.739789in}}%
\pgfpathlineto{\pgfqpoint{2.484369in}{1.748884in}}%
\pgfpathlineto{\pgfqpoint{2.484836in}{1.870495in}}%
\pgfpathlineto{\pgfqpoint{2.485538in}{1.836511in}}%
\pgfpathlineto{\pgfqpoint{2.487643in}{1.553056in}}%
\pgfpathlineto{\pgfqpoint{2.488344in}{1.595982in}}%
\pgfpathlineto{\pgfqpoint{2.490917in}{1.877835in}}%
\pgfpathlineto{\pgfqpoint{2.491384in}{1.852807in}}%
\pgfpathlineto{\pgfqpoint{2.493255in}{1.686911in}}%
\pgfpathlineto{\pgfqpoint{2.493489in}{1.695071in}}%
\pgfpathlineto{\pgfqpoint{2.494658in}{1.572209in}}%
\pgfpathlineto{\pgfqpoint{2.495126in}{1.594382in}}%
\pgfpathlineto{\pgfqpoint{2.496296in}{1.677863in}}%
\pgfpathlineto{\pgfqpoint{2.496529in}{1.670162in}}%
\pgfpathlineto{\pgfqpoint{2.497699in}{1.717180in}}%
\pgfpathlineto{\pgfqpoint{2.497933in}{1.713080in}}%
\pgfpathlineto{\pgfqpoint{2.499336in}{1.607588in}}%
\pgfpathlineto{\pgfqpoint{2.498634in}{1.729195in}}%
\pgfpathlineto{\pgfqpoint{2.499570in}{1.674993in}}%
\pgfpathlineto{\pgfqpoint{2.501440in}{1.795285in}}%
\pgfpathlineto{\pgfqpoint{2.503077in}{1.659212in}}%
\pgfpathlineto{\pgfqpoint{2.503545in}{1.670909in}}%
\pgfpathlineto{\pgfqpoint{2.504013in}{1.843113in}}%
\pgfpathlineto{\pgfqpoint{2.504481in}{1.748029in}}%
\pgfpathlineto{\pgfqpoint{2.505416in}{1.626188in}}%
\pgfpathlineto{\pgfqpoint{2.505650in}{1.639048in}}%
\pgfpathlineto{\pgfqpoint{2.506118in}{1.733583in}}%
\pgfpathlineto{\pgfqpoint{2.507053in}{1.692414in}}%
\pgfpathlineto{\pgfqpoint{2.507988in}{1.473325in}}%
\pgfpathlineto{\pgfqpoint{2.508456in}{1.560602in}}%
\pgfpathlineto{\pgfqpoint{2.508924in}{1.639862in}}%
\pgfpathlineto{\pgfqpoint{2.509392in}{1.774257in}}%
\pgfpathlineto{\pgfqpoint{2.510093in}{1.706067in}}%
\pgfpathlineto{\pgfqpoint{2.510327in}{1.693800in}}%
\pgfpathlineto{\pgfqpoint{2.510561in}{1.717690in}}%
\pgfpathlineto{\pgfqpoint{2.512198in}{1.828253in}}%
\pgfpathlineto{\pgfqpoint{2.512432in}{1.822332in}}%
\pgfpathlineto{\pgfqpoint{2.513835in}{1.558049in}}%
\pgfpathlineto{\pgfqpoint{2.514303in}{1.582439in}}%
\pgfpathlineto{\pgfqpoint{2.515004in}{1.680192in}}%
\pgfpathlineto{\pgfqpoint{2.515472in}{1.634109in}}%
\pgfpathlineto{\pgfqpoint{2.516407in}{1.530791in}}%
\pgfpathlineto{\pgfqpoint{2.517109in}{1.545718in}}%
\pgfpathlineto{\pgfqpoint{2.518746in}{1.752737in}}%
\pgfpathlineto{\pgfqpoint{2.519448in}{1.844328in}}%
\pgfpathlineto{\pgfqpoint{2.519915in}{1.792056in}}%
\pgfpathlineto{\pgfqpoint{2.521085in}{1.567989in}}%
\pgfpathlineto{\pgfqpoint{2.521318in}{1.494013in}}%
\pgfpathlineto{\pgfqpoint{2.522020in}{1.576079in}}%
\pgfpathlineto{\pgfqpoint{2.522254in}{1.594131in}}%
\pgfpathlineto{\pgfqpoint{2.522955in}{1.567365in}}%
\pgfpathlineto{\pgfqpoint{2.523189in}{1.569934in}}%
\pgfpathlineto{\pgfqpoint{2.523423in}{1.567938in}}%
\pgfpathlineto{\pgfqpoint{2.525762in}{1.841967in}}%
\pgfpathlineto{\pgfqpoint{2.525996in}{1.836285in}}%
\pgfpathlineto{\pgfqpoint{2.527165in}{1.477848in}}%
\pgfpathlineto{\pgfqpoint{2.527867in}{1.562676in}}%
\pgfpathlineto{\pgfqpoint{2.528568in}{1.626220in}}%
\pgfpathlineto{\pgfqpoint{2.529270in}{1.785859in}}%
\pgfpathlineto{\pgfqpoint{2.529971in}{1.749608in}}%
\pgfpathlineto{\pgfqpoint{2.530439in}{1.758122in}}%
\pgfpathlineto{\pgfqpoint{2.531374in}{1.705041in}}%
\pgfpathlineto{\pgfqpoint{2.532544in}{1.460329in}}%
\pgfpathlineto{\pgfqpoint{2.533011in}{1.546421in}}%
\pgfpathlineto{\pgfqpoint{2.535818in}{1.772034in}}%
\pgfpathlineto{\pgfqpoint{2.533479in}{1.539563in}}%
\pgfpathlineto{\pgfqpoint{2.536052in}{1.719668in}}%
\pgfpathlineto{\pgfqpoint{2.536519in}{1.560201in}}%
\pgfpathlineto{\pgfqpoint{2.537221in}{1.674163in}}%
\pgfpathlineto{\pgfqpoint{2.537922in}{1.717433in}}%
\pgfpathlineto{\pgfqpoint{2.538156in}{1.729544in}}%
\pgfpathlineto{\pgfqpoint{2.538390in}{1.699696in}}%
\pgfpathlineto{\pgfqpoint{2.539326in}{1.529850in}}%
\pgfpathlineto{\pgfqpoint{2.540495in}{1.604702in}}%
\pgfpathlineto{\pgfqpoint{2.541196in}{1.774524in}}%
\pgfpathlineto{\pgfqpoint{2.541898in}{1.716229in}}%
\pgfpathlineto{\pgfqpoint{2.542132in}{1.688696in}}%
\pgfpathlineto{\pgfqpoint{2.542600in}{1.714520in}}%
\pgfpathlineto{\pgfqpoint{2.542834in}{1.759107in}}%
\pgfpathlineto{\pgfqpoint{2.543301in}{1.725525in}}%
\pgfpathlineto{\pgfqpoint{2.544938in}{1.477802in}}%
\pgfpathlineto{\pgfqpoint{2.545172in}{1.505786in}}%
\pgfpathlineto{\pgfqpoint{2.547277in}{1.759455in}}%
\pgfpathlineto{\pgfqpoint{2.547745in}{1.705564in}}%
\pgfpathlineto{\pgfqpoint{2.548212in}{1.783384in}}%
\pgfpathlineto{\pgfqpoint{2.548446in}{1.806019in}}%
\pgfpathlineto{\pgfqpoint{2.548914in}{1.765880in}}%
\pgfpathlineto{\pgfqpoint{2.549849in}{1.595697in}}%
\pgfpathlineto{\pgfqpoint{2.551019in}{1.540235in}}%
\pgfpathlineto{\pgfqpoint{2.551252in}{1.554595in}}%
\pgfpathlineto{\pgfqpoint{2.552188in}{1.700107in}}%
\pgfpathlineto{\pgfqpoint{2.552656in}{1.793897in}}%
\pgfpathlineto{\pgfqpoint{2.553123in}{1.723199in}}%
\pgfpathlineto{\pgfqpoint{2.554293in}{1.572958in}}%
\pgfpathlineto{\pgfqpoint{2.554760in}{1.636680in}}%
\pgfpathlineto{\pgfqpoint{2.556163in}{1.737345in}}%
\pgfpathlineto{\pgfqpoint{2.556397in}{1.782707in}}%
\pgfpathlineto{\pgfqpoint{2.557099in}{1.721600in}}%
\pgfpathlineto{\pgfqpoint{2.558502in}{1.795887in}}%
\pgfpathlineto{\pgfqpoint{2.558736in}{1.782153in}}%
\pgfpathlineto{\pgfqpoint{2.559204in}{1.701860in}}%
\pgfpathlineto{\pgfqpoint{2.559905in}{1.486979in}}%
\pgfpathlineto{\pgfqpoint{2.560607in}{1.563428in}}%
\pgfpathlineto{\pgfqpoint{2.562010in}{1.817414in}}%
\pgfpathlineto{\pgfqpoint{2.562945in}{1.777530in}}%
\pgfpathlineto{\pgfqpoint{2.563647in}{1.745067in}}%
\pgfpathlineto{\pgfqpoint{2.564115in}{1.630589in}}%
\pgfpathlineto{\pgfqpoint{2.564816in}{1.684672in}}%
\pgfpathlineto{\pgfqpoint{2.565050in}{1.667059in}}%
\pgfpathlineto{\pgfqpoint{2.565284in}{1.673848in}}%
\pgfpathlineto{\pgfqpoint{2.566219in}{1.769938in}}%
\pgfpathlineto{\pgfqpoint{2.566453in}{1.738304in}}%
\pgfpathlineto{\pgfqpoint{2.566921in}{1.690346in}}%
\pgfpathlineto{\pgfqpoint{2.567389in}{1.748984in}}%
\pgfpathlineto{\pgfqpoint{2.567623in}{1.747129in}}%
\pgfpathlineto{\pgfqpoint{2.570195in}{1.585268in}}%
\pgfpathlineto{\pgfqpoint{2.572767in}{1.919760in}}%
\pgfpathlineto{\pgfqpoint{2.573001in}{1.850055in}}%
\pgfpathlineto{\pgfqpoint{2.573937in}{1.700700in}}%
\pgfpathlineto{\pgfqpoint{2.574405in}{1.700774in}}%
\pgfpathlineto{\pgfqpoint{2.575808in}{1.630044in}}%
\pgfpathlineto{\pgfqpoint{2.576042in}{1.593973in}}%
\pgfpathlineto{\pgfqpoint{2.576509in}{1.683826in}}%
\pgfpathlineto{\pgfqpoint{2.577912in}{1.785975in}}%
\pgfpathlineto{\pgfqpoint{2.578848in}{1.680201in}}%
\pgfpathlineto{\pgfqpoint{2.579316in}{1.718785in}}%
\pgfpathlineto{\pgfqpoint{2.579783in}{1.748551in}}%
\pgfpathlineto{\pgfqpoint{2.580017in}{1.694511in}}%
\pgfpathlineto{\pgfqpoint{2.580251in}{1.718240in}}%
\pgfpathlineto{\pgfqpoint{2.582122in}{1.570603in}}%
\pgfpathlineto{\pgfqpoint{2.580953in}{1.724166in}}%
\pgfpathlineto{\pgfqpoint{2.582823in}{1.615460in}}%
\pgfpathlineto{\pgfqpoint{2.583993in}{1.857640in}}%
\pgfpathlineto{\pgfqpoint{2.584694in}{1.842360in}}%
\pgfpathlineto{\pgfqpoint{2.587033in}{1.548119in}}%
\pgfpathlineto{\pgfqpoint{2.587734in}{1.611672in}}%
\pgfpathlineto{\pgfqpoint{2.588436in}{1.577225in}}%
\pgfpathlineto{\pgfqpoint{2.588670in}{1.577443in}}%
\pgfpathlineto{\pgfqpoint{2.589371in}{1.621731in}}%
\pgfpathlineto{\pgfqpoint{2.590775in}{1.826224in}}%
\pgfpathlineto{\pgfqpoint{2.591009in}{1.808958in}}%
\pgfpathlineto{\pgfqpoint{2.592646in}{1.709772in}}%
\pgfpathlineto{\pgfqpoint{2.593113in}{1.727060in}}%
\pgfpathlineto{\pgfqpoint{2.593581in}{1.774920in}}%
\pgfpathlineto{\pgfqpoint{2.594049in}{1.731755in}}%
\pgfpathlineto{\pgfqpoint{2.594750in}{1.672659in}}%
\pgfpathlineto{\pgfqpoint{2.595920in}{1.530428in}}%
\pgfpathlineto{\pgfqpoint{2.596153in}{1.561588in}}%
\pgfpathlineto{\pgfqpoint{2.597323in}{1.665301in}}%
\pgfpathlineto{\pgfqpoint{2.597557in}{1.640456in}}%
\pgfpathlineto{\pgfqpoint{2.597790in}{1.614633in}}%
\pgfpathlineto{\pgfqpoint{2.598024in}{1.682535in}}%
\pgfpathlineto{\pgfqpoint{2.599194in}{1.869885in}}%
\pgfpathlineto{\pgfqpoint{2.599427in}{1.857610in}}%
\pgfpathlineto{\pgfqpoint{2.599661in}{1.853778in}}%
\pgfpathlineto{\pgfqpoint{2.601766in}{1.528629in}}%
\pgfpathlineto{\pgfqpoint{2.602468in}{1.590536in}}%
\pgfpathlineto{\pgfqpoint{2.605508in}{1.793804in}}%
\pgfpathlineto{\pgfqpoint{2.606443in}{1.706001in}}%
\pgfpathlineto{\pgfqpoint{2.606677in}{1.683149in}}%
\pgfpathlineto{\pgfqpoint{2.606911in}{1.729061in}}%
\pgfpathlineto{\pgfqpoint{2.607379in}{1.704796in}}%
\pgfpathlineto{\pgfqpoint{2.607613in}{1.703545in}}%
\pgfpathlineto{\pgfqpoint{2.607846in}{1.711555in}}%
\pgfpathlineto{\pgfqpoint{2.608080in}{1.695450in}}%
\pgfpathlineto{\pgfqpoint{2.609250in}{1.617791in}}%
\pgfpathlineto{\pgfqpoint{2.609483in}{1.638243in}}%
\pgfpathlineto{\pgfqpoint{2.610653in}{1.794975in}}%
\pgfpathlineto{\pgfqpoint{2.610887in}{1.782631in}}%
\pgfpathlineto{\pgfqpoint{2.611354in}{1.706842in}}%
\pgfpathlineto{\pgfqpoint{2.611822in}{1.812884in}}%
\pgfpathlineto{\pgfqpoint{2.612524in}{1.893643in}}%
\pgfpathlineto{\pgfqpoint{2.612757in}{1.839172in}}%
\pgfpathlineto{\pgfqpoint{2.614394in}{1.598366in}}%
\pgfpathlineto{\pgfqpoint{2.615564in}{1.682681in}}%
\pgfpathlineto{\pgfqpoint{2.616031in}{1.648093in}}%
\pgfpathlineto{\pgfqpoint{2.616265in}{1.635363in}}%
\pgfpathlineto{\pgfqpoint{2.617668in}{1.921392in}}%
\pgfpathlineto{\pgfqpoint{2.617902in}{1.900358in}}%
\pgfpathlineto{\pgfqpoint{2.619072in}{1.758102in}}%
\pgfpathlineto{\pgfqpoint{2.619305in}{1.807454in}}%
\pgfpathlineto{\pgfqpoint{2.619773in}{1.908822in}}%
\pgfpathlineto{\pgfqpoint{2.620475in}{1.828031in}}%
\pgfpathlineto{\pgfqpoint{2.622112in}{1.561769in}}%
\pgfpathlineto{\pgfqpoint{2.622346in}{1.571058in}}%
\pgfpathlineto{\pgfqpoint{2.624918in}{1.983791in}}%
\pgfpathlineto{\pgfqpoint{2.625152in}{1.978755in}}%
\pgfpathlineto{\pgfqpoint{2.625386in}{2.001073in}}%
\pgfpathlineto{\pgfqpoint{2.625620in}{1.980140in}}%
\pgfpathlineto{\pgfqpoint{2.628192in}{1.686166in}}%
\pgfpathlineto{\pgfqpoint{2.630063in}{1.813665in}}%
\pgfpathlineto{\pgfqpoint{2.632635in}{1.996122in}}%
\pgfpathlineto{\pgfqpoint{2.632869in}{1.980778in}}%
\pgfpathlineto{\pgfqpoint{2.633571in}{1.856749in}}%
\pgfpathlineto{\pgfqpoint{2.634272in}{1.897499in}}%
\pgfpathlineto{\pgfqpoint{2.634740in}{1.910257in}}%
\pgfpathlineto{\pgfqpoint{2.635909in}{1.782753in}}%
\pgfpathlineto{\pgfqpoint{2.636143in}{1.807080in}}%
\pgfpathlineto{\pgfqpoint{2.636377in}{1.859330in}}%
\pgfpathlineto{\pgfqpoint{2.636845in}{1.794572in}}%
\pgfpathlineto{\pgfqpoint{2.637547in}{1.699179in}}%
\pgfpathlineto{\pgfqpoint{2.638014in}{1.740812in}}%
\pgfpathlineto{\pgfqpoint{2.639184in}{1.892501in}}%
\pgfpathlineto{\pgfqpoint{2.639417in}{1.834247in}}%
\pgfpathlineto{\pgfqpoint{2.639651in}{1.788321in}}%
\pgfpathlineto{\pgfqpoint{2.640119in}{1.881714in}}%
\pgfpathlineto{\pgfqpoint{2.640587in}{1.877140in}}%
\pgfpathlineto{\pgfqpoint{2.641522in}{1.963163in}}%
\pgfpathlineto{\pgfqpoint{2.641990in}{2.091828in}}%
\pgfpathlineto{\pgfqpoint{2.642458in}{2.024710in}}%
\pgfpathlineto{\pgfqpoint{2.643627in}{1.703742in}}%
\pgfpathlineto{\pgfqpoint{2.644328in}{1.787799in}}%
\pgfpathlineto{\pgfqpoint{2.644562in}{1.808468in}}%
\pgfpathlineto{\pgfqpoint{2.644796in}{1.759913in}}%
\pgfpathlineto{\pgfqpoint{2.645030in}{1.721963in}}%
\pgfpathlineto{\pgfqpoint{2.645498in}{1.779353in}}%
\pgfpathlineto{\pgfqpoint{2.646901in}{1.886620in}}%
\pgfpathlineto{\pgfqpoint{2.647836in}{1.960367in}}%
\pgfpathlineto{\pgfqpoint{2.648304in}{1.931661in}}%
\pgfpathlineto{\pgfqpoint{2.649473in}{1.991425in}}%
\pgfpathlineto{\pgfqpoint{2.649707in}{1.990643in}}%
\pgfpathlineto{\pgfqpoint{2.650175in}{1.916199in}}%
\pgfpathlineto{\pgfqpoint{2.651110in}{1.923945in}}%
\pgfpathlineto{\pgfqpoint{2.652981in}{1.748738in}}%
\pgfpathlineto{\pgfqpoint{2.654384in}{1.888935in}}%
\pgfpathlineto{\pgfqpoint{2.653683in}{1.743857in}}%
\pgfpathlineto{\pgfqpoint{2.654618in}{1.857206in}}%
\pgfpathlineto{\pgfqpoint{2.655320in}{1.879909in}}%
\pgfpathlineto{\pgfqpoint{2.656255in}{1.814676in}}%
\pgfpathlineto{\pgfqpoint{2.656723in}{1.829724in}}%
\pgfpathlineto{\pgfqpoint{2.656957in}{1.845503in}}%
\pgfpathlineto{\pgfqpoint{2.657191in}{1.816475in}}%
\pgfpathlineto{\pgfqpoint{2.657425in}{1.781572in}}%
\pgfpathlineto{\pgfqpoint{2.657892in}{1.843544in}}%
\pgfpathlineto{\pgfqpoint{2.658828in}{1.954393in}}%
\pgfpathlineto{\pgfqpoint{2.659062in}{2.022200in}}%
\pgfpathlineto{\pgfqpoint{2.659763in}{1.989720in}}%
\pgfpathlineto{\pgfqpoint{2.662336in}{1.739444in}}%
\pgfpathlineto{\pgfqpoint{2.662569in}{1.787552in}}%
\pgfpathlineto{\pgfqpoint{2.663505in}{1.862121in}}%
\pgfpathlineto{\pgfqpoint{2.663739in}{1.826626in}}%
\pgfpathlineto{\pgfqpoint{2.664674in}{1.754918in}}%
\pgfpathlineto{\pgfqpoint{2.664908in}{1.772762in}}%
\pgfpathlineto{\pgfqpoint{2.665376in}{1.846501in}}%
\pgfpathlineto{\pgfqpoint{2.665843in}{1.764571in}}%
\pgfpathlineto{\pgfqpoint{2.666545in}{1.780754in}}%
\pgfpathlineto{\pgfqpoint{2.667948in}{2.012848in}}%
\pgfpathlineto{\pgfqpoint{2.668416in}{1.979680in}}%
\pgfpathlineto{\pgfqpoint{2.668884in}{2.014688in}}%
\pgfpathlineto{\pgfqpoint{2.671222in}{1.661401in}}%
\pgfpathlineto{\pgfqpoint{2.674496in}{2.054348in}}%
\pgfpathlineto{\pgfqpoint{2.676133in}{1.837974in}}%
\pgfpathlineto{\pgfqpoint{2.676601in}{1.723285in}}%
\pgfpathlineto{\pgfqpoint{2.677536in}{1.776922in}}%
\pgfpathlineto{\pgfqpoint{2.678004in}{1.824666in}}%
\pgfpathlineto{\pgfqpoint{2.678238in}{1.778531in}}%
\pgfpathlineto{\pgfqpoint{2.678940in}{1.667502in}}%
\pgfpathlineto{\pgfqpoint{2.679407in}{1.693227in}}%
\pgfpathlineto{\pgfqpoint{2.681044in}{2.035966in}}%
\pgfpathlineto{\pgfqpoint{2.681746in}{1.964657in}}%
\pgfpathlineto{\pgfqpoint{2.685020in}{1.690154in}}%
\pgfpathlineto{\pgfqpoint{2.685488in}{1.663735in}}%
\pgfpathlineto{\pgfqpoint{2.685955in}{1.688058in}}%
\pgfpathlineto{\pgfqpoint{2.686189in}{1.691724in}}%
\pgfpathlineto{\pgfqpoint{2.686423in}{1.688130in}}%
\pgfpathlineto{\pgfqpoint{2.688528in}{2.041904in}}%
\pgfpathlineto{\pgfqpoint{2.689229in}{1.904318in}}%
\pgfpathlineto{\pgfqpoint{2.689463in}{1.897228in}}%
\pgfpathlineto{\pgfqpoint{2.691100in}{1.712014in}}%
\pgfpathlineto{\pgfqpoint{2.691334in}{1.712773in}}%
\pgfpathlineto{\pgfqpoint{2.691568in}{1.722666in}}%
\pgfpathlineto{\pgfqpoint{2.691802in}{1.700611in}}%
\pgfpathlineto{\pgfqpoint{2.692036in}{1.709195in}}%
\pgfpathlineto{\pgfqpoint{2.692503in}{1.670462in}}%
\pgfpathlineto{\pgfqpoint{2.692971in}{1.695890in}}%
\pgfpathlineto{\pgfqpoint{2.693205in}{1.725064in}}%
\pgfpathlineto{\pgfqpoint{2.693673in}{1.693254in}}%
\pgfpathlineto{\pgfqpoint{2.694140in}{1.703934in}}%
\pgfpathlineto{\pgfqpoint{2.694608in}{1.767641in}}%
\pgfpathlineto{\pgfqpoint{2.695777in}{1.958021in}}%
\pgfpathlineto{\pgfqpoint{2.696011in}{1.916920in}}%
\pgfpathlineto{\pgfqpoint{2.698116in}{1.707516in}}%
\pgfpathlineto{\pgfqpoint{2.699052in}{1.703397in}}%
\pgfpathlineto{\pgfqpoint{2.699519in}{1.790249in}}%
\pgfpathlineto{\pgfqpoint{2.701624in}{1.621861in}}%
\pgfpathlineto{\pgfqpoint{2.701858in}{1.629599in}}%
\pgfpathlineto{\pgfqpoint{2.702092in}{1.624760in}}%
\pgfpathlineto{\pgfqpoint{2.703963in}{1.856641in}}%
\pgfpathlineto{\pgfqpoint{2.704196in}{1.855778in}}%
\pgfpathlineto{\pgfqpoint{2.704664in}{1.820287in}}%
\pgfpathlineto{\pgfqpoint{2.705366in}{1.847945in}}%
\pgfpathlineto{\pgfqpoint{2.705600in}{1.878119in}}%
\pgfpathlineto{\pgfqpoint{2.705833in}{1.839535in}}%
\pgfpathlineto{\pgfqpoint{2.707704in}{1.603912in}}%
\pgfpathlineto{\pgfqpoint{2.707938in}{1.578694in}}%
\pgfpathlineto{\pgfqpoint{2.708406in}{1.642490in}}%
\pgfpathlineto{\pgfqpoint{2.710277in}{1.813433in}}%
\pgfpathlineto{\pgfqpoint{2.710978in}{1.746069in}}%
\pgfpathlineto{\pgfqpoint{2.711446in}{1.660391in}}%
\pgfpathlineto{\pgfqpoint{2.712148in}{1.706019in}}%
\pgfpathlineto{\pgfqpoint{2.713785in}{1.796832in}}%
\pgfpathlineto{\pgfqpoint{2.714954in}{1.595474in}}%
\pgfpathlineto{\pgfqpoint{2.715889in}{1.643191in}}%
\pgfpathlineto{\pgfqpoint{2.716591in}{1.740418in}}%
\pgfpathlineto{\pgfqpoint{2.717293in}{1.734830in}}%
\pgfpathlineto{\pgfqpoint{2.717526in}{1.731433in}}%
\pgfpathlineto{\pgfqpoint{2.717760in}{1.756085in}}%
\pgfpathlineto{\pgfqpoint{2.718228in}{1.711516in}}%
\pgfpathlineto{\pgfqpoint{2.720567in}{1.542113in}}%
\pgfpathlineto{\pgfqpoint{2.720800in}{1.590980in}}%
\pgfpathlineto{\pgfqpoint{2.721268in}{1.639427in}}%
\pgfpathlineto{\pgfqpoint{2.721736in}{1.619299in}}%
\pgfpathlineto{\pgfqpoint{2.721970in}{1.586440in}}%
\pgfpathlineto{\pgfqpoint{2.722437in}{1.661816in}}%
\pgfpathlineto{\pgfqpoint{2.722671in}{1.666302in}}%
\pgfpathlineto{\pgfqpoint{2.723841in}{1.821018in}}%
\pgfpathlineto{\pgfqpoint{2.724308in}{1.788992in}}%
\pgfpathlineto{\pgfqpoint{2.725244in}{1.707996in}}%
\pgfpathlineto{\pgfqpoint{2.726881in}{1.469387in}}%
\pgfpathlineto{\pgfqpoint{2.727115in}{1.462931in}}%
\pgfpathlineto{\pgfqpoint{2.728985in}{1.785180in}}%
\pgfpathlineto{\pgfqpoint{2.730389in}{1.753293in}}%
\pgfpathlineto{\pgfqpoint{2.732961in}{1.570573in}}%
\pgfpathlineto{\pgfqpoint{2.733195in}{1.614966in}}%
\pgfpathlineto{\pgfqpoint{2.733663in}{1.661756in}}%
\pgfpathlineto{\pgfqpoint{2.733897in}{1.629646in}}%
\pgfpathlineto{\pgfqpoint{2.734598in}{1.447116in}}%
\pgfpathlineto{\pgfqpoint{2.735300in}{1.519212in}}%
\pgfpathlineto{\pgfqpoint{2.735534in}{1.508813in}}%
\pgfpathlineto{\pgfqpoint{2.735767in}{1.538500in}}%
\pgfpathlineto{\pgfqpoint{2.737872in}{1.791453in}}%
\pgfpathlineto{\pgfqpoint{2.738106in}{1.740609in}}%
\pgfpathlineto{\pgfqpoint{2.738340in}{1.762727in}}%
\pgfpathlineto{\pgfqpoint{2.738808in}{1.704385in}}%
\pgfpathlineto{\pgfqpoint{2.740678in}{1.402785in}}%
\pgfpathlineto{\pgfqpoint{2.740912in}{1.405813in}}%
\pgfpathlineto{\pgfqpoint{2.742082in}{1.584324in}}%
\pgfpathlineto{\pgfqpoint{2.742783in}{1.554777in}}%
\pgfpathlineto{\pgfqpoint{2.743017in}{1.542883in}}%
\pgfpathlineto{\pgfqpoint{2.743251in}{1.571725in}}%
\pgfpathlineto{\pgfqpoint{2.743952in}{1.740148in}}%
\pgfpathlineto{\pgfqpoint{2.744654in}{1.648350in}}%
\pgfpathlineto{\pgfqpoint{2.745122in}{1.644359in}}%
\pgfpathlineto{\pgfqpoint{2.746525in}{1.513430in}}%
\pgfpathlineto{\pgfqpoint{2.746759in}{1.477325in}}%
\pgfpathlineto{\pgfqpoint{2.747227in}{1.521991in}}%
\pgfpathlineto{\pgfqpoint{2.748162in}{1.686881in}}%
\pgfpathlineto{\pgfqpoint{2.748630in}{1.613142in}}%
\pgfpathlineto{\pgfqpoint{2.749799in}{1.414374in}}%
\pgfpathlineto{\pgfqpoint{2.750501in}{1.431765in}}%
\pgfpathlineto{\pgfqpoint{2.752371in}{1.737315in}}%
\pgfpathlineto{\pgfqpoint{2.752605in}{1.723710in}}%
\pgfpathlineto{\pgfqpoint{2.753073in}{1.692486in}}%
\pgfpathlineto{\pgfqpoint{2.754242in}{1.486172in}}%
\pgfpathlineto{\pgfqpoint{2.754710in}{1.514546in}}%
\pgfpathlineto{\pgfqpoint{2.755412in}{1.584678in}}%
\pgfpathlineto{\pgfqpoint{2.755645in}{1.566020in}}%
\pgfpathlineto{\pgfqpoint{2.756113in}{1.501458in}}%
\pgfpathlineto{\pgfqpoint{2.756815in}{1.518231in}}%
\pgfpathlineto{\pgfqpoint{2.758452in}{1.461603in}}%
\pgfpathlineto{\pgfqpoint{2.759153in}{1.505272in}}%
\pgfpathlineto{\pgfqpoint{2.760323in}{1.655900in}}%
\pgfpathlineto{\pgfqpoint{2.760790in}{1.635530in}}%
\pgfpathlineto{\pgfqpoint{2.762194in}{1.486204in}}%
\pgfpathlineto{\pgfqpoint{2.762661in}{1.528352in}}%
\pgfpathlineto{\pgfqpoint{2.764064in}{1.650999in}}%
\pgfpathlineto{\pgfqpoint{2.764532in}{1.646986in}}%
\pgfpathlineto{\pgfqpoint{2.764766in}{1.693289in}}%
\pgfpathlineto{\pgfqpoint{2.766637in}{1.359609in}}%
\pgfpathlineto{\pgfqpoint{2.767105in}{1.420477in}}%
\pgfpathlineto{\pgfqpoint{2.768508in}{1.523672in}}%
\pgfpathlineto{\pgfqpoint{2.770612in}{1.697922in}}%
\pgfpathlineto{\pgfqpoint{2.770846in}{1.650207in}}%
\pgfpathlineto{\pgfqpoint{2.772717in}{1.374789in}}%
\pgfpathlineto{\pgfqpoint{2.773185in}{1.435612in}}%
\pgfpathlineto{\pgfqpoint{2.773653in}{1.397647in}}%
\pgfpathlineto{\pgfqpoint{2.774354in}{1.349086in}}%
\pgfpathlineto{\pgfqpoint{2.774822in}{1.365034in}}%
\pgfpathlineto{\pgfqpoint{2.776927in}{1.539876in}}%
\pgfpathlineto{\pgfqpoint{2.777161in}{1.514212in}}%
\pgfpathlineto{\pgfqpoint{2.777394in}{1.539521in}}%
\pgfpathlineto{\pgfqpoint{2.778096in}{1.662022in}}%
\pgfpathlineto{\pgfqpoint{2.778564in}{1.589962in}}%
\pgfpathlineto{\pgfqpoint{2.780902in}{1.270047in}}%
\pgfpathlineto{\pgfqpoint{2.783475in}{1.509924in}}%
\pgfpathlineto{\pgfqpoint{2.784644in}{1.604017in}}%
\pgfpathlineto{\pgfqpoint{2.784878in}{1.595490in}}%
\pgfpathlineto{\pgfqpoint{2.786281in}{1.478989in}}%
\pgfpathlineto{\pgfqpoint{2.788386in}{1.282163in}}%
\pgfpathlineto{\pgfqpoint{2.788620in}{1.326827in}}%
\pgfpathlineto{\pgfqpoint{2.790257in}{1.548349in}}%
\pgfpathlineto{\pgfqpoint{2.790490in}{1.543704in}}%
\pgfpathlineto{\pgfqpoint{2.791660in}{1.308729in}}%
\pgfpathlineto{\pgfqpoint{2.792361in}{1.374493in}}%
\pgfpathlineto{\pgfqpoint{2.793063in}{1.496324in}}%
\pgfpathlineto{\pgfqpoint{2.793765in}{1.473407in}}%
\pgfpathlineto{\pgfqpoint{2.794700in}{1.349202in}}%
\pgfpathlineto{\pgfqpoint{2.794934in}{1.371906in}}%
\pgfpathlineto{\pgfqpoint{2.795635in}{1.546926in}}%
\pgfpathlineto{\pgfqpoint{2.796805in}{1.462791in}}%
\pgfpathlineto{\pgfqpoint{2.800780in}{1.267434in}}%
\pgfpathlineto{\pgfqpoint{2.797506in}{1.466661in}}%
\pgfpathlineto{\pgfqpoint{2.801014in}{1.292985in}}%
\pgfpathlineto{\pgfqpoint{2.802183in}{1.519364in}}%
\pgfpathlineto{\pgfqpoint{2.802651in}{1.467798in}}%
\pgfpathlineto{\pgfqpoint{2.802885in}{1.445301in}}%
\pgfpathlineto{\pgfqpoint{2.803353in}{1.490415in}}%
\pgfpathlineto{\pgfqpoint{2.804054in}{1.598469in}}%
\pgfpathlineto{\pgfqpoint{2.804522in}{1.531629in}}%
\pgfpathlineto{\pgfqpoint{2.805457in}{1.321114in}}%
\pgfpathlineto{\pgfqpoint{2.806627in}{1.385213in}}%
\pgfpathlineto{\pgfqpoint{2.806861in}{1.378398in}}%
\pgfpathlineto{\pgfqpoint{2.807562in}{1.239495in}}%
\pgfpathlineto{\pgfqpoint{2.808030in}{1.277359in}}%
\pgfpathlineto{\pgfqpoint{2.808498in}{1.326035in}}%
\pgfpathlineto{\pgfqpoint{2.810369in}{1.621460in}}%
\pgfpathlineto{\pgfqpoint{2.812239in}{1.346703in}}%
\pgfpathlineto{\pgfqpoint{2.813175in}{1.461981in}}%
\pgfpathlineto{\pgfqpoint{2.813409in}{1.426701in}}%
\pgfpathlineto{\pgfqpoint{2.814344in}{1.306644in}}%
\pgfpathlineto{\pgfqpoint{2.814812in}{1.340937in}}%
\pgfpathlineto{\pgfqpoint{2.815046in}{1.351456in}}%
\pgfpathlineto{\pgfqpoint{2.815280in}{1.325876in}}%
\pgfpathlineto{\pgfqpoint{2.815513in}{1.267573in}}%
\pgfpathlineto{\pgfqpoint{2.815981in}{1.383211in}}%
\pgfpathlineto{\pgfqpoint{2.817150in}{1.488424in}}%
\pgfpathlineto{\pgfqpoint{2.817384in}{1.480207in}}%
\pgfpathlineto{\pgfqpoint{2.817852in}{1.526634in}}%
\pgfpathlineto{\pgfqpoint{2.818554in}{1.588610in}}%
\pgfpathlineto{\pgfqpoint{2.819021in}{1.557928in}}%
\pgfpathlineto{\pgfqpoint{2.819957in}{1.437932in}}%
\pgfpathlineto{\pgfqpoint{2.820658in}{1.443397in}}%
\pgfpathlineto{\pgfqpoint{2.821360in}{1.333487in}}%
\pgfpathlineto{\pgfqpoint{2.821594in}{1.337972in}}%
\pgfpathlineto{\pgfqpoint{2.823465in}{1.622790in}}%
\pgfpathlineto{\pgfqpoint{2.823932in}{1.520294in}}%
\pgfpathlineto{\pgfqpoint{2.824868in}{1.401549in}}%
\pgfpathlineto{\pgfqpoint{2.825336in}{1.428356in}}%
\pgfpathlineto{\pgfqpoint{2.825569in}{1.440145in}}%
\pgfpathlineto{\pgfqpoint{2.825803in}{1.412241in}}%
\pgfpathlineto{\pgfqpoint{2.826505in}{1.302875in}}%
\pgfpathlineto{\pgfqpoint{2.826739in}{1.340048in}}%
\pgfpathlineto{\pgfqpoint{2.829077in}{1.689557in}}%
\pgfpathlineto{\pgfqpoint{2.829779in}{1.559250in}}%
\pgfpathlineto{\pgfqpoint{2.830714in}{1.407987in}}%
\pgfpathlineto{\pgfqpoint{2.831182in}{1.445776in}}%
\pgfpathlineto{\pgfqpoint{2.832819in}{1.364182in}}%
\pgfpathlineto{\pgfqpoint{2.833053in}{1.373007in}}%
\pgfpathlineto{\pgfqpoint{2.834924in}{1.594377in}}%
\pgfpathlineto{\pgfqpoint{2.835158in}{1.618358in}}%
\pgfpathlineto{\pgfqpoint{2.835625in}{1.562670in}}%
\pgfpathlineto{\pgfqpoint{2.835859in}{1.596240in}}%
\pgfpathlineto{\pgfqpoint{2.836327in}{1.534974in}}%
\pgfpathlineto{\pgfqpoint{2.836795in}{1.601764in}}%
\pgfpathlineto{\pgfqpoint{2.837496in}{1.647582in}}%
\pgfpathlineto{\pgfqpoint{2.837730in}{1.644032in}}%
\pgfpathlineto{\pgfqpoint{2.838432in}{1.647558in}}%
\pgfpathlineto{\pgfqpoint{2.839601in}{1.522636in}}%
\pgfpathlineto{\pgfqpoint{2.839835in}{1.520878in}}%
\pgfpathlineto{\pgfqpoint{2.840069in}{1.527046in}}%
\pgfpathlineto{\pgfqpoint{2.840770in}{1.378373in}}%
\pgfpathlineto{\pgfqpoint{2.841472in}{1.482421in}}%
\pgfpathlineto{\pgfqpoint{2.842641in}{1.578827in}}%
\pgfpathlineto{\pgfqpoint{2.842875in}{1.558287in}}%
\pgfpathlineto{\pgfqpoint{2.843343in}{1.534335in}}%
\pgfpathlineto{\pgfqpoint{2.843577in}{1.567533in}}%
\pgfpathlineto{\pgfqpoint{2.843810in}{1.556384in}}%
\pgfpathlineto{\pgfqpoint{2.845214in}{1.674514in}}%
\pgfpathlineto{\pgfqpoint{2.845681in}{1.632647in}}%
\pgfpathlineto{\pgfqpoint{2.845915in}{1.677357in}}%
\pgfpathlineto{\pgfqpoint{2.846851in}{1.723963in}}%
\pgfpathlineto{\pgfqpoint{2.847084in}{1.697149in}}%
\pgfpathlineto{\pgfqpoint{2.848254in}{1.477030in}}%
\pgfpathlineto{\pgfqpoint{2.848955in}{1.482110in}}%
\pgfpathlineto{\pgfqpoint{2.850125in}{1.634383in}}%
\pgfpathlineto{\pgfqpoint{2.850358in}{1.589051in}}%
\pgfpathlineto{\pgfqpoint{2.850592in}{1.588406in}}%
\pgfpathlineto{\pgfqpoint{2.850826in}{1.584254in}}%
\pgfpathlineto{\pgfqpoint{2.851060in}{1.588697in}}%
\pgfpathlineto{\pgfqpoint{2.851762in}{1.596805in}}%
\pgfpathlineto{\pgfqpoint{2.852229in}{1.523448in}}%
\pgfpathlineto{\pgfqpoint{2.853165in}{1.653196in}}%
\pgfpathlineto{\pgfqpoint{2.853633in}{1.593136in}}%
\pgfpathlineto{\pgfqpoint{2.853866in}{1.569553in}}%
\pgfpathlineto{\pgfqpoint{2.854334in}{1.600111in}}%
\pgfpathlineto{\pgfqpoint{2.855503in}{1.750976in}}%
\pgfpathlineto{\pgfqpoint{2.855737in}{1.815176in}}%
\pgfpathlineto{\pgfqpoint{2.856439in}{1.754815in}}%
\pgfpathlineto{\pgfqpoint{2.857374in}{1.563586in}}%
\pgfpathlineto{\pgfqpoint{2.858076in}{1.637052in}}%
\pgfpathlineto{\pgfqpoint{2.860181in}{1.536583in}}%
\pgfpathlineto{\pgfqpoint{2.860648in}{1.555491in}}%
\pgfpathlineto{\pgfqpoint{2.861350in}{1.673240in}}%
\pgfpathlineto{\pgfqpoint{2.861818in}{1.641924in}}%
\pgfpathlineto{\pgfqpoint{2.862051in}{1.649554in}}%
\pgfpathlineto{\pgfqpoint{2.863221in}{1.744681in}}%
\pgfpathlineto{\pgfqpoint{2.864390in}{1.604650in}}%
\pgfpathlineto{\pgfqpoint{2.865325in}{1.661495in}}%
\pgfpathlineto{\pgfqpoint{2.866027in}{1.634350in}}%
\pgfpathlineto{\pgfqpoint{2.866962in}{1.741598in}}%
\pgfpathlineto{\pgfqpoint{2.868599in}{1.526861in}}%
\pgfpathlineto{\pgfqpoint{2.870470in}{1.784134in}}%
\pgfpathlineto{\pgfqpoint{2.873277in}{1.583647in}}%
\pgfpathlineto{\pgfqpoint{2.870938in}{1.791315in}}%
\pgfpathlineto{\pgfqpoint{2.873744in}{1.644744in}}%
\pgfpathlineto{\pgfqpoint{2.873978in}{1.660960in}}%
\pgfpathlineto{\pgfqpoint{2.874680in}{1.512870in}}%
\pgfpathlineto{\pgfqpoint{2.875381in}{1.558776in}}%
\pgfpathlineto{\pgfqpoint{2.877252in}{1.865585in}}%
\pgfpathlineto{\pgfqpoint{2.877954in}{1.751689in}}%
\pgfpathlineto{\pgfqpoint{2.879591in}{1.619625in}}%
\pgfpathlineto{\pgfqpoint{2.879825in}{1.635325in}}%
\pgfpathlineto{\pgfqpoint{2.880292in}{1.618062in}}%
\pgfpathlineto{\pgfqpoint{2.880526in}{1.624822in}}%
\pgfpathlineto{\pgfqpoint{2.880994in}{1.593067in}}%
\pgfpathlineto{\pgfqpoint{2.881228in}{1.550837in}}%
\pgfpathlineto{\pgfqpoint{2.881929in}{1.614339in}}%
\pgfpathlineto{\pgfqpoint{2.884268in}{1.813644in}}%
\pgfpathlineto{\pgfqpoint{2.884970in}{1.848720in}}%
\pgfpathlineto{\pgfqpoint{2.886841in}{1.684638in}}%
\pgfpathlineto{\pgfqpoint{2.887074in}{1.721650in}}%
\pgfpathlineto{\pgfqpoint{2.887542in}{1.646530in}}%
\pgfpathlineto{\pgfqpoint{2.888010in}{1.511608in}}%
\pgfpathlineto{\pgfqpoint{2.888711in}{1.594206in}}%
\pgfpathlineto{\pgfqpoint{2.889881in}{1.752756in}}%
\pgfpathlineto{\pgfqpoint{2.890348in}{1.739768in}}%
\pgfpathlineto{\pgfqpoint{2.890816in}{1.720941in}}%
\pgfpathlineto{\pgfqpoint{2.891284in}{1.731592in}}%
\pgfpathlineto{\pgfqpoint{2.892219in}{1.788909in}}%
\pgfpathlineto{\pgfqpoint{2.892921in}{1.778779in}}%
\pgfpathlineto{\pgfqpoint{2.895727in}{1.513759in}}%
\pgfpathlineto{\pgfqpoint{2.896195in}{1.621921in}}%
\pgfpathlineto{\pgfqpoint{2.897598in}{1.830694in}}%
\pgfpathlineto{\pgfqpoint{2.897832in}{1.820621in}}%
\pgfpathlineto{\pgfqpoint{2.898300in}{1.757517in}}%
\pgfpathlineto{\pgfqpoint{2.898767in}{1.806902in}}%
\pgfpathlineto{\pgfqpoint{2.899001in}{1.840419in}}%
\pgfpathlineto{\pgfqpoint{2.899469in}{1.810497in}}%
\pgfpathlineto{\pgfqpoint{2.901574in}{1.590294in}}%
\pgfpathlineto{\pgfqpoint{2.902743in}{1.741889in}}%
\pgfpathlineto{\pgfqpoint{2.902977in}{1.704641in}}%
\pgfpathlineto{\pgfqpoint{2.903912in}{1.754738in}}%
\pgfpathlineto{\pgfqpoint{2.904380in}{1.814732in}}%
\pgfpathlineto{\pgfqpoint{2.904614in}{1.738996in}}%
\pgfpathlineto{\pgfqpoint{2.905549in}{1.617269in}}%
\pgfpathlineto{\pgfqpoint{2.905783in}{1.642402in}}%
\pgfpathlineto{\pgfqpoint{2.906719in}{1.812071in}}%
\pgfpathlineto{\pgfqpoint{2.907186in}{1.767724in}}%
\pgfpathlineto{\pgfqpoint{2.907654in}{1.713375in}}%
\pgfpathlineto{\pgfqpoint{2.908589in}{1.597408in}}%
\pgfpathlineto{\pgfqpoint{2.909291in}{1.652724in}}%
\pgfpathlineto{\pgfqpoint{2.911162in}{1.879185in}}%
\pgfpathlineto{\pgfqpoint{2.911396in}{1.830118in}}%
\pgfpathlineto{\pgfqpoint{2.912331in}{1.812799in}}%
\pgfpathlineto{\pgfqpoint{2.912565in}{1.840037in}}%
\pgfpathlineto{\pgfqpoint{2.912799in}{1.784007in}}%
\pgfpathlineto{\pgfqpoint{2.914670in}{1.518648in}}%
\pgfpathlineto{\pgfqpoint{2.917476in}{1.945391in}}%
\pgfpathlineto{\pgfqpoint{2.917944in}{1.810293in}}%
\pgfpathlineto{\pgfqpoint{2.918412in}{1.722152in}}%
\pgfpathlineto{\pgfqpoint{2.919815in}{1.535411in}}%
\pgfpathlineto{\pgfqpoint{2.920049in}{1.556675in}}%
\pgfpathlineto{\pgfqpoint{2.920984in}{1.765952in}}%
\pgfpathlineto{\pgfqpoint{2.921452in}{1.652979in}}%
\pgfpathlineto{\pgfqpoint{2.921686in}{1.639196in}}%
\pgfpathlineto{\pgfqpoint{2.921919in}{1.659575in}}%
\pgfpathlineto{\pgfqpoint{2.924258in}{1.827221in}}%
\pgfpathlineto{\pgfqpoint{2.924492in}{1.771568in}}%
\pgfpathlineto{\pgfqpoint{2.925193in}{1.864493in}}%
\pgfpathlineto{\pgfqpoint{2.925427in}{1.803104in}}%
\pgfpathlineto{\pgfqpoint{2.925661in}{1.819442in}}%
\pgfpathlineto{\pgfqpoint{2.926830in}{1.700907in}}%
\pgfpathlineto{\pgfqpoint{2.927064in}{1.751212in}}%
\pgfpathlineto{\pgfqpoint{2.927532in}{1.683721in}}%
\pgfpathlineto{\pgfqpoint{2.928234in}{1.579927in}}%
\pgfpathlineto{\pgfqpoint{2.928701in}{1.628942in}}%
\pgfpathlineto{\pgfqpoint{2.928935in}{1.626713in}}%
\pgfpathlineto{\pgfqpoint{2.929169in}{1.597512in}}%
\pgfpathlineto{\pgfqpoint{2.929403in}{1.662260in}}%
\pgfpathlineto{\pgfqpoint{2.929637in}{1.658423in}}%
\pgfpathlineto{\pgfqpoint{2.930104in}{1.711140in}}%
\pgfpathlineto{\pgfqpoint{2.931040in}{1.891926in}}%
\pgfpathlineto{\pgfqpoint{2.931274in}{1.832779in}}%
\pgfpathlineto{\pgfqpoint{2.931975in}{1.612422in}}%
\pgfpathlineto{\pgfqpoint{2.932677in}{1.657763in}}%
\pgfpathlineto{\pgfqpoint{2.932911in}{1.640739in}}%
\pgfpathlineto{\pgfqpoint{2.933145in}{1.681550in}}%
\pgfpathlineto{\pgfqpoint{2.934548in}{1.822908in}}%
\pgfpathlineto{\pgfqpoint{2.934782in}{1.804830in}}%
\pgfpathlineto{\pgfqpoint{2.935016in}{1.798673in}}%
\pgfpathlineto{\pgfqpoint{2.935483in}{1.869468in}}%
\pgfpathlineto{\pgfqpoint{2.935951in}{1.817312in}}%
\pgfpathlineto{\pgfqpoint{2.937588in}{1.561559in}}%
\pgfpathlineto{\pgfqpoint{2.937822in}{1.579351in}}%
\pgfpathlineto{\pgfqpoint{2.938290in}{1.590366in}}%
\pgfpathlineto{\pgfqpoint{2.939225in}{1.726292in}}%
\pgfpathlineto{\pgfqpoint{2.939459in}{1.690099in}}%
\pgfpathlineto{\pgfqpoint{2.939927in}{1.626800in}}%
\pgfpathlineto{\pgfqpoint{2.940394in}{1.714320in}}%
\pgfpathlineto{\pgfqpoint{2.941096in}{1.742097in}}%
\pgfpathlineto{\pgfqpoint{2.941330in}{1.722882in}}%
\pgfpathlineto{\pgfqpoint{2.941797in}{1.733660in}}%
\pgfpathlineto{\pgfqpoint{2.942265in}{1.711879in}}%
\pgfpathlineto{\pgfqpoint{2.943201in}{1.818875in}}%
\pgfpathlineto{\pgfqpoint{2.943434in}{1.782595in}}%
\pgfpathlineto{\pgfqpoint{2.943902in}{1.723897in}}%
\pgfpathlineto{\pgfqpoint{2.944370in}{1.789677in}}%
\pgfpathlineto{\pgfqpoint{2.944838in}{1.785176in}}%
\pgfpathlineto{\pgfqpoint{2.946241in}{1.523833in}}%
\pgfpathlineto{\pgfqpoint{2.946942in}{1.593274in}}%
\pgfpathlineto{\pgfqpoint{2.948813in}{1.775525in}}%
\pgfpathlineto{\pgfqpoint{2.949047in}{1.750933in}}%
\pgfpathlineto{\pgfqpoint{2.950918in}{1.601942in}}%
\pgfpathlineto{\pgfqpoint{2.952321in}{1.770482in}}%
\pgfpathlineto{\pgfqpoint{2.952789in}{1.763608in}}%
\pgfpathlineto{\pgfqpoint{2.953023in}{1.745844in}}%
\pgfpathlineto{\pgfqpoint{2.953257in}{1.770408in}}%
\pgfpathlineto{\pgfqpoint{2.953490in}{1.805592in}}%
\pgfpathlineto{\pgfqpoint{2.953958in}{1.714511in}}%
\pgfpathlineto{\pgfqpoint{2.954660in}{1.608823in}}%
\pgfpathlineto{\pgfqpoint{2.955361in}{1.633185in}}%
\pgfpathlineto{\pgfqpoint{2.955829in}{1.588243in}}%
\pgfpathlineto{\pgfqpoint{2.956297in}{1.523134in}}%
\pgfpathlineto{\pgfqpoint{2.956998in}{1.565693in}}%
\pgfpathlineto{\pgfqpoint{2.958168in}{1.649218in}}%
\pgfpathlineto{\pgfqpoint{2.958401in}{1.640384in}}%
\pgfpathlineto{\pgfqpoint{2.958635in}{1.625637in}}%
\pgfpathlineto{\pgfqpoint{2.959571in}{1.824356in}}%
\pgfpathlineto{\pgfqpoint{2.960038in}{1.754252in}}%
\pgfpathlineto{\pgfqpoint{2.962143in}{1.561212in}}%
\pgfpathlineto{\pgfqpoint{2.962845in}{1.551836in}}%
\pgfpathlineto{\pgfqpoint{2.963780in}{1.657825in}}%
\pgfpathlineto{\pgfqpoint{2.964248in}{1.665785in}}%
\pgfpathlineto{\pgfqpoint{2.964482in}{1.714894in}}%
\pgfpathlineto{\pgfqpoint{2.964950in}{1.673843in}}%
\pgfpathlineto{\pgfqpoint{2.966587in}{1.507287in}}%
\pgfpathlineto{\pgfqpoint{2.968925in}{1.857031in}}%
\pgfpathlineto{\pgfqpoint{2.971030in}{1.496067in}}%
\pgfpathlineto{\pgfqpoint{2.971264in}{1.549104in}}%
\pgfpathlineto{\pgfqpoint{2.971965in}{1.635418in}}%
\pgfpathlineto{\pgfqpoint{2.972667in}{1.587701in}}%
\pgfpathlineto{\pgfqpoint{2.973836in}{1.404562in}}%
\pgfpathlineto{\pgfqpoint{2.974070in}{1.472483in}}%
\pgfpathlineto{\pgfqpoint{2.975239in}{1.697980in}}%
\pgfpathlineto{\pgfqpoint{2.975941in}{1.655408in}}%
\pgfpathlineto{\pgfqpoint{2.976175in}{1.655418in}}%
\pgfpathlineto{\pgfqpoint{2.976876in}{1.694253in}}%
\pgfpathlineto{\pgfqpoint{2.977110in}{1.687343in}}%
\pgfpathlineto{\pgfqpoint{2.977578in}{1.569779in}}%
\pgfpathlineto{\pgfqpoint{2.978280in}{1.626271in}}%
\pgfpathlineto{\pgfqpoint{2.978513in}{1.651515in}}%
\pgfpathlineto{\pgfqpoint{2.978981in}{1.638653in}}%
\pgfpathlineto{\pgfqpoint{2.981320in}{1.367904in}}%
\pgfpathlineto{\pgfqpoint{2.982021in}{1.418422in}}%
\pgfpathlineto{\pgfqpoint{2.983658in}{1.627914in}}%
\pgfpathlineto{\pgfqpoint{2.984126in}{1.764821in}}%
\pgfpathlineto{\pgfqpoint{2.984594in}{1.685128in}}%
\pgfpathlineto{\pgfqpoint{2.986465in}{1.405620in}}%
\pgfpathlineto{\pgfqpoint{2.986932in}{1.363575in}}%
\pgfpathlineto{\pgfqpoint{2.987400in}{1.302073in}}%
\pgfpathlineto{\pgfqpoint{2.987634in}{1.324769in}}%
\pgfpathlineto{\pgfqpoint{2.989972in}{1.622281in}}%
\pgfpathlineto{\pgfqpoint{2.990206in}{1.597514in}}%
\pgfpathlineto{\pgfqpoint{2.990674in}{1.627889in}}%
\pgfpathlineto{\pgfqpoint{2.990908in}{1.650654in}}%
\pgfpathlineto{\pgfqpoint{2.991142in}{1.620168in}}%
\pgfpathlineto{\pgfqpoint{2.991609in}{1.630619in}}%
\pgfpathlineto{\pgfqpoint{2.993480in}{1.257118in}}%
\pgfpathlineto{\pgfqpoint{2.993948in}{1.340729in}}%
\pgfpathlineto{\pgfqpoint{2.995351in}{1.536077in}}%
\pgfpathlineto{\pgfqpoint{2.995819in}{1.521523in}}%
\pgfpathlineto{\pgfqpoint{2.996521in}{1.576765in}}%
\pgfpathlineto{\pgfqpoint{2.996754in}{1.522100in}}%
\pgfpathlineto{\pgfqpoint{2.996988in}{1.522860in}}%
\pgfpathlineto{\pgfqpoint{2.997924in}{1.637721in}}%
\pgfpathlineto{\pgfqpoint{2.998158in}{1.589030in}}%
\pgfpathlineto{\pgfqpoint{2.999561in}{1.427900in}}%
\pgfpathlineto{\pgfqpoint{2.999795in}{1.457545in}}%
\pgfpathlineto{\pgfqpoint{3.000028in}{1.493268in}}%
\pgfpathlineto{\pgfqpoint{3.000496in}{1.397051in}}%
\pgfpathlineto{\pgfqpoint{3.002133in}{1.347455in}}%
\pgfpathlineto{\pgfqpoint{3.002601in}{1.271750in}}%
\pgfpathlineto{\pgfqpoint{3.002835in}{1.314341in}}%
\pgfpathlineto{\pgfqpoint{3.004472in}{1.526540in}}%
\pgfpathlineto{\pgfqpoint{3.004706in}{1.587281in}}%
\pgfpathlineto{\pgfqpoint{3.005407in}{1.551632in}}%
\pgfpathlineto{\pgfqpoint{3.005875in}{1.489185in}}%
\pgfpathlineto{\pgfqpoint{3.006576in}{1.504089in}}%
\pgfpathlineto{\pgfqpoint{3.006810in}{1.517185in}}%
\pgfpathlineto{\pgfqpoint{3.007278in}{1.505176in}}%
\pgfpathlineto{\pgfqpoint{3.009149in}{1.333316in}}%
\pgfpathlineto{\pgfqpoint{3.009851in}{1.335891in}}%
\pgfpathlineto{\pgfqpoint{3.010084in}{1.338763in}}%
\pgfpathlineto{\pgfqpoint{3.010786in}{1.237504in}}%
\pgfpathlineto{\pgfqpoint{3.011254in}{1.301388in}}%
\pgfpathlineto{\pgfqpoint{3.012891in}{1.570794in}}%
\pgfpathlineto{\pgfqpoint{3.013358in}{1.532576in}}%
\pgfpathlineto{\pgfqpoint{3.013592in}{1.536898in}}%
\pgfpathlineto{\pgfqpoint{3.014762in}{1.320382in}}%
\pgfpathlineto{\pgfqpoint{3.015463in}{1.402963in}}%
\pgfpathlineto{\pgfqpoint{3.015697in}{1.408616in}}%
\pgfpathlineto{\pgfqpoint{3.015931in}{1.395784in}}%
\pgfpathlineto{\pgfqpoint{3.017334in}{1.211132in}}%
\pgfpathlineto{\pgfqpoint{3.018503in}{1.242589in}}%
\pgfpathlineto{\pgfqpoint{3.020608in}{1.549159in}}%
\pgfpathlineto{\pgfqpoint{3.021076in}{1.512231in}}%
\pgfpathlineto{\pgfqpoint{3.022245in}{1.331875in}}%
\pgfpathlineto{\pgfqpoint{3.022713in}{1.354925in}}%
\pgfpathlineto{\pgfqpoint{3.022947in}{1.382362in}}%
\pgfpathlineto{\pgfqpoint{3.023180in}{1.354147in}}%
\pgfpathlineto{\pgfqpoint{3.024116in}{1.125286in}}%
\pgfpathlineto{\pgfqpoint{3.024584in}{1.175889in}}%
\pgfpathlineto{\pgfqpoint{3.026221in}{1.426287in}}%
\pgfpathlineto{\pgfqpoint{3.026455in}{1.429233in}}%
\pgfpathlineto{\pgfqpoint{3.026688in}{1.358201in}}%
\pgfpathlineto{\pgfqpoint{3.027390in}{1.448252in}}%
\pgfpathlineto{\pgfqpoint{3.028092in}{1.514299in}}%
\pgfpathlineto{\pgfqpoint{3.028793in}{1.476717in}}%
\pgfpathlineto{\pgfqpoint{3.030196in}{1.127599in}}%
\pgfpathlineto{\pgfqpoint{3.030664in}{1.182179in}}%
\pgfpathlineto{\pgfqpoint{3.032067in}{1.357518in}}%
\pgfpathlineto{\pgfqpoint{3.032535in}{1.330873in}}%
\pgfpathlineto{\pgfqpoint{3.033470in}{1.216067in}}%
\pgfpathlineto{\pgfqpoint{3.033938in}{1.272229in}}%
\pgfpathlineto{\pgfqpoint{3.035809in}{1.500949in}}%
\pgfpathlineto{\pgfqpoint{3.036043in}{1.498214in}}%
\pgfpathlineto{\pgfqpoint{3.036978in}{1.185940in}}%
\pgfpathlineto{\pgfqpoint{3.038147in}{1.225687in}}%
\pgfpathlineto{\pgfqpoint{3.038381in}{1.226484in}}%
\pgfpathlineto{\pgfqpoint{3.040486in}{1.380553in}}%
\pgfpathlineto{\pgfqpoint{3.040720in}{1.382046in}}%
\pgfpathlineto{\pgfqpoint{3.041422in}{1.297392in}}%
\pgfpathlineto{\pgfqpoint{3.041655in}{1.345125in}}%
\pgfpathlineto{\pgfqpoint{3.042591in}{1.386282in}}%
\pgfpathlineto{\pgfqpoint{3.042825in}{1.372247in}}%
\pgfpathlineto{\pgfqpoint{3.043760in}{1.239078in}}%
\pgfpathlineto{\pgfqpoint{3.044228in}{1.293423in}}%
\pgfpathlineto{\pgfqpoint{3.044462in}{1.297125in}}%
\pgfpathlineto{\pgfqpoint{3.045163in}{1.231064in}}%
\pgfpathlineto{\pgfqpoint{3.045631in}{1.264680in}}%
\pgfpathlineto{\pgfqpoint{3.046566in}{1.423877in}}%
\pgfpathlineto{\pgfqpoint{3.047034in}{1.397599in}}%
\pgfpathlineto{\pgfqpoint{3.047502in}{1.316471in}}%
\pgfpathlineto{\pgfqpoint{3.047970in}{1.364901in}}%
\pgfpathlineto{\pgfqpoint{3.048437in}{1.455674in}}%
\pgfpathlineto{\pgfqpoint{3.048671in}{1.432117in}}%
\pgfpathlineto{\pgfqpoint{3.050074in}{1.221822in}}%
\pgfpathlineto{\pgfqpoint{3.051244in}{1.285926in}}%
\pgfpathlineto{\pgfqpoint{3.051477in}{1.273774in}}%
\pgfpathlineto{\pgfqpoint{3.051711in}{1.274036in}}%
\pgfpathlineto{\pgfqpoint{3.051945in}{1.307772in}}%
\pgfpathlineto{\pgfqpoint{3.052647in}{1.305183in}}%
\pgfpathlineto{\pgfqpoint{3.052881in}{1.280235in}}%
\pgfpathlineto{\pgfqpoint{3.053348in}{1.348790in}}%
\pgfpathlineto{\pgfqpoint{3.054751in}{1.442487in}}%
\pgfpathlineto{\pgfqpoint{3.055453in}{1.434648in}}%
\pgfpathlineto{\pgfqpoint{3.056622in}{1.250726in}}%
\pgfpathlineto{\pgfqpoint{3.056856in}{1.264906in}}%
\pgfpathlineto{\pgfqpoint{3.057792in}{1.412637in}}%
\pgfpathlineto{\pgfqpoint{3.058493in}{1.328066in}}%
\pgfpathlineto{\pgfqpoint{3.058961in}{1.332683in}}%
\pgfpathlineto{\pgfqpoint{3.059663in}{1.285247in}}%
\pgfpathlineto{\pgfqpoint{3.060364in}{1.382352in}}%
\pgfpathlineto{\pgfqpoint{3.060832in}{1.318620in}}%
\pgfpathlineto{\pgfqpoint{3.061300in}{1.240846in}}%
\pgfpathlineto{\pgfqpoint{3.062001in}{1.299535in}}%
\pgfpathlineto{\pgfqpoint{3.062235in}{1.299266in}}%
\pgfpathlineto{\pgfqpoint{3.064106in}{1.428799in}}%
\pgfpathlineto{\pgfqpoint{3.064574in}{1.410493in}}%
\pgfpathlineto{\pgfqpoint{3.065041in}{1.416259in}}%
\pgfpathlineto{\pgfqpoint{3.065743in}{1.460286in}}%
\pgfpathlineto{\pgfqpoint{3.065977in}{1.448680in}}%
\pgfpathlineto{\pgfqpoint{3.068315in}{1.178382in}}%
\pgfpathlineto{\pgfqpoint{3.068549in}{1.201305in}}%
\pgfpathlineto{\pgfqpoint{3.069485in}{1.347490in}}%
\pgfpathlineto{\pgfqpoint{3.070186in}{1.301039in}}%
\pgfpathlineto{\pgfqpoint{3.070420in}{1.300000in}}%
\pgfpathlineto{\pgfqpoint{3.071589in}{1.437123in}}%
\pgfpathlineto{\pgfqpoint{3.072057in}{1.392219in}}%
\pgfpathlineto{\pgfqpoint{3.073694in}{1.325313in}}%
\pgfpathlineto{\pgfqpoint{3.074630in}{1.416035in}}%
\pgfpathlineto{\pgfqpoint{3.074863in}{1.414270in}}%
\pgfpathlineto{\pgfqpoint{3.076500in}{1.199470in}}%
\pgfpathlineto{\pgfqpoint{3.080008in}{1.427716in}}%
\pgfpathlineto{\pgfqpoint{3.080242in}{1.472384in}}%
\pgfpathlineto{\pgfqpoint{3.080944in}{1.419592in}}%
\pgfpathlineto{\pgfqpoint{3.081879in}{1.484373in}}%
\pgfpathlineto{\pgfqpoint{3.082815in}{1.465922in}}%
\pgfpathlineto{\pgfqpoint{3.084919in}{1.171379in}}%
\pgfpathlineto{\pgfqpoint{3.083282in}{1.470022in}}%
\pgfpathlineto{\pgfqpoint{3.085621in}{1.277105in}}%
\pgfpathlineto{\pgfqpoint{3.086556in}{1.372810in}}%
\pgfpathlineto{\pgfqpoint{3.087024in}{1.365838in}}%
\pgfpathlineto{\pgfqpoint{3.087258in}{1.352675in}}%
\pgfpathlineto{\pgfqpoint{3.088895in}{1.525074in}}%
\pgfpathlineto{\pgfqpoint{3.091000in}{1.226180in}}%
\pgfpathlineto{\pgfqpoint{3.091701in}{1.307389in}}%
\pgfpathlineto{\pgfqpoint{3.092637in}{1.458373in}}%
\pgfpathlineto{\pgfqpoint{3.093104in}{1.433871in}}%
\pgfpathlineto{\pgfqpoint{3.095209in}{1.371567in}}%
\pgfpathlineto{\pgfqpoint{3.095443in}{1.400776in}}%
\pgfpathlineto{\pgfqpoint{3.096145in}{1.368399in}}%
\pgfpathlineto{\pgfqpoint{3.096378in}{1.334843in}}%
\pgfpathlineto{\pgfqpoint{3.097080in}{1.376837in}}%
\pgfpathlineto{\pgfqpoint{3.097314in}{1.355071in}}%
\pgfpathlineto{\pgfqpoint{3.098717in}{1.307060in}}%
\pgfpathlineto{\pgfqpoint{3.100120in}{1.535277in}}%
\pgfpathlineto{\pgfqpoint{3.100588in}{1.637830in}}%
\pgfpathlineto{\pgfqpoint{3.101056in}{1.542972in}}%
\pgfpathlineto{\pgfqpoint{3.102225in}{1.246478in}}%
\pgfpathlineto{\pgfqpoint{3.102927in}{1.275921in}}%
\pgfpathlineto{\pgfqpoint{3.105265in}{1.624329in}}%
\pgfpathlineto{\pgfqpoint{3.105499in}{1.592472in}}%
\pgfpathlineto{\pgfqpoint{3.106434in}{1.456285in}}%
\pgfpathlineto{\pgfqpoint{3.107136in}{1.482588in}}%
\pgfpathlineto{\pgfqpoint{3.107604in}{1.413110in}}%
\pgfpathlineto{\pgfqpoint{3.107838in}{1.359095in}}%
\pgfpathlineto{\pgfqpoint{3.108305in}{1.483534in}}%
\pgfpathlineto{\pgfqpoint{3.109007in}{1.538568in}}%
\pgfpathlineto{\pgfqpoint{3.109241in}{1.480343in}}%
\pgfpathlineto{\pgfqpoint{3.109475in}{1.474420in}}%
\pgfpathlineto{\pgfqpoint{3.110176in}{1.324785in}}%
\pgfpathlineto{\pgfqpoint{3.111112in}{1.357808in}}%
\pgfpathlineto{\pgfqpoint{3.113684in}{1.658662in}}%
\pgfpathlineto{\pgfqpoint{3.116023in}{1.428442in}}%
\pgfpathlineto{\pgfqpoint{3.117192in}{1.313225in}}%
\pgfpathlineto{\pgfqpoint{3.117660in}{1.368216in}}%
\pgfpathlineto{\pgfqpoint{3.119063in}{1.698926in}}%
\pgfpathlineto{\pgfqpoint{3.119531in}{1.684560in}}%
\pgfpathlineto{\pgfqpoint{3.119764in}{1.679860in}}%
\pgfpathlineto{\pgfqpoint{3.121869in}{1.312269in}}%
\pgfpathlineto{\pgfqpoint{3.122337in}{1.377390in}}%
\pgfpathlineto{\pgfqpoint{3.123272in}{1.526513in}}%
\pgfpathlineto{\pgfqpoint{3.123740in}{1.495007in}}%
\pgfpathlineto{\pgfqpoint{3.124442in}{1.572656in}}%
\pgfpathlineto{\pgfqpoint{3.125377in}{1.714336in}}%
\pgfpathlineto{\pgfqpoint{3.125611in}{1.670077in}}%
\pgfpathlineto{\pgfqpoint{3.127949in}{1.494838in}}%
\pgfpathlineto{\pgfqpoint{3.128651in}{1.481480in}}%
\pgfpathlineto{\pgfqpoint{3.129820in}{1.584107in}}%
\pgfpathlineto{\pgfqpoint{3.131925in}{1.474230in}}%
\pgfpathlineto{\pgfqpoint{3.132159in}{1.486348in}}%
\pgfpathlineto{\pgfqpoint{3.133094in}{1.642326in}}%
\pgfpathlineto{\pgfqpoint{3.134030in}{1.593735in}}%
\pgfpathlineto{\pgfqpoint{3.134965in}{1.483203in}}%
\pgfpathlineto{\pgfqpoint{3.135433in}{1.517073in}}%
\pgfpathlineto{\pgfqpoint{3.137070in}{1.648411in}}%
\pgfpathlineto{\pgfqpoint{3.138005in}{1.732034in}}%
\pgfpathlineto{\pgfqpoint{3.138239in}{1.729334in}}%
\pgfpathlineto{\pgfqpoint{3.139876in}{1.535931in}}%
\pgfpathlineto{\pgfqpoint{3.140110in}{1.565905in}}%
\pgfpathlineto{\pgfqpoint{3.140578in}{1.503227in}}%
\pgfpathlineto{\pgfqpoint{3.141046in}{1.479316in}}%
\pgfpathlineto{\pgfqpoint{3.141279in}{1.509248in}}%
\pgfpathlineto{\pgfqpoint{3.143852in}{1.653062in}}%
\pgfpathlineto{\pgfqpoint{3.141747in}{1.502324in}}%
\pgfpathlineto{\pgfqpoint{3.144320in}{1.641383in}}%
\pgfpathlineto{\pgfqpoint{3.145021in}{1.581562in}}%
\pgfpathlineto{\pgfqpoint{3.145489in}{1.611264in}}%
\pgfpathlineto{\pgfqpoint{3.146190in}{1.599271in}}%
\pgfpathlineto{\pgfqpoint{3.145957in}{1.622389in}}%
\pgfpathlineto{\pgfqpoint{3.146424in}{1.609106in}}%
\pgfpathlineto{\pgfqpoint{3.146658in}{1.609871in}}%
\pgfpathlineto{\pgfqpoint{3.147126in}{1.615991in}}%
\pgfpathlineto{\pgfqpoint{3.147360in}{1.589891in}}%
\pgfpathlineto{\pgfqpoint{3.148763in}{1.796564in}}%
\pgfpathlineto{\pgfqpoint{3.150400in}{1.484862in}}%
\pgfpathlineto{\pgfqpoint{3.151102in}{1.708046in}}%
\pgfpathlineto{\pgfqpoint{3.151569in}{1.650847in}}%
\pgfpathlineto{\pgfqpoint{3.152505in}{1.443533in}}%
\pgfpathlineto{\pgfqpoint{3.152972in}{1.564610in}}%
\pgfpathlineto{\pgfqpoint{3.154142in}{1.755225in}}%
\pgfpathlineto{\pgfqpoint{3.154609in}{1.811412in}}%
\pgfpathlineto{\pgfqpoint{3.155311in}{1.783235in}}%
\pgfpathlineto{\pgfqpoint{3.156948in}{1.557733in}}%
\pgfpathlineto{\pgfqpoint{3.157416in}{1.569128in}}%
\pgfpathlineto{\pgfqpoint{3.157650in}{1.560441in}}%
\pgfpathlineto{\pgfqpoint{3.157883in}{1.548650in}}%
\pgfpathlineto{\pgfqpoint{3.159520in}{1.720477in}}%
\pgfpathlineto{\pgfqpoint{3.159754in}{1.719610in}}%
\pgfpathlineto{\pgfqpoint{3.161157in}{1.513223in}}%
\pgfpathlineto{\pgfqpoint{3.161391in}{1.545850in}}%
\pgfpathlineto{\pgfqpoint{3.163028in}{1.837899in}}%
\pgfpathlineto{\pgfqpoint{3.163496in}{1.804644in}}%
\pgfpathlineto{\pgfqpoint{3.163730in}{1.826193in}}%
\pgfpathlineto{\pgfqpoint{3.163964in}{1.755598in}}%
\pgfpathlineto{\pgfqpoint{3.165601in}{1.665886in}}%
\pgfpathlineto{\pgfqpoint{3.166302in}{1.587127in}}%
\pgfpathlineto{\pgfqpoint{3.167004in}{1.587594in}}%
\pgfpathlineto{\pgfqpoint{3.167238in}{1.592192in}}%
\pgfpathlineto{\pgfqpoint{3.167472in}{1.563707in}}%
\pgfpathlineto{\pgfqpoint{3.167706in}{1.601996in}}%
\pgfpathlineto{\pgfqpoint{3.167939in}{1.645250in}}%
\pgfpathlineto{\pgfqpoint{3.168641in}{1.590061in}}%
\pgfpathlineto{\pgfqpoint{3.169576in}{1.723647in}}%
\pgfpathlineto{\pgfqpoint{3.170278in}{1.865926in}}%
\pgfpathlineto{\pgfqpoint{3.170980in}{1.797844in}}%
\pgfpathlineto{\pgfqpoint{3.171447in}{1.747700in}}%
\pgfpathlineto{\pgfqpoint{3.172149in}{1.773304in}}%
\pgfpathlineto{\pgfqpoint{3.172383in}{1.781929in}}%
\pgfpathlineto{\pgfqpoint{3.172850in}{1.768944in}}%
\pgfpathlineto{\pgfqpoint{3.173084in}{1.770776in}}%
\pgfpathlineto{\pgfqpoint{3.174721in}{1.535094in}}%
\pgfpathlineto{\pgfqpoint{3.174955in}{1.557466in}}%
\pgfpathlineto{\pgfqpoint{3.176358in}{1.627187in}}%
\pgfpathlineto{\pgfqpoint{3.178229in}{1.883795in}}%
\pgfpathlineto{\pgfqpoint{3.178697in}{1.807491in}}%
\pgfpathlineto{\pgfqpoint{3.179866in}{1.620814in}}%
\pgfpathlineto{\pgfqpoint{3.180568in}{1.655591in}}%
\pgfpathlineto{\pgfqpoint{3.180802in}{1.675258in}}%
\pgfpathlineto{\pgfqpoint{3.181036in}{1.635856in}}%
\pgfpathlineto{\pgfqpoint{3.181503in}{1.646824in}}%
\pgfpathlineto{\pgfqpoint{3.181971in}{1.622003in}}%
\pgfpathlineto{\pgfqpoint{3.182205in}{1.633548in}}%
\pgfpathlineto{\pgfqpoint{3.183608in}{1.784632in}}%
\pgfpathlineto{\pgfqpoint{3.183842in}{1.786462in}}%
\pgfpathlineto{\pgfqpoint{3.184543in}{1.645145in}}%
\pgfpathlineto{\pgfqpoint{3.185011in}{1.747335in}}%
\pgfpathlineto{\pgfqpoint{3.185245in}{1.779597in}}%
\pgfpathlineto{\pgfqpoint{3.185947in}{1.777008in}}%
\pgfpathlineto{\pgfqpoint{3.186180in}{1.689221in}}%
\pgfpathlineto{\pgfqpoint{3.187116in}{1.722544in}}%
\pgfpathlineto{\pgfqpoint{3.187350in}{1.714825in}}%
\pgfpathlineto{\pgfqpoint{3.187584in}{1.723230in}}%
\pgfpathlineto{\pgfqpoint{3.187817in}{1.739184in}}%
\pgfpathlineto{\pgfqpoint{3.189221in}{1.608710in}}%
\pgfpathlineto{\pgfqpoint{3.191091in}{1.844622in}}%
\pgfpathlineto{\pgfqpoint{3.191559in}{1.913389in}}%
\pgfpathlineto{\pgfqpoint{3.192027in}{1.839628in}}%
\pgfpathlineto{\pgfqpoint{3.193898in}{1.579060in}}%
\pgfpathlineto{\pgfqpoint{3.194599in}{1.564603in}}%
\pgfpathlineto{\pgfqpoint{3.195769in}{1.701539in}}%
\pgfpathlineto{\pgfqpoint{3.196704in}{1.910940in}}%
\pgfpathlineto{\pgfqpoint{3.197406in}{1.868254in}}%
\pgfpathlineto{\pgfqpoint{3.199043in}{1.734421in}}%
\pgfpathlineto{\pgfqpoint{3.199277in}{1.746309in}}%
\pgfpathlineto{\pgfqpoint{3.199510in}{1.748141in}}%
\pgfpathlineto{\pgfqpoint{3.199744in}{1.725329in}}%
\pgfpathlineto{\pgfqpoint{3.199978in}{1.754253in}}%
\pgfpathlineto{\pgfqpoint{3.200446in}{1.743291in}}%
\pgfpathlineto{\pgfqpoint{3.200680in}{1.744218in}}%
\pgfpathlineto{\pgfqpoint{3.201849in}{1.589640in}}%
\pgfpathlineto{\pgfqpoint{3.202083in}{1.610289in}}%
\pgfpathlineto{\pgfqpoint{3.205123in}{1.905982in}}%
\pgfpathlineto{\pgfqpoint{3.205357in}{1.834352in}}%
\pgfpathlineto{\pgfqpoint{3.206058in}{1.727491in}}%
\pgfpathlineto{\pgfqpoint{3.206526in}{1.783885in}}%
\pgfpathlineto{\pgfqpoint{3.206994in}{1.800600in}}%
\pgfpathlineto{\pgfqpoint{3.207929in}{1.705055in}}%
\pgfpathlineto{\pgfqpoint{3.208397in}{1.732709in}}%
\pgfpathlineto{\pgfqpoint{3.209566in}{1.622356in}}%
\pgfpathlineto{\pgfqpoint{3.210970in}{1.869082in}}%
\pgfpathlineto{\pgfqpoint{3.211203in}{1.854865in}}%
\pgfpathlineto{\pgfqpoint{3.211905in}{1.815300in}}%
\pgfpathlineto{\pgfqpoint{3.212139in}{1.861033in}}%
\pgfpathlineto{\pgfqpoint{3.212373in}{1.874193in}}%
\pgfpathlineto{\pgfqpoint{3.212607in}{1.829158in}}%
\pgfpathlineto{\pgfqpoint{3.212840in}{1.833150in}}%
\pgfpathlineto{\pgfqpoint{3.213542in}{1.744532in}}%
\pgfpathlineto{\pgfqpoint{3.214244in}{1.574874in}}%
\pgfpathlineto{\pgfqpoint{3.214711in}{1.660643in}}%
\pgfpathlineto{\pgfqpoint{3.215179in}{1.735343in}}%
\pgfpathlineto{\pgfqpoint{3.215881in}{1.684263in}}%
\pgfpathlineto{\pgfqpoint{3.216114in}{1.673560in}}%
\pgfpathlineto{\pgfqpoint{3.216348in}{1.704824in}}%
\pgfpathlineto{\pgfqpoint{3.217751in}{1.906922in}}%
\pgfpathlineto{\pgfqpoint{3.217985in}{1.880847in}}%
\pgfpathlineto{\pgfqpoint{3.220792in}{1.651722in}}%
\pgfpathlineto{\pgfqpoint{3.221259in}{1.705826in}}%
\pgfpathlineto{\pgfqpoint{3.221727in}{1.656910in}}%
\pgfpathlineto{\pgfqpoint{3.222195in}{1.581900in}}%
\pgfpathlineto{\pgfqpoint{3.223130in}{1.592065in}}%
\pgfpathlineto{\pgfqpoint{3.225936in}{1.850532in}}%
\pgfpathlineto{\pgfqpoint{3.226638in}{1.815328in}}%
\pgfpathlineto{\pgfqpoint{3.227106in}{1.817437in}}%
\pgfpathlineto{\pgfqpoint{3.228743in}{1.608131in}}%
\pgfpathlineto{\pgfqpoint{3.229444in}{1.502046in}}%
\pgfpathlineto{\pgfqpoint{3.229912in}{1.557057in}}%
\pgfpathlineto{\pgfqpoint{3.230146in}{1.554613in}}%
\pgfpathlineto{\pgfqpoint{3.230380in}{1.525582in}}%
\pgfpathlineto{\pgfqpoint{3.230614in}{1.564664in}}%
\pgfpathlineto{\pgfqpoint{3.232718in}{1.771091in}}%
\pgfpathlineto{\pgfqpoint{3.232952in}{1.779524in}}%
\pgfpathlineto{\pgfqpoint{3.233654in}{1.791203in}}%
\pgfpathlineto{\pgfqpoint{3.234355in}{1.727352in}}%
\pgfpathlineto{\pgfqpoint{3.234589in}{1.728141in}}%
\pgfpathlineto{\pgfqpoint{3.235057in}{1.747572in}}%
\pgfpathlineto{\pgfqpoint{3.235759in}{1.616403in}}%
\pgfpathlineto{\pgfqpoint{3.236460in}{1.645561in}}%
\pgfpathlineto{\pgfqpoint{3.236694in}{1.650271in}}%
\pgfpathlineto{\pgfqpoint{3.237162in}{1.686503in}}%
\pgfpathlineto{\pgfqpoint{3.237396in}{1.617610in}}%
\pgfpathlineto{\pgfqpoint{3.238565in}{1.482065in}}%
\pgfpathlineto{\pgfqpoint{3.238799in}{1.535748in}}%
\pgfpathlineto{\pgfqpoint{3.239033in}{1.527572in}}%
\pgfpathlineto{\pgfqpoint{3.239266in}{1.550320in}}%
\pgfpathlineto{\pgfqpoint{3.239734in}{1.657865in}}%
\pgfpathlineto{\pgfqpoint{3.240436in}{1.567054in}}%
\pgfpathlineto{\pgfqpoint{3.240670in}{1.563074in}}%
\pgfpathlineto{\pgfqpoint{3.240903in}{1.573754in}}%
\pgfpathlineto{\pgfqpoint{3.242541in}{1.838422in}}%
\pgfpathlineto{\pgfqpoint{3.242774in}{1.773937in}}%
\pgfpathlineto{\pgfqpoint{3.244879in}{1.459637in}}%
\pgfpathlineto{\pgfqpoint{3.246750in}{1.528233in}}%
\pgfpathlineto{\pgfqpoint{3.246984in}{1.533460in}}%
\pgfpathlineto{\pgfqpoint{3.249089in}{1.716257in}}%
\pgfpathlineto{\pgfqpoint{3.249322in}{1.699337in}}%
\pgfpathlineto{\pgfqpoint{3.249556in}{1.750538in}}%
\pgfpathlineto{\pgfqpoint{3.249790in}{1.758948in}}%
\pgfpathlineto{\pgfqpoint{3.251427in}{1.470559in}}%
\pgfpathlineto{\pgfqpoint{3.251895in}{1.482505in}}%
\pgfpathlineto{\pgfqpoint{3.253064in}{1.422298in}}%
\pgfpathlineto{\pgfqpoint{3.253298in}{1.453411in}}%
\pgfpathlineto{\pgfqpoint{3.255403in}{1.700171in}}%
\pgfpathlineto{\pgfqpoint{3.255637in}{1.695456in}}%
\pgfpathlineto{\pgfqpoint{3.257274in}{1.535547in}}%
\pgfpathlineto{\pgfqpoint{3.257741in}{1.543269in}}%
\pgfpathlineto{\pgfqpoint{3.257975in}{1.538212in}}%
\pgfpathlineto{\pgfqpoint{3.258677in}{1.588608in}}%
\pgfpathlineto{\pgfqpoint{3.258911in}{1.552511in}}%
\pgfpathlineto{\pgfqpoint{3.260080in}{1.450915in}}%
\pgfpathlineto{\pgfqpoint{3.260314in}{1.497350in}}%
\pgfpathlineto{\pgfqpoint{3.260782in}{1.544552in}}%
\pgfpathlineto{\pgfqpoint{3.261249in}{1.481852in}}%
\pgfpathlineto{\pgfqpoint{3.263120in}{1.370164in}}%
\pgfpathlineto{\pgfqpoint{3.263354in}{1.391893in}}%
\pgfpathlineto{\pgfqpoint{3.264757in}{1.532620in}}%
\pgfpathlineto{\pgfqpoint{3.265459in}{1.467507in}}%
\pgfpathlineto{\pgfqpoint{3.265693in}{1.454827in}}%
\pgfpathlineto{\pgfqpoint{3.266160in}{1.487898in}}%
\pgfpathlineto{\pgfqpoint{3.267563in}{1.636100in}}%
\pgfpathlineto{\pgfqpoint{3.268031in}{1.623409in}}%
\pgfpathlineto{\pgfqpoint{3.268499in}{1.564520in}}%
\pgfpathlineto{\pgfqpoint{3.269902in}{1.258566in}}%
\pgfpathlineto{\pgfqpoint{3.270370in}{1.330969in}}%
\pgfpathlineto{\pgfqpoint{3.270604in}{1.316851in}}%
\pgfpathlineto{\pgfqpoint{3.271071in}{1.354973in}}%
\pgfpathlineto{\pgfqpoint{3.272942in}{1.665498in}}%
\pgfpathlineto{\pgfqpoint{3.273176in}{1.600636in}}%
\pgfpathlineto{\pgfqpoint{3.275047in}{1.243178in}}%
\pgfpathlineto{\pgfqpoint{3.275515in}{1.284354in}}%
\pgfpathlineto{\pgfqpoint{3.276450in}{1.402473in}}%
\pgfpathlineto{\pgfqpoint{3.276918in}{1.324728in}}%
\pgfpathlineto{\pgfqpoint{3.277152in}{1.327474in}}%
\pgfpathlineto{\pgfqpoint{3.277386in}{1.292419in}}%
\pgfpathlineto{\pgfqpoint{3.277853in}{1.364331in}}%
\pgfpathlineto{\pgfqpoint{3.279256in}{1.540936in}}%
\pgfpathlineto{\pgfqpoint{3.279490in}{1.527341in}}%
\pgfpathlineto{\pgfqpoint{3.279958in}{1.389208in}}%
\pgfpathlineto{\pgfqpoint{3.280660in}{1.491392in}}%
\pgfpathlineto{\pgfqpoint{3.282764in}{1.266092in}}%
\pgfpathlineto{\pgfqpoint{3.283232in}{1.277176in}}%
\pgfpathlineto{\pgfqpoint{3.284167in}{1.347180in}}%
\pgfpathlineto{\pgfqpoint{3.284401in}{1.377320in}}%
\pgfpathlineto{\pgfqpoint{3.284869in}{1.330042in}}%
\pgfpathlineto{\pgfqpoint{3.285337in}{1.332750in}}%
\pgfpathlineto{\pgfqpoint{3.285804in}{1.293116in}}%
\pgfpathlineto{\pgfqpoint{3.286038in}{1.256054in}}%
\pgfpathlineto{\pgfqpoint{3.286506in}{1.301347in}}%
\pgfpathlineto{\pgfqpoint{3.286740in}{1.292977in}}%
\pgfpathlineto{\pgfqpoint{3.288143in}{1.508416in}}%
\pgfpathlineto{\pgfqpoint{3.288611in}{1.455636in}}%
\pgfpathlineto{\pgfqpoint{3.288845in}{1.451983in}}%
\pgfpathlineto{\pgfqpoint{3.291417in}{1.200103in}}%
\pgfpathlineto{\pgfqpoint{3.292119in}{1.176344in}}%
\pgfpathlineto{\pgfqpoint{3.294691in}{1.405432in}}%
\pgfpathlineto{\pgfqpoint{3.295627in}{1.360286in}}%
\pgfpathlineto{\pgfqpoint{3.295860in}{1.422955in}}%
\pgfpathlineto{\pgfqpoint{3.296562in}{1.342306in}}%
\pgfpathlineto{\pgfqpoint{3.298199in}{1.237275in}}%
\pgfpathlineto{\pgfqpoint{3.298667in}{1.182459in}}%
\pgfpathlineto{\pgfqpoint{3.299134in}{1.250356in}}%
\pgfpathlineto{\pgfqpoint{3.299368in}{1.248396in}}%
\pgfpathlineto{\pgfqpoint{3.300304in}{1.159428in}}%
\pgfpathlineto{\pgfqpoint{3.300771in}{1.196632in}}%
\pgfpathlineto{\pgfqpoint{3.301239in}{1.296272in}}%
\pgfpathlineto{\pgfqpoint{3.303110in}{1.489986in}}%
\pgfpathlineto{\pgfqpoint{3.303344in}{1.460762in}}%
\pgfpathlineto{\pgfqpoint{3.305215in}{1.227088in}}%
\pgfpathlineto{\pgfqpoint{3.305916in}{1.181746in}}%
\pgfpathlineto{\pgfqpoint{3.306150in}{1.228541in}}%
\pgfpathlineto{\pgfqpoint{3.306384in}{1.234542in}}%
\pgfpathlineto{\pgfqpoint{3.306852in}{1.166332in}}%
\pgfpathlineto{\pgfqpoint{3.307320in}{1.260850in}}%
\pgfpathlineto{\pgfqpoint{3.308957in}{1.358549in}}%
\pgfpathlineto{\pgfqpoint{3.309658in}{1.392613in}}%
\pgfpathlineto{\pgfqpoint{3.309424in}{1.355529in}}%
\pgfpathlineto{\pgfqpoint{3.310360in}{1.391855in}}%
\pgfpathlineto{\pgfqpoint{3.311295in}{1.167156in}}%
\pgfpathlineto{\pgfqpoint{3.311997in}{1.219881in}}%
\pgfpathlineto{\pgfqpoint{3.313166in}{1.400607in}}%
\pgfpathlineto{\pgfqpoint{3.313634in}{1.348225in}}%
\pgfpathlineto{\pgfqpoint{3.313868in}{1.294446in}}%
\pgfpathlineto{\pgfqpoint{3.314335in}{1.419510in}}%
\pgfpathlineto{\pgfqpoint{3.314803in}{1.344783in}}%
\pgfpathlineto{\pgfqpoint{3.315037in}{1.349757in}}%
\pgfpathlineto{\pgfqpoint{3.316674in}{1.216891in}}%
\pgfpathlineto{\pgfqpoint{3.317843in}{1.371077in}}%
\pgfpathlineto{\pgfqpoint{3.318077in}{1.355621in}}%
\pgfpathlineto{\pgfqpoint{3.318779in}{1.149949in}}%
\pgfpathlineto{\pgfqpoint{3.319480in}{1.209322in}}%
\pgfpathlineto{\pgfqpoint{3.319714in}{1.189685in}}%
\pgfpathlineto{\pgfqpoint{3.320182in}{1.240795in}}%
\pgfpathlineto{\pgfqpoint{3.320416in}{1.245850in}}%
\pgfpathlineto{\pgfqpoint{3.321585in}{1.540221in}}%
\pgfpathlineto{\pgfqpoint{3.322053in}{1.444207in}}%
\pgfpathlineto{\pgfqpoint{3.322754in}{1.312979in}}%
\pgfpathlineto{\pgfqpoint{3.323690in}{1.146710in}}%
\pgfpathlineto{\pgfqpoint{3.324157in}{1.175475in}}%
\pgfpathlineto{\pgfqpoint{3.324391in}{1.197072in}}%
\pgfpathlineto{\pgfqpoint{3.324859in}{1.153697in}}%
\pgfpathlineto{\pgfqpoint{3.325093in}{1.116963in}}%
\pgfpathlineto{\pgfqpoint{3.325561in}{1.206194in}}%
\pgfpathlineto{\pgfqpoint{3.326730in}{1.503095in}}%
\pgfpathlineto{\pgfqpoint{3.327431in}{1.401084in}}%
\pgfpathlineto{\pgfqpoint{3.327899in}{1.358950in}}%
\pgfpathlineto{\pgfqpoint{3.329770in}{1.105064in}}%
\pgfpathlineto{\pgfqpoint{3.330004in}{1.106490in}}%
\pgfpathlineto{\pgfqpoint{3.330939in}{1.379973in}}%
\pgfpathlineto{\pgfqpoint{3.331641in}{1.345637in}}%
\pgfpathlineto{\pgfqpoint{3.333278in}{1.152750in}}%
\pgfpathlineto{\pgfqpoint{3.334213in}{1.399709in}}%
\pgfpathlineto{\pgfqpoint{3.334915in}{1.363492in}}%
\pgfpathlineto{\pgfqpoint{3.336786in}{1.208998in}}%
\pgfpathlineto{\pgfqpoint{3.337020in}{1.192480in}}%
\pgfpathlineto{\pgfqpoint{3.337254in}{1.229537in}}%
\pgfpathlineto{\pgfqpoint{3.337487in}{1.225888in}}%
\pgfpathlineto{\pgfqpoint{3.337721in}{1.239288in}}%
\pgfpathlineto{\pgfqpoint{3.337955in}{1.211289in}}%
\pgfpathlineto{\pgfqpoint{3.338657in}{1.169913in}}%
\pgfpathlineto{\pgfqpoint{3.339124in}{1.183971in}}%
\pgfpathlineto{\pgfqpoint{3.340761in}{1.327151in}}%
\pgfpathlineto{\pgfqpoint{3.341463in}{1.202294in}}%
\pgfpathlineto{\pgfqpoint{3.342165in}{1.208798in}}%
\pgfpathlineto{\pgfqpoint{3.343100in}{1.342222in}}%
\pgfpathlineto{\pgfqpoint{3.343802in}{1.334045in}}%
\pgfpathlineto{\pgfqpoint{3.344035in}{1.353635in}}%
\pgfpathlineto{\pgfqpoint{3.344503in}{1.316374in}}%
\pgfpathlineto{\pgfqpoint{3.345205in}{1.209223in}}%
\pgfpathlineto{\pgfqpoint{3.345906in}{1.233835in}}%
\pgfpathlineto{\pgfqpoint{3.347309in}{1.380887in}}%
\pgfpathlineto{\pgfqpoint{3.350584in}{1.139001in}}%
\pgfpathlineto{\pgfqpoint{3.351051in}{1.193164in}}%
\pgfpathlineto{\pgfqpoint{3.351285in}{1.191350in}}%
\pgfpathlineto{\pgfqpoint{3.353624in}{1.412621in}}%
\pgfpathlineto{\pgfqpoint{3.353858in}{1.394403in}}%
\pgfpathlineto{\pgfqpoint{3.354793in}{1.399306in}}%
\pgfpathlineto{\pgfqpoint{3.355495in}{1.295909in}}%
\pgfpathlineto{\pgfqpoint{3.355728in}{1.262282in}}%
\pgfpathlineto{\pgfqpoint{3.355962in}{1.305350in}}%
\pgfpathlineto{\pgfqpoint{3.356664in}{1.268745in}}%
\pgfpathlineto{\pgfqpoint{3.357833in}{1.312336in}}%
\pgfpathlineto{\pgfqpoint{3.358067in}{1.262093in}}%
\pgfpathlineto{\pgfqpoint{3.358769in}{1.289820in}}%
\pgfpathlineto{\pgfqpoint{3.359704in}{1.349853in}}%
\pgfpathlineto{\pgfqpoint{3.359938in}{1.297592in}}%
\pgfpathlineto{\pgfqpoint{3.361107in}{1.233591in}}%
\pgfpathlineto{\pgfqpoint{3.361341in}{1.262868in}}%
\pgfpathlineto{\pgfqpoint{3.361575in}{1.262247in}}%
\pgfpathlineto{\pgfqpoint{3.363212in}{1.366008in}}%
\pgfpathlineto{\pgfqpoint{3.363446in}{1.368355in}}%
\pgfpathlineto{\pgfqpoint{3.363913in}{1.436367in}}%
\pgfpathlineto{\pgfqpoint{3.364615in}{1.394446in}}%
\pgfpathlineto{\pgfqpoint{3.365083in}{1.347847in}}%
\pgfpathlineto{\pgfqpoint{3.365550in}{1.351609in}}%
\pgfpathlineto{\pgfqpoint{3.367188in}{1.492587in}}%
\pgfpathlineto{\pgfqpoint{3.368825in}{1.268344in}}%
\pgfpathlineto{\pgfqpoint{3.369292in}{1.171004in}}%
\pgfpathlineto{\pgfqpoint{3.369994in}{1.225753in}}%
\pgfpathlineto{\pgfqpoint{3.373034in}{1.634826in}}%
\pgfpathlineto{\pgfqpoint{3.374671in}{1.333667in}}%
\pgfpathlineto{\pgfqpoint{3.374905in}{1.374531in}}%
\pgfpathlineto{\pgfqpoint{3.375373in}{1.407006in}}%
\pgfpathlineto{\pgfqpoint{3.375840in}{1.361917in}}%
\pgfpathlineto{\pgfqpoint{3.376074in}{1.394920in}}%
\pgfpathlineto{\pgfqpoint{3.376308in}{1.400634in}}%
\pgfpathlineto{\pgfqpoint{3.377477in}{1.283979in}}%
\pgfpathlineto{\pgfqpoint{3.377711in}{1.316862in}}%
\pgfpathlineto{\pgfqpoint{3.378413in}{1.414318in}}%
\pgfpathlineto{\pgfqpoint{3.380050in}{1.597597in}}%
\pgfpathlineto{\pgfqpoint{3.380284in}{1.598983in}}%
\pgfpathlineto{\pgfqpoint{3.382622in}{1.379714in}}%
\pgfpathlineto{\pgfqpoint{3.383558in}{1.395263in}}%
\pgfpathlineto{\pgfqpoint{3.384493in}{1.545027in}}%
\pgfpathlineto{\pgfqpoint{3.384961in}{1.486700in}}%
\pgfpathlineto{\pgfqpoint{3.385195in}{1.439847in}}%
\pgfpathlineto{\pgfqpoint{3.385662in}{1.555532in}}%
\pgfpathlineto{\pgfqpoint{3.385896in}{1.566891in}}%
\pgfpathlineto{\pgfqpoint{3.386130in}{1.558961in}}%
\pgfpathlineto{\pgfqpoint{3.388703in}{1.302161in}}%
\pgfpathlineto{\pgfqpoint{3.388936in}{1.335987in}}%
\pgfpathlineto{\pgfqpoint{3.391275in}{1.727800in}}%
\pgfpathlineto{\pgfqpoint{3.391743in}{1.658388in}}%
\pgfpathlineto{\pgfqpoint{3.393380in}{1.423306in}}%
\pgfpathlineto{\pgfqpoint{3.393614in}{1.453401in}}%
\pgfpathlineto{\pgfqpoint{3.394081in}{1.539525in}}%
\pgfpathlineto{\pgfqpoint{3.394783in}{1.472668in}}%
\pgfpathlineto{\pgfqpoint{3.395718in}{1.382721in}}%
\pgfpathlineto{\pgfqpoint{3.396420in}{1.417420in}}%
\pgfpathlineto{\pgfqpoint{3.396888in}{1.454724in}}%
\pgfpathlineto{\pgfqpoint{3.397355in}{1.580965in}}%
\pgfpathlineto{\pgfqpoint{3.398057in}{1.739537in}}%
\pgfpathlineto{\pgfqpoint{3.398525in}{1.685626in}}%
\pgfpathlineto{\pgfqpoint{3.398759in}{1.686004in}}%
\pgfpathlineto{\pgfqpoint{3.400396in}{1.478868in}}%
\pgfpathlineto{\pgfqpoint{3.400629in}{1.483635in}}%
\pgfpathlineto{\pgfqpoint{3.401565in}{1.567046in}}%
\pgfpathlineto{\pgfqpoint{3.401799in}{1.526643in}}%
\pgfpathlineto{\pgfqpoint{3.402968in}{1.439142in}}%
\pgfpathlineto{\pgfqpoint{3.403202in}{1.451300in}}%
\pgfpathlineto{\pgfqpoint{3.404839in}{1.632723in}}%
\pgfpathlineto{\pgfqpoint{3.405540in}{1.622987in}}%
\pgfpathlineto{\pgfqpoint{3.406710in}{1.721851in}}%
\pgfpathlineto{\pgfqpoint{3.406944in}{1.708240in}}%
\pgfpathlineto{\pgfqpoint{3.407645in}{1.572611in}}%
\pgfpathlineto{\pgfqpoint{3.408347in}{1.589071in}}%
\pgfpathlineto{\pgfqpoint{3.408814in}{1.643595in}}%
\pgfpathlineto{\pgfqpoint{3.409282in}{1.596230in}}%
\pgfpathlineto{\pgfqpoint{3.410919in}{1.430556in}}%
\pgfpathlineto{\pgfqpoint{3.411153in}{1.442975in}}%
\pgfpathlineto{\pgfqpoint{3.412556in}{1.569725in}}%
\pgfpathlineto{\pgfqpoint{3.412790in}{1.563394in}}%
\pgfpathlineto{\pgfqpoint{3.413726in}{1.823235in}}%
\pgfpathlineto{\pgfqpoint{3.414193in}{1.718015in}}%
\pgfpathlineto{\pgfqpoint{3.415596in}{1.539326in}}%
\pgfpathlineto{\pgfqpoint{3.416532in}{1.461866in}}%
\pgfpathlineto{\pgfqpoint{3.417000in}{1.475229in}}%
\pgfpathlineto{\pgfqpoint{3.418169in}{1.583908in}}%
\pgfpathlineto{\pgfqpoint{3.418637in}{1.558551in}}%
\pgfpathlineto{\pgfqpoint{3.420741in}{1.750934in}}%
\pgfpathlineto{\pgfqpoint{3.420975in}{1.688869in}}%
\pgfpathlineto{\pgfqpoint{3.421677in}{1.738332in}}%
\pgfpathlineto{\pgfqpoint{3.421911in}{1.741306in}}%
\pgfpathlineto{\pgfqpoint{3.422144in}{1.733689in}}%
\pgfpathlineto{\pgfqpoint{3.422612in}{1.677054in}}%
\pgfpathlineto{\pgfqpoint{3.422846in}{1.564159in}}%
\pgfpathlineto{\pgfqpoint{3.423781in}{1.624654in}}%
\pgfpathlineto{\pgfqpoint{3.424015in}{1.625673in}}%
\pgfpathlineto{\pgfqpoint{3.424249in}{1.621524in}}%
\pgfpathlineto{\pgfqpoint{3.425418in}{1.475452in}}%
\pgfpathlineto{\pgfqpoint{3.425652in}{1.530363in}}%
\pgfpathlineto{\pgfqpoint{3.426822in}{1.611562in}}%
\pgfpathlineto{\pgfqpoint{3.427055in}{1.594880in}}%
\pgfpathlineto{\pgfqpoint{3.427523in}{1.539810in}}%
\pgfpathlineto{\pgfqpoint{3.427991in}{1.608165in}}%
\pgfpathlineto{\pgfqpoint{3.428926in}{1.759754in}}%
\pgfpathlineto{\pgfqpoint{3.429160in}{1.741596in}}%
\pgfpathlineto{\pgfqpoint{3.429628in}{1.689278in}}%
\pgfpathlineto{\pgfqpoint{3.430096in}{1.742672in}}%
\pgfpathlineto{\pgfqpoint{3.430330in}{1.778146in}}%
\pgfpathlineto{\pgfqpoint{3.431031in}{1.732098in}}%
\pgfpathlineto{\pgfqpoint{3.431265in}{1.735888in}}%
\pgfpathlineto{\pgfqpoint{3.431967in}{1.499248in}}%
\pgfpathlineto{\pgfqpoint{3.432668in}{1.568354in}}%
\pgfpathlineto{\pgfqpoint{3.433136in}{1.616464in}}%
\pgfpathlineto{\pgfqpoint{3.433604in}{1.550405in}}%
\pgfpathlineto{\pgfqpoint{3.434071in}{1.553594in}}%
\pgfpathlineto{\pgfqpoint{3.434305in}{1.558646in}}%
\pgfpathlineto{\pgfqpoint{3.434773in}{1.515057in}}%
\pgfpathlineto{\pgfqpoint{3.435007in}{1.532960in}}%
\pgfpathlineto{\pgfqpoint{3.435942in}{1.753413in}}%
\pgfpathlineto{\pgfqpoint{3.436644in}{1.707699in}}%
\pgfpathlineto{\pgfqpoint{3.437111in}{1.677481in}}%
\pgfpathlineto{\pgfqpoint{3.437345in}{1.687456in}}%
\pgfpathlineto{\pgfqpoint{3.438047in}{1.788280in}}%
\pgfpathlineto{\pgfqpoint{3.438515in}{1.750642in}}%
\pgfpathlineto{\pgfqpoint{3.438982in}{1.697568in}}%
\pgfpathlineto{\pgfqpoint{3.439450in}{1.770545in}}%
\pgfpathlineto{\pgfqpoint{3.442256in}{1.469165in}}%
\pgfpathlineto{\pgfqpoint{3.442724in}{1.497472in}}%
\pgfpathlineto{\pgfqpoint{3.445764in}{1.824520in}}%
\pgfpathlineto{\pgfqpoint{3.445998in}{1.816588in}}%
\pgfpathlineto{\pgfqpoint{3.449272in}{1.474386in}}%
\pgfpathlineto{\pgfqpoint{3.449506in}{1.515350in}}%
\pgfpathlineto{\pgfqpoint{3.451143in}{1.819142in}}%
\pgfpathlineto{\pgfqpoint{3.451611in}{1.776498in}}%
\pgfpathlineto{\pgfqpoint{3.451845in}{1.758174in}}%
\pgfpathlineto{\pgfqpoint{3.452312in}{1.794298in}}%
\pgfpathlineto{\pgfqpoint{3.452546in}{1.798050in}}%
\pgfpathlineto{\pgfqpoint{3.455119in}{1.493086in}}%
\pgfpathlineto{\pgfqpoint{3.455352in}{1.547568in}}%
\pgfpathlineto{\pgfqpoint{3.456756in}{1.777799in}}%
\pgfpathlineto{\pgfqpoint{3.457223in}{1.739932in}}%
\pgfpathlineto{\pgfqpoint{3.457925in}{1.644520in}}%
\pgfpathlineto{\pgfqpoint{3.458393in}{1.721386in}}%
\pgfpathlineto{\pgfqpoint{3.458626in}{1.727405in}}%
\pgfpathlineto{\pgfqpoint{3.460264in}{1.492340in}}%
\pgfpathlineto{\pgfqpoint{3.460731in}{1.599901in}}%
\pgfpathlineto{\pgfqpoint{3.463304in}{1.766231in}}%
\pgfpathlineto{\pgfqpoint{3.463538in}{1.748848in}}%
\pgfpathlineto{\pgfqpoint{3.464473in}{1.699589in}}%
\pgfpathlineto{\pgfqpoint{3.464005in}{1.764566in}}%
\pgfpathlineto{\pgfqpoint{3.464941in}{1.719144in}}%
\pgfpathlineto{\pgfqpoint{3.465175in}{1.719468in}}%
\pgfpathlineto{\pgfqpoint{3.468215in}{1.468421in}}%
\pgfpathlineto{\pgfqpoint{3.466110in}{1.724289in}}%
\pgfpathlineto{\pgfqpoint{3.468449in}{1.488217in}}%
\pgfpathlineto{\pgfqpoint{3.470086in}{1.824303in}}%
\pgfpathlineto{\pgfqpoint{3.470553in}{1.770140in}}%
\pgfpathlineto{\pgfqpoint{3.472658in}{1.589155in}}%
\pgfpathlineto{\pgfqpoint{3.474529in}{1.736818in}}%
\pgfpathlineto{\pgfqpoint{3.474763in}{1.709030in}}%
\pgfpathlineto{\pgfqpoint{3.476868in}{1.488681in}}%
\pgfpathlineto{\pgfqpoint{3.478271in}{1.644992in}}%
\pgfpathlineto{\pgfqpoint{3.478738in}{1.620661in}}%
\pgfpathlineto{\pgfqpoint{3.478972in}{1.590072in}}%
\pgfpathlineto{\pgfqpoint{3.479206in}{1.673455in}}%
\pgfpathlineto{\pgfqpoint{3.480142in}{1.766488in}}%
\pgfpathlineto{\pgfqpoint{3.479674in}{1.668213in}}%
\pgfpathlineto{\pgfqpoint{3.481311in}{1.760016in}}%
\pgfpathlineto{\pgfqpoint{3.484351in}{1.452170in}}%
\pgfpathlineto{\pgfqpoint{3.484585in}{1.456986in}}%
\pgfpathlineto{\pgfqpoint{3.485520in}{1.526121in}}%
\pgfpathlineto{\pgfqpoint{3.485988in}{1.489775in}}%
\pgfpathlineto{\pgfqpoint{3.486222in}{1.484145in}}%
\pgfpathlineto{\pgfqpoint{3.487391in}{1.634259in}}%
\pgfpathlineto{\pgfqpoint{3.487859in}{1.630083in}}%
\pgfpathlineto{\pgfqpoint{3.489262in}{1.740463in}}%
\pgfpathlineto{\pgfqpoint{3.489496in}{1.701694in}}%
\pgfpathlineto{\pgfqpoint{3.491835in}{1.426024in}}%
\pgfpathlineto{\pgfqpoint{3.493238in}{1.576203in}}%
\pgfpathlineto{\pgfqpoint{3.493705in}{1.572131in}}%
\pgfpathlineto{\pgfqpoint{3.493939in}{1.580155in}}%
\pgfpathlineto{\pgfqpoint{3.495109in}{1.397266in}}%
\pgfpathlineto{\pgfqpoint{3.495342in}{1.460877in}}%
\pgfpathlineto{\pgfqpoint{3.496512in}{1.738470in}}%
\pgfpathlineto{\pgfqpoint{3.496746in}{1.723926in}}%
\pgfpathlineto{\pgfqpoint{3.497915in}{1.474734in}}%
\pgfpathlineto{\pgfqpoint{3.498850in}{1.323769in}}%
\pgfpathlineto{\pgfqpoint{3.499084in}{1.381838in}}%
\pgfpathlineto{\pgfqpoint{3.500253in}{1.610142in}}%
\pgfpathlineto{\pgfqpoint{3.500487in}{1.556575in}}%
\pgfpathlineto{\pgfqpoint{3.500955in}{1.468907in}}%
\pgfpathlineto{\pgfqpoint{3.501423in}{1.523260in}}%
\pgfpathlineto{\pgfqpoint{3.502592in}{1.699574in}}%
\pgfpathlineto{\pgfqpoint{3.502826in}{1.675613in}}%
\pgfpathlineto{\pgfqpoint{3.505398in}{1.374514in}}%
\pgfpathlineto{\pgfqpoint{3.505866in}{1.380802in}}%
\pgfpathlineto{\pgfqpoint{3.507971in}{1.514303in}}%
\pgfpathlineto{\pgfqpoint{3.509374in}{1.683225in}}%
\pgfpathlineto{\pgfqpoint{3.509608in}{1.676544in}}%
\pgfpathlineto{\pgfqpoint{3.510777in}{1.439777in}}%
\pgfpathlineto{\pgfqpoint{3.511245in}{1.339797in}}%
\pgfpathlineto{\pgfqpoint{3.511946in}{1.390381in}}%
\pgfpathlineto{\pgfqpoint{3.512180in}{1.389266in}}%
\pgfpathlineto{\pgfqpoint{3.512414in}{1.426911in}}%
\pgfpathlineto{\pgfqpoint{3.512882in}{1.376963in}}%
\pgfpathlineto{\pgfqpoint{3.513116in}{1.349674in}}%
\pgfpathlineto{\pgfqpoint{3.513583in}{1.401925in}}%
\pgfpathlineto{\pgfqpoint{3.515454in}{1.685825in}}%
\pgfpathlineto{\pgfqpoint{3.515688in}{1.647290in}}%
\pgfpathlineto{\pgfqpoint{3.517091in}{1.359674in}}%
\pgfpathlineto{\pgfqpoint{3.517793in}{1.397721in}}%
\pgfpathlineto{\pgfqpoint{3.518027in}{1.392580in}}%
\pgfpathlineto{\pgfqpoint{3.518728in}{1.477606in}}%
\pgfpathlineto{\pgfqpoint{3.519196in}{1.455553in}}%
\pgfpathlineto{\pgfqpoint{3.520365in}{1.507372in}}%
\pgfpathlineto{\pgfqpoint{3.520599in}{1.493872in}}%
\pgfpathlineto{\pgfqpoint{3.522236in}{1.422758in}}%
\pgfpathlineto{\pgfqpoint{3.522470in}{1.432004in}}%
\pgfpathlineto{\pgfqpoint{3.523172in}{1.517678in}}%
\pgfpathlineto{\pgfqpoint{3.523873in}{1.493406in}}%
\pgfpathlineto{\pgfqpoint{3.524341in}{1.530542in}}%
\pgfpathlineto{\pgfqpoint{3.524575in}{1.522600in}}%
\pgfpathlineto{\pgfqpoint{3.525978in}{1.329283in}}%
\pgfpathlineto{\pgfqpoint{3.526212in}{1.354988in}}%
\pgfpathlineto{\pgfqpoint{3.526446in}{1.354786in}}%
\pgfpathlineto{\pgfqpoint{3.526913in}{1.299752in}}%
\pgfpathlineto{\pgfqpoint{3.527147in}{1.370044in}}%
\pgfpathlineto{\pgfqpoint{3.527615in}{1.406094in}}%
\pgfpathlineto{\pgfqpoint{3.529720in}{1.558599in}}%
\pgfpathlineto{\pgfqpoint{3.530187in}{1.556497in}}%
\pgfpathlineto{\pgfqpoint{3.532058in}{1.312146in}}%
\pgfpathlineto{\pgfqpoint{3.532526in}{1.357048in}}%
\pgfpathlineto{\pgfqpoint{3.533228in}{1.533485in}}%
\pgfpathlineto{\pgfqpoint{3.533695in}{1.423033in}}%
\pgfpathlineto{\pgfqpoint{3.535332in}{1.220684in}}%
\pgfpathlineto{\pgfqpoint{3.535566in}{1.252826in}}%
\pgfpathlineto{\pgfqpoint{3.537203in}{1.559539in}}%
\pgfpathlineto{\pgfqpoint{3.538373in}{1.530526in}}%
\pgfpathlineto{\pgfqpoint{3.540945in}{1.278419in}}%
\pgfpathlineto{\pgfqpoint{3.541179in}{1.323260in}}%
\pgfpathlineto{\pgfqpoint{3.542348in}{1.414189in}}%
\pgfpathlineto{\pgfqpoint{3.542582in}{1.384779in}}%
\pgfpathlineto{\pgfqpoint{3.543284in}{1.419862in}}%
\pgfpathlineto{\pgfqpoint{3.543517in}{1.405000in}}%
\pgfpathlineto{\pgfqpoint{3.544687in}{1.183275in}}%
\pgfpathlineto{\pgfqpoint{3.545154in}{1.239240in}}%
\pgfpathlineto{\pgfqpoint{3.546324in}{1.443718in}}%
\pgfpathlineto{\pgfqpoint{3.546558in}{1.416337in}}%
\pgfpathlineto{\pgfqpoint{3.547025in}{1.445369in}}%
\pgfpathlineto{\pgfqpoint{3.547259in}{1.479695in}}%
\pgfpathlineto{\pgfqpoint{3.547961in}{1.451051in}}%
\pgfpathlineto{\pgfqpoint{3.549364in}{1.273476in}}%
\pgfpathlineto{\pgfqpoint{3.549598in}{1.290020in}}%
\pgfpathlineto{\pgfqpoint{3.549832in}{1.295324in}}%
\pgfpathlineto{\pgfqpoint{3.550299in}{1.282864in}}%
\pgfpathlineto{\pgfqpoint{3.550533in}{1.288065in}}%
\pgfpathlineto{\pgfqpoint{3.551235in}{1.399673in}}%
\pgfpathlineto{\pgfqpoint{3.551702in}{1.306489in}}%
\pgfpathlineto{\pgfqpoint{3.553106in}{1.189525in}}%
\pgfpathlineto{\pgfqpoint{3.553340in}{1.214883in}}%
\pgfpathlineto{\pgfqpoint{3.554509in}{1.388521in}}%
\pgfpathlineto{\pgfqpoint{3.554743in}{1.376214in}}%
\pgfpathlineto{\pgfqpoint{3.556380in}{1.194532in}}%
\pgfpathlineto{\pgfqpoint{3.558484in}{1.425110in}}%
\pgfpathlineto{\pgfqpoint{3.559186in}{1.360680in}}%
\pgfpathlineto{\pgfqpoint{3.559420in}{1.383166in}}%
\pgfpathlineto{\pgfqpoint{3.559654in}{1.421369in}}%
\pgfpathlineto{\pgfqpoint{3.560121in}{1.390862in}}%
\pgfpathlineto{\pgfqpoint{3.561992in}{1.153662in}}%
\pgfpathlineto{\pgfqpoint{3.562226in}{1.177146in}}%
\pgfpathlineto{\pgfqpoint{3.562460in}{1.194402in}}%
\pgfpathlineto{\pgfqpoint{3.562694in}{1.143097in}}%
\pgfpathlineto{\pgfqpoint{3.562928in}{1.140015in}}%
\pgfpathlineto{\pgfqpoint{3.564097in}{1.278272in}}%
\pgfpathlineto{\pgfqpoint{3.564331in}{1.217767in}}%
\pgfpathlineto{\pgfqpoint{3.564565in}{1.207067in}}%
\pgfpathlineto{\pgfqpoint{3.564799in}{1.212852in}}%
\pgfpathlineto{\pgfqpoint{3.565734in}{1.415888in}}%
\pgfpathlineto{\pgfqpoint{3.566436in}{1.385187in}}%
\pgfpathlineto{\pgfqpoint{3.566669in}{1.397982in}}%
\pgfpathlineto{\pgfqpoint{3.566903in}{1.361669in}}%
\pgfpathlineto{\pgfqpoint{3.568307in}{1.222488in}}%
\pgfpathlineto{\pgfqpoint{3.568774in}{1.233724in}}%
\pgfpathlineto{\pgfqpoint{3.569008in}{1.240539in}}%
\pgfpathlineto{\pgfqpoint{3.569242in}{1.240360in}}%
\pgfpathlineto{\pgfqpoint{3.570879in}{1.085395in}}%
\pgfpathlineto{\pgfqpoint{3.571347in}{1.122420in}}%
\pgfpathlineto{\pgfqpoint{3.572984in}{1.316612in}}%
\pgfpathlineto{\pgfqpoint{3.573218in}{1.328280in}}%
\pgfpathlineto{\pgfqpoint{3.573451in}{1.316980in}}%
\pgfpathlineto{\pgfqpoint{3.574153in}{1.233363in}}%
\pgfpathlineto{\pgfqpoint{3.574621in}{1.301804in}}%
\pgfpathlineto{\pgfqpoint{3.574855in}{1.355351in}}%
\pgfpathlineto{\pgfqpoint{3.575556in}{1.262764in}}%
\pgfpathlineto{\pgfqpoint{3.577661in}{1.130349in}}%
\pgfpathlineto{\pgfqpoint{3.577895in}{1.148224in}}%
\pgfpathlineto{\pgfqpoint{3.578362in}{1.205144in}}%
\pgfpathlineto{\pgfqpoint{3.578830in}{1.165306in}}%
\pgfpathlineto{\pgfqpoint{3.579298in}{1.117219in}}%
\pgfpathlineto{\pgfqpoint{3.579999in}{1.060038in}}%
\pgfpathlineto{\pgfqpoint{3.580233in}{1.111945in}}%
\pgfpathlineto{\pgfqpoint{3.581636in}{1.325952in}}%
\pgfpathlineto{\pgfqpoint{3.581870in}{1.303995in}}%
\pgfpathlineto{\pgfqpoint{3.582806in}{1.189595in}}%
\pgfpathlineto{\pgfqpoint{3.584443in}{1.072230in}}%
\pgfpathlineto{\pgfqpoint{3.586314in}{1.283405in}}%
\pgfpathlineto{\pgfqpoint{3.586548in}{1.282694in}}%
\pgfpathlineto{\pgfqpoint{3.587015in}{1.256471in}}%
\pgfpathlineto{\pgfqpoint{3.587249in}{1.297627in}}%
\pgfpathlineto{\pgfqpoint{3.587483in}{1.281685in}}%
\pgfpathlineto{\pgfqpoint{3.587717in}{1.299677in}}%
\pgfpathlineto{\pgfqpoint{3.587951in}{1.255911in}}%
\pgfpathlineto{\pgfqpoint{3.588418in}{1.197343in}}%
\pgfpathlineto{\pgfqpoint{3.588652in}{1.215036in}}%
\pgfpathlineto{\pgfqpoint{3.589822in}{1.044024in}}%
\pgfpathlineto{\pgfqpoint{3.590055in}{1.044612in}}%
\pgfpathlineto{\pgfqpoint{3.591926in}{1.224446in}}%
\pgfpathlineto{\pgfqpoint{3.592160in}{1.218804in}}%
\pgfpathlineto{\pgfqpoint{3.592394in}{1.205745in}}%
\pgfpathlineto{\pgfqpoint{3.592862in}{1.233907in}}%
\pgfpathlineto{\pgfqpoint{3.594265in}{1.323147in}}%
\pgfpathlineto{\pgfqpoint{3.594499in}{1.321723in}}%
\pgfpathlineto{\pgfqpoint{3.594966in}{1.270353in}}%
\pgfpathlineto{\pgfqpoint{3.595902in}{1.287993in}}%
\pgfpathlineto{\pgfqpoint{3.596136in}{1.298518in}}%
\pgfpathlineto{\pgfqpoint{3.597305in}{1.095383in}}%
\pgfpathlineto{\pgfqpoint{3.597539in}{1.123377in}}%
\pgfpathlineto{\pgfqpoint{3.597773in}{1.171675in}}%
\pgfpathlineto{\pgfqpoint{3.598474in}{1.127310in}}%
\pgfpathlineto{\pgfqpoint{3.598942in}{1.068568in}}%
\pgfpathlineto{\pgfqpoint{3.599410in}{1.129245in}}%
\pgfpathlineto{\pgfqpoint{3.599644in}{1.124777in}}%
\pgfpathlineto{\pgfqpoint{3.599878in}{1.136087in}}%
\pgfpathlineto{\pgfqpoint{3.602918in}{1.417226in}}%
\pgfpathlineto{\pgfqpoint{3.603152in}{1.387064in}}%
\pgfpathlineto{\pgfqpoint{3.604321in}{1.138652in}}%
\pgfpathlineto{\pgfqpoint{3.605022in}{1.148298in}}%
\pgfpathlineto{\pgfqpoint{3.605256in}{1.180801in}}%
\pgfpathlineto{\pgfqpoint{3.605724in}{1.140532in}}%
\pgfpathlineto{\pgfqpoint{3.606192in}{1.158892in}}%
\pgfpathlineto{\pgfqpoint{3.606426in}{1.161056in}}%
\pgfpathlineto{\pgfqpoint{3.606893in}{1.106996in}}%
\pgfpathlineto{\pgfqpoint{3.607361in}{1.173097in}}%
\pgfpathlineto{\pgfqpoint{3.610167in}{1.484634in}}%
\pgfpathlineto{\pgfqpoint{3.611804in}{1.332564in}}%
\pgfpathlineto{\pgfqpoint{3.612974in}{1.095083in}}%
\pgfpathlineto{\pgfqpoint{3.613207in}{1.125185in}}%
\pgfpathlineto{\pgfqpoint{3.614143in}{1.303407in}}%
\pgfpathlineto{\pgfqpoint{3.614611in}{1.255952in}}%
\pgfpathlineto{\pgfqpoint{3.615078in}{1.293344in}}%
\pgfpathlineto{\pgfqpoint{3.615312in}{1.251893in}}%
\pgfpathlineto{\pgfqpoint{3.616014in}{1.236734in}}%
\pgfpathlineto{\pgfqpoint{3.616248in}{1.259944in}}%
\pgfpathlineto{\pgfqpoint{3.617417in}{1.500556in}}%
\pgfpathlineto{\pgfqpoint{3.617885in}{1.453438in}}%
\pgfpathlineto{\pgfqpoint{3.618119in}{1.460303in}}%
\pgfpathlineto{\pgfqpoint{3.620223in}{1.230413in}}%
\pgfpathlineto{\pgfqpoint{3.620457in}{1.264717in}}%
\pgfpathlineto{\pgfqpoint{3.620691in}{1.324346in}}%
\pgfpathlineto{\pgfqpoint{3.621393in}{1.267036in}}%
\pgfpathlineto{\pgfqpoint{3.622562in}{1.198482in}}%
\pgfpathlineto{\pgfqpoint{3.622796in}{1.237914in}}%
\pgfpathlineto{\pgfqpoint{3.624433in}{1.519722in}}%
\pgfpathlineto{\pgfqpoint{3.624900in}{1.454770in}}%
\pgfpathlineto{\pgfqpoint{3.625368in}{1.429630in}}%
\pgfpathlineto{\pgfqpoint{3.625602in}{1.372595in}}%
\pgfpathlineto{\pgfqpoint{3.626304in}{1.428793in}}%
\pgfpathlineto{\pgfqpoint{3.626771in}{1.535569in}}%
\pgfpathlineto{\pgfqpoint{3.627239in}{1.476166in}}%
\pgfpathlineto{\pgfqpoint{3.628408in}{1.240586in}}%
\pgfpathlineto{\pgfqpoint{3.628876in}{1.267076in}}%
\pgfpathlineto{\pgfqpoint{3.629344in}{1.331748in}}%
\pgfpathlineto{\pgfqpoint{3.630045in}{1.316815in}}%
\pgfpathlineto{\pgfqpoint{3.630279in}{1.311451in}}%
\pgfpathlineto{\pgfqpoint{3.630513in}{1.331390in}}%
\pgfpathlineto{\pgfqpoint{3.633086in}{1.472423in}}%
\pgfpathlineto{\pgfqpoint{3.634021in}{1.555884in}}%
\pgfpathlineto{\pgfqpoint{3.634489in}{1.539215in}}%
\pgfpathlineto{\pgfqpoint{3.636827in}{1.282775in}}%
\pgfpathlineto{\pgfqpoint{3.637295in}{1.295122in}}%
\pgfpathlineto{\pgfqpoint{3.638698in}{1.516137in}}%
\pgfpathlineto{\pgfqpoint{3.638932in}{1.491060in}}%
\pgfpathlineto{\pgfqpoint{3.639166in}{1.509708in}}%
\pgfpathlineto{\pgfqpoint{3.639634in}{1.477727in}}%
\pgfpathlineto{\pgfqpoint{3.639867in}{1.457217in}}%
\pgfpathlineto{\pgfqpoint{3.640335in}{1.500741in}}%
\pgfpathlineto{\pgfqpoint{3.640569in}{1.492704in}}%
\pgfpathlineto{\pgfqpoint{3.641271in}{1.415407in}}%
\pgfpathlineto{\pgfqpoint{3.641738in}{1.444006in}}%
\pgfpathlineto{\pgfqpoint{3.642206in}{1.533622in}}%
\pgfpathlineto{\pgfqpoint{3.642908in}{1.465894in}}%
\pgfpathlineto{\pgfqpoint{3.643141in}{1.461963in}}%
\pgfpathlineto{\pgfqpoint{3.643609in}{1.402699in}}%
\pgfpathlineto{\pgfqpoint{3.644311in}{1.408424in}}%
\pgfpathlineto{\pgfqpoint{3.644545in}{1.426935in}}%
\pgfpathlineto{\pgfqpoint{3.644778in}{1.388982in}}%
\pgfpathlineto{\pgfqpoint{3.645246in}{1.349132in}}%
\pgfpathlineto{\pgfqpoint{3.645480in}{1.369022in}}%
\pgfpathlineto{\pgfqpoint{3.646649in}{1.603014in}}%
\pgfpathlineto{\pgfqpoint{3.646883in}{1.598681in}}%
\pgfpathlineto{\pgfqpoint{3.648520in}{1.427623in}}%
\pgfpathlineto{\pgfqpoint{3.648754in}{1.430817in}}%
\pgfpathlineto{\pgfqpoint{3.649923in}{1.318616in}}%
\pgfpathlineto{\pgfqpoint{3.652028in}{1.599318in}}%
\pgfpathlineto{\pgfqpoint{3.652496in}{1.620006in}}%
\pgfpathlineto{\pgfqpoint{3.652730in}{1.644975in}}%
\pgfpathlineto{\pgfqpoint{3.653197in}{1.596826in}}%
\pgfpathlineto{\pgfqpoint{3.653431in}{1.602518in}}%
\pgfpathlineto{\pgfqpoint{3.654834in}{1.312390in}}%
\pgfpathlineto{\pgfqpoint{3.655536in}{1.351892in}}%
\pgfpathlineto{\pgfqpoint{3.656705in}{1.479447in}}%
\pgfpathlineto{\pgfqpoint{3.658342in}{1.664353in}}%
\pgfpathlineto{\pgfqpoint{3.658576in}{1.649913in}}%
\pgfpathlineto{\pgfqpoint{3.659044in}{1.670340in}}%
\pgfpathlineto{\pgfqpoint{3.659278in}{1.684806in}}%
\pgfpathlineto{\pgfqpoint{3.659512in}{1.665902in}}%
\pgfpathlineto{\pgfqpoint{3.662318in}{1.384529in}}%
\pgfpathlineto{\pgfqpoint{3.662552in}{1.429356in}}%
\pgfpathlineto{\pgfqpoint{3.663253in}{1.425589in}}%
\pgfpathlineto{\pgfqpoint{3.664423in}{1.631100in}}%
\pgfpathlineto{\pgfqpoint{3.665124in}{1.758425in}}%
\pgfpathlineto{\pgfqpoint{3.665592in}{1.709857in}}%
\pgfpathlineto{\pgfqpoint{3.668164in}{1.416718in}}%
\pgfpathlineto{\pgfqpoint{3.668632in}{1.447068in}}%
\pgfpathlineto{\pgfqpoint{3.669568in}{1.547272in}}%
\pgfpathlineto{\pgfqpoint{3.670035in}{1.538507in}}%
\pgfpathlineto{\pgfqpoint{3.670503in}{1.458440in}}%
\pgfpathlineto{\pgfqpoint{3.670971in}{1.544200in}}%
\pgfpathlineto{\pgfqpoint{3.673777in}{1.743558in}}%
\pgfpathlineto{\pgfqpoint{3.674245in}{1.695329in}}%
\pgfpathlineto{\pgfqpoint{3.676583in}{1.489941in}}%
\pgfpathlineto{\pgfqpoint{3.676817in}{1.501085in}}%
\pgfpathlineto{\pgfqpoint{3.677051in}{1.471584in}}%
\pgfpathlineto{\pgfqpoint{3.677285in}{1.457756in}}%
\pgfpathlineto{\pgfqpoint{3.677753in}{1.493565in}}%
\pgfpathlineto{\pgfqpoint{3.677987in}{1.502093in}}%
\pgfpathlineto{\pgfqpoint{3.679857in}{1.746179in}}%
\pgfpathlineto{\pgfqpoint{3.681261in}{1.577948in}}%
\pgfpathlineto{\pgfqpoint{3.681494in}{1.613276in}}%
\pgfpathlineto{\pgfqpoint{3.683365in}{1.698684in}}%
\pgfpathlineto{\pgfqpoint{3.683599in}{1.684613in}}%
\pgfpathlineto{\pgfqpoint{3.683833in}{1.708730in}}%
\pgfpathlineto{\pgfqpoint{3.684067in}{1.706180in}}%
\pgfpathlineto{\pgfqpoint{3.684301in}{1.716352in}}%
\pgfpathlineto{\pgfqpoint{3.686172in}{1.479863in}}%
\pgfpathlineto{\pgfqpoint{3.686639in}{1.499597in}}%
\pgfpathlineto{\pgfqpoint{3.686873in}{1.572025in}}%
\pgfpathlineto{\pgfqpoint{3.687575in}{1.493361in}}%
\pgfpathlineto{\pgfqpoint{3.688276in}{1.635143in}}%
\pgfpathlineto{\pgfqpoint{3.688510in}{1.700681in}}%
\pgfpathlineto{\pgfqpoint{3.689446in}{1.688175in}}%
\pgfpathlineto{\pgfqpoint{3.690147in}{1.773776in}}%
\pgfpathlineto{\pgfqpoint{3.690849in}{1.717401in}}%
\pgfpathlineto{\pgfqpoint{3.691316in}{1.662095in}}%
\pgfpathlineto{\pgfqpoint{3.691784in}{1.699983in}}%
\pgfpathlineto{\pgfqpoint{3.692252in}{1.722995in}}%
\pgfpathlineto{\pgfqpoint{3.692720in}{1.696699in}}%
\pgfpathlineto{\pgfqpoint{3.694357in}{1.488045in}}%
\pgfpathlineto{\pgfqpoint{3.694824in}{1.489573in}}%
\pgfpathlineto{\pgfqpoint{3.696461in}{1.780239in}}%
\pgfpathlineto{\pgfqpoint{3.696929in}{1.778197in}}%
\pgfpathlineto{\pgfqpoint{3.698098in}{1.598428in}}%
\pgfpathlineto{\pgfqpoint{3.698800in}{1.688731in}}%
\pgfpathlineto{\pgfqpoint{3.699735in}{1.725381in}}%
\pgfpathlineto{\pgfqpoint{3.699969in}{1.719592in}}%
\pgfpathlineto{\pgfqpoint{3.701840in}{1.523938in}}%
\pgfpathlineto{\pgfqpoint{3.702074in}{1.530157in}}%
\pgfpathlineto{\pgfqpoint{3.702308in}{1.522716in}}%
\pgfpathlineto{\pgfqpoint{3.702542in}{1.537043in}}%
\pgfpathlineto{\pgfqpoint{3.705582in}{1.769846in}}%
\pgfpathlineto{\pgfqpoint{3.706050in}{1.750451in}}%
\pgfpathlineto{\pgfqpoint{3.706985in}{1.750752in}}%
\pgfpathlineto{\pgfqpoint{3.707453in}{1.689336in}}%
\pgfpathlineto{\pgfqpoint{3.708388in}{1.433988in}}%
\pgfpathlineto{\pgfqpoint{3.709090in}{1.472286in}}%
\pgfpathlineto{\pgfqpoint{3.709558in}{1.524584in}}%
\pgfpathlineto{\pgfqpoint{3.710727in}{1.725871in}}%
\pgfpathlineto{\pgfqpoint{3.710961in}{1.697486in}}%
\pgfpathlineto{\pgfqpoint{3.711662in}{1.726480in}}%
\pgfpathlineto{\pgfqpoint{3.712130in}{1.661609in}}%
\pgfpathlineto{\pgfqpoint{3.713299in}{1.730155in}}%
\pgfpathlineto{\pgfqpoint{3.713533in}{1.710976in}}%
\pgfpathlineto{\pgfqpoint{3.714001in}{1.652266in}}%
\pgfpathlineto{\pgfqpoint{3.715872in}{1.539381in}}%
\pgfpathlineto{\pgfqpoint{3.716106in}{1.530417in}}%
\pgfpathlineto{\pgfqpoint{3.716807in}{1.415950in}}%
\pgfpathlineto{\pgfqpoint{3.717509in}{1.449211in}}%
\pgfpathlineto{\pgfqpoint{3.717976in}{1.512324in}}%
\pgfpathlineto{\pgfqpoint{3.719380in}{1.780841in}}%
\pgfpathlineto{\pgfqpoint{3.720081in}{1.731236in}}%
\pgfpathlineto{\pgfqpoint{3.722654in}{1.351088in}}%
\pgfpathlineto{\pgfqpoint{3.722887in}{1.386877in}}%
\pgfpathlineto{\pgfqpoint{3.723823in}{1.628510in}}%
\pgfpathlineto{\pgfqpoint{3.724992in}{1.563053in}}%
\pgfpathlineto{\pgfqpoint{3.725694in}{1.492998in}}%
\pgfpathlineto{\pgfqpoint{3.725928in}{1.509598in}}%
\pgfpathlineto{\pgfqpoint{3.726395in}{1.584063in}}%
\pgfpathlineto{\pgfqpoint{3.726863in}{1.539205in}}%
\pgfpathlineto{\pgfqpoint{3.727565in}{1.445804in}}%
\pgfpathlineto{\pgfqpoint{3.727799in}{1.497923in}}%
\pgfpathlineto{\pgfqpoint{3.728968in}{1.590753in}}%
\pgfpathlineto{\pgfqpoint{3.729202in}{1.554488in}}%
\pgfpathlineto{\pgfqpoint{3.729669in}{1.518905in}}%
\pgfpathlineto{\pgfqpoint{3.729903in}{1.591665in}}%
\pgfpathlineto{\pgfqpoint{3.730137in}{1.616737in}}%
\pgfpathlineto{\pgfqpoint{3.730605in}{1.571360in}}%
\pgfpathlineto{\pgfqpoint{3.732008in}{1.344785in}}%
\pgfpathlineto{\pgfqpoint{3.732476in}{1.462980in}}%
\pgfpathlineto{\pgfqpoint{3.733411in}{1.676543in}}%
\pgfpathlineto{\pgfqpoint{3.733879in}{1.608180in}}%
\pgfpathlineto{\pgfqpoint{3.735750in}{1.315325in}}%
\pgfpathlineto{\pgfqpoint{3.735984in}{1.323769in}}%
\pgfpathlineto{\pgfqpoint{3.736919in}{1.541502in}}%
\pgfpathlineto{\pgfqpoint{3.737387in}{1.452766in}}%
\pgfpathlineto{\pgfqpoint{3.738088in}{1.598272in}}%
\pgfpathlineto{\pgfqpoint{3.739258in}{1.572234in}}%
\pgfpathlineto{\pgfqpoint{3.740661in}{1.334339in}}%
\pgfpathlineto{\pgfqpoint{3.741129in}{1.372045in}}%
\pgfpathlineto{\pgfqpoint{3.741362in}{1.439217in}}%
\pgfpathlineto{\pgfqpoint{3.742064in}{1.360690in}}%
\pgfpathlineto{\pgfqpoint{3.742999in}{1.512289in}}%
\pgfpathlineto{\pgfqpoint{3.743701in}{1.462129in}}%
\pgfpathlineto{\pgfqpoint{3.745104in}{1.406798in}}%
\pgfpathlineto{\pgfqpoint{3.746040in}{1.557438in}}%
\pgfpathlineto{\pgfqpoint{3.746741in}{1.668243in}}%
\pgfpathlineto{\pgfqpoint{3.746975in}{1.591180in}}%
\pgfpathlineto{\pgfqpoint{3.748612in}{1.284072in}}%
\pgfpathlineto{\pgfqpoint{3.748846in}{1.308292in}}%
\pgfpathlineto{\pgfqpoint{3.750483in}{1.492186in}}%
\pgfpathlineto{\pgfqpoint{3.750717in}{1.475082in}}%
\pgfpathlineto{\pgfqpoint{3.751184in}{1.399446in}}%
\pgfpathlineto{\pgfqpoint{3.751418in}{1.467405in}}%
\pgfpathlineto{\pgfqpoint{3.752120in}{1.629967in}}%
\pgfpathlineto{\pgfqpoint{3.752588in}{1.488737in}}%
\pgfpathlineto{\pgfqpoint{3.753523in}{1.304491in}}%
\pgfpathlineto{\pgfqpoint{3.754225in}{1.369850in}}%
\pgfpathlineto{\pgfqpoint{3.754692in}{1.472333in}}%
\pgfpathlineto{\pgfqpoint{3.755628in}{1.439820in}}%
\pgfpathlineto{\pgfqpoint{3.756096in}{1.410232in}}%
\pgfpathlineto{\pgfqpoint{3.756329in}{1.446405in}}%
\pgfpathlineto{\pgfqpoint{3.756797in}{1.496299in}}%
\pgfpathlineto{\pgfqpoint{3.757031in}{1.485508in}}%
\pgfpathlineto{\pgfqpoint{3.757966in}{1.275861in}}%
\pgfpathlineto{\pgfqpoint{3.758434in}{1.332235in}}%
\pgfpathlineto{\pgfqpoint{3.758668in}{1.347102in}}%
\pgfpathlineto{\pgfqpoint{3.759136in}{1.321241in}}%
\pgfpathlineto{\pgfqpoint{3.759370in}{1.318094in}}%
\pgfpathlineto{\pgfqpoint{3.760071in}{1.445221in}}%
\pgfpathlineto{\pgfqpoint{3.760539in}{1.367278in}}%
\pgfpathlineto{\pgfqpoint{3.761708in}{1.302077in}}%
\pgfpathlineto{\pgfqpoint{3.761942in}{1.318547in}}%
\pgfpathlineto{\pgfqpoint{3.763345in}{1.556480in}}%
\pgfpathlineto{\pgfqpoint{3.764047in}{1.516839in}}%
\pgfpathlineto{\pgfqpoint{3.766151in}{1.310604in}}%
\pgfpathlineto{\pgfqpoint{3.766853in}{1.402061in}}%
\pgfpathlineto{\pgfqpoint{3.767321in}{1.333462in}}%
\pgfpathlineto{\pgfqpoint{3.768022in}{1.218559in}}%
\pgfpathlineto{\pgfqpoint{3.768490in}{1.298332in}}%
\pgfpathlineto{\pgfqpoint{3.769192in}{1.383556in}}%
\pgfpathlineto{\pgfqpoint{3.771063in}{1.520849in}}%
\pgfpathlineto{\pgfqpoint{3.771530in}{1.462032in}}%
\pgfpathlineto{\pgfqpoint{3.772933in}{1.296164in}}%
\pgfpathlineto{\pgfqpoint{3.773401in}{1.316794in}}%
\pgfpathlineto{\pgfqpoint{3.774570in}{1.455138in}}%
\pgfpathlineto{\pgfqpoint{3.775038in}{1.430703in}}%
\pgfpathlineto{\pgfqpoint{3.776675in}{1.233684in}}%
\pgfpathlineto{\pgfqpoint{3.777377in}{1.256058in}}%
\pgfpathlineto{\pgfqpoint{3.778078in}{1.294540in}}%
\pgfpathlineto{\pgfqpoint{3.778312in}{1.285619in}}%
\pgfpathlineto{\pgfqpoint{3.779248in}{1.176148in}}%
\pgfpathlineto{\pgfqpoint{3.779481in}{1.264954in}}%
\pgfpathlineto{\pgfqpoint{3.780885in}{1.422134in}}%
\pgfpathlineto{\pgfqpoint{3.781352in}{1.433861in}}%
\pgfpathlineto{\pgfqpoint{3.783223in}{1.307995in}}%
\pgfpathlineto{\pgfqpoint{3.783457in}{1.281818in}}%
\pgfpathlineto{\pgfqpoint{3.783925in}{1.350996in}}%
\pgfpathlineto{\pgfqpoint{3.784159in}{1.354147in}}%
\pgfpathlineto{\pgfqpoint{3.785328in}{1.122703in}}%
\pgfpathlineto{\pgfqpoint{3.786030in}{1.178680in}}%
\pgfpathlineto{\pgfqpoint{3.787199in}{1.340794in}}%
\pgfpathlineto{\pgfqpoint{3.787667in}{1.284013in}}%
\pgfpathlineto{\pgfqpoint{3.788134in}{1.266900in}}%
\pgfpathlineto{\pgfqpoint{3.788368in}{1.280141in}}%
\pgfpathlineto{\pgfqpoint{3.790005in}{1.436865in}}%
\pgfpathlineto{\pgfqpoint{3.790239in}{1.402314in}}%
\pgfpathlineto{\pgfqpoint{3.791876in}{1.235630in}}%
\pgfpathlineto{\pgfqpoint{3.793045in}{1.069565in}}%
\pgfpathlineto{\pgfqpoint{3.793513in}{1.170353in}}%
\pgfpathlineto{\pgfqpoint{3.794916in}{1.273164in}}%
\pgfpathlineto{\pgfqpoint{3.795150in}{1.272520in}}%
\pgfpathlineto{\pgfqpoint{3.796085in}{1.157198in}}%
\pgfpathlineto{\pgfqpoint{3.796319in}{1.260569in}}%
\pgfpathlineto{\pgfqpoint{3.797255in}{1.429055in}}%
\pgfpathlineto{\pgfqpoint{3.797722in}{1.410259in}}%
\pgfpathlineto{\pgfqpoint{3.799593in}{1.159953in}}%
\pgfpathlineto{\pgfqpoint{3.799827in}{1.159599in}}%
\pgfpathlineto{\pgfqpoint{3.800061in}{1.189384in}}%
\pgfpathlineto{\pgfqpoint{3.800529in}{1.160390in}}%
\pgfpathlineto{\pgfqpoint{3.801698in}{1.100990in}}%
\pgfpathlineto{\pgfqpoint{3.803101in}{1.239846in}}%
\pgfpathlineto{\pgfqpoint{3.804037in}{1.364789in}}%
\pgfpathlineto{\pgfqpoint{3.804271in}{1.336155in}}%
\pgfpathlineto{\pgfqpoint{3.804738in}{1.342087in}}%
\pgfpathlineto{\pgfqpoint{3.806141in}{1.200661in}}%
\pgfpathlineto{\pgfqpoint{3.806609in}{1.246568in}}%
\pgfpathlineto{\pgfqpoint{3.806843in}{1.206278in}}%
\pgfpathlineto{\pgfqpoint{3.808012in}{1.025744in}}%
\pgfpathlineto{\pgfqpoint{3.808480in}{1.098947in}}%
\pgfpathlineto{\pgfqpoint{3.809883in}{1.218664in}}%
\pgfpathlineto{\pgfqpoint{3.810585in}{1.155255in}}%
\pgfpathlineto{\pgfqpoint{3.810819in}{1.193936in}}%
\pgfpathlineto{\pgfqpoint{3.811286in}{1.235555in}}%
\pgfpathlineto{\pgfqpoint{3.811754in}{1.218937in}}%
\pgfpathlineto{\pgfqpoint{3.811988in}{1.191987in}}%
\pgfpathlineto{\pgfqpoint{3.812222in}{1.226123in}}%
\pgfpathlineto{\pgfqpoint{3.813157in}{1.325084in}}%
\pgfpathlineto{\pgfqpoint{3.813391in}{1.265904in}}%
\pgfpathlineto{\pgfqpoint{3.813625in}{1.279296in}}%
\pgfpathlineto{\pgfqpoint{3.813859in}{1.262751in}}%
\pgfpathlineto{\pgfqpoint{3.814093in}{1.265831in}}%
\pgfpathlineto{\pgfqpoint{3.816899in}{0.952571in}}%
\pgfpathlineto{\pgfqpoint{3.814794in}{1.283437in}}%
\pgfpathlineto{\pgfqpoint{3.817367in}{1.064379in}}%
\pgfpathlineto{\pgfqpoint{3.818068in}{1.165609in}}%
\pgfpathlineto{\pgfqpoint{3.819471in}{1.295930in}}%
\pgfpathlineto{\pgfqpoint{3.819705in}{1.277664in}}%
\pgfpathlineto{\pgfqpoint{3.819939in}{1.272270in}}%
\pgfpathlineto{\pgfqpoint{3.820173in}{1.225861in}}%
\pgfpathlineto{\pgfqpoint{3.820875in}{1.303668in}}%
\pgfpathlineto{\pgfqpoint{3.821108in}{1.245499in}}%
\pgfpathlineto{\pgfqpoint{3.821342in}{1.242647in}}%
\pgfpathlineto{\pgfqpoint{3.823915in}{1.017544in}}%
\pgfpathlineto{\pgfqpoint{3.825552in}{1.108218in}}%
\pgfpathlineto{\pgfqpoint{3.826019in}{1.107318in}}%
\pgfpathlineto{\pgfqpoint{3.826721in}{1.079236in}}%
\pgfpathlineto{\pgfqpoint{3.826955in}{1.100134in}}%
\pgfpathlineto{\pgfqpoint{3.827890in}{1.250173in}}%
\pgfpathlineto{\pgfqpoint{3.828358in}{1.166894in}}%
\pgfpathlineto{\pgfqpoint{3.828592in}{1.131082in}}%
\pgfpathlineto{\pgfqpoint{3.829060in}{1.220371in}}%
\pgfpathlineto{\pgfqpoint{3.829293in}{1.229319in}}%
\pgfpathlineto{\pgfqpoint{3.830930in}{1.140012in}}%
\pgfpathlineto{\pgfqpoint{3.831398in}{1.153003in}}%
\pgfpathlineto{\pgfqpoint{3.832568in}{1.076546in}}%
\pgfpathlineto{\pgfqpoint{3.835140in}{1.240638in}}%
\pgfpathlineto{\pgfqpoint{3.835608in}{1.206284in}}%
\pgfpathlineto{\pgfqpoint{3.836309in}{1.171352in}}%
\pgfpathlineto{\pgfqpoint{3.837479in}{0.964567in}}%
\pgfpathlineto{\pgfqpoint{3.837712in}{0.974452in}}%
\pgfpathlineto{\pgfqpoint{3.840051in}{1.333884in}}%
\pgfpathlineto{\pgfqpoint{3.843559in}{0.978764in}}%
\pgfpathlineto{\pgfqpoint{3.844027in}{1.063808in}}%
\pgfpathlineto{\pgfqpoint{3.844494in}{1.149899in}}%
\pgfpathlineto{\pgfqpoint{3.844728in}{1.177215in}}%
\pgfpathlineto{\pgfqpoint{3.844962in}{1.103490in}}%
\pgfpathlineto{\pgfqpoint{3.845196in}{1.095891in}}%
\pgfpathlineto{\pgfqpoint{3.845664in}{1.115696in}}%
\pgfpathlineto{\pgfqpoint{3.847067in}{1.237564in}}%
\pgfpathlineto{\pgfqpoint{3.847535in}{1.384895in}}%
\pgfpathlineto{\pgfqpoint{3.848236in}{1.285676in}}%
\pgfpathlineto{\pgfqpoint{3.850107in}{1.085811in}}%
\pgfpathlineto{\pgfqpoint{3.850809in}{1.145130in}}%
\pgfpathlineto{\pgfqpoint{3.851042in}{1.151893in}}%
\pgfpathlineto{\pgfqpoint{3.851510in}{1.269516in}}%
\pgfpathlineto{\pgfqpoint{3.852212in}{1.231121in}}%
\pgfpathlineto{\pgfqpoint{3.853147in}{1.146182in}}%
\pgfpathlineto{\pgfqpoint{3.853381in}{1.162714in}}%
\pgfpathlineto{\pgfqpoint{3.854784in}{1.367525in}}%
\pgfpathlineto{\pgfqpoint{3.855252in}{1.283173in}}%
\pgfpathlineto{\pgfqpoint{3.856889in}{1.033284in}}%
\pgfpathlineto{\pgfqpoint{3.857357in}{1.059873in}}%
\pgfpathlineto{\pgfqpoint{3.859227in}{1.377886in}}%
\pgfpathlineto{\pgfqpoint{3.859929in}{1.334517in}}%
\pgfpathlineto{\pgfqpoint{3.860397in}{1.331623in}}%
\pgfpathlineto{\pgfqpoint{3.860864in}{1.351017in}}%
\pgfpathlineto{\pgfqpoint{3.862501in}{1.124725in}}%
\pgfpathlineto{\pgfqpoint{3.862969in}{1.063962in}}%
\pgfpathlineto{\pgfqpoint{3.863437in}{1.129331in}}%
\pgfpathlineto{\pgfqpoint{3.863671in}{1.127723in}}%
\pgfpathlineto{\pgfqpoint{3.863905in}{1.132019in}}%
\pgfpathlineto{\pgfqpoint{3.864139in}{1.137365in}}%
\pgfpathlineto{\pgfqpoint{3.865776in}{1.479466in}}%
\pgfpathlineto{\pgfqpoint{3.866243in}{1.405651in}}%
\pgfpathlineto{\pgfqpoint{3.868348in}{1.211898in}}%
\pgfpathlineto{\pgfqpoint{3.869283in}{1.086781in}}%
\pgfpathlineto{\pgfqpoint{3.869517in}{1.148248in}}%
\pgfpathlineto{\pgfqpoint{3.871154in}{1.406971in}}%
\pgfpathlineto{\pgfqpoint{3.871388in}{1.374302in}}%
\pgfpathlineto{\pgfqpoint{3.871856in}{1.425007in}}%
\pgfpathlineto{\pgfqpoint{3.872324in}{1.387205in}}%
\pgfpathlineto{\pgfqpoint{3.872557in}{1.409365in}}%
\pgfpathlineto{\pgfqpoint{3.873025in}{1.352446in}}%
\pgfpathlineto{\pgfqpoint{3.873259in}{1.359565in}}%
\pgfpathlineto{\pgfqpoint{3.873493in}{1.343553in}}%
\pgfpathlineto{\pgfqpoint{3.875364in}{1.134591in}}%
\pgfpathlineto{\pgfqpoint{3.876065in}{1.186130in}}%
\pgfpathlineto{\pgfqpoint{3.877468in}{1.390428in}}%
\pgfpathlineto{\pgfqpoint{3.879106in}{1.376165in}}%
\pgfpathlineto{\pgfqpoint{3.880743in}{1.458611in}}%
\pgfpathlineto{\pgfqpoint{3.880976in}{1.431324in}}%
\pgfpathlineto{\pgfqpoint{3.881678in}{1.225443in}}%
\pgfpathlineto{\pgfqpoint{3.882380in}{1.250924in}}%
\pgfpathlineto{\pgfqpoint{3.882613in}{1.275135in}}%
\pgfpathlineto{\pgfqpoint{3.882847in}{1.227799in}}%
\pgfpathlineto{\pgfqpoint{3.883315in}{1.234920in}}%
\pgfpathlineto{\pgfqpoint{3.883549in}{1.196369in}}%
\pgfpathlineto{\pgfqpoint{3.884017in}{1.287006in}}%
\pgfpathlineto{\pgfqpoint{3.884718in}{1.368649in}}%
\pgfpathlineto{\pgfqpoint{3.885186in}{1.352158in}}%
\pgfpathlineto{\pgfqpoint{3.885420in}{1.352924in}}%
\pgfpathlineto{\pgfqpoint{3.887057in}{1.438946in}}%
\pgfpathlineto{\pgfqpoint{3.887524in}{1.341741in}}%
\pgfpathlineto{\pgfqpoint{3.888460in}{1.381900in}}%
\pgfpathlineto{\pgfqpoint{3.888928in}{1.371301in}}%
\pgfpathlineto{\pgfqpoint{3.891266in}{1.506115in}}%
\pgfpathlineto{\pgfqpoint{3.892435in}{1.275409in}}%
\pgfpathlineto{\pgfqpoint{3.893137in}{1.278807in}}%
\pgfpathlineto{\pgfqpoint{3.893371in}{1.273060in}}%
\pgfpathlineto{\pgfqpoint{3.893605in}{1.289815in}}%
\pgfpathlineto{\pgfqpoint{3.893839in}{1.289030in}}%
\pgfpathlineto{\pgfqpoint{3.894306in}{1.311962in}}%
\pgfpathlineto{\pgfqpoint{3.896177in}{1.575773in}}%
\pgfpathlineto{\pgfqpoint{3.896411in}{1.550496in}}%
\pgfpathlineto{\pgfqpoint{3.898048in}{1.446346in}}%
\pgfpathlineto{\pgfqpoint{3.898750in}{1.500590in}}%
\pgfpathlineto{\pgfqpoint{3.898984in}{1.433601in}}%
\pgfpathlineto{\pgfqpoint{3.900387in}{1.339606in}}%
\pgfpathlineto{\pgfqpoint{3.900854in}{1.363898in}}%
\pgfpathlineto{\pgfqpoint{3.901088in}{1.360411in}}%
\pgfpathlineto{\pgfqpoint{3.901556in}{1.277606in}}%
\pgfpathlineto{\pgfqpoint{3.902024in}{1.407494in}}%
\pgfpathlineto{\pgfqpoint{3.903427in}{1.563314in}}%
\pgfpathlineto{\pgfqpoint{3.903895in}{1.489929in}}%
\pgfpathlineto{\pgfqpoint{3.904596in}{1.386836in}}%
\pgfpathlineto{\pgfqpoint{3.905064in}{1.425603in}}%
\pgfpathlineto{\pgfqpoint{3.906467in}{1.522438in}}%
\pgfpathlineto{\pgfqpoint{3.906935in}{1.481611in}}%
\pgfpathlineto{\pgfqpoint{3.907169in}{1.472469in}}%
\pgfpathlineto{\pgfqpoint{3.907402in}{1.506370in}}%
\pgfpathlineto{\pgfqpoint{3.907636in}{1.501398in}}%
\pgfpathlineto{\pgfqpoint{3.908806in}{1.371290in}}%
\pgfpathlineto{\pgfqpoint{3.909039in}{1.389656in}}%
\pgfpathlineto{\pgfqpoint{3.909507in}{1.443960in}}%
\pgfpathlineto{\pgfqpoint{3.911144in}{1.577736in}}%
\pgfpathlineto{\pgfqpoint{3.911378in}{1.591995in}}%
\pgfpathlineto{\pgfqpoint{3.911612in}{1.569053in}}%
\pgfpathlineto{\pgfqpoint{3.913015in}{1.351201in}}%
\pgfpathlineto{\pgfqpoint{3.913483in}{1.361085in}}%
\pgfpathlineto{\pgfqpoint{3.917225in}{1.611147in}}%
\pgfpathlineto{\pgfqpoint{3.917458in}{1.592378in}}%
\pgfpathlineto{\pgfqpoint{3.918160in}{1.507294in}}%
\pgfpathlineto{\pgfqpoint{3.918628in}{1.552999in}}%
\pgfpathlineto{\pgfqpoint{3.921668in}{1.354039in}}%
\pgfpathlineto{\pgfqpoint{3.922136in}{1.402000in}}%
\pgfpathlineto{\pgfqpoint{3.922369in}{1.405163in}}%
\pgfpathlineto{\pgfqpoint{3.922603in}{1.393507in}}%
\pgfpathlineto{\pgfqpoint{3.922837in}{1.392431in}}%
\pgfpathlineto{\pgfqpoint{3.923071in}{1.357549in}}%
\pgfpathlineto{\pgfqpoint{3.923305in}{1.476255in}}%
\pgfpathlineto{\pgfqpoint{3.924474in}{1.718815in}}%
\pgfpathlineto{\pgfqpoint{3.924942in}{1.658127in}}%
\pgfpathlineto{\pgfqpoint{3.925877in}{1.497745in}}%
\pgfpathlineto{\pgfqpoint{3.926345in}{1.508631in}}%
\pgfpathlineto{\pgfqpoint{3.926579in}{1.525085in}}%
\pgfpathlineto{\pgfqpoint{3.927047in}{1.480355in}}%
\pgfpathlineto{\pgfqpoint{3.928684in}{1.323853in}}%
\pgfpathlineto{\pgfqpoint{3.928918in}{1.340138in}}%
\pgfpathlineto{\pgfqpoint{3.930087in}{1.579621in}}%
\pgfpathlineto{\pgfqpoint{3.931022in}{1.564458in}}%
\pgfpathlineto{\pgfqpoint{3.932192in}{1.419988in}}%
\pgfpathlineto{\pgfqpoint{3.932425in}{1.448595in}}%
\pgfpathlineto{\pgfqpoint{3.933829in}{1.645829in}}%
\pgfpathlineto{\pgfqpoint{3.934296in}{1.630703in}}%
\pgfpathlineto{\pgfqpoint{3.934998in}{1.648209in}}%
\pgfpathlineto{\pgfqpoint{3.936401in}{1.356386in}}%
\pgfpathlineto{\pgfqpoint{3.936635in}{1.265693in}}%
\pgfpathlineto{\pgfqpoint{3.937336in}{1.369766in}}%
\pgfpathlineto{\pgfqpoint{3.940377in}{1.662817in}}%
\pgfpathlineto{\pgfqpoint{3.944118in}{1.331200in}}%
\pgfpathlineto{\pgfqpoint{3.946925in}{1.690507in}}%
\pgfpathlineto{\pgfqpoint{3.944586in}{1.327451in}}%
\pgfpathlineto{\pgfqpoint{3.947392in}{1.648747in}}%
\pgfpathlineto{\pgfqpoint{3.949965in}{1.300353in}}%
\pgfpathlineto{\pgfqpoint{3.950199in}{1.326605in}}%
\pgfpathlineto{\pgfqpoint{3.951134in}{1.453380in}}%
\pgfpathlineto{\pgfqpoint{3.951602in}{1.432963in}}%
\pgfpathlineto{\pgfqpoint{3.951836in}{1.435289in}}%
\pgfpathlineto{\pgfqpoint{3.952537in}{1.582554in}}%
\pgfpathlineto{\pgfqpoint{3.953239in}{1.561405in}}%
\pgfpathlineto{\pgfqpoint{3.954408in}{1.414562in}}%
\pgfpathlineto{\pgfqpoint{3.954642in}{1.461111in}}%
\pgfpathlineto{\pgfqpoint{3.955811in}{1.648303in}}%
\pgfpathlineto{\pgfqpoint{3.956279in}{1.594757in}}%
\pgfpathlineto{\pgfqpoint{3.957916in}{1.447627in}}%
\pgfpathlineto{\pgfqpoint{3.958150in}{1.407098in}}%
\pgfpathlineto{\pgfqpoint{3.958618in}{1.487143in}}%
\pgfpathlineto{\pgfqpoint{3.959787in}{1.545631in}}%
\pgfpathlineto{\pgfqpoint{3.960255in}{1.496156in}}%
\pgfpathlineto{\pgfqpoint{3.961424in}{1.340363in}}%
\pgfpathlineto{\pgfqpoint{3.961658in}{1.351167in}}%
\pgfpathlineto{\pgfqpoint{3.962827in}{1.517593in}}%
\pgfpathlineto{\pgfqpoint{3.963061in}{1.443010in}}%
\pgfpathlineto{\pgfqpoint{3.963529in}{1.409504in}}%
\pgfpathlineto{\pgfqpoint{3.963996in}{1.450663in}}%
\pgfpathlineto{\pgfqpoint{3.964464in}{1.445698in}}%
\pgfpathlineto{\pgfqpoint{3.965633in}{1.618938in}}%
\pgfpathlineto{\pgfqpoint{3.965867in}{1.607781in}}%
\pgfpathlineto{\pgfqpoint{3.966569in}{1.461092in}}%
\pgfpathlineto{\pgfqpoint{3.967270in}{1.465857in}}%
\pgfpathlineto{\pgfqpoint{3.967504in}{1.482726in}}%
\pgfpathlineto{\pgfqpoint{3.967972in}{1.442595in}}%
\pgfpathlineto{\pgfqpoint{3.968206in}{1.446318in}}%
\pgfpathlineto{\pgfqpoint{3.969375in}{1.321031in}}%
\pgfpathlineto{\pgfqpoint{3.969609in}{1.380430in}}%
\pgfpathlineto{\pgfqpoint{3.969843in}{1.355267in}}%
\pgfpathlineto{\pgfqpoint{3.970077in}{1.381164in}}%
\pgfpathlineto{\pgfqpoint{3.970311in}{1.379078in}}%
\pgfpathlineto{\pgfqpoint{3.971480in}{1.510067in}}%
\pgfpathlineto{\pgfqpoint{3.972182in}{1.465413in}}%
\pgfpathlineto{\pgfqpoint{3.972883in}{1.361156in}}%
\pgfpathlineto{\pgfqpoint{3.973351in}{1.406071in}}%
\pgfpathlineto{\pgfqpoint{3.973819in}{1.492009in}}%
\pgfpathlineto{\pgfqpoint{3.974520in}{1.438816in}}%
\pgfpathlineto{\pgfqpoint{3.975456in}{1.344576in}}%
\pgfpathlineto{\pgfqpoint{3.976157in}{1.382540in}}%
\pgfpathlineto{\pgfqpoint{3.977326in}{1.556311in}}%
\pgfpathlineto{\pgfqpoint{3.977560in}{1.528869in}}%
\pgfpathlineto{\pgfqpoint{3.978262in}{1.331111in}}%
\pgfpathlineto{\pgfqpoint{3.978963in}{1.409055in}}%
\pgfpathlineto{\pgfqpoint{3.979197in}{1.412208in}}%
\pgfpathlineto{\pgfqpoint{3.979431in}{1.474729in}}%
\pgfpathlineto{\pgfqpoint{3.980133in}{1.380054in}}%
\pgfpathlineto{\pgfqpoint{3.980367in}{1.375340in}}%
\pgfpathlineto{\pgfqpoint{3.980600in}{1.381958in}}%
\pgfpathlineto{\pgfqpoint{3.981068in}{1.412916in}}%
\pgfpathlineto{\pgfqpoint{3.982471in}{1.288348in}}%
\pgfpathlineto{\pgfqpoint{3.982939in}{1.308790in}}%
\pgfpathlineto{\pgfqpoint{3.984342in}{1.521706in}}%
\pgfpathlineto{\pgfqpoint{3.984576in}{1.465425in}}%
\pgfpathlineto{\pgfqpoint{3.985511in}{1.428159in}}%
\pgfpathlineto{\pgfqpoint{3.988084in}{1.186886in}}%
\pgfpathlineto{\pgfqpoint{3.988552in}{1.194885in}}%
\pgfpathlineto{\pgfqpoint{3.989253in}{1.342216in}}%
\pgfpathlineto{\pgfqpoint{3.989955in}{1.298820in}}%
\pgfpathlineto{\pgfqpoint{3.990423in}{1.233142in}}%
\pgfpathlineto{\pgfqpoint{3.990890in}{1.295140in}}%
\pgfpathlineto{\pgfqpoint{3.992060in}{1.472183in}}%
\pgfpathlineto{\pgfqpoint{3.992527in}{1.416051in}}%
\pgfpathlineto{\pgfqpoint{3.993229in}{1.491936in}}%
\pgfpathlineto{\pgfqpoint{3.993697in}{1.431929in}}%
\pgfpathlineto{\pgfqpoint{3.995567in}{1.196617in}}%
\pgfpathlineto{\pgfqpoint{3.996269in}{1.116275in}}%
\pgfpathlineto{\pgfqpoint{3.996503in}{1.168021in}}%
\pgfpathlineto{\pgfqpoint{3.998374in}{1.373535in}}%
\pgfpathlineto{\pgfqpoint{3.999075in}{1.332999in}}%
\pgfpathlineto{\pgfqpoint{3.999777in}{1.344293in}}%
\pgfpathlineto{\pgfqpoint{4.000946in}{1.448888in}}%
\pgfpathlineto{\pgfqpoint{4.001414in}{1.393140in}}%
\pgfpathlineto{\pgfqpoint{4.001882in}{1.303124in}}%
\pgfpathlineto{\pgfqpoint{4.002583in}{1.346939in}}%
\pgfpathlineto{\pgfqpoint{4.002817in}{1.356617in}}%
\pgfpathlineto{\pgfqpoint{4.003051in}{1.344946in}}%
\pgfpathlineto{\pgfqpoint{4.003986in}{1.172903in}}%
\pgfpathlineto{\pgfqpoint{4.004688in}{1.184204in}}%
\pgfpathlineto{\pgfqpoint{4.005156in}{1.239707in}}%
\pgfpathlineto{\pgfqpoint{4.005390in}{1.211829in}}%
\pgfpathlineto{\pgfqpoint{4.006091in}{1.088156in}}%
\pgfpathlineto{\pgfqpoint{4.006559in}{1.189799in}}%
\pgfpathlineto{\pgfqpoint{4.007260in}{1.350276in}}%
\pgfpathlineto{\pgfqpoint{4.007728in}{1.496774in}}%
\pgfpathlineto{\pgfqpoint{4.008430in}{1.408158in}}%
\pgfpathlineto{\pgfqpoint{4.009833in}{1.179190in}}%
\pgfpathlineto{\pgfqpoint{4.010067in}{1.207794in}}%
\pgfpathlineto{\pgfqpoint{4.010301in}{1.251005in}}%
\pgfpathlineto{\pgfqpoint{4.011002in}{1.192167in}}%
\pgfpathlineto{\pgfqpoint{4.012171in}{1.074293in}}%
\pgfpathlineto{\pgfqpoint{4.012405in}{1.103950in}}%
\pgfpathlineto{\pgfqpoint{4.014276in}{1.372605in}}%
\pgfpathlineto{\pgfqpoint{4.014978in}{1.284981in}}%
\pgfpathlineto{\pgfqpoint{4.015212in}{1.344329in}}%
\pgfpathlineto{\pgfqpoint{4.015445in}{1.371599in}}%
\pgfpathlineto{\pgfqpoint{4.015913in}{1.297477in}}%
\pgfpathlineto{\pgfqpoint{4.016147in}{1.301953in}}%
\pgfpathlineto{\pgfqpoint{4.016381in}{1.288545in}}%
\pgfpathlineto{\pgfqpoint{4.016849in}{1.242569in}}%
\pgfpathlineto{\pgfqpoint{4.017082in}{1.309640in}}%
\pgfpathlineto{\pgfqpoint{4.017784in}{1.338880in}}%
\pgfpathlineto{\pgfqpoint{4.018018in}{1.326770in}}%
\pgfpathlineto{\pgfqpoint{4.020123in}{1.034763in}}%
\pgfpathlineto{\pgfqpoint{4.020357in}{1.087097in}}%
\pgfpathlineto{\pgfqpoint{4.021526in}{1.375755in}}%
\pgfpathlineto{\pgfqpoint{4.021994in}{1.318725in}}%
\pgfpathlineto{\pgfqpoint{4.022461in}{1.274537in}}%
\pgfpathlineto{\pgfqpoint{4.022929in}{1.186432in}}%
\pgfpathlineto{\pgfqpoint{4.023864in}{1.215886in}}%
\pgfpathlineto{\pgfqpoint{4.024098in}{1.224463in}}%
\pgfpathlineto{\pgfqpoint{4.024332in}{1.200282in}}%
\pgfpathlineto{\pgfqpoint{4.025735in}{1.091353in}}%
\pgfpathlineto{\pgfqpoint{4.025969in}{1.099806in}}%
\pgfpathlineto{\pgfqpoint{4.027138in}{1.312656in}}%
\pgfpathlineto{\pgfqpoint{4.027840in}{1.232488in}}%
\pgfpathlineto{\pgfqpoint{4.028074in}{1.233105in}}%
\pgfpathlineto{\pgfqpoint{4.028775in}{1.144942in}}%
\pgfpathlineto{\pgfqpoint{4.029243in}{1.177744in}}%
\pgfpathlineto{\pgfqpoint{4.029711in}{1.229326in}}%
\pgfpathlineto{\pgfqpoint{4.030179in}{1.277701in}}%
\pgfpathlineto{\pgfqpoint{4.030412in}{1.254071in}}%
\pgfpathlineto{\pgfqpoint{4.031816in}{0.992876in}}%
\pgfpathlineto{\pgfqpoint{4.032049in}{1.001477in}}%
\pgfpathlineto{\pgfqpoint{4.034388in}{1.379561in}}%
\pgfpathlineto{\pgfqpoint{4.035090in}{1.310205in}}%
\pgfpathlineto{\pgfqpoint{4.037662in}{0.996332in}}%
\pgfpathlineto{\pgfqpoint{4.037896in}{0.997419in}}%
\pgfpathlineto{\pgfqpoint{4.039299in}{1.130347in}}%
\pgfpathlineto{\pgfqpoint{4.039533in}{1.115732in}}%
\pgfpathlineto{\pgfqpoint{4.040235in}{1.245510in}}%
\pgfpathlineto{\pgfqpoint{4.040936in}{1.207131in}}%
\pgfpathlineto{\pgfqpoint{4.041404in}{1.253633in}}%
\pgfpathlineto{\pgfqpoint{4.042105in}{1.231836in}}%
\pgfpathlineto{\pgfqpoint{4.046315in}{0.924547in}}%
\pgfpathlineto{\pgfqpoint{4.046549in}{0.950759in}}%
\pgfpathlineto{\pgfqpoint{4.047250in}{1.065716in}}%
\pgfpathlineto{\pgfqpoint{4.048420in}{1.259751in}}%
\pgfpathlineto{\pgfqpoint{4.048887in}{1.231523in}}%
\pgfpathlineto{\pgfqpoint{4.050524in}{1.128954in}}%
\pgfpathlineto{\pgfqpoint{4.052161in}{0.836835in}}%
\pgfpathlineto{\pgfqpoint{4.052395in}{0.877526in}}%
\pgfpathlineto{\pgfqpoint{4.055435in}{1.240163in}}%
\pgfpathlineto{\pgfqpoint{4.056137in}{1.149952in}}%
\pgfpathlineto{\pgfqpoint{4.057540in}{0.862448in}}%
\pgfpathlineto{\pgfqpoint{4.057774in}{0.893762in}}%
\pgfpathlineto{\pgfqpoint{4.058008in}{0.943275in}}%
\pgfpathlineto{\pgfqpoint{4.058709in}{0.859890in}}%
\pgfpathlineto{\pgfqpoint{4.058943in}{0.925903in}}%
\pgfpathlineto{\pgfqpoint{4.061282in}{1.240009in}}%
\pgfpathlineto{\pgfqpoint{4.061983in}{1.174531in}}%
\pgfpathlineto{\pgfqpoint{4.062685in}{1.049935in}}%
\pgfpathlineto{\pgfqpoint{4.064322in}{0.905874in}}%
\pgfpathlineto{\pgfqpoint{4.065491in}{0.861446in}}%
\pgfpathlineto{\pgfqpoint{4.067362in}{1.103942in}}%
\pgfpathlineto{\pgfqpoint{4.068064in}{1.174076in}}%
\pgfpathlineto{\pgfqpoint{4.068298in}{1.218549in}}%
\pgfpathlineto{\pgfqpoint{4.068532in}{1.159039in}}%
\pgfpathlineto{\pgfqpoint{4.069935in}{0.960234in}}%
\pgfpathlineto{\pgfqpoint{4.071104in}{1.002765in}}%
\pgfpathlineto{\pgfqpoint{4.071572in}{0.909608in}}%
\pgfpathlineto{\pgfqpoint{4.072273in}{0.982940in}}%
\pgfpathlineto{\pgfqpoint{4.072507in}{0.981228in}}%
\pgfpathlineto{\pgfqpoint{4.073209in}{0.886477in}}%
\pgfpathlineto{\pgfqpoint{4.073910in}{0.941301in}}%
\pgfpathlineto{\pgfqpoint{4.074144in}{0.944340in}}%
\pgfpathlineto{\pgfqpoint{4.075781in}{1.152969in}}%
\pgfpathlineto{\pgfqpoint{4.076015in}{1.129842in}}%
\pgfpathlineto{\pgfqpoint{4.076249in}{1.140050in}}%
\pgfpathlineto{\pgfqpoint{4.076483in}{1.126141in}}%
\pgfpathlineto{\pgfqpoint{4.078587in}{0.916455in}}%
\pgfpathlineto{\pgfqpoint{4.078821in}{0.940914in}}%
\pgfpathlineto{\pgfqpoint{4.079523in}{1.094466in}}%
\pgfpathlineto{\pgfqpoint{4.080224in}{1.048914in}}%
\pgfpathlineto{\pgfqpoint{4.080692in}{1.027410in}}%
\pgfpathlineto{\pgfqpoint{4.080926in}{1.039515in}}%
\pgfpathlineto{\pgfqpoint{4.081160in}{1.128338in}}%
\pgfpathlineto{\pgfqpoint{4.081862in}{1.043927in}}%
\pgfpathlineto{\pgfqpoint{4.082095in}{1.034933in}}%
\pgfpathlineto{\pgfqpoint{4.082563in}{0.897061in}}%
\pgfpathlineto{\pgfqpoint{4.083265in}{0.979071in}}%
\pgfpathlineto{\pgfqpoint{4.084902in}{1.187312in}}%
\pgfpathlineto{\pgfqpoint{4.085136in}{1.153580in}}%
\pgfpathlineto{\pgfqpoint{4.086539in}{0.953178in}}%
\pgfpathlineto{\pgfqpoint{4.087708in}{1.080856in}}%
\pgfpathlineto{\pgfqpoint{4.088176in}{1.067998in}}%
\pgfpathlineto{\pgfqpoint{4.088877in}{0.939491in}}%
\pgfpathlineto{\pgfqpoint{4.089111in}{1.011103in}}%
\pgfpathlineto{\pgfqpoint{4.090280in}{1.156994in}}%
\pgfpathlineto{\pgfqpoint{4.090514in}{1.112833in}}%
\pgfpathlineto{\pgfqpoint{4.090982in}{1.051760in}}%
\pgfpathlineto{\pgfqpoint{4.091450in}{1.095294in}}%
\pgfpathlineto{\pgfqpoint{4.091684in}{1.104418in}}%
\pgfpathlineto{\pgfqpoint{4.091917in}{1.090749in}}%
\pgfpathlineto{\pgfqpoint{4.094490in}{0.939557in}}%
\pgfpathlineto{\pgfqpoint{4.094724in}{0.972571in}}%
\pgfpathlineto{\pgfqpoint{4.094958in}{0.971140in}}%
\pgfpathlineto{\pgfqpoint{4.096829in}{1.170347in}}%
\pgfpathlineto{\pgfqpoint{4.097062in}{1.168622in}}%
\pgfpathlineto{\pgfqpoint{4.097998in}{1.261483in}}%
\pgfpathlineto{\pgfqpoint{4.098232in}{1.194918in}}%
\pgfpathlineto{\pgfqpoint{4.100336in}{1.006346in}}%
\pgfpathlineto{\pgfqpoint{4.100570in}{1.008098in}}%
\pgfpathlineto{\pgfqpoint{4.102441in}{1.136133in}}%
\pgfpathlineto{\pgfqpoint{4.102909in}{1.067535in}}%
\pgfpathlineto{\pgfqpoint{4.103377in}{1.119323in}}%
\pgfpathlineto{\pgfqpoint{4.104312in}{1.312979in}}%
\pgfpathlineto{\pgfqpoint{4.104780in}{1.244872in}}%
\pgfpathlineto{\pgfqpoint{4.105247in}{1.229121in}}%
\pgfpathlineto{\pgfqpoint{4.105481in}{1.133575in}}%
\pgfpathlineto{\pgfqpoint{4.106417in}{1.204870in}}%
\pgfpathlineto{\pgfqpoint{4.106884in}{1.194137in}}%
\pgfpathlineto{\pgfqpoint{4.108989in}{1.031316in}}%
\pgfpathlineto{\pgfqpoint{4.110860in}{1.216440in}}%
\pgfpathlineto{\pgfqpoint{4.111094in}{1.194377in}}%
\pgfpathlineto{\pgfqpoint{4.112029in}{1.114704in}}%
\pgfpathlineto{\pgfqpoint{4.112263in}{1.163248in}}%
\pgfpathlineto{\pgfqpoint{4.113900in}{1.388451in}}%
\pgfpathlineto{\pgfqpoint{4.114368in}{1.271860in}}%
\pgfpathlineto{\pgfqpoint{4.116005in}{1.033480in}}%
\pgfpathlineto{\pgfqpoint{4.116239in}{1.053124in}}%
\pgfpathlineto{\pgfqpoint{4.116473in}{1.049707in}}%
\pgfpathlineto{\pgfqpoint{4.119513in}{1.376840in}}%
\pgfpathlineto{\pgfqpoint{4.119747in}{1.316777in}}%
\pgfpathlineto{\pgfqpoint{4.119981in}{1.256200in}}%
\pgfpathlineto{\pgfqpoint{4.120682in}{1.315444in}}%
\pgfpathlineto{\pgfqpoint{4.120916in}{1.324494in}}%
\pgfpathlineto{\pgfqpoint{4.122319in}{1.229479in}}%
\pgfpathlineto{\pgfqpoint{4.122553in}{1.267017in}}%
\pgfpathlineto{\pgfqpoint{4.123021in}{1.243001in}}%
\pgfpathlineto{\pgfqpoint{4.123722in}{1.081878in}}%
\pgfpathlineto{\pgfqpoint{4.124424in}{1.111638in}}%
\pgfpathlineto{\pgfqpoint{4.125593in}{1.245957in}}%
\pgfpathlineto{\pgfqpoint{4.126529in}{1.416709in}}%
\pgfpathlineto{\pgfqpoint{4.126996in}{1.374476in}}%
\pgfpathlineto{\pgfqpoint{4.127230in}{1.372575in}}%
\pgfpathlineto{\pgfqpoint{4.128867in}{1.173977in}}%
\pgfpathlineto{\pgfqpoint{4.129569in}{1.202364in}}%
\pgfpathlineto{\pgfqpoint{4.130738in}{1.262995in}}%
\pgfpathlineto{\pgfqpoint{4.130972in}{1.262894in}}%
\pgfpathlineto{\pgfqpoint{4.131674in}{1.310421in}}%
\pgfpathlineto{\pgfqpoint{4.131907in}{1.255946in}}%
\pgfpathlineto{\pgfqpoint{4.133544in}{1.105015in}}%
\pgfpathlineto{\pgfqpoint{4.133778in}{1.142022in}}%
\pgfpathlineto{\pgfqpoint{4.136351in}{1.442961in}}%
\pgfpathlineto{\pgfqpoint{4.137988in}{1.361917in}}%
\pgfpathlineto{\pgfqpoint{4.138455in}{1.369280in}}%
\pgfpathlineto{\pgfqpoint{4.138689in}{1.289999in}}%
\pgfpathlineto{\pgfqpoint{4.139859in}{1.106293in}}%
\pgfpathlineto{\pgfqpoint{4.140092in}{1.153293in}}%
\pgfpathlineto{\pgfqpoint{4.142899in}{1.500909in}}%
\pgfpathlineto{\pgfqpoint{4.143367in}{1.469893in}}%
\pgfpathlineto{\pgfqpoint{4.145237in}{1.245544in}}%
\pgfpathlineto{\pgfqpoint{4.145471in}{1.247582in}}%
\pgfpathlineto{\pgfqpoint{4.146641in}{1.282894in}}%
\pgfpathlineto{\pgfqpoint{4.145939in}{1.237357in}}%
\pgfpathlineto{\pgfqpoint{4.146874in}{1.278190in}}%
\pgfpathlineto{\pgfqpoint{4.147108in}{1.266875in}}%
\pgfpathlineto{\pgfqpoint{4.147342in}{1.304851in}}%
\pgfpathlineto{\pgfqpoint{4.148278in}{1.361646in}}%
\pgfpathlineto{\pgfqpoint{4.148511in}{1.347539in}}%
\pgfpathlineto{\pgfqpoint{4.149447in}{1.360089in}}%
\pgfpathlineto{\pgfqpoint{4.150148in}{1.285199in}}%
\pgfpathlineto{\pgfqpoint{4.150382in}{1.338268in}}%
\pgfpathlineto{\pgfqpoint{4.151084in}{1.256950in}}%
\pgfpathlineto{\pgfqpoint{4.152487in}{1.466027in}}%
\pgfpathlineto{\pgfqpoint{4.154124in}{1.448342in}}%
\pgfpathlineto{\pgfqpoint{4.154358in}{1.446019in}}%
\pgfpathlineto{\pgfqpoint{4.154592in}{1.467091in}}%
\pgfpathlineto{\pgfqpoint{4.154826in}{1.430316in}}%
\pgfpathlineto{\pgfqpoint{4.155293in}{1.352643in}}%
\pgfpathlineto{\pgfqpoint{4.155995in}{1.399587in}}%
\pgfpathlineto{\pgfqpoint{4.156229in}{1.399977in}}%
\pgfpathlineto{\pgfqpoint{4.157632in}{1.203929in}}%
\pgfpathlineto{\pgfqpoint{4.157866in}{1.210446in}}%
\pgfpathlineto{\pgfqpoint{4.160438in}{1.488868in}}%
\pgfpathlineto{\pgfqpoint{4.160906in}{1.399925in}}%
\pgfpathlineto{\pgfqpoint{4.161140in}{1.314474in}}%
\pgfpathlineto{\pgfqpoint{4.161841in}{1.341506in}}%
\pgfpathlineto{\pgfqpoint{4.163245in}{1.440149in}}%
\pgfpathlineto{\pgfqpoint{4.164648in}{1.366082in}}%
\pgfpathlineto{\pgfqpoint{4.164882in}{1.397698in}}%
\pgfpathlineto{\pgfqpoint{4.165349in}{1.341489in}}%
\pgfpathlineto{\pgfqpoint{4.165583in}{1.347049in}}%
\pgfpathlineto{\pgfqpoint{4.166051in}{1.436437in}}%
\pgfpathlineto{\pgfqpoint{4.166752in}{1.561032in}}%
\pgfpathlineto{\pgfqpoint{4.167220in}{1.461605in}}%
\pgfpathlineto{\pgfqpoint{4.167922in}{1.386736in}}%
\pgfpathlineto{\pgfqpoint{4.168389in}{1.441412in}}%
\pgfpathlineto{\pgfqpoint{4.168623in}{1.432158in}}%
\pgfpathlineto{\pgfqpoint{4.168857in}{1.447582in}}%
\pgfpathlineto{\pgfqpoint{4.169091in}{1.474453in}}%
\pgfpathlineto{\pgfqpoint{4.169559in}{1.420911in}}%
\pgfpathlineto{\pgfqpoint{4.170728in}{1.315506in}}%
\pgfpathlineto{\pgfqpoint{4.171430in}{1.387367in}}%
\pgfpathlineto{\pgfqpoint{4.171897in}{1.365790in}}%
\pgfpathlineto{\pgfqpoint{4.172131in}{1.330903in}}%
\pgfpathlineto{\pgfqpoint{4.172365in}{1.394350in}}%
\pgfpathlineto{\pgfqpoint{4.172833in}{1.521799in}}%
\pgfpathlineto{\pgfqpoint{4.173534in}{1.430912in}}%
\pgfpathlineto{\pgfqpoint{4.173768in}{1.437413in}}%
\pgfpathlineto{\pgfqpoint{4.174002in}{1.430724in}}%
\pgfpathlineto{\pgfqpoint{4.174938in}{1.578029in}}%
\pgfpathlineto{\pgfqpoint{4.175405in}{1.509976in}}%
\pgfpathlineto{\pgfqpoint{4.175639in}{1.507154in}}%
\pgfpathlineto{\pgfqpoint{4.175873in}{1.520546in}}%
\pgfpathlineto{\pgfqpoint{4.177510in}{1.353596in}}%
\pgfpathlineto{\pgfqpoint{4.179615in}{1.483754in}}%
\pgfpathlineto{\pgfqpoint{4.180082in}{1.449647in}}%
\pgfpathlineto{\pgfqpoint{4.180316in}{1.478313in}}%
\pgfpathlineto{\pgfqpoint{4.181018in}{1.550937in}}%
\pgfpathlineto{\pgfqpoint{4.181486in}{1.507325in}}%
\pgfpathlineto{\pgfqpoint{4.182421in}{1.373137in}}%
\pgfpathlineto{\pgfqpoint{4.183123in}{1.384708in}}%
\pgfpathlineto{\pgfqpoint{4.183824in}{1.487858in}}%
\pgfpathlineto{\pgfqpoint{4.184526in}{1.457258in}}%
\pgfpathlineto{\pgfqpoint{4.185227in}{1.296404in}}%
\pgfpathlineto{\pgfqpoint{4.185695in}{1.381513in}}%
\pgfpathlineto{\pgfqpoint{4.187332in}{1.568875in}}%
\pgfpathlineto{\pgfqpoint{4.188267in}{1.499604in}}%
\pgfpathlineto{\pgfqpoint{4.189437in}{1.383221in}}%
\pgfpathlineto{\pgfqpoint{4.189671in}{1.392535in}}%
\pgfpathlineto{\pgfqpoint{4.190840in}{1.538307in}}%
\pgfpathlineto{\pgfqpoint{4.191074in}{1.523663in}}%
\pgfpathlineto{\pgfqpoint{4.192477in}{1.414177in}}%
\pgfpathlineto{\pgfqpoint{4.193179in}{1.453354in}}%
\pgfpathlineto{\pgfqpoint{4.193646in}{1.418282in}}%
\pgfpathlineto{\pgfqpoint{4.193880in}{1.416929in}}%
\pgfpathlineto{\pgfqpoint{4.194816in}{1.511201in}}%
\pgfpathlineto{\pgfqpoint{4.195049in}{1.484689in}}%
\pgfpathlineto{\pgfqpoint{4.196219in}{1.361452in}}%
\pgfpathlineto{\pgfqpoint{4.196453in}{1.389059in}}%
\pgfpathlineto{\pgfqpoint{4.197622in}{1.448525in}}%
\pgfpathlineto{\pgfqpoint{4.198791in}{1.352042in}}%
\pgfpathlineto{\pgfqpoint{4.199259in}{1.394919in}}%
\pgfpathlineto{\pgfqpoint{4.201364in}{1.592794in}}%
\pgfpathlineto{\pgfqpoint{4.201597in}{1.558781in}}%
\pgfpathlineto{\pgfqpoint{4.204872in}{1.278654in}}%
\pgfpathlineto{\pgfqpoint{4.207912in}{1.535011in}}%
\pgfpathlineto{\pgfqpoint{4.209081in}{1.419561in}}%
\pgfpathlineto{\pgfqpoint{4.209315in}{1.443523in}}%
\pgfpathlineto{\pgfqpoint{4.210016in}{1.608731in}}%
\pgfpathlineto{\pgfqpoint{4.210952in}{1.527868in}}%
\pgfpathlineto{\pgfqpoint{4.212589in}{1.204092in}}%
\pgfpathlineto{\pgfqpoint{4.213057in}{1.243943in}}%
\pgfpathlineto{\pgfqpoint{4.214460in}{1.526229in}}%
\pgfpathlineto{\pgfqpoint{4.214694in}{1.487860in}}%
\pgfpathlineto{\pgfqpoint{4.215161in}{1.480971in}}%
\pgfpathlineto{\pgfqpoint{4.215395in}{1.464159in}}%
\pgfpathlineto{\pgfqpoint{4.215629in}{1.506446in}}%
\pgfpathlineto{\pgfqpoint{4.215863in}{1.519923in}}%
\pgfpathlineto{\pgfqpoint{4.216097in}{1.482508in}}%
\pgfpathlineto{\pgfqpoint{4.216331in}{1.482219in}}%
\pgfpathlineto{\pgfqpoint{4.216798in}{1.536903in}}%
\pgfpathlineto{\pgfqpoint{4.217032in}{1.484002in}}%
\pgfpathlineto{\pgfqpoint{4.217968in}{1.338624in}}%
\pgfpathlineto{\pgfqpoint{4.218435in}{1.270208in}}%
\pgfpathlineto{\pgfqpoint{4.219137in}{1.275248in}}%
\pgfpathlineto{\pgfqpoint{4.220540in}{1.361553in}}%
\pgfpathlineto{\pgfqpoint{4.221008in}{1.338685in}}%
\pgfpathlineto{\pgfqpoint{4.221476in}{1.311247in}}%
\pgfpathlineto{\pgfqpoint{4.221709in}{1.339237in}}%
\pgfpathlineto{\pgfqpoint{4.221943in}{1.333671in}}%
\pgfpathlineto{\pgfqpoint{4.222879in}{1.464210in}}%
\pgfpathlineto{\pgfqpoint{4.223346in}{1.416640in}}%
\pgfpathlineto{\pgfqpoint{4.223580in}{1.392061in}}%
\pgfpathlineto{\pgfqpoint{4.223814in}{1.449991in}}%
\pgfpathlineto{\pgfqpoint{4.224750in}{1.535437in}}%
\pgfpathlineto{\pgfqpoint{4.224983in}{1.515621in}}%
\pgfpathlineto{\pgfqpoint{4.226854in}{1.270677in}}%
\pgfpathlineto{\pgfqpoint{4.227790in}{1.200507in}}%
\pgfpathlineto{\pgfqpoint{4.228024in}{1.248576in}}%
\pgfpathlineto{\pgfqpoint{4.228257in}{1.249651in}}%
\pgfpathlineto{\pgfqpoint{4.228725in}{1.197673in}}%
\pgfpathlineto{\pgfqpoint{4.228959in}{1.174420in}}%
\pgfpathlineto{\pgfqpoint{4.230596in}{1.505553in}}%
\pgfpathlineto{\pgfqpoint{4.230830in}{1.494684in}}%
\pgfpathlineto{\pgfqpoint{4.231298in}{1.522230in}}%
\pgfpathlineto{\pgfqpoint{4.231531in}{1.476805in}}%
\pgfpathlineto{\pgfqpoint{4.232701in}{1.305999in}}%
\pgfpathlineto{\pgfqpoint{4.234104in}{1.159262in}}%
\pgfpathlineto{\pgfqpoint{4.234338in}{1.211805in}}%
\pgfpathlineto{\pgfqpoint{4.236443in}{1.386098in}}%
\pgfpathlineto{\pgfqpoint{4.236676in}{1.371274in}}%
\pgfpathlineto{\pgfqpoint{4.237378in}{1.393191in}}%
\pgfpathlineto{\pgfqpoint{4.237612in}{1.376932in}}%
\pgfpathlineto{\pgfqpoint{4.237846in}{1.383627in}}%
\pgfpathlineto{\pgfqpoint{4.238080in}{1.348990in}}%
\pgfpathlineto{\pgfqpoint{4.238313in}{1.388357in}}%
\pgfpathlineto{\pgfqpoint{4.238781in}{1.377997in}}%
\pgfpathlineto{\pgfqpoint{4.239483in}{1.471875in}}%
\pgfpathlineto{\pgfqpoint{4.239717in}{1.424098in}}%
\pgfpathlineto{\pgfqpoint{4.241587in}{1.279149in}}%
\pgfpathlineto{\pgfqpoint{4.242523in}{1.152150in}}%
\pgfpathlineto{\pgfqpoint{4.242991in}{1.173820in}}%
\pgfpathlineto{\pgfqpoint{4.243926in}{1.293482in}}%
\pgfpathlineto{\pgfqpoint{4.244160in}{1.234290in}}%
\pgfpathlineto{\pgfqpoint{4.244628in}{1.282381in}}%
\pgfpathlineto{\pgfqpoint{4.245797in}{1.397563in}}%
\pgfpathlineto{\pgfqpoint{4.246031in}{1.350912in}}%
\pgfpathlineto{\pgfqpoint{4.247200in}{1.475735in}}%
\pgfpathlineto{\pgfqpoint{4.247434in}{1.430522in}}%
\pgfpathlineto{\pgfqpoint{4.249305in}{1.144939in}}%
\pgfpathlineto{\pgfqpoint{4.249539in}{1.148959in}}%
\pgfpathlineto{\pgfqpoint{4.250474in}{1.230621in}}%
\pgfpathlineto{\pgfqpoint{4.250708in}{1.192971in}}%
\pgfpathlineto{\pgfqpoint{4.251643in}{1.116029in}}%
\pgfpathlineto{\pgfqpoint{4.251877in}{1.147604in}}%
\pgfpathlineto{\pgfqpoint{4.253748in}{1.489520in}}%
\pgfpathlineto{\pgfqpoint{4.253982in}{1.462958in}}%
\pgfpathlineto{\pgfqpoint{4.255151in}{0.990769in}}%
\pgfpathlineto{\pgfqpoint{4.255853in}{1.059824in}}%
\pgfpathlineto{\pgfqpoint{4.257490in}{1.457972in}}%
\pgfpathlineto{\pgfqpoint{4.258191in}{1.380354in}}%
\pgfpathlineto{\pgfqpoint{4.260062in}{1.001958in}}%
\pgfpathlineto{\pgfqpoint{4.260296in}{1.033265in}}%
\pgfpathlineto{\pgfqpoint{4.260530in}{1.035228in}}%
\pgfpathlineto{\pgfqpoint{4.262167in}{1.370884in}}%
\pgfpathlineto{\pgfqpoint{4.262869in}{1.318728in}}%
\pgfpathlineto{\pgfqpoint{4.265441in}{0.977703in}}%
\pgfpathlineto{\pgfqpoint{4.265909in}{0.982460in}}%
\pgfpathlineto{\pgfqpoint{4.266376in}{1.013388in}}%
\pgfpathlineto{\pgfqpoint{4.266844in}{1.109076in}}%
\pgfpathlineto{\pgfqpoint{4.268247in}{1.335149in}}%
\pgfpathlineto{\pgfqpoint{4.271989in}{0.940108in}}%
\pgfpathlineto{\pgfqpoint{4.272223in}{0.949929in}}%
\pgfpathlineto{\pgfqpoint{4.272457in}{0.958146in}}%
\pgfpathlineto{\pgfqpoint{4.274795in}{1.274523in}}%
\pgfpathlineto{\pgfqpoint{4.275497in}{1.158519in}}%
\pgfpathlineto{\pgfqpoint{4.277134in}{0.999258in}}%
\pgfpathlineto{\pgfqpoint{4.275965in}{1.178215in}}%
\pgfpathlineto{\pgfqpoint{4.277368in}{1.032640in}}%
\pgfpathlineto{\pgfqpoint{4.279005in}{1.155390in}}%
\pgfpathlineto{\pgfqpoint{4.279940in}{1.009250in}}%
\pgfpathlineto{\pgfqpoint{4.280408in}{1.072486in}}%
\pgfpathlineto{\pgfqpoint{4.281110in}{0.998737in}}%
\pgfpathlineto{\pgfqpoint{4.281577in}{1.056353in}}%
\pgfpathlineto{\pgfqpoint{4.283214in}{1.139926in}}%
\pgfpathlineto{\pgfqpoint{4.283916in}{1.111001in}}%
\pgfpathlineto{\pgfqpoint{4.284851in}{1.049247in}}%
\pgfpathlineto{\pgfqpoint{4.285319in}{1.065572in}}%
\pgfpathlineto{\pgfqpoint{4.285553in}{1.069317in}}%
\pgfpathlineto{\pgfqpoint{4.286956in}{1.173852in}}%
\pgfpathlineto{\pgfqpoint{4.288359in}{0.910753in}}%
\pgfpathlineto{\pgfqpoint{4.289295in}{0.976934in}}%
\pgfpathlineto{\pgfqpoint{4.289996in}{0.975638in}}%
\pgfpathlineto{\pgfqpoint{4.291166in}{1.088851in}}%
\pgfpathlineto{\pgfqpoint{4.291399in}{1.086320in}}%
\pgfpathlineto{\pgfqpoint{4.292335in}{1.026587in}}%
\pgfpathlineto{\pgfqpoint{4.292569in}{1.052027in}}%
\pgfpathlineto{\pgfqpoint{4.293738in}{1.104535in}}%
\pgfpathlineto{\pgfqpoint{4.293972in}{1.072874in}}%
\pgfpathlineto{\pgfqpoint{4.294206in}{1.068162in}}%
\pgfpathlineto{\pgfqpoint{4.295609in}{0.943484in}}%
\pgfpathlineto{\pgfqpoint{4.297012in}{1.019573in}}%
\pgfpathlineto{\pgfqpoint{4.297246in}{0.969141in}}%
\pgfpathlineto{\pgfqpoint{4.297948in}{1.056762in}}%
\pgfpathlineto{\pgfqpoint{4.298649in}{1.195052in}}%
\pgfpathlineto{\pgfqpoint{4.299351in}{1.122646in}}%
\pgfpathlineto{\pgfqpoint{4.300286in}{0.956884in}}%
\pgfpathlineto{\pgfqpoint{4.300754in}{1.040225in}}%
\pgfpathlineto{\pgfqpoint{4.300988in}{1.032712in}}%
\pgfpathlineto{\pgfqpoint{4.301923in}{0.884161in}}%
\pgfpathlineto{\pgfqpoint{4.302625in}{0.913980in}}%
\pgfpathlineto{\pgfqpoint{4.302859in}{0.917103in}}%
\pgfpathlineto{\pgfqpoint{4.303794in}{0.989748in}}%
\pgfpathlineto{\pgfqpoint{4.304028in}{0.959868in}}%
\pgfpathlineto{\pgfqpoint{4.304262in}{0.921771in}}%
\pgfpathlineto{\pgfqpoint{4.304729in}{1.021610in}}%
\pgfpathlineto{\pgfqpoint{4.305665in}{1.158776in}}%
\pgfpathlineto{\pgfqpoint{4.306366in}{1.124042in}}%
\pgfpathlineto{\pgfqpoint{4.306600in}{1.134490in}}%
\pgfpathlineto{\pgfqpoint{4.306834in}{1.109548in}}%
\pgfpathlineto{\pgfqpoint{4.307068in}{1.120010in}}%
\pgfpathlineto{\pgfqpoint{4.308471in}{1.053237in}}%
\pgfpathlineto{\pgfqpoint{4.308705in}{1.038635in}}%
\pgfpathlineto{\pgfqpoint{4.309640in}{0.804323in}}%
\pgfpathlineto{\pgfqpoint{4.310108in}{0.864877in}}%
\pgfpathlineto{\pgfqpoint{4.312213in}{1.115613in}}%
\pgfpathlineto{\pgfqpoint{4.312681in}{1.129192in}}%
\pgfpathlineto{\pgfqpoint{4.313148in}{1.083530in}}%
\pgfpathlineto{\pgfqpoint{4.313616in}{1.131891in}}%
\pgfpathlineto{\pgfqpoint{4.313850in}{1.129290in}}%
\pgfpathlineto{\pgfqpoint{4.314552in}{0.982651in}}%
\pgfpathlineto{\pgfqpoint{4.315487in}{0.988276in}}%
\pgfpathlineto{\pgfqpoint{4.316890in}{0.891473in}}%
\pgfpathlineto{\pgfqpoint{4.317124in}{0.931470in}}%
\pgfpathlineto{\pgfqpoint{4.318293in}{0.994015in}}%
\pgfpathlineto{\pgfqpoint{4.318995in}{0.984484in}}%
\pgfpathlineto{\pgfqpoint{4.319229in}{1.035514in}}%
\pgfpathlineto{\pgfqpoint{4.319930in}{0.975258in}}%
\pgfpathlineto{\pgfqpoint{4.320164in}{0.998090in}}%
\pgfpathlineto{\pgfqpoint{4.320398in}{0.971702in}}%
\pgfpathlineto{\pgfqpoint{4.320866in}{1.030183in}}%
\pgfpathlineto{\pgfqpoint{4.322035in}{1.251346in}}%
\pgfpathlineto{\pgfqpoint{4.322503in}{1.143255in}}%
\pgfpathlineto{\pgfqpoint{4.324607in}{0.984051in}}%
\pgfpathlineto{\pgfqpoint{4.324841in}{0.986822in}}%
\pgfpathlineto{\pgfqpoint{4.325075in}{0.956305in}}%
\pgfpathlineto{\pgfqpoint{4.325309in}{0.991307in}}%
\pgfpathlineto{\pgfqpoint{4.325543in}{0.987548in}}%
\pgfpathlineto{\pgfqpoint{4.325777in}{1.046171in}}%
\pgfpathlineto{\pgfqpoint{4.326478in}{0.948515in}}%
\pgfpathlineto{\pgfqpoint{4.326712in}{0.954567in}}%
\pgfpathlineto{\pgfqpoint{4.326946in}{0.952760in}}%
\pgfpathlineto{\pgfqpoint{4.327414in}{1.066331in}}%
\pgfpathlineto{\pgfqpoint{4.328349in}{1.039845in}}%
\pgfpathlineto{\pgfqpoint{4.329285in}{1.099282in}}%
\pgfpathlineto{\pgfqpoint{4.329986in}{1.206469in}}%
\pgfpathlineto{\pgfqpoint{4.330454in}{1.171943in}}%
\pgfpathlineto{\pgfqpoint{4.331389in}{1.063807in}}%
\pgfpathlineto{\pgfqpoint{4.331623in}{1.091901in}}%
\pgfpathlineto{\pgfqpoint{4.332091in}{1.170255in}}%
\pgfpathlineto{\pgfqpoint{4.332559in}{1.093939in}}%
\pgfpathlineto{\pgfqpoint{4.332793in}{1.052802in}}%
\pgfpathlineto{\pgfqpoint{4.333260in}{1.141997in}}%
\pgfpathlineto{\pgfqpoint{4.333962in}{1.148988in}}%
\pgfpathlineto{\pgfqpoint{4.334663in}{1.068814in}}%
\pgfpathlineto{\pgfqpoint{4.335131in}{1.134282in}}%
\pgfpathlineto{\pgfqpoint{4.335599in}{1.134637in}}%
\pgfpathlineto{\pgfqpoint{4.335833in}{1.123876in}}%
\pgfpathlineto{\pgfqpoint{4.336768in}{0.860993in}}%
\pgfpathlineto{\pgfqpoint{4.337236in}{0.913061in}}%
\pgfpathlineto{\pgfqpoint{4.339107in}{1.298353in}}%
\pgfpathlineto{\pgfqpoint{4.339341in}{1.293402in}}%
\pgfpathlineto{\pgfqpoint{4.340276in}{1.012483in}}%
\pgfpathlineto{\pgfqpoint{4.341211in}{1.068794in}}%
\pgfpathlineto{\pgfqpoint{4.341445in}{1.068835in}}%
\pgfpathlineto{\pgfqpoint{4.341679in}{1.071358in}}%
\pgfpathlineto{\pgfqpoint{4.341913in}{1.051440in}}%
\pgfpathlineto{\pgfqpoint{4.342848in}{1.054462in}}%
\pgfpathlineto{\pgfqpoint{4.343082in}{1.047644in}}%
\pgfpathlineto{\pgfqpoint{4.343316in}{1.067682in}}%
\pgfpathlineto{\pgfqpoint{4.343550in}{1.056645in}}%
\pgfpathlineto{\pgfqpoint{4.344018in}{1.044907in}}%
\pgfpathlineto{\pgfqpoint{4.344953in}{1.171003in}}%
\pgfpathlineto{\pgfqpoint{4.345187in}{1.235516in}}%
\pgfpathlineto{\pgfqpoint{4.345889in}{1.195559in}}%
\pgfpathlineto{\pgfqpoint{4.347526in}{1.012485in}}%
\pgfpathlineto{\pgfqpoint{4.349163in}{1.280466in}}%
\pgfpathlineto{\pgfqpoint{4.349630in}{1.262929in}}%
\pgfpathlineto{\pgfqpoint{4.350566in}{1.164833in}}%
\pgfpathlineto{\pgfqpoint{4.351034in}{1.174435in}}%
\pgfpathlineto{\pgfqpoint{4.351267in}{1.189410in}}%
\pgfpathlineto{\pgfqpoint{4.351501in}{1.188456in}}%
\pgfpathlineto{\pgfqpoint{4.353138in}{1.075922in}}%
\pgfpathlineto{\pgfqpoint{4.353372in}{1.071882in}}%
\pgfpathlineto{\pgfqpoint{4.353606in}{1.081211in}}%
\pgfpathlineto{\pgfqpoint{4.354541in}{1.198855in}}%
\pgfpathlineto{\pgfqpoint{4.354775in}{1.165293in}}%
\pgfpathlineto{\pgfqpoint{4.355477in}{1.118636in}}%
\pgfpathlineto{\pgfqpoint{4.355945in}{1.151106in}}%
\pgfpathlineto{\pgfqpoint{4.356178in}{1.188023in}}%
\pgfpathlineto{\pgfqpoint{4.356880in}{1.131589in}}%
\pgfpathlineto{\pgfqpoint{4.358283in}{1.056035in}}%
\pgfpathlineto{\pgfqpoint{4.359920in}{1.235845in}}%
\pgfpathlineto{\pgfqpoint{4.360388in}{1.198093in}}%
\pgfpathlineto{\pgfqpoint{4.361323in}{1.161203in}}%
\pgfpathlineto{\pgfqpoint{4.362025in}{1.161862in}}%
\pgfpathlineto{\pgfqpoint{4.363896in}{1.270003in}}%
\pgfpathlineto{\pgfqpoint{4.365299in}{0.972426in}}%
\pgfpathlineto{\pgfqpoint{4.366001in}{1.025094in}}%
\pgfpathlineto{\pgfqpoint{4.366702in}{1.173560in}}%
\pgfpathlineto{\pgfqpoint{4.367170in}{1.150974in}}%
\pgfpathlineto{\pgfqpoint{4.367404in}{1.117423in}}%
\pgfpathlineto{\pgfqpoint{4.367871in}{1.156003in}}%
\pgfpathlineto{\pgfqpoint{4.369041in}{1.272627in}}%
\pgfpathlineto{\pgfqpoint{4.369742in}{1.251511in}}%
\pgfpathlineto{\pgfqpoint{4.369976in}{1.255462in}}%
\pgfpathlineto{\pgfqpoint{4.370912in}{1.394885in}}%
\pgfpathlineto{\pgfqpoint{4.371379in}{1.296871in}}%
\pgfpathlineto{\pgfqpoint{4.373484in}{1.050428in}}%
\pgfpathlineto{\pgfqpoint{4.374186in}{1.116977in}}%
\pgfpathlineto{\pgfqpoint{4.376057in}{1.243901in}}%
\pgfpathlineto{\pgfqpoint{4.376290in}{1.197803in}}%
\pgfpathlineto{\pgfqpoint{4.376758in}{1.228747in}}%
\pgfpathlineto{\pgfqpoint{4.377694in}{1.401857in}}%
\pgfpathlineto{\pgfqpoint{4.378395in}{1.374564in}}%
\pgfpathlineto{\pgfqpoint{4.379331in}{1.199492in}}%
\pgfpathlineto{\pgfqpoint{4.380032in}{1.252207in}}%
\pgfpathlineto{\pgfqpoint{4.380968in}{1.167772in}}%
\pgfpathlineto{\pgfqpoint{4.381201in}{1.211372in}}%
\pgfpathlineto{\pgfqpoint{4.381435in}{1.246013in}}%
\pgfpathlineto{\pgfqpoint{4.381669in}{1.208787in}}%
\pgfpathlineto{\pgfqpoint{4.382371in}{1.143041in}}%
\pgfpathlineto{\pgfqpoint{4.382838in}{1.147469in}}%
\pgfpathlineto{\pgfqpoint{4.384943in}{1.340338in}}%
\pgfpathlineto{\pgfqpoint{4.385645in}{1.309480in}}%
\pgfpathlineto{\pgfqpoint{4.386112in}{1.277506in}}%
\pgfpathlineto{\pgfqpoint{4.387048in}{1.182194in}}%
\pgfpathlineto{\pgfqpoint{4.387282in}{1.196079in}}%
\pgfpathlineto{\pgfqpoint{4.388217in}{1.309742in}}%
\pgfpathlineto{\pgfqpoint{4.388685in}{1.273826in}}%
\pgfpathlineto{\pgfqpoint{4.388919in}{1.282547in}}%
\pgfpathlineto{\pgfqpoint{4.390556in}{1.431338in}}%
\pgfpathlineto{\pgfqpoint{4.392894in}{1.094838in}}%
\pgfpathlineto{\pgfqpoint{4.393830in}{1.188012in}}%
\pgfpathlineto{\pgfqpoint{4.396168in}{1.475049in}}%
\pgfpathlineto{\pgfqpoint{4.396402in}{1.457012in}}%
\pgfpathlineto{\pgfqpoint{4.398039in}{1.247081in}}%
\pgfpathlineto{\pgfqpoint{4.398273in}{1.254396in}}%
\pgfpathlineto{\pgfqpoint{4.399209in}{1.339427in}}%
\pgfpathlineto{\pgfqpoint{4.399676in}{1.333104in}}%
\pgfpathlineto{\pgfqpoint{4.400846in}{1.203400in}}%
\pgfpathlineto{\pgfqpoint{4.401313in}{1.236224in}}%
\pgfpathlineto{\pgfqpoint{4.402950in}{1.397671in}}%
\pgfpathlineto{\pgfqpoint{4.403184in}{1.378129in}}%
\pgfpathlineto{\pgfqpoint{4.405055in}{1.492644in}}%
\pgfpathlineto{\pgfqpoint{4.405289in}{1.452881in}}%
\pgfpathlineto{\pgfqpoint{4.408095in}{1.238169in}}%
\pgfpathlineto{\pgfqpoint{4.408563in}{1.308443in}}%
\pgfpathlineto{\pgfqpoint{4.409265in}{1.401446in}}%
\pgfpathlineto{\pgfqpoint{4.409732in}{1.321640in}}%
\pgfpathlineto{\pgfqpoint{4.409966in}{1.313414in}}%
\pgfpathlineto{\pgfqpoint{4.410200in}{1.338857in}}%
\pgfpathlineto{\pgfqpoint{4.410434in}{1.418403in}}%
\pgfpathlineto{\pgfqpoint{4.411369in}{1.409578in}}%
\pgfpathlineto{\pgfqpoint{4.411603in}{1.392955in}}%
\pgfpathlineto{\pgfqpoint{4.411837in}{1.416716in}}%
\pgfpathlineto{\pgfqpoint{4.412772in}{1.481991in}}%
\pgfpathlineto{\pgfqpoint{4.413006in}{1.440520in}}%
\pgfpathlineto{\pgfqpoint{4.413240in}{1.438475in}}%
\pgfpathlineto{\pgfqpoint{4.413708in}{1.509653in}}%
\pgfpathlineto{\pgfqpoint{4.414643in}{1.496796in}}%
\pgfpathlineto{\pgfqpoint{4.416748in}{1.214333in}}%
\pgfpathlineto{\pgfqpoint{4.419087in}{1.508801in}}%
\pgfpathlineto{\pgfqpoint{4.420256in}{1.426927in}}%
\pgfpathlineto{\pgfqpoint{4.420724in}{1.449794in}}%
\pgfpathlineto{\pgfqpoint{4.421191in}{1.578950in}}%
\pgfpathlineto{\pgfqpoint{4.421893in}{1.544606in}}%
\pgfpathlineto{\pgfqpoint{4.424933in}{1.301428in}}%
\pgfpathlineto{\pgfqpoint{4.426570in}{1.427859in}}%
\pgfpathlineto{\pgfqpoint{4.426804in}{1.412059in}}%
\pgfpathlineto{\pgfqpoint{4.427272in}{1.452718in}}%
\pgfpathlineto{\pgfqpoint{4.427506in}{1.460152in}}%
\pgfpathlineto{\pgfqpoint{4.428441in}{1.618922in}}%
\pgfpathlineto{\pgfqpoint{4.428675in}{1.602636in}}%
\pgfpathlineto{\pgfqpoint{4.430312in}{1.311332in}}%
\pgfpathlineto{\pgfqpoint{4.430780in}{1.343937in}}%
\pgfpathlineto{\pgfqpoint{4.431013in}{1.340466in}}%
\pgfpathlineto{\pgfqpoint{4.431247in}{1.299238in}}%
\pgfpathlineto{\pgfqpoint{4.431715in}{1.331609in}}%
\pgfpathlineto{\pgfqpoint{4.432884in}{1.457687in}}%
\pgfpathlineto{\pgfqpoint{4.433118in}{1.448367in}}%
\pgfpathlineto{\pgfqpoint{4.433586in}{1.448312in}}%
\pgfpathlineto{\pgfqpoint{4.433820in}{1.451570in}}%
\pgfpathlineto{\pgfqpoint{4.434755in}{1.531479in}}%
\pgfpathlineto{\pgfqpoint{4.434989in}{1.521512in}}%
\pgfpathlineto{\pgfqpoint{4.436626in}{1.373124in}}%
\pgfpathlineto{\pgfqpoint{4.437328in}{1.309585in}}%
\pgfpathlineto{\pgfqpoint{4.437562in}{1.345506in}}%
\pgfpathlineto{\pgfqpoint{4.438497in}{1.540359in}}%
\pgfpathlineto{\pgfqpoint{4.439432in}{1.524034in}}%
\pgfpathlineto{\pgfqpoint{4.439666in}{1.533273in}}%
\pgfpathlineto{\pgfqpoint{4.440134in}{1.450485in}}%
\pgfpathlineto{\pgfqpoint{4.440836in}{1.519677in}}%
\pgfpathlineto{\pgfqpoint{4.441303in}{1.520460in}}%
\pgfpathlineto{\pgfqpoint{4.443174in}{1.296982in}}%
\pgfpathlineto{\pgfqpoint{4.443642in}{1.246073in}}%
\pgfpathlineto{\pgfqpoint{4.444110in}{1.287196in}}%
\pgfpathlineto{\pgfqpoint{4.445980in}{1.625236in}}%
\pgfpathlineto{\pgfqpoint{4.448553in}{1.347178in}}%
\pgfpathlineto{\pgfqpoint{4.449021in}{1.363955in}}%
\pgfpathlineto{\pgfqpoint{4.449722in}{1.410261in}}%
\pgfpathlineto{\pgfqpoint{4.450190in}{1.378828in}}%
\pgfpathlineto{\pgfqpoint{4.450424in}{1.374246in}}%
\pgfpathlineto{\pgfqpoint{4.451125in}{1.422156in}}%
\pgfpathlineto{\pgfqpoint{4.451359in}{1.373122in}}%
\pgfpathlineto{\pgfqpoint{4.451593in}{1.368284in}}%
\pgfpathlineto{\pgfqpoint{4.451827in}{1.340144in}}%
\pgfpathlineto{\pgfqpoint{4.452528in}{1.379641in}}%
\pgfpathlineto{\pgfqpoint{4.453698in}{1.594788in}}%
\pgfpathlineto{\pgfqpoint{4.454633in}{1.479560in}}%
\pgfpathlineto{\pgfqpoint{4.455335in}{1.504056in}}%
\pgfpathlineto{\pgfqpoint{4.455569in}{1.476480in}}%
\pgfpathlineto{\pgfqpoint{4.458141in}{1.221256in}}%
\pgfpathlineto{\pgfqpoint{4.458609in}{1.328353in}}%
\pgfpathlineto{\pgfqpoint{4.459310in}{1.385703in}}%
\pgfpathlineto{\pgfqpoint{4.459544in}{1.371676in}}%
\pgfpathlineto{\pgfqpoint{4.460714in}{1.582267in}}%
\pgfpathlineto{\pgfqpoint{4.461181in}{1.492826in}}%
\pgfpathlineto{\pgfqpoint{4.461415in}{1.483077in}}%
\pgfpathlineto{\pgfqpoint{4.461649in}{1.525608in}}%
\pgfpathlineto{\pgfqpoint{4.461883in}{1.481903in}}%
\pgfpathlineto{\pgfqpoint{4.462351in}{1.523446in}}%
\pgfpathlineto{\pgfqpoint{4.464221in}{1.229485in}}%
\pgfpathlineto{\pgfqpoint{4.465157in}{1.268339in}}%
\pgfpathlineto{\pgfqpoint{4.467028in}{1.589649in}}%
\pgfpathlineto{\pgfqpoint{4.467729in}{1.478362in}}%
\pgfpathlineto{\pgfqpoint{4.469834in}{1.228761in}}%
\pgfpathlineto{\pgfqpoint{4.470302in}{1.108454in}}%
\pgfpathlineto{\pgfqpoint{4.470770in}{1.273268in}}%
\pgfpathlineto{\pgfqpoint{4.472173in}{1.492501in}}%
\pgfpathlineto{\pgfqpoint{4.472407in}{1.480642in}}%
\pgfpathlineto{\pgfqpoint{4.474044in}{1.397436in}}%
\pgfpathlineto{\pgfqpoint{4.475213in}{1.083962in}}%
\pgfpathlineto{\pgfqpoint{4.475447in}{1.122962in}}%
\pgfpathlineto{\pgfqpoint{4.477785in}{1.493203in}}%
\pgfpathlineto{\pgfqpoint{4.479188in}{1.297576in}}%
\pgfpathlineto{\pgfqpoint{4.480358in}{1.037541in}}%
\pgfpathlineto{\pgfqpoint{4.480592in}{1.091172in}}%
\pgfpathlineto{\pgfqpoint{4.481995in}{1.388096in}}%
\pgfpathlineto{\pgfqpoint{4.482930in}{1.375115in}}%
\pgfpathlineto{\pgfqpoint{4.483632in}{1.405507in}}%
\pgfpathlineto{\pgfqpoint{4.485970in}{1.133680in}}%
\pgfpathlineto{\pgfqpoint{4.487607in}{1.292251in}}%
\pgfpathlineto{\pgfqpoint{4.490180in}{1.156632in}}%
\pgfpathlineto{\pgfqpoint{4.490414in}{1.164463in}}%
\pgfpathlineto{\pgfqpoint{4.492051in}{1.315137in}}%
\pgfpathlineto{\pgfqpoint{4.492518in}{1.263741in}}%
\pgfpathlineto{\pgfqpoint{4.493220in}{1.140713in}}%
\pgfpathlineto{\pgfqpoint{4.493688in}{1.181001in}}%
\pgfpathlineto{\pgfqpoint{4.494155in}{1.298197in}}%
\pgfpathlineto{\pgfqpoint{4.494857in}{1.209422in}}%
\pgfpathlineto{\pgfqpoint{4.495792in}{1.064938in}}%
\pgfpathlineto{\pgfqpoint{4.496026in}{1.078419in}}%
\pgfpathlineto{\pgfqpoint{4.496494in}{1.172151in}}%
\pgfpathlineto{\pgfqpoint{4.497429in}{1.350703in}}%
\pgfpathlineto{\pgfqpoint{4.497897in}{1.326792in}}%
\pgfpathlineto{\pgfqpoint{4.498833in}{1.151601in}}%
\pgfpathlineto{\pgfqpoint{4.499300in}{1.199673in}}%
\pgfpathlineto{\pgfqpoint{4.500236in}{1.250470in}}%
\pgfpathlineto{\pgfqpoint{4.501873in}{1.067230in}}%
\pgfpathlineto{\pgfqpoint{4.502107in}{1.086318in}}%
\pgfpathlineto{\pgfqpoint{4.502574in}{1.055773in}}%
\pgfpathlineto{\pgfqpoint{4.502808in}{1.039926in}}%
\pgfpathlineto{\pgfqpoint{4.503042in}{1.096870in}}%
\pgfpathlineto{\pgfqpoint{4.503276in}{1.072283in}}%
\pgfpathlineto{\pgfqpoint{4.503510in}{1.095522in}}%
\pgfpathlineto{\pgfqpoint{4.503978in}{1.054236in}}%
\pgfpathlineto{\pgfqpoint{4.504211in}{1.054588in}}%
\pgfpathlineto{\pgfqpoint{4.506082in}{1.324973in}}%
\pgfpathlineto{\pgfqpoint{4.506550in}{1.273108in}}%
\pgfpathlineto{\pgfqpoint{4.507719in}{1.156645in}}%
\pgfpathlineto{\pgfqpoint{4.507953in}{1.171235in}}%
\pgfpathlineto{\pgfqpoint{4.508655in}{1.183699in}}%
\pgfpathlineto{\pgfqpoint{4.510526in}{1.018702in}}%
\pgfpathlineto{\pgfqpoint{4.510993in}{0.992022in}}%
\pgfpathlineto{\pgfqpoint{4.511461in}{0.973683in}}%
\pgfpathlineto{\pgfqpoint{4.511695in}{0.982789in}}%
\pgfpathlineto{\pgfqpoint{4.514033in}{1.280450in}}%
\pgfpathlineto{\pgfqpoint{4.514969in}{1.268898in}}%
\pgfpathlineto{\pgfqpoint{4.517074in}{0.898854in}}%
\pgfpathlineto{\pgfqpoint{4.518477in}{1.104719in}}%
\pgfpathlineto{\pgfqpoint{4.518945in}{1.055350in}}%
\pgfpathlineto{\pgfqpoint{4.519178in}{1.013459in}}%
\pgfpathlineto{\pgfqpoint{4.519880in}{1.065445in}}%
\pgfpathlineto{\pgfqpoint{4.521049in}{1.172343in}}%
\pgfpathlineto{\pgfqpoint{4.521517in}{1.146428in}}%
\pgfpathlineto{\pgfqpoint{4.522219in}{1.071322in}}%
\pgfpathlineto{\pgfqpoint{4.522686in}{1.090709in}}%
\pgfpathlineto{\pgfqpoint{4.523622in}{1.215728in}}%
\pgfpathlineto{\pgfqpoint{4.523856in}{1.164744in}}%
\pgfpathlineto{\pgfqpoint{4.525493in}{1.026236in}}%
\pgfpathlineto{\pgfqpoint{4.526194in}{1.018159in}}%
\pgfpathlineto{\pgfqpoint{4.526662in}{1.062399in}}%
\pgfpathlineto{\pgfqpoint{4.527363in}{0.980649in}}%
\pgfpathlineto{\pgfqpoint{4.527831in}{1.016129in}}%
\pgfpathlineto{\pgfqpoint{4.528065in}{1.049541in}}%
\pgfpathlineto{\pgfqpoint{4.528533in}{0.996741in}}%
\pgfpathlineto{\pgfqpoint{4.528767in}{1.009178in}}%
\pgfpathlineto{\pgfqpoint{4.529000in}{1.013126in}}%
\pgfpathlineto{\pgfqpoint{4.529468in}{0.972954in}}%
\pgfpathlineto{\pgfqpoint{4.529936in}{1.015024in}}%
\pgfpathlineto{\pgfqpoint{4.531105in}{1.123605in}}%
\pgfpathlineto{\pgfqpoint{4.531339in}{1.106322in}}%
\pgfpathlineto{\pgfqpoint{4.532275in}{1.043496in}}%
\pgfpathlineto{\pgfqpoint{4.532508in}{1.074618in}}%
\pgfpathlineto{\pgfqpoint{4.532976in}{1.055454in}}%
\pgfpathlineto{\pgfqpoint{4.533210in}{1.056856in}}%
\pgfpathlineto{\pgfqpoint{4.534145in}{1.157105in}}%
\pgfpathlineto{\pgfqpoint{4.534613in}{1.107406in}}%
\pgfpathlineto{\pgfqpoint{4.534847in}{1.080707in}}%
\pgfpathlineto{\pgfqpoint{4.535081in}{1.131360in}}%
\pgfpathlineto{\pgfqpoint{4.535315in}{1.155075in}}%
\pgfpathlineto{\pgfqpoint{4.535782in}{1.097289in}}%
\pgfpathlineto{\pgfqpoint{4.537186in}{0.891448in}}%
\pgfpathlineto{\pgfqpoint{4.537419in}{0.893458in}}%
\pgfpathlineto{\pgfqpoint{4.538589in}{1.026865in}}%
\pgfpathlineto{\pgfqpoint{4.538823in}{1.018683in}}%
\pgfpathlineto{\pgfqpoint{4.539290in}{0.923432in}}%
\pgfpathlineto{\pgfqpoint{4.539758in}{0.972207in}}%
\pgfpathlineto{\pgfqpoint{4.540460in}{1.088509in}}%
\pgfpathlineto{\pgfqpoint{4.540927in}{1.076043in}}%
\pgfpathlineto{\pgfqpoint{4.541395in}{1.005358in}}%
\pgfpathlineto{\pgfqpoint{4.541629in}{1.029142in}}%
\pgfpathlineto{\pgfqpoint{4.543032in}{1.151825in}}%
\pgfpathlineto{\pgfqpoint{4.544669in}{0.911384in}}%
\pgfpathlineto{\pgfqpoint{4.546072in}{1.004194in}}%
\pgfpathlineto{\pgfqpoint{4.546306in}{0.983877in}}%
\pgfpathlineto{\pgfqpoint{4.547475in}{1.084080in}}%
\pgfpathlineto{\pgfqpoint{4.547008in}{0.976863in}}%
\pgfpathlineto{\pgfqpoint{4.547943in}{1.036128in}}%
\pgfpathlineto{\pgfqpoint{4.549112in}{0.914307in}}%
\pgfpathlineto{\pgfqpoint{4.549346in}{0.935455in}}%
\pgfpathlineto{\pgfqpoint{4.549580in}{0.921306in}}%
\pgfpathlineto{\pgfqpoint{4.549814in}{0.942832in}}%
\pgfpathlineto{\pgfqpoint{4.550983in}{1.117930in}}%
\pgfpathlineto{\pgfqpoint{4.551217in}{1.089916in}}%
\pgfpathlineto{\pgfqpoint{4.551451in}{1.048719in}}%
\pgfpathlineto{\pgfqpoint{4.552153in}{1.085526in}}%
\pgfpathlineto{\pgfqpoint{4.552620in}{1.118533in}}%
\pgfpathlineto{\pgfqpoint{4.552854in}{1.113506in}}%
\pgfpathlineto{\pgfqpoint{4.554257in}{0.949961in}}%
\pgfpathlineto{\pgfqpoint{4.554959in}{0.962597in}}%
\pgfpathlineto{\pgfqpoint{4.555193in}{0.947856in}}%
\pgfpathlineto{\pgfqpoint{4.555660in}{0.971039in}}%
\pgfpathlineto{\pgfqpoint{4.555894in}{0.960956in}}%
\pgfpathlineto{\pgfqpoint{4.556596in}{0.976708in}}%
\pgfpathlineto{\pgfqpoint{4.556830in}{0.968028in}}%
\pgfpathlineto{\pgfqpoint{4.557297in}{0.907854in}}%
\pgfpathlineto{\pgfqpoint{4.557999in}{0.954294in}}%
\pgfpathlineto{\pgfqpoint{4.558934in}{0.898546in}}%
\pgfpathlineto{\pgfqpoint{4.559168in}{0.941132in}}%
\pgfpathlineto{\pgfqpoint{4.561039in}{1.209188in}}%
\pgfpathlineto{\pgfqpoint{4.562676in}{0.952877in}}%
\pgfpathlineto{\pgfqpoint{4.562910in}{0.987081in}}%
\pgfpathlineto{\pgfqpoint{4.563144in}{0.947920in}}%
\pgfpathlineto{\pgfqpoint{4.563846in}{0.986862in}}%
\pgfpathlineto{\pgfqpoint{4.564781in}{0.774145in}}%
\pgfpathlineto{\pgfqpoint{4.565483in}{0.890365in}}%
\pgfpathlineto{\pgfqpoint{4.566184in}{1.088559in}}%
\pgfpathlineto{\pgfqpoint{4.567120in}{1.075495in}}%
\pgfpathlineto{\pgfqpoint{4.567587in}{1.133999in}}%
\pgfpathlineto{\pgfqpoint{4.567821in}{1.097025in}}%
\pgfpathlineto{\pgfqpoint{4.568289in}{1.041428in}}%
\pgfpathlineto{\pgfqpoint{4.568990in}{1.083924in}}%
\pgfpathlineto{\pgfqpoint{4.569458in}{1.105271in}}%
\pgfpathlineto{\pgfqpoint{4.571095in}{0.869995in}}%
\pgfpathlineto{\pgfqpoint{4.571329in}{0.921428in}}%
\pgfpathlineto{\pgfqpoint{4.571797in}{0.974610in}}%
\pgfpathlineto{\pgfqpoint{4.572732in}{0.970703in}}%
\pgfpathlineto{\pgfqpoint{4.572966in}{0.965855in}}%
\pgfpathlineto{\pgfqpoint{4.573200in}{0.972082in}}%
\pgfpathlineto{\pgfqpoint{4.573901in}{0.955863in}}%
\pgfpathlineto{\pgfqpoint{4.574603in}{1.078835in}}%
\pgfpathlineto{\pgfqpoint{4.575071in}{1.150323in}}%
\pgfpathlineto{\pgfqpoint{4.575538in}{1.099317in}}%
\pgfpathlineto{\pgfqpoint{4.577176in}{0.917440in}}%
\pgfpathlineto{\pgfqpoint{4.578111in}{0.939488in}}%
\pgfpathlineto{\pgfqpoint{4.578813in}{1.040968in}}%
\pgfpathlineto{\pgfqpoint{4.579280in}{0.980419in}}%
\pgfpathlineto{\pgfqpoint{4.579514in}{0.980024in}}%
\pgfpathlineto{\pgfqpoint{4.581619in}{1.118541in}}%
\pgfpathlineto{\pgfqpoint{4.581853in}{1.104491in}}%
\pgfpathlineto{\pgfqpoint{4.583724in}{0.997807in}}%
\pgfpathlineto{\pgfqpoint{4.583957in}{1.027810in}}%
\pgfpathlineto{\pgfqpoint{4.584191in}{1.027314in}}%
\pgfpathlineto{\pgfqpoint{4.585594in}{0.914204in}}%
\pgfpathlineto{\pgfqpoint{4.585828in}{0.940901in}}%
\pgfpathlineto{\pgfqpoint{4.587699in}{1.088276in}}%
\pgfpathlineto{\pgfqpoint{4.588401in}{1.247885in}}%
\pgfpathlineto{\pgfqpoint{4.588868in}{1.197432in}}%
\pgfpathlineto{\pgfqpoint{4.590739in}{0.977929in}}%
\pgfpathlineto{\pgfqpoint{4.591207in}{0.901724in}}%
\pgfpathlineto{\pgfqpoint{4.592142in}{0.914484in}}%
\pgfpathlineto{\pgfqpoint{4.593312in}{1.164220in}}%
\pgfpathlineto{\pgfqpoint{4.594247in}{1.134026in}}%
\pgfpathlineto{\pgfqpoint{4.596586in}{0.995040in}}%
\pgfpathlineto{\pgfqpoint{4.596820in}{0.997954in}}%
\pgfpathlineto{\pgfqpoint{4.597521in}{1.097021in}}%
\pgfpathlineto{\pgfqpoint{4.598457in}{1.260343in}}%
\pgfpathlineto{\pgfqpoint{4.599158in}{1.173440in}}%
\pgfpathlineto{\pgfqpoint{4.600328in}{0.902045in}}%
\pgfpathlineto{\pgfqpoint{4.601263in}{0.978075in}}%
\pgfpathlineto{\pgfqpoint{4.602900in}{1.214169in}}%
\pgfpathlineto{\pgfqpoint{4.603134in}{1.188093in}}%
\pgfpathlineto{\pgfqpoint{4.603602in}{1.133633in}}%
\pgfpathlineto{\pgfqpoint{4.604303in}{1.172205in}}%
\pgfpathlineto{\pgfqpoint{4.606174in}{1.254765in}}%
\pgfpathlineto{\pgfqpoint{4.608980in}{0.895991in}}%
\pgfpathlineto{\pgfqpoint{4.610617in}{1.259996in}}%
\pgfpathlineto{\pgfqpoint{4.611085in}{1.212664in}}%
\pgfpathlineto{\pgfqpoint{4.611787in}{1.130525in}}%
\pgfpathlineto{\pgfqpoint{4.612021in}{1.171179in}}%
\pgfpathlineto{\pgfqpoint{4.612722in}{1.245313in}}%
\pgfpathlineto{\pgfqpoint{4.613190in}{1.198914in}}%
\pgfpathlineto{\pgfqpoint{4.613891in}{1.225524in}}%
\pgfpathlineto{\pgfqpoint{4.615528in}{1.075430in}}%
\pgfpathlineto{\pgfqpoint{4.615762in}{1.102437in}}%
\pgfpathlineto{\pgfqpoint{4.615996in}{1.122527in}}%
\pgfpathlineto{\pgfqpoint{4.616464in}{1.075461in}}%
\pgfpathlineto{\pgfqpoint{4.616698in}{1.070304in}}%
\pgfpathlineto{\pgfqpoint{4.617399in}{1.017183in}}%
\pgfpathlineto{\pgfqpoint{4.617867in}{1.035175in}}%
\pgfpathlineto{\pgfqpoint{4.618335in}{1.057040in}}%
\pgfpathlineto{\pgfqpoint{4.620439in}{1.285014in}}%
\pgfpathlineto{\pgfqpoint{4.620907in}{1.248709in}}%
\pgfpathlineto{\pgfqpoint{4.621375in}{1.241048in}}%
\pgfpathlineto{\pgfqpoint{4.621843in}{1.188120in}}%
\pgfpathlineto{\pgfqpoint{4.622310in}{1.254717in}}%
\pgfpathlineto{\pgfqpoint{4.622544in}{1.214991in}}%
\pgfpathlineto{\pgfqpoint{4.624181in}{1.112931in}}%
\pgfpathlineto{\pgfqpoint{4.624415in}{1.113255in}}%
\pgfpathlineto{\pgfqpoint{4.625351in}{1.211798in}}%
\pgfpathlineto{\pgfqpoint{4.625818in}{1.170682in}}%
\pgfpathlineto{\pgfqpoint{4.626052in}{1.163470in}}%
\pgfpathlineto{\pgfqpoint{4.626520in}{1.084735in}}%
\pgfpathlineto{\pgfqpoint{4.627221in}{1.140364in}}%
\pgfpathlineto{\pgfqpoint{4.629326in}{1.299017in}}%
\pgfpathlineto{\pgfqpoint{4.630028in}{1.251735in}}%
\pgfpathlineto{\pgfqpoint{4.631197in}{1.067279in}}%
\pgfpathlineto{\pgfqpoint{4.631899in}{1.162673in}}%
\pgfpathlineto{\pgfqpoint{4.632600in}{1.296527in}}%
\pgfpathlineto{\pgfqpoint{4.633068in}{1.221412in}}%
\pgfpathlineto{\pgfqpoint{4.633302in}{1.186103in}}%
\pgfpathlineto{\pgfqpoint{4.633536in}{1.263429in}}%
\pgfpathlineto{\pgfqpoint{4.634003in}{1.345123in}}%
\pgfpathlineto{\pgfqpoint{4.634705in}{1.279162in}}%
\pgfpathlineto{\pgfqpoint{4.635640in}{1.231452in}}%
\pgfpathlineto{\pgfqpoint{4.635874in}{1.268044in}}%
\pgfpathlineto{\pgfqpoint{4.636108in}{1.264543in}}%
\pgfpathlineto{\pgfqpoint{4.636342in}{1.290584in}}%
\pgfpathlineto{\pgfqpoint{4.636576in}{1.253224in}}%
\pgfpathlineto{\pgfqpoint{4.638213in}{1.089289in}}%
\pgfpathlineto{\pgfqpoint{4.638447in}{1.103511in}}%
\pgfpathlineto{\pgfqpoint{4.638914in}{1.120419in}}%
\pgfpathlineto{\pgfqpoint{4.639382in}{1.182447in}}%
\pgfpathlineto{\pgfqpoint{4.641253in}{1.392340in}}%
\pgfpathlineto{\pgfqpoint{4.641721in}{1.397167in}}%
\pgfpathlineto{\pgfqpoint{4.645229in}{1.116724in}}%
\pgfpathlineto{\pgfqpoint{4.647099in}{1.309600in}}%
\pgfpathlineto{\pgfqpoint{4.647333in}{1.301034in}}%
\pgfpathlineto{\pgfqpoint{4.647567in}{1.313973in}}%
\pgfpathlineto{\pgfqpoint{4.648970in}{1.470257in}}%
\pgfpathlineto{\pgfqpoint{4.649204in}{1.464327in}}%
\pgfpathlineto{\pgfqpoint{4.651777in}{1.143154in}}%
\pgfpathlineto{\pgfqpoint{4.652010in}{1.139887in}}%
\pgfpathlineto{\pgfqpoint{4.654115in}{1.531419in}}%
\pgfpathlineto{\pgfqpoint{4.654583in}{1.460690in}}%
\pgfpathlineto{\pgfqpoint{4.655051in}{1.435521in}}%
\pgfpathlineto{\pgfqpoint{4.657389in}{1.190614in}}%
\pgfpathlineto{\pgfqpoint{4.658559in}{1.322078in}}%
\pgfpathlineto{\pgfqpoint{4.659260in}{1.320243in}}%
\pgfpathlineto{\pgfqpoint{4.659494in}{1.337532in}}%
\pgfpathlineto{\pgfqpoint{4.660429in}{1.543306in}}%
\pgfpathlineto{\pgfqpoint{4.660897in}{1.513746in}}%
\pgfpathlineto{\pgfqpoint{4.663236in}{1.270483in}}%
\pgfpathlineto{\pgfqpoint{4.663470in}{1.276463in}}%
\pgfpathlineto{\pgfqpoint{4.664873in}{1.369214in}}%
\pgfpathlineto{\pgfqpoint{4.666042in}{1.312263in}}%
\pgfpathlineto{\pgfqpoint{4.666977in}{1.497685in}}%
\pgfpathlineto{\pgfqpoint{4.667679in}{1.443819in}}%
\pgfpathlineto{\pgfqpoint{4.668147in}{1.467160in}}%
\pgfpathlineto{\pgfqpoint{4.668848in}{1.463393in}}%
\pgfpathlineto{\pgfqpoint{4.669316in}{1.460665in}}%
\pgfpathlineto{\pgfqpoint{4.669550in}{1.478457in}}%
\pgfpathlineto{\pgfqpoint{4.670485in}{1.416439in}}%
\pgfpathlineto{\pgfqpoint{4.670719in}{1.451536in}}%
\pgfpathlineto{\pgfqpoint{4.671421in}{1.382258in}}%
\pgfpathlineto{\pgfqpoint{4.672590in}{1.189964in}}%
\pgfpathlineto{\pgfqpoint{4.672824in}{1.215195in}}%
\pgfpathlineto{\pgfqpoint{4.674929in}{1.590121in}}%
\pgfpathlineto{\pgfqpoint{4.675163in}{1.585608in}}%
\pgfpathlineto{\pgfqpoint{4.675396in}{1.589976in}}%
\pgfpathlineto{\pgfqpoint{4.678670in}{1.188152in}}%
\pgfpathlineto{\pgfqpoint{4.678904in}{1.189131in}}%
\pgfpathlineto{\pgfqpoint{4.680307in}{1.669772in}}%
\pgfpathlineto{\pgfqpoint{4.680775in}{1.569845in}}%
\pgfpathlineto{\pgfqpoint{4.682178in}{1.353130in}}%
\pgfpathlineto{\pgfqpoint{4.682646in}{1.280221in}}%
\pgfpathlineto{\pgfqpoint{4.683114in}{1.359817in}}%
\pgfpathlineto{\pgfqpoint{4.683348in}{1.366879in}}%
\pgfpathlineto{\pgfqpoint{4.683581in}{1.344911in}}%
\pgfpathlineto{\pgfqpoint{4.683815in}{1.322380in}}%
\pgfpathlineto{\pgfqpoint{4.684283in}{1.326278in}}%
\pgfpathlineto{\pgfqpoint{4.685920in}{1.525817in}}%
\pgfpathlineto{\pgfqpoint{4.686154in}{1.518029in}}%
\pgfpathlineto{\pgfqpoint{4.687791in}{1.219510in}}%
\pgfpathlineto{\pgfqpoint{4.688025in}{1.275870in}}%
\pgfpathlineto{\pgfqpoint{4.689896in}{1.519279in}}%
\pgfpathlineto{\pgfqpoint{4.690130in}{1.547062in}}%
\pgfpathlineto{\pgfqpoint{4.690597in}{1.491174in}}%
\pgfpathlineto{\pgfqpoint{4.692468in}{1.296767in}}%
\pgfpathlineto{\pgfqpoint{4.693637in}{1.248508in}}%
\pgfpathlineto{\pgfqpoint{4.695274in}{1.448446in}}%
\pgfpathlineto{\pgfqpoint{4.695508in}{1.427532in}}%
\pgfpathlineto{\pgfqpoint{4.697145in}{1.298192in}}%
\pgfpathlineto{\pgfqpoint{4.697379in}{1.319307in}}%
\pgfpathlineto{\pgfqpoint{4.699484in}{1.471194in}}%
\pgfpathlineto{\pgfqpoint{4.700185in}{1.370296in}}%
\pgfpathlineto{\pgfqpoint{4.700419in}{1.335813in}}%
\pgfpathlineto{\pgfqpoint{4.700887in}{1.401060in}}%
\pgfpathlineto{\pgfqpoint{4.701355in}{1.477318in}}%
\pgfpathlineto{\pgfqpoint{4.701823in}{1.379668in}}%
\pgfpathlineto{\pgfqpoint{4.703927in}{1.188591in}}%
\pgfpathlineto{\pgfqpoint{4.704395in}{1.225759in}}%
\pgfpathlineto{\pgfqpoint{4.707201in}{1.551258in}}%
\pgfpathlineto{\pgfqpoint{4.707435in}{1.496815in}}%
\pgfpathlineto{\pgfqpoint{4.709306in}{1.190801in}}%
\pgfpathlineto{\pgfqpoint{4.709774in}{1.250524in}}%
\pgfpathlineto{\pgfqpoint{4.710475in}{1.228492in}}%
\pgfpathlineto{\pgfqpoint{4.710709in}{1.249386in}}%
\pgfpathlineto{\pgfqpoint{4.710943in}{1.263714in}}%
\pgfpathlineto{\pgfqpoint{4.711177in}{1.241172in}}%
\pgfpathlineto{\pgfqpoint{4.712112in}{1.175690in}}%
\pgfpathlineto{\pgfqpoint{4.712346in}{1.176461in}}%
\pgfpathlineto{\pgfqpoint{4.714451in}{1.461869in}}%
\pgfpathlineto{\pgfqpoint{4.714919in}{1.417556in}}%
\pgfpathlineto{\pgfqpoint{4.715620in}{1.249577in}}%
\pgfpathlineto{\pgfqpoint{4.716789in}{1.268141in}}%
\pgfpathlineto{\pgfqpoint{4.717023in}{1.315891in}}%
\pgfpathlineto{\pgfqpoint{4.717959in}{1.305401in}}%
\pgfpathlineto{\pgfqpoint{4.718894in}{1.075273in}}%
\pgfpathlineto{\pgfqpoint{4.719362in}{1.191261in}}%
\pgfpathlineto{\pgfqpoint{4.720531in}{1.324361in}}%
\pgfpathlineto{\pgfqpoint{4.721467in}{1.286887in}}%
\pgfpathlineto{\pgfqpoint{4.721934in}{1.298428in}}%
\pgfpathlineto{\pgfqpoint{4.722402in}{1.231128in}}%
\pgfpathlineto{\pgfqpoint{4.722870in}{1.301618in}}%
\pgfpathlineto{\pgfqpoint{4.723571in}{1.426588in}}%
\pgfpathlineto{\pgfqpoint{4.724039in}{1.372880in}}%
\pgfpathlineto{\pgfqpoint{4.725442in}{1.118859in}}%
\pgfpathlineto{\pgfqpoint{4.726144in}{1.195394in}}%
\pgfpathlineto{\pgfqpoint{4.726378in}{1.196822in}}%
\pgfpathlineto{\pgfqpoint{4.727079in}{1.303616in}}%
\pgfpathlineto{\pgfqpoint{4.727547in}{1.245439in}}%
\pgfpathlineto{\pgfqpoint{4.728950in}{1.094224in}}%
\pgfpathlineto{\pgfqpoint{4.731055in}{1.304071in}}%
\pgfpathlineto{\pgfqpoint{4.731289in}{1.269751in}}%
\pgfpathlineto{\pgfqpoint{4.731990in}{1.311668in}}%
\pgfpathlineto{\pgfqpoint{4.732224in}{1.310963in}}%
\pgfpathlineto{\pgfqpoint{4.732458in}{1.344753in}}%
\pgfpathlineto{\pgfqpoint{4.733160in}{1.303001in}}%
\pgfpathlineto{\pgfqpoint{4.733627in}{1.328480in}}%
\pgfpathlineto{\pgfqpoint{4.733861in}{1.290707in}}%
\pgfpathlineto{\pgfqpoint{4.734797in}{1.164159in}}%
\pgfpathlineto{\pgfqpoint{4.735264in}{1.205321in}}%
\pgfpathlineto{\pgfqpoint{4.735498in}{1.263906in}}%
\pgfpathlineto{\pgfqpoint{4.735966in}{1.227115in}}%
\pgfpathlineto{\pgfqpoint{4.737135in}{1.133131in}}%
\pgfpathlineto{\pgfqpoint{4.737369in}{1.154979in}}%
\pgfpathlineto{\pgfqpoint{4.737603in}{1.110021in}}%
\pgfpathlineto{\pgfqpoint{4.737837in}{1.081460in}}%
\pgfpathlineto{\pgfqpoint{4.738305in}{1.112724in}}%
\pgfpathlineto{\pgfqpoint{4.739708in}{1.224723in}}%
\pgfpathlineto{\pgfqpoint{4.738772in}{1.108509in}}%
\pgfpathlineto{\pgfqpoint{4.739942in}{1.222561in}}%
\pgfpathlineto{\pgfqpoint{4.740175in}{1.177106in}}%
\pgfpathlineto{\pgfqpoint{4.740877in}{1.244926in}}%
\pgfpathlineto{\pgfqpoint{4.741579in}{1.324734in}}%
\pgfpathlineto{\pgfqpoint{4.741812in}{1.253806in}}%
\pgfpathlineto{\pgfqpoint{4.743216in}{1.133179in}}%
\pgfpathlineto{\pgfqpoint{4.743917in}{1.220391in}}%
\pgfpathlineto{\pgfqpoint{4.744385in}{1.182862in}}%
\pgfpathlineto{\pgfqpoint{4.745554in}{1.143734in}}%
\pgfpathlineto{\pgfqpoint{4.746022in}{1.191573in}}%
\pgfpathlineto{\pgfqpoint{4.746490in}{1.137579in}}%
\pgfpathlineto{\pgfqpoint{4.746723in}{1.168735in}}%
\pgfpathlineto{\pgfqpoint{4.747425in}{1.058471in}}%
\pgfpathlineto{\pgfqpoint{4.748127in}{1.120560in}}%
\pgfpathlineto{\pgfqpoint{4.749062in}{1.016693in}}%
\pgfpathlineto{\pgfqpoint{4.749296in}{1.082210in}}%
\pgfpathlineto{\pgfqpoint{4.749764in}{1.163604in}}%
\pgfpathlineto{\pgfqpoint{4.750465in}{1.129096in}}%
\pgfpathlineto{\pgfqpoint{4.750699in}{1.105236in}}%
\pgfpathlineto{\pgfqpoint{4.751167in}{1.145824in}}%
\pgfpathlineto{\pgfqpoint{4.752102in}{1.175364in}}%
\pgfpathlineto{\pgfqpoint{4.752570in}{1.114170in}}%
\pgfpathlineto{\pgfqpoint{4.753038in}{1.177889in}}%
\pgfpathlineto{\pgfqpoint{4.753739in}{1.249062in}}%
\pgfpathlineto{\pgfqpoint{4.754207in}{1.206843in}}%
\pgfpathlineto{\pgfqpoint{4.754909in}{1.142701in}}%
\pgfpathlineto{\pgfqpoint{4.755844in}{0.975746in}}%
\pgfpathlineto{\pgfqpoint{4.756078in}{1.007566in}}%
\pgfpathlineto{\pgfqpoint{4.756779in}{1.099283in}}%
\pgfpathlineto{\pgfqpoint{4.757247in}{1.068650in}}%
\pgfpathlineto{\pgfqpoint{4.757481in}{1.061780in}}%
\pgfpathlineto{\pgfqpoint{4.757715in}{1.089570in}}%
\pgfpathlineto{\pgfqpoint{4.758183in}{1.059988in}}%
\pgfpathlineto{\pgfqpoint{4.758650in}{1.087415in}}%
\pgfpathlineto{\pgfqpoint{4.758884in}{1.068459in}}%
\pgfpathlineto{\pgfqpoint{4.759118in}{1.094071in}}%
\pgfpathlineto{\pgfqpoint{4.759352in}{1.134694in}}%
\pgfpathlineto{\pgfqpoint{4.759820in}{1.083139in}}%
\pgfpathlineto{\pgfqpoint{4.760053in}{1.098176in}}%
\pgfpathlineto{\pgfqpoint{4.760521in}{1.126442in}}%
\pgfpathlineto{\pgfqpoint{4.761924in}{1.288096in}}%
\pgfpathlineto{\pgfqpoint{4.762158in}{1.199078in}}%
\pgfpathlineto{\pgfqpoint{4.763795in}{1.038418in}}%
\pgfpathlineto{\pgfqpoint{4.764965in}{0.970938in}}%
\pgfpathlineto{\pgfqpoint{4.765198in}{0.986648in}}%
\pgfpathlineto{\pgfqpoint{4.765432in}{1.006554in}}%
\pgfpathlineto{\pgfqpoint{4.765900in}{0.986940in}}%
\pgfpathlineto{\pgfqpoint{4.766368in}{0.903762in}}%
\pgfpathlineto{\pgfqpoint{4.766835in}{0.973361in}}%
\pgfpathlineto{\pgfqpoint{4.768472in}{1.206986in}}%
\pgfpathlineto{\pgfqpoint{4.768940in}{1.173671in}}%
\pgfpathlineto{\pgfqpoint{4.770343in}{1.099982in}}%
\pgfpathlineto{\pgfqpoint{4.770577in}{1.116201in}}%
\pgfpathlineto{\pgfqpoint{4.770811in}{1.085282in}}%
\pgfpathlineto{\pgfqpoint{4.773617in}{0.873914in}}%
\pgfpathlineto{\pgfqpoint{4.774085in}{0.899329in}}%
\pgfpathlineto{\pgfqpoint{4.774553in}{0.900561in}}%
\pgfpathlineto{\pgfqpoint{4.774787in}{0.893172in}}%
\pgfpathlineto{\pgfqpoint{4.776424in}{1.184238in}}%
\pgfpathlineto{\pgfqpoint{4.776657in}{1.151901in}}%
\pgfpathlineto{\pgfqpoint{4.777359in}{1.067221in}}%
\pgfpathlineto{\pgfqpoint{4.778061in}{1.068503in}}%
\pgfpathlineto{\pgfqpoint{4.778294in}{1.069364in}}%
\pgfpathlineto{\pgfqpoint{4.778528in}{1.108298in}}%
\pgfpathlineto{\pgfqpoint{4.778762in}{1.048752in}}%
\pgfpathlineto{\pgfqpoint{4.780165in}{0.888898in}}%
\pgfpathlineto{\pgfqpoint{4.780399in}{0.877988in}}%
\pgfpathlineto{\pgfqpoint{4.780867in}{0.965331in}}%
\pgfpathlineto{\pgfqpoint{4.781335in}{0.854981in}}%
\pgfpathlineto{\pgfqpoint{4.783439in}{1.003494in}}%
\pgfpathlineto{\pgfqpoint{4.783673in}{0.970986in}}%
\pgfpathlineto{\pgfqpoint{4.783907in}{0.951159in}}%
\pgfpathlineto{\pgfqpoint{4.784141in}{0.994910in}}%
\pgfpathlineto{\pgfqpoint{4.784609in}{1.102616in}}%
\pgfpathlineto{\pgfqpoint{4.785076in}{1.202579in}}%
\pgfpathlineto{\pgfqpoint{4.785544in}{1.134369in}}%
\pgfpathlineto{\pgfqpoint{4.786012in}{1.046333in}}%
\pgfpathlineto{\pgfqpoint{4.787181in}{0.856204in}}%
\pgfpathlineto{\pgfqpoint{4.787649in}{0.868092in}}%
\pgfpathlineto{\pgfqpoint{4.787883in}{0.903954in}}%
\pgfpathlineto{\pgfqpoint{4.788117in}{0.839350in}}%
\pgfpathlineto{\pgfqpoint{4.788350in}{0.846871in}}%
\pgfpathlineto{\pgfqpoint{4.788584in}{0.825676in}}%
\pgfpathlineto{\pgfqpoint{4.789052in}{0.863547in}}%
\pgfpathlineto{\pgfqpoint{4.790221in}{1.150551in}}%
\pgfpathlineto{\pgfqpoint{4.791157in}{1.029601in}}%
\pgfpathlineto{\pgfqpoint{4.791391in}{1.024787in}}%
\pgfpathlineto{\pgfqpoint{4.791624in}{1.025320in}}%
\pgfpathlineto{\pgfqpoint{4.792092in}{1.057085in}}%
\pgfpathlineto{\pgfqpoint{4.792326in}{1.008416in}}%
\pgfpathlineto{\pgfqpoint{4.793028in}{0.902596in}}%
\pgfpathlineto{\pgfqpoint{4.793495in}{0.784375in}}%
\pgfpathlineto{\pgfqpoint{4.793963in}{0.867074in}}%
\pgfpathlineto{\pgfqpoint{4.795600in}{0.939478in}}%
\pgfpathlineto{\pgfqpoint{4.795834in}{0.939802in}}%
\pgfpathlineto{\pgfqpoint{4.797237in}{1.104500in}}%
\pgfpathlineto{\pgfqpoint{4.797471in}{1.068702in}}%
\pgfpathlineto{\pgfqpoint{4.799342in}{0.861895in}}%
\pgfpathlineto{\pgfqpoint{4.799810in}{0.877486in}}%
\pgfpathlineto{\pgfqpoint{4.800043in}{0.891250in}}%
\pgfpathlineto{\pgfqpoint{4.801213in}{1.125304in}}%
\pgfpathlineto{\pgfqpoint{4.801447in}{1.097918in}}%
\pgfpathlineto{\pgfqpoint{4.803785in}{0.799452in}}%
\pgfpathlineto{\pgfqpoint{4.804253in}{0.866181in}}%
\pgfpathlineto{\pgfqpoint{4.805656in}{1.141560in}}%
\pgfpathlineto{\pgfqpoint{4.806358in}{1.034981in}}%
\pgfpathlineto{\pgfqpoint{4.806825in}{1.075722in}}%
\pgfpathlineto{\pgfqpoint{4.807059in}{1.088702in}}%
\pgfpathlineto{\pgfqpoint{4.807527in}{1.079932in}}%
\pgfpathlineto{\pgfqpoint{4.809398in}{0.985957in}}%
\pgfpathlineto{\pgfqpoint{4.809865in}{1.035569in}}%
\pgfpathlineto{\pgfqpoint{4.810099in}{1.001508in}}%
\pgfpathlineto{\pgfqpoint{4.810567in}{0.883949in}}%
\pgfpathlineto{\pgfqpoint{4.811503in}{0.889056in}}%
\pgfpathlineto{\pgfqpoint{4.811736in}{0.874883in}}%
\pgfpathlineto{\pgfqpoint{4.812204in}{0.902891in}}%
\pgfpathlineto{\pgfqpoint{4.815244in}{1.225188in}}%
\pgfpathlineto{\pgfqpoint{4.815478in}{1.198291in}}%
\pgfpathlineto{\pgfqpoint{4.816647in}{0.889806in}}%
\pgfpathlineto{\pgfqpoint{4.817583in}{0.941579in}}%
\pgfpathlineto{\pgfqpoint{4.818284in}{1.016494in}}%
\pgfpathlineto{\pgfqpoint{4.818518in}{0.994798in}}%
\pgfpathlineto{\pgfqpoint{4.818986in}{0.897895in}}%
\pgfpathlineto{\pgfqpoint{4.819921in}{0.918431in}}%
\pgfpathlineto{\pgfqpoint{4.820389in}{0.999848in}}%
\pgfpathlineto{\pgfqpoint{4.821091in}{0.950960in}}%
\pgfpathlineto{\pgfqpoint{4.822260in}{1.196714in}}%
\pgfpathlineto{\pgfqpoint{4.823663in}{1.169432in}}%
\pgfpathlineto{\pgfqpoint{4.825768in}{0.893393in}}%
\pgfpathlineto{\pgfqpoint{4.826236in}{0.933486in}}%
\pgfpathlineto{\pgfqpoint{4.826470in}{1.003295in}}%
\pgfpathlineto{\pgfqpoint{4.826937in}{0.929954in}}%
\pgfpathlineto{\pgfqpoint{4.827171in}{0.936715in}}%
\pgfpathlineto{\pgfqpoint{4.827405in}{0.936563in}}%
\pgfpathlineto{\pgfqpoint{4.827873in}{0.885324in}}%
\pgfpathlineto{\pgfqpoint{4.828340in}{0.954178in}}%
\pgfpathlineto{\pgfqpoint{4.829744in}{1.145220in}}%
\pgfpathlineto{\pgfqpoint{4.830211in}{1.108717in}}%
\pgfpathlineto{\pgfqpoint{4.830445in}{1.059291in}}%
\pgfpathlineto{\pgfqpoint{4.831381in}{1.066548in}}%
\pgfpathlineto{\pgfqpoint{4.832550in}{1.148724in}}%
\pgfpathlineto{\pgfqpoint{4.834187in}{0.953027in}}%
\pgfpathlineto{\pgfqpoint{4.835356in}{1.021936in}}%
\pgfpathlineto{\pgfqpoint{4.835590in}{1.054850in}}%
\pgfpathlineto{\pgfqpoint{4.836292in}{1.018689in}}%
\pgfpathlineto{\pgfqpoint{4.836993in}{0.934154in}}%
\pgfpathlineto{\pgfqpoint{4.837461in}{0.958018in}}%
\pgfpathlineto{\pgfqpoint{4.839566in}{1.219425in}}%
\pgfpathlineto{\pgfqpoint{4.839799in}{1.210881in}}%
\pgfpathlineto{\pgfqpoint{4.841437in}{1.009117in}}%
\pgfpathlineto{\pgfqpoint{4.841904in}{0.924035in}}%
\pgfpathlineto{\pgfqpoint{4.842606in}{0.982339in}}%
\pgfpathlineto{\pgfqpoint{4.842840in}{0.968248in}}%
\pgfpathlineto{\pgfqpoint{4.843074in}{0.970961in}}%
\pgfpathlineto{\pgfqpoint{4.844243in}{1.296517in}}%
\pgfpathlineto{\pgfqpoint{4.844944in}{1.166468in}}%
\pgfpathlineto{\pgfqpoint{4.846114in}{0.913855in}}%
\pgfpathlineto{\pgfqpoint{4.846348in}{0.918402in}}%
\pgfpathlineto{\pgfqpoint{4.846581in}{0.921846in}}%
\pgfpathlineto{\pgfqpoint{4.848452in}{1.269631in}}%
\pgfpathlineto{\pgfqpoint{4.848686in}{1.250113in}}%
\pgfpathlineto{\pgfqpoint{4.849388in}{1.114319in}}%
\pgfpathlineto{\pgfqpoint{4.849855in}{1.147059in}}%
\pgfpathlineto{\pgfqpoint{4.850089in}{1.213745in}}%
\pgfpathlineto{\pgfqpoint{4.850557in}{1.138192in}}%
\pgfpathlineto{\pgfqpoint{4.851726in}{0.952995in}}%
\pgfpathlineto{\pgfqpoint{4.851960in}{0.980206in}}%
\pgfpathlineto{\pgfqpoint{4.854065in}{1.238639in}}%
\pgfpathlineto{\pgfqpoint{4.854299in}{1.237157in}}%
\pgfpathlineto{\pgfqpoint{4.855702in}{1.173479in}}%
\pgfpathlineto{\pgfqpoint{4.855936in}{1.189323in}}%
\pgfpathlineto{\pgfqpoint{4.856170in}{1.187551in}}%
\pgfpathlineto{\pgfqpoint{4.856403in}{1.209211in}}%
\pgfpathlineto{\pgfqpoint{4.856637in}{1.154815in}}%
\pgfpathlineto{\pgfqpoint{4.857807in}{1.025755in}}%
\pgfpathlineto{\pgfqpoint{4.858041in}{1.064015in}}%
\pgfpathlineto{\pgfqpoint{4.858508in}{1.049523in}}%
\pgfpathlineto{\pgfqpoint{4.858742in}{1.066403in}}%
\pgfpathlineto{\pgfqpoint{4.860847in}{1.331714in}}%
\pgfpathlineto{\pgfqpoint{4.863653in}{0.996889in}}%
\pgfpathlineto{\pgfqpoint{4.863887in}{1.011596in}}%
\pgfpathlineto{\pgfqpoint{4.866927in}{1.379437in}}%
\pgfpathlineto{\pgfqpoint{4.867161in}{1.372628in}}%
\pgfpathlineto{\pgfqpoint{4.868798in}{1.154604in}}%
\pgfpathlineto{\pgfqpoint{4.870435in}{1.051628in}}%
\pgfpathlineto{\pgfqpoint{4.872306in}{1.302145in}}%
\pgfpathlineto{\pgfqpoint{4.872540in}{1.261574in}}%
\pgfpathlineto{\pgfqpoint{4.873475in}{1.304804in}}%
\pgfpathlineto{\pgfqpoint{4.873943in}{1.303044in}}%
\pgfpathlineto{\pgfqpoint{4.874177in}{1.286267in}}%
\pgfpathlineto{\pgfqpoint{4.874645in}{1.321779in}}%
\pgfpathlineto{\pgfqpoint{4.874878in}{1.321673in}}%
\pgfpathlineto{\pgfqpoint{4.875112in}{1.323590in}}%
\pgfpathlineto{\pgfqpoint{4.876048in}{1.134090in}}%
\pgfpathlineto{\pgfqpoint{4.876749in}{1.192246in}}%
\pgfpathlineto{\pgfqpoint{4.876983in}{1.255154in}}%
\pgfpathlineto{\pgfqpoint{4.877919in}{1.242157in}}%
\pgfpathlineto{\pgfqpoint{4.879088in}{1.147495in}}%
\pgfpathlineto{\pgfqpoint{4.879322in}{1.175259in}}%
\pgfpathlineto{\pgfqpoint{4.879789in}{1.221974in}}%
\pgfpathlineto{\pgfqpoint{4.881426in}{1.377878in}}%
\pgfpathlineto{\pgfqpoint{4.881660in}{1.378767in}}%
\pgfpathlineto{\pgfqpoint{4.883999in}{1.150740in}}%
\pgfpathlineto{\pgfqpoint{4.884233in}{1.157341in}}%
\pgfpathlineto{\pgfqpoint{4.885402in}{1.366257in}}%
\pgfpathlineto{\pgfqpoint{4.885636in}{1.329747in}}%
\pgfpathlineto{\pgfqpoint{4.885870in}{1.284826in}}%
\pgfpathlineto{\pgfqpoint{4.886337in}{1.350123in}}%
\pgfpathlineto{\pgfqpoint{4.886571in}{1.448179in}}%
\pgfpathlineto{\pgfqpoint{4.887507in}{1.378967in}}%
\pgfpathlineto{\pgfqpoint{4.887741in}{1.384688in}}%
\pgfpathlineto{\pgfqpoint{4.889144in}{1.136471in}}%
\pgfpathlineto{\pgfqpoint{4.890079in}{1.148179in}}%
\pgfpathlineto{\pgfqpoint{4.891249in}{1.367561in}}%
\pgfpathlineto{\pgfqpoint{4.892652in}{1.505481in}}%
\pgfpathlineto{\pgfqpoint{4.892886in}{1.498512in}}%
\pgfpathlineto{\pgfqpoint{4.893119in}{1.493459in}}%
\pgfpathlineto{\pgfqpoint{4.894990in}{1.176802in}}%
\pgfpathlineto{\pgfqpoint{4.895458in}{1.143305in}}%
\pgfpathlineto{\pgfqpoint{4.895692in}{1.175205in}}%
\pgfpathlineto{\pgfqpoint{4.897563in}{1.410615in}}%
\pgfpathlineto{\pgfqpoint{4.898030in}{1.387338in}}%
\pgfpathlineto{\pgfqpoint{4.898264in}{1.372947in}}%
\pgfpathlineto{\pgfqpoint{4.898732in}{1.404802in}}%
\pgfpathlineto{\pgfqpoint{4.899434in}{1.403231in}}%
\pgfpathlineto{\pgfqpoint{4.899901in}{1.426178in}}%
\pgfpathlineto{\pgfqpoint{4.901772in}{1.298359in}}%
\pgfpathlineto{\pgfqpoint{4.900603in}{1.433460in}}%
\pgfpathlineto{\pgfqpoint{4.902474in}{1.340482in}}%
\pgfpathlineto{\pgfqpoint{4.903643in}{1.405247in}}%
\pgfpathlineto{\pgfqpoint{4.904345in}{1.402003in}}%
\pgfpathlineto{\pgfqpoint{4.906449in}{1.212004in}}%
\pgfpathlineto{\pgfqpoint{4.909490in}{1.515641in}}%
\pgfpathlineto{\pgfqpoint{4.909723in}{1.501227in}}%
\pgfpathlineto{\pgfqpoint{4.910659in}{1.329891in}}%
\pgfpathlineto{\pgfqpoint{4.912062in}{1.359759in}}%
\pgfpathlineto{\pgfqpoint{4.912530in}{1.384565in}}%
\pgfpathlineto{\pgfqpoint{4.912764in}{1.354714in}}%
\pgfpathlineto{\pgfqpoint{4.913465in}{1.287754in}}%
\pgfpathlineto{\pgfqpoint{4.913933in}{1.318685in}}%
\pgfpathlineto{\pgfqpoint{4.914167in}{1.339743in}}%
\pgfpathlineto{\pgfqpoint{4.914634in}{1.295788in}}%
\pgfpathlineto{\pgfqpoint{4.914868in}{1.272973in}}%
\pgfpathlineto{\pgfqpoint{4.915102in}{1.299302in}}%
\pgfpathlineto{\pgfqpoint{4.916505in}{1.426227in}}%
\pgfpathlineto{\pgfqpoint{4.916739in}{1.439871in}}%
\pgfpathlineto{\pgfqpoint{4.917441in}{1.540078in}}%
\pgfpathlineto{\pgfqpoint{4.917908in}{1.531324in}}%
\pgfpathlineto{\pgfqpoint{4.920013in}{1.391968in}}%
\pgfpathlineto{\pgfqpoint{4.920481in}{1.429517in}}%
\pgfpathlineto{\pgfqpoint{4.920715in}{1.445969in}}%
\pgfpathlineto{\pgfqpoint{4.920949in}{1.422479in}}%
\pgfpathlineto{\pgfqpoint{4.922352in}{1.249038in}}%
\pgfpathlineto{\pgfqpoint{4.922820in}{1.253211in}}%
\pgfpathlineto{\pgfqpoint{4.926327in}{1.544881in}}%
\pgfpathlineto{\pgfqpoint{4.926561in}{1.542481in}}%
\pgfpathlineto{\pgfqpoint{4.928198in}{1.295739in}}%
\pgfpathlineto{\pgfqpoint{4.930303in}{1.441032in}}%
\pgfpathlineto{\pgfqpoint{4.930537in}{1.382471in}}%
\pgfpathlineto{\pgfqpoint{4.931005in}{1.255744in}}%
\pgfpathlineto{\pgfqpoint{4.931706in}{1.334067in}}%
\pgfpathlineto{\pgfqpoint{4.932408in}{1.454553in}}%
\pgfpathlineto{\pgfqpoint{4.933109in}{1.392031in}}%
\pgfpathlineto{\pgfqpoint{4.933577in}{1.332336in}}%
\pgfpathlineto{\pgfqpoint{4.934045in}{1.383861in}}%
\pgfpathlineto{\pgfqpoint{4.934279in}{1.414075in}}%
\pgfpathlineto{\pgfqpoint{4.934746in}{1.362643in}}%
\pgfpathlineto{\pgfqpoint{4.935214in}{1.408539in}}%
\pgfpathlineto{\pgfqpoint{4.935682in}{1.470610in}}%
\pgfpathlineto{\pgfqpoint{4.936851in}{1.543615in}}%
\pgfpathlineto{\pgfqpoint{4.936150in}{1.469603in}}%
\pgfpathlineto{\pgfqpoint{4.937085in}{1.495821in}}%
\pgfpathlineto{\pgfqpoint{4.939891in}{1.160654in}}%
\pgfpathlineto{\pgfqpoint{4.940125in}{1.179695in}}%
\pgfpathlineto{\pgfqpoint{4.941294in}{1.506119in}}%
\pgfpathlineto{\pgfqpoint{4.941996in}{1.434575in}}%
\pgfpathlineto{\pgfqpoint{4.942698in}{1.330438in}}%
\pgfpathlineto{\pgfqpoint{4.943399in}{1.368629in}}%
\pgfpathlineto{\pgfqpoint{4.944335in}{1.514899in}}%
\pgfpathlineto{\pgfqpoint{4.944802in}{1.413032in}}%
\pgfpathlineto{\pgfqpoint{4.945972in}{1.325313in}}%
\pgfpathlineto{\pgfqpoint{4.946439in}{1.353807in}}%
\pgfpathlineto{\pgfqpoint{4.947141in}{1.476227in}}%
\pgfpathlineto{\pgfqpoint{4.947842in}{1.425931in}}%
\pgfpathlineto{\pgfqpoint{4.948076in}{1.423650in}}%
\pgfpathlineto{\pgfqpoint{4.949246in}{1.222994in}}%
\pgfpathlineto{\pgfqpoint{4.949479in}{1.263452in}}%
\pgfpathlineto{\pgfqpoint{4.951350in}{1.422460in}}%
\pgfpathlineto{\pgfqpoint{4.951584in}{1.400272in}}%
\pgfpathlineto{\pgfqpoint{4.952052in}{1.352186in}}%
\pgfpathlineto{\pgfqpoint{4.952286in}{1.397188in}}%
\pgfpathlineto{\pgfqpoint{4.953221in}{1.555661in}}%
\pgfpathlineto{\pgfqpoint{4.953689in}{1.494165in}}%
\pgfpathlineto{\pgfqpoint{4.955092in}{1.251361in}}%
\pgfpathlineto{\pgfqpoint{4.955794in}{1.328062in}}%
\pgfpathlineto{\pgfqpoint{4.956729in}{1.385737in}}%
\pgfpathlineto{\pgfqpoint{4.956963in}{1.357677in}}%
\pgfpathlineto{\pgfqpoint{4.957665in}{1.233417in}}%
\pgfpathlineto{\pgfqpoint{4.958132in}{1.314071in}}%
\pgfpathlineto{\pgfqpoint{4.958834in}{1.455861in}}%
\pgfpathlineto{\pgfqpoint{4.959535in}{1.442778in}}%
\pgfpathlineto{\pgfqpoint{4.959769in}{1.512694in}}%
\pgfpathlineto{\pgfqpoint{4.960471in}{1.428694in}}%
\pgfpathlineto{\pgfqpoint{4.961640in}{1.339025in}}%
\pgfpathlineto{\pgfqpoint{4.961874in}{1.369634in}}%
\pgfpathlineto{\pgfqpoint{4.962342in}{1.452494in}}%
\pgfpathlineto{\pgfqpoint{4.963043in}{1.388240in}}%
\pgfpathlineto{\pgfqpoint{4.963511in}{1.306092in}}%
\pgfpathlineto{\pgfqpoint{4.964446in}{1.338038in}}%
\pgfpathlineto{\pgfqpoint{4.964680in}{1.359022in}}%
\pgfpathlineto{\pgfqpoint{4.965148in}{1.318883in}}%
\pgfpathlineto{\pgfqpoint{4.965616in}{1.276004in}}%
\pgfpathlineto{\pgfqpoint{4.965850in}{1.324498in}}%
\pgfpathlineto{\pgfqpoint{4.966317in}{1.288924in}}%
\pgfpathlineto{\pgfqpoint{4.967487in}{1.390667in}}%
\pgfpathlineto{\pgfqpoint{4.968656in}{1.348809in}}%
\pgfpathlineto{\pgfqpoint{4.968890in}{1.351865in}}%
\pgfpathlineto{\pgfqpoint{4.969124in}{1.372231in}}%
\pgfpathlineto{\pgfqpoint{4.969358in}{1.341729in}}%
\pgfpathlineto{\pgfqpoint{4.970995in}{1.219763in}}%
\pgfpathlineto{\pgfqpoint{4.971228in}{1.239095in}}%
\pgfpathlineto{\pgfqpoint{4.971462in}{1.215669in}}%
\pgfpathlineto{\pgfqpoint{4.972164in}{1.254886in}}%
\pgfpathlineto{\pgfqpoint{4.974035in}{1.467532in}}%
\pgfpathlineto{\pgfqpoint{4.976139in}{1.194391in}}%
\pgfpathlineto{\pgfqpoint{4.976607in}{1.225299in}}%
\pgfpathlineto{\pgfqpoint{4.977075in}{1.281157in}}%
\pgfpathlineto{\pgfqpoint{4.977543in}{1.254124in}}%
\pgfpathlineto{\pgfqpoint{4.978244in}{1.172946in}}%
\pgfpathlineto{\pgfqpoint{4.978712in}{1.233142in}}%
\pgfpathlineto{\pgfqpoint{4.978946in}{1.297233in}}%
\pgfpathlineto{\pgfqpoint{4.979647in}{1.251220in}}%
\pgfpathlineto{\pgfqpoint{4.979881in}{1.231227in}}%
\pgfpathlineto{\pgfqpoint{4.980115in}{1.258613in}}%
\pgfpathlineto{\pgfqpoint{4.982220in}{1.450002in}}%
\pgfpathlineto{\pgfqpoint{4.985026in}{1.104312in}}%
\pgfpathlineto{\pgfqpoint{4.986195in}{1.120012in}}%
\pgfpathlineto{\pgfqpoint{4.987365in}{1.262888in}}%
\pgfpathlineto{\pgfqpoint{4.988066in}{1.230720in}}%
\pgfpathlineto{\pgfqpoint{4.988534in}{1.247759in}}%
\pgfpathlineto{\pgfqpoint{4.989002in}{1.198604in}}%
\pgfpathlineto{\pgfqpoint{4.990639in}{1.325811in}}%
\pgfpathlineto{\pgfqpoint{4.991106in}{1.340973in}}%
\pgfpathlineto{\pgfqpoint{4.992977in}{0.999095in}}%
\pgfpathlineto{\pgfqpoint{4.993211in}{1.002325in}}%
\pgfpathlineto{\pgfqpoint{4.993679in}{1.061093in}}%
\pgfpathlineto{\pgfqpoint{4.995550in}{1.352723in}}%
\pgfpathlineto{\pgfqpoint{4.995784in}{1.385378in}}%
\pgfpathlineto{\pgfqpoint{4.996017in}{1.352135in}}%
\pgfpathlineto{\pgfqpoint{4.997655in}{1.119017in}}%
\pgfpathlineto{\pgfqpoint{4.997888in}{1.134711in}}%
\pgfpathlineto{\pgfqpoint{4.999292in}{0.989188in}}%
\pgfpathlineto{\pgfqpoint{5.000227in}{1.013290in}}%
\pgfpathlineto{\pgfqpoint{5.002332in}{1.337427in}}%
\pgfpathlineto{\pgfqpoint{5.003033in}{1.254723in}}%
\pgfpathlineto{\pgfqpoint{5.005372in}{0.943407in}}%
\pgfpathlineto{\pgfqpoint{5.006073in}{1.016227in}}%
\pgfpathlineto{\pgfqpoint{5.006307in}{1.042814in}}%
\pgfpathlineto{\pgfqpoint{5.006775in}{0.982779in}}%
\pgfpathlineto{\pgfqpoint{5.007009in}{0.985891in}}%
\pgfpathlineto{\pgfqpoint{5.008880in}{1.218324in}}%
\pgfpathlineto{\pgfqpoint{5.009114in}{1.199451in}}%
\pgfpathlineto{\pgfqpoint{5.011218in}{1.018605in}}%
\pgfpathlineto{\pgfqpoint{5.012154in}{1.083715in}}%
\pgfpathlineto{\pgfqpoint{5.012855in}{1.083302in}}%
\pgfpathlineto{\pgfqpoint{5.013323in}{1.149120in}}%
\pgfpathlineto{\pgfqpoint{5.014025in}{1.126327in}}%
\pgfpathlineto{\pgfqpoint{5.015428in}{1.024591in}}%
\pgfpathlineto{\pgfqpoint{5.015896in}{1.088276in}}%
\pgfpathlineto{\pgfqpoint{5.016363in}{1.050882in}}%
\pgfpathlineto{\pgfqpoint{5.016831in}{1.006533in}}%
\pgfpathlineto{\pgfqpoint{5.017299in}{1.052455in}}%
\pgfpathlineto{\pgfqpoint{5.017533in}{1.041605in}}%
\pgfpathlineto{\pgfqpoint{5.017766in}{1.072226in}}%
\pgfpathlineto{\pgfqpoint{5.018234in}{1.013480in}}%
\pgfpathlineto{\pgfqpoint{5.018702in}{1.052555in}}%
\pgfpathlineto{\pgfqpoint{5.018936in}{1.047084in}}%
\pgfpathlineto{\pgfqpoint{5.019637in}{0.958912in}}%
\pgfpathlineto{\pgfqpoint{5.020105in}{0.996018in}}%
\pgfpathlineto{\pgfqpoint{5.021274in}{1.289140in}}%
\pgfpathlineto{\pgfqpoint{5.021508in}{1.196849in}}%
\pgfpathlineto{\pgfqpoint{5.023379in}{1.014161in}}%
\pgfpathlineto{\pgfqpoint{5.023613in}{1.017168in}}%
\pgfpathlineto{\pgfqpoint{5.023847in}{0.994360in}}%
\pgfpathlineto{\pgfqpoint{5.024314in}{1.020129in}}%
\pgfpathlineto{\pgfqpoint{5.024782in}{1.009724in}}%
\pgfpathlineto{\pgfqpoint{5.025016in}{1.002122in}}%
\pgfpathlineto{\pgfqpoint{5.025250in}{0.953935in}}%
\pgfpathlineto{\pgfqpoint{5.025718in}{1.005883in}}%
\pgfpathlineto{\pgfqpoint{5.026185in}{0.961037in}}%
\pgfpathlineto{\pgfqpoint{5.026887in}{1.041844in}}%
\pgfpathlineto{\pgfqpoint{5.027121in}{0.992866in}}%
\pgfpathlineto{\pgfqpoint{5.027355in}{0.959767in}}%
\pgfpathlineto{\pgfqpoint{5.027822in}{1.049367in}}%
\pgfpathlineto{\pgfqpoint{5.029226in}{1.214452in}}%
\pgfpathlineto{\pgfqpoint{5.029459in}{1.194255in}}%
\pgfpathlineto{\pgfqpoint{5.030395in}{1.037750in}}%
\pgfpathlineto{\pgfqpoint{5.031096in}{1.083436in}}%
\pgfpathlineto{\pgfqpoint{5.031330in}{1.066013in}}%
\pgfpathlineto{\pgfqpoint{5.032032in}{1.091799in}}%
\pgfpathlineto{\pgfqpoint{5.032500in}{1.068427in}}%
\pgfpathlineto{\pgfqpoint{5.033903in}{0.864684in}}%
\pgfpathlineto{\pgfqpoint{5.034137in}{0.867802in}}%
\pgfpathlineto{\pgfqpoint{5.036007in}{1.169057in}}%
\pgfpathlineto{\pgfqpoint{5.036241in}{1.159998in}}%
\pgfpathlineto{\pgfqpoint{5.036943in}{1.141017in}}%
\pgfpathlineto{\pgfqpoint{5.037177in}{1.166075in}}%
\pgfpathlineto{\pgfqpoint{5.037411in}{1.182446in}}%
\pgfpathlineto{\pgfqpoint{5.037644in}{1.144804in}}%
\pgfpathlineto{\pgfqpoint{5.041386in}{0.885029in}}%
\pgfpathlineto{\pgfqpoint{5.038112in}{1.159265in}}%
\pgfpathlineto{\pgfqpoint{5.041620in}{0.910167in}}%
\pgfpathlineto{\pgfqpoint{5.042555in}{1.204108in}}%
\pgfpathlineto{\pgfqpoint{5.043491in}{1.199625in}}%
\pgfpathlineto{\pgfqpoint{5.043725in}{1.206069in}}%
\pgfpathlineto{\pgfqpoint{5.043959in}{1.200419in}}%
\pgfpathlineto{\pgfqpoint{5.046063in}{0.953238in}}%
\pgfpathlineto{\pgfqpoint{5.046297in}{1.015057in}}%
\pgfpathlineto{\pgfqpoint{5.046531in}{1.043260in}}%
\pgfpathlineto{\pgfqpoint{5.047233in}{1.002275in}}%
\pgfpathlineto{\pgfqpoint{5.047700in}{0.991696in}}%
\pgfpathlineto{\pgfqpoint{5.047934in}{0.958279in}}%
\pgfpathlineto{\pgfqpoint{5.048636in}{0.995955in}}%
\pgfpathlineto{\pgfqpoint{5.049104in}{1.026751in}}%
\pgfpathlineto{\pgfqpoint{5.049571in}{1.000187in}}%
\pgfpathlineto{\pgfqpoint{5.050039in}{0.936615in}}%
\pgfpathlineto{\pgfqpoint{5.050741in}{0.979441in}}%
\pgfpathlineto{\pgfqpoint{5.052611in}{1.165863in}}%
\pgfpathlineto{\pgfqpoint{5.053079in}{1.225775in}}%
\pgfpathlineto{\pgfqpoint{5.053547in}{1.181658in}}%
\pgfpathlineto{\pgfqpoint{5.054950in}{0.867242in}}%
\pgfpathlineto{\pgfqpoint{5.055418in}{0.905984in}}%
\pgfpathlineto{\pgfqpoint{5.055652in}{0.905062in}}%
\pgfpathlineto{\pgfqpoint{5.056119in}{0.890102in}}%
\pgfpathlineto{\pgfqpoint{5.056353in}{0.925007in}}%
\pgfpathlineto{\pgfqpoint{5.058224in}{1.213718in}}%
\pgfpathlineto{\pgfqpoint{5.058458in}{1.250809in}}%
\pgfpathlineto{\pgfqpoint{5.059160in}{1.241000in}}%
\pgfpathlineto{\pgfqpoint{5.061030in}{0.878867in}}%
\pgfpathlineto{\pgfqpoint{5.061498in}{0.838335in}}%
\pgfpathlineto{\pgfqpoint{5.063135in}{1.101843in}}%
\pgfpathlineto{\pgfqpoint{5.064071in}{1.241392in}}%
\pgfpathlineto{\pgfqpoint{5.064538in}{1.164327in}}%
\pgfpathlineto{\pgfqpoint{5.064772in}{1.158561in}}%
\pgfpathlineto{\pgfqpoint{5.065941in}{1.041267in}}%
\pgfpathlineto{\pgfqpoint{5.066175in}{1.068120in}}%
\pgfpathlineto{\pgfqpoint{5.066409in}{1.073370in}}%
\pgfpathlineto{\pgfqpoint{5.067812in}{0.876659in}}%
\pgfpathlineto{\pgfqpoint{5.068514in}{0.906370in}}%
\pgfpathlineto{\pgfqpoint{5.071086in}{1.303854in}}%
\pgfpathlineto{\pgfqpoint{5.072723in}{0.881970in}}%
\pgfpathlineto{\pgfqpoint{5.073893in}{0.896089in}}%
\pgfpathlineto{\pgfqpoint{5.076231in}{1.263786in}}%
\pgfpathlineto{\pgfqpoint{5.076699in}{1.295115in}}%
\pgfpathlineto{\pgfqpoint{5.076933in}{1.283102in}}%
\pgfpathlineto{\pgfqpoint{5.078804in}{0.913247in}}%
\pgfpathlineto{\pgfqpoint{5.079038in}{0.928088in}}%
\pgfpathlineto{\pgfqpoint{5.081142in}{1.203231in}}%
\pgfpathlineto{\pgfqpoint{5.082078in}{1.237480in}}%
\pgfpathlineto{\pgfqpoint{5.082312in}{1.174694in}}%
\pgfpathlineto{\pgfqpoint{5.082779in}{1.228167in}}%
\pgfpathlineto{\pgfqpoint{5.083013in}{1.290677in}}%
\pgfpathlineto{\pgfqpoint{5.083715in}{1.242509in}}%
\pgfpathlineto{\pgfqpoint{5.085352in}{1.076301in}}%
\pgfpathlineto{\pgfqpoint{5.085586in}{1.124495in}}%
\pgfpathlineto{\pgfqpoint{5.085819in}{1.159364in}}%
\pgfpathlineto{\pgfqpoint{5.086287in}{1.097163in}}%
\pgfpathlineto{\pgfqpoint{5.086989in}{1.001661in}}%
\pgfpathlineto{\pgfqpoint{5.087456in}{1.073415in}}%
\pgfpathlineto{\pgfqpoint{5.087690in}{1.082538in}}%
\pgfpathlineto{\pgfqpoint{5.087924in}{1.056211in}}%
\pgfpathlineto{\pgfqpoint{5.088158in}{1.026161in}}%
\pgfpathlineto{\pgfqpoint{5.088626in}{1.089235in}}%
\pgfpathlineto{\pgfqpoint{5.089561in}{1.206520in}}%
\pgfpathlineto{\pgfqpoint{5.090029in}{1.129655in}}%
\pgfpathlineto{\pgfqpoint{5.090497in}{1.103624in}}%
\pgfpathlineto{\pgfqpoint{5.090731in}{1.128138in}}%
\pgfpathlineto{\pgfqpoint{5.092134in}{1.342471in}}%
\pgfpathlineto{\pgfqpoint{5.092601in}{1.278355in}}%
\pgfpathlineto{\pgfqpoint{5.094472in}{1.138666in}}%
\pgfpathlineto{\pgfqpoint{5.094940in}{1.160074in}}%
\pgfpathlineto{\pgfqpoint{5.095642in}{1.275331in}}%
\pgfpathlineto{\pgfqpoint{5.096109in}{1.175332in}}%
\pgfpathlineto{\pgfqpoint{5.096343in}{1.155855in}}%
\pgfpathlineto{\pgfqpoint{5.096811in}{1.177470in}}%
\pgfpathlineto{\pgfqpoint{5.097045in}{1.213916in}}%
\pgfpathlineto{\pgfqpoint{5.097746in}{1.194728in}}%
\pgfpathlineto{\pgfqpoint{5.098448in}{1.097667in}}%
\pgfpathlineto{\pgfqpoint{5.098916in}{1.115274in}}%
\pgfpathlineto{\pgfqpoint{5.101020in}{1.320405in}}%
\pgfpathlineto{\pgfqpoint{5.101254in}{1.261759in}}%
\pgfpathlineto{\pgfqpoint{5.102190in}{1.281927in}}%
\pgfpathlineto{\pgfqpoint{5.102423in}{1.299775in}}%
\pgfpathlineto{\pgfqpoint{5.102657in}{1.255150in}}%
\pgfpathlineto{\pgfqpoint{5.102891in}{1.244223in}}%
\pgfpathlineto{\pgfqpoint{5.103125in}{1.256718in}}%
\pgfpathlineto{\pgfqpoint{5.103359in}{1.247854in}}%
\pgfpathlineto{\pgfqpoint{5.103593in}{1.313141in}}%
\pgfpathlineto{\pgfqpoint{5.104294in}{1.245603in}}%
\pgfpathlineto{\pgfqpoint{5.104996in}{1.140690in}}%
\pgfpathlineto{\pgfqpoint{5.105464in}{1.229419in}}%
\pgfpathlineto{\pgfqpoint{5.106867in}{1.306201in}}%
\pgfpathlineto{\pgfqpoint{5.107802in}{1.150398in}}%
\pgfpathlineto{\pgfqpoint{5.108270in}{1.230015in}}%
\pgfpathlineto{\pgfqpoint{5.109907in}{1.331724in}}%
\pgfpathlineto{\pgfqpoint{5.110141in}{1.331356in}}%
\pgfpathlineto{\pgfqpoint{5.112012in}{1.086052in}}%
\pgfpathlineto{\pgfqpoint{5.112246in}{1.086127in}}%
\pgfpathlineto{\pgfqpoint{5.112479in}{1.085803in}}%
\pgfpathlineto{\pgfqpoint{5.115052in}{1.403228in}}%
\pgfpathlineto{\pgfqpoint{5.115286in}{1.396985in}}%
\pgfpathlineto{\pgfqpoint{5.118326in}{1.111494in}}%
\pgfpathlineto{\pgfqpoint{5.121600in}{1.394295in}}%
\pgfpathlineto{\pgfqpoint{5.122068in}{1.351941in}}%
\pgfpathlineto{\pgfqpoint{5.122769in}{1.252620in}}%
\pgfpathlineto{\pgfqpoint{5.123237in}{1.343332in}}%
\pgfpathlineto{\pgfqpoint{5.123471in}{1.378774in}}%
\pgfpathlineto{\pgfqpoint{5.123939in}{1.324742in}}%
\pgfpathlineto{\pgfqpoint{5.124874in}{1.208178in}}%
\pgfpathlineto{\pgfqpoint{5.125108in}{1.230115in}}%
\pgfpathlineto{\pgfqpoint{5.126043in}{1.312081in}}%
\pgfpathlineto{\pgfqpoint{5.126277in}{1.285220in}}%
\pgfpathlineto{\pgfqpoint{5.126745in}{1.256595in}}%
\pgfpathlineto{\pgfqpoint{5.127213in}{1.286409in}}%
\pgfpathlineto{\pgfqpoint{5.127914in}{1.388331in}}%
\pgfpathlineto{\pgfqpoint{5.128616in}{1.333700in}}%
\pgfpathlineto{\pgfqpoint{5.129317in}{1.228927in}}%
\pgfpathlineto{\pgfqpoint{5.129785in}{1.278526in}}%
\pgfpathlineto{\pgfqpoint{5.130019in}{1.245803in}}%
\pgfpathlineto{\pgfqpoint{5.130487in}{1.317289in}}%
\pgfpathlineto{\pgfqpoint{5.130720in}{1.295997in}}%
\pgfpathlineto{\pgfqpoint{5.130954in}{1.279137in}}%
\pgfpathlineto{\pgfqpoint{5.131422in}{1.316980in}}%
\pgfpathlineto{\pgfqpoint{5.132357in}{1.497251in}}%
\pgfpathlineto{\pgfqpoint{5.132825in}{1.436254in}}%
\pgfpathlineto{\pgfqpoint{5.133761in}{1.299700in}}%
\pgfpathlineto{\pgfqpoint{5.134228in}{1.318340in}}%
\pgfpathlineto{\pgfqpoint{5.134930in}{1.302402in}}%
\pgfpathlineto{\pgfqpoint{5.135164in}{1.311049in}}%
\pgfpathlineto{\pgfqpoint{5.135865in}{1.341758in}}%
\pgfpathlineto{\pgfqpoint{5.136099in}{1.338921in}}%
\pgfpathlineto{\pgfqpoint{5.136801in}{1.232457in}}%
\pgfpathlineto{\pgfqpoint{5.137269in}{1.261010in}}%
\pgfpathlineto{\pgfqpoint{5.138906in}{1.369075in}}%
\pgfpathlineto{\pgfqpoint{5.139139in}{1.354786in}}%
\pgfpathlineto{\pgfqpoint{5.140543in}{1.423031in}}%
\pgfpathlineto{\pgfqpoint{5.141712in}{1.168325in}}%
\pgfpathlineto{\pgfqpoint{5.142180in}{1.283780in}}%
\pgfpathlineto{\pgfqpoint{5.144284in}{1.426471in}}%
\pgfpathlineto{\pgfqpoint{5.144986in}{1.360485in}}%
\pgfpathlineto{\pgfqpoint{5.145220in}{1.392711in}}%
\pgfpathlineto{\pgfqpoint{5.146155in}{1.513051in}}%
\pgfpathlineto{\pgfqpoint{5.146389in}{1.448767in}}%
\pgfpathlineto{\pgfqpoint{5.147792in}{1.225497in}}%
\pgfpathlineto{\pgfqpoint{5.148026in}{1.256865in}}%
\pgfpathlineto{\pgfqpoint{5.148260in}{1.230841in}}%
\pgfpathlineto{\pgfqpoint{5.148494in}{1.292863in}}%
\pgfpathlineto{\pgfqpoint{5.148728in}{1.275424in}}%
\pgfpathlineto{\pgfqpoint{5.149195in}{1.309104in}}%
\pgfpathlineto{\pgfqpoint{5.150131in}{1.306445in}}%
\pgfpathlineto{\pgfqpoint{5.150365in}{1.308109in}}%
\pgfpathlineto{\pgfqpoint{5.150832in}{1.377418in}}%
\pgfpathlineto{\pgfqpoint{5.151768in}{1.372829in}}%
\pgfpathlineto{\pgfqpoint{5.152469in}{1.492362in}}%
\pgfpathlineto{\pgfqpoint{5.153405in}{1.483900in}}%
\pgfpathlineto{\pgfqpoint{5.155276in}{1.322370in}}%
\pgfpathlineto{\pgfqpoint{5.156445in}{1.496996in}}%
\pgfpathlineto{\pgfqpoint{5.157147in}{1.414921in}}%
\pgfpathlineto{\pgfqpoint{5.158316in}{1.306885in}}%
\pgfpathlineto{\pgfqpoint{5.158550in}{1.339393in}}%
\pgfpathlineto{\pgfqpoint{5.158784in}{1.372510in}}%
\pgfpathlineto{\pgfqpoint{5.159251in}{1.298307in}}%
\pgfpathlineto{\pgfqpoint{5.159485in}{1.271800in}}%
\pgfpathlineto{\pgfqpoint{5.159953in}{1.276858in}}%
\pgfpathlineto{\pgfqpoint{5.162058in}{1.527865in}}%
\pgfpathlineto{\pgfqpoint{5.162291in}{1.521797in}}%
\pgfpathlineto{\pgfqpoint{5.163461in}{1.468145in}}%
\pgfpathlineto{\pgfqpoint{5.163227in}{1.530893in}}%
\pgfpathlineto{\pgfqpoint{5.163928in}{1.473648in}}%
\pgfpathlineto{\pgfqpoint{5.164396in}{1.528465in}}%
\pgfpathlineto{\pgfqpoint{5.164864in}{1.492736in}}%
\pgfpathlineto{\pgfqpoint{5.167670in}{1.244540in}}%
\pgfpathlineto{\pgfqpoint{5.171178in}{1.617323in}}%
\pgfpathlineto{\pgfqpoint{5.173049in}{1.166557in}}%
\pgfpathlineto{\pgfqpoint{5.173517in}{1.235767in}}%
\pgfpathlineto{\pgfqpoint{5.175855in}{1.593112in}}%
\pgfpathlineto{\pgfqpoint{5.178895in}{1.290834in}}%
\pgfpathlineto{\pgfqpoint{5.179129in}{1.305561in}}%
\pgfpathlineto{\pgfqpoint{5.180299in}{1.364846in}}%
\pgfpathlineto{\pgfqpoint{5.181468in}{1.215608in}}%
\pgfpathlineto{\pgfqpoint{5.181702in}{1.279781in}}%
\pgfpathlineto{\pgfqpoint{5.183573in}{1.566147in}}%
\pgfpathlineto{\pgfqpoint{5.186847in}{1.281481in}}%
\pgfpathlineto{\pgfqpoint{5.187314in}{1.313212in}}%
\pgfpathlineto{\pgfqpoint{5.187548in}{1.333393in}}%
\pgfpathlineto{\pgfqpoint{5.187782in}{1.315435in}}%
\pgfpathlineto{\pgfqpoint{5.188250in}{1.250224in}}%
\pgfpathlineto{\pgfqpoint{5.188718in}{1.342247in}}%
\pgfpathlineto{\pgfqpoint{5.188951in}{1.341718in}}%
\pgfpathlineto{\pgfqpoint{5.190822in}{1.246772in}}%
\pgfpathlineto{\pgfqpoint{5.189419in}{1.350606in}}%
\pgfpathlineto{\pgfqpoint{5.191056in}{1.277537in}}%
\pgfpathlineto{\pgfqpoint{5.192693in}{1.474646in}}%
\pgfpathlineto{\pgfqpoint{5.193161in}{1.449320in}}%
\pgfpathlineto{\pgfqpoint{5.194330in}{1.151695in}}%
\pgfpathlineto{\pgfqpoint{5.195032in}{1.250504in}}%
\pgfpathlineto{\pgfqpoint{5.196201in}{1.375495in}}%
\pgfpathlineto{\pgfqpoint{5.196669in}{1.334158in}}%
\pgfpathlineto{\pgfqpoint{5.196903in}{1.275243in}}%
\pgfpathlineto{\pgfqpoint{5.197604in}{1.319780in}}%
\pgfpathlineto{\pgfqpoint{5.198072in}{1.347189in}}%
\pgfpathlineto{\pgfqpoint{5.198306in}{1.312731in}}%
\pgfpathlineto{\pgfqpoint{5.198540in}{1.289531in}}%
\pgfpathlineto{\pgfqpoint{5.198774in}{1.338088in}}%
\pgfpathlineto{\pgfqpoint{5.199007in}{1.337505in}}%
\pgfpathlineto{\pgfqpoint{5.199943in}{1.377287in}}%
\pgfpathlineto{\pgfqpoint{5.200177in}{1.353833in}}%
\pgfpathlineto{\pgfqpoint{5.202515in}{1.109583in}}%
\pgfpathlineto{\pgfqpoint{5.202983in}{1.125114in}}%
\pgfpathlineto{\pgfqpoint{5.203685in}{1.233212in}}%
\pgfpathlineto{\pgfqpoint{5.204620in}{1.456588in}}%
\pgfpathlineto{\pgfqpoint{5.205322in}{1.338065in}}%
\pgfpathlineto{\pgfqpoint{5.205555in}{1.348342in}}%
\pgfpathlineto{\pgfqpoint{5.206023in}{1.336129in}}%
\pgfpathlineto{\pgfqpoint{5.206257in}{1.320754in}}%
\pgfpathlineto{\pgfqpoint{5.206491in}{1.392471in}}%
\pgfpathlineto{\pgfqpoint{5.206959in}{1.296346in}}%
\pgfpathlineto{\pgfqpoint{5.208362in}{1.163875in}}%
\pgfpathlineto{\pgfqpoint{5.208596in}{1.166613in}}%
\pgfpathlineto{\pgfqpoint{5.208829in}{1.156847in}}%
\pgfpathlineto{\pgfqpoint{5.209531in}{1.128221in}}%
\pgfpathlineto{\pgfqpoint{5.210934in}{1.385778in}}%
\pgfpathlineto{\pgfqpoint{5.211402in}{1.316098in}}%
\pgfpathlineto{\pgfqpoint{5.211870in}{1.223001in}}%
\pgfpathlineto{\pgfqpoint{5.212571in}{1.287170in}}%
\pgfpathlineto{\pgfqpoint{5.212805in}{1.296573in}}%
\pgfpathlineto{\pgfqpoint{5.213273in}{1.277566in}}%
\pgfpathlineto{\pgfqpoint{5.214676in}{1.139042in}}%
\pgfpathlineto{\pgfqpoint{5.216313in}{1.256823in}}%
\pgfpathlineto{\pgfqpoint{5.216547in}{1.273603in}}%
\pgfpathlineto{\pgfqpoint{5.216781in}{1.211442in}}%
\pgfpathlineto{\pgfqpoint{5.217015in}{1.182182in}}%
\pgfpathlineto{\pgfqpoint{5.217482in}{1.232468in}}%
\pgfpathlineto{\pgfqpoint{5.218418in}{1.305136in}}%
\pgfpathlineto{\pgfqpoint{5.217950in}{1.222399in}}%
\pgfpathlineto{\pgfqpoint{5.218652in}{1.295489in}}%
\pgfpathlineto{\pgfqpoint{5.219353in}{1.185095in}}%
\pgfpathlineto{\pgfqpoint{5.220055in}{1.207367in}}%
\pgfpathlineto{\pgfqpoint{5.220289in}{1.207707in}}%
\pgfpathlineto{\pgfqpoint{5.220522in}{1.238136in}}%
\pgfpathlineto{\pgfqpoint{5.220990in}{1.194690in}}%
\pgfpathlineto{\pgfqpoint{5.221224in}{1.159257in}}%
\pgfpathlineto{\pgfqpoint{5.222159in}{1.182088in}}%
\pgfpathlineto{\pgfqpoint{5.222393in}{1.157632in}}%
\pgfpathlineto{\pgfqpoint{5.222627in}{1.203008in}}%
\pgfpathlineto{\pgfqpoint{5.223095in}{1.311772in}}%
\pgfpathlineto{\pgfqpoint{5.224030in}{1.307044in}}%
\pgfpathlineto{\pgfqpoint{5.225200in}{1.211273in}}%
\pgfpathlineto{\pgfqpoint{5.225667in}{1.227345in}}%
\pgfpathlineto{\pgfqpoint{5.226135in}{1.364994in}}%
\pgfpathlineto{\pgfqpoint{5.226837in}{1.253464in}}%
\pgfpathlineto{\pgfqpoint{5.228240in}{1.071947in}}%
\pgfpathlineto{\pgfqpoint{5.228707in}{1.135946in}}%
\pgfpathlineto{\pgfqpoint{5.228941in}{1.141664in}}%
\pgfpathlineto{\pgfqpoint{5.230345in}{1.310255in}}%
\pgfpathlineto{\pgfqpoint{5.230578in}{1.306834in}}%
\pgfpathlineto{\pgfqpoint{5.230812in}{1.311236in}}%
\pgfpathlineto{\pgfqpoint{5.231280in}{1.360899in}}%
\pgfpathlineto{\pgfqpoint{5.231748in}{1.290488in}}%
\pgfpathlineto{\pgfqpoint{5.233151in}{1.178629in}}%
\pgfpathlineto{\pgfqpoint{5.233385in}{1.201471in}}%
\pgfpathlineto{\pgfqpoint{5.234086in}{1.309204in}}%
\pgfpathlineto{\pgfqpoint{5.234320in}{1.244212in}}%
\pgfpathlineto{\pgfqpoint{5.234788in}{1.158450in}}%
\pgfpathlineto{\pgfqpoint{5.235489in}{1.176629in}}%
\pgfpathlineto{\pgfqpoint{5.236893in}{1.251806in}}%
\pgfpathlineto{\pgfqpoint{5.237126in}{1.202791in}}%
\pgfpathlineto{\pgfqpoint{5.237828in}{1.245288in}}%
\pgfpathlineto{\pgfqpoint{5.238763in}{1.240226in}}%
\pgfpathlineto{\pgfqpoint{5.239465in}{1.320418in}}%
\pgfpathlineto{\pgfqpoint{5.241336in}{1.069066in}}%
\pgfpathlineto{\pgfqpoint{5.241804in}{1.106389in}}%
\pgfpathlineto{\pgfqpoint{5.242739in}{1.348570in}}%
\pgfpathlineto{\pgfqpoint{5.243207in}{1.280010in}}%
\pgfpathlineto{\pgfqpoint{5.243441in}{1.282183in}}%
\pgfpathlineto{\pgfqpoint{5.244610in}{1.398025in}}%
\pgfpathlineto{\pgfqpoint{5.245078in}{1.352007in}}%
\pgfpathlineto{\pgfqpoint{5.247650in}{1.104006in}}%
\pgfpathlineto{\pgfqpoint{5.249521in}{1.245688in}}%
\pgfpathlineto{\pgfqpoint{5.249755in}{1.240707in}}%
\pgfpathlineto{\pgfqpoint{5.251626in}{1.433819in}}%
\pgfpathlineto{\pgfqpoint{5.253964in}{1.087941in}}%
\pgfpathlineto{\pgfqpoint{5.254666in}{1.155532in}}%
\pgfpathlineto{\pgfqpoint{5.255367in}{1.252984in}}%
\pgfpathlineto{\pgfqpoint{5.256069in}{1.444669in}}%
\pgfpathlineto{\pgfqpoint{5.256537in}{1.410587in}}%
\pgfpathlineto{\pgfqpoint{5.257238in}{1.250650in}}%
\pgfpathlineto{\pgfqpoint{5.257940in}{1.272313in}}%
\pgfpathlineto{\pgfqpoint{5.258641in}{1.236334in}}%
\pgfpathlineto{\pgfqpoint{5.258875in}{1.240023in}}%
\pgfpathlineto{\pgfqpoint{5.259811in}{1.305870in}}%
\pgfpathlineto{\pgfqpoint{5.260045in}{1.294654in}}%
\pgfpathlineto{\pgfqpoint{5.262149in}{1.021326in}}%
\pgfpathlineto{\pgfqpoint{5.264722in}{1.421620in}}%
\pgfpathlineto{\pgfqpoint{5.265190in}{1.358803in}}%
\pgfpathlineto{\pgfqpoint{5.267528in}{1.061480in}}%
\pgfpathlineto{\pgfqpoint{5.270101in}{1.336716in}}%
\pgfpathlineto{\pgfqpoint{5.270568in}{1.367462in}}%
\pgfpathlineto{\pgfqpoint{5.271036in}{1.341098in}}%
\pgfpathlineto{\pgfqpoint{5.271270in}{1.338676in}}%
\pgfpathlineto{\pgfqpoint{5.271738in}{1.425015in}}%
\pgfpathlineto{\pgfqpoint{5.272205in}{1.354185in}}%
\pgfpathlineto{\pgfqpoint{5.274310in}{1.071192in}}%
\pgfpathlineto{\pgfqpoint{5.274544in}{1.080019in}}%
\pgfpathlineto{\pgfqpoint{5.276415in}{1.380164in}}%
\pgfpathlineto{\pgfqpoint{5.277818in}{1.272894in}}%
\pgfpathlineto{\pgfqpoint{5.278753in}{1.343277in}}%
\pgfpathlineto{\pgfqpoint{5.278987in}{1.298112in}}%
\pgfpathlineto{\pgfqpoint{5.279923in}{1.027995in}}%
\pgfpathlineto{\pgfqpoint{5.280624in}{1.092649in}}%
\pgfpathlineto{\pgfqpoint{5.281092in}{1.085023in}}%
\pgfpathlineto{\pgfqpoint{5.281794in}{1.169615in}}%
\pgfpathlineto{\pgfqpoint{5.282729in}{1.356445in}}%
\pgfpathlineto{\pgfqpoint{5.282963in}{1.307790in}}%
\pgfpathlineto{\pgfqpoint{5.283197in}{1.268488in}}%
\pgfpathlineto{\pgfqpoint{5.283898in}{1.325346in}}%
\pgfpathlineto{\pgfqpoint{5.284132in}{1.340012in}}%
\pgfpathlineto{\pgfqpoint{5.284600in}{1.319028in}}%
\pgfpathlineto{\pgfqpoint{5.288342in}{1.073076in}}%
\pgfpathlineto{\pgfqpoint{5.288809in}{1.087346in}}%
\pgfpathlineto{\pgfqpoint{5.290680in}{1.263089in}}%
\pgfpathlineto{\pgfqpoint{5.290914in}{1.260795in}}%
\pgfpathlineto{\pgfqpoint{5.291616in}{1.304500in}}%
\pgfpathlineto{\pgfqpoint{5.291850in}{1.283358in}}%
\pgfpathlineto{\pgfqpoint{5.292083in}{1.197552in}}%
\pgfpathlineto{\pgfqpoint{5.292785in}{1.254411in}}%
\pgfpathlineto{\pgfqpoint{5.293954in}{1.379529in}}%
\pgfpathlineto{\pgfqpoint{5.296293in}{1.064170in}}%
\pgfpathlineto{\pgfqpoint{5.296761in}{1.050038in}}%
\pgfpathlineto{\pgfqpoint{5.298865in}{1.303742in}}%
\pgfpathlineto{\pgfqpoint{5.300268in}{1.141978in}}%
\pgfpathlineto{\pgfqpoint{5.300736in}{1.160202in}}%
\pgfpathlineto{\pgfqpoint{5.301204in}{1.263073in}}%
\pgfpathlineto{\pgfqpoint{5.301905in}{1.387353in}}%
\pgfpathlineto{\pgfqpoint{5.302607in}{1.362560in}}%
\pgfpathlineto{\pgfqpoint{5.305413in}{0.977274in}}%
\pgfpathlineto{\pgfqpoint{5.307752in}{1.404020in}}%
\pgfpathlineto{\pgfqpoint{5.308454in}{1.320224in}}%
\pgfpathlineto{\pgfqpoint{5.309623in}{1.135821in}}%
\pgfpathlineto{\pgfqpoint{5.310091in}{1.043933in}}%
\pgfpathlineto{\pgfqpoint{5.310792in}{1.055466in}}%
\pgfpathlineto{\pgfqpoint{5.311260in}{1.053898in}}%
\pgfpathlineto{\pgfqpoint{5.312195in}{1.092578in}}%
\pgfpathlineto{\pgfqpoint{5.312429in}{1.072229in}}%
\pgfpathlineto{\pgfqpoint{5.312897in}{1.109592in}}%
\pgfpathlineto{\pgfqpoint{5.314066in}{1.279036in}}%
\pgfpathlineto{\pgfqpoint{5.314300in}{1.275539in}}%
\pgfpathlineto{\pgfqpoint{5.315002in}{1.124034in}}%
\pgfpathlineto{\pgfqpoint{5.315469in}{1.218400in}}%
\pgfpathlineto{\pgfqpoint{5.315937in}{1.307451in}}%
\pgfpathlineto{\pgfqpoint{5.316405in}{1.243133in}}%
\pgfpathlineto{\pgfqpoint{5.316872in}{1.121280in}}%
\pgfpathlineto{\pgfqpoint{5.317808in}{1.124015in}}%
\pgfpathlineto{\pgfqpoint{5.318509in}{1.114367in}}%
\pgfpathlineto{\pgfqpoint{5.318977in}{1.209106in}}%
\pgfpathlineto{\pgfqpoint{5.319679in}{1.152089in}}%
\pgfpathlineto{\pgfqpoint{5.320614in}{1.070002in}}%
\pgfpathlineto{\pgfqpoint{5.321082in}{1.088816in}}%
\pgfpathlineto{\pgfqpoint{5.321316in}{1.094407in}}%
\pgfpathlineto{\pgfqpoint{5.322017in}{1.144478in}}%
\pgfpathlineto{\pgfqpoint{5.322251in}{1.141776in}}%
\pgfpathlineto{\pgfqpoint{5.322953in}{0.974779in}}%
\pgfpathlineto{\pgfqpoint{5.323421in}{1.071800in}}%
\pgfpathlineto{\pgfqpoint{5.326695in}{1.248120in}}%
\pgfpathlineto{\pgfqpoint{5.324122in}{1.060824in}}%
\pgfpathlineto{\pgfqpoint{5.326928in}{1.247185in}}%
\pgfpathlineto{\pgfqpoint{5.327396in}{1.141368in}}%
\pgfpathlineto{\pgfqpoint{5.328332in}{1.158531in}}%
\pgfpathlineto{\pgfqpoint{5.329969in}{0.980836in}}%
\pgfpathlineto{\pgfqpoint{5.330202in}{1.032832in}}%
\pgfpathlineto{\pgfqpoint{5.330670in}{1.116297in}}%
\pgfpathlineto{\pgfqpoint{5.331138in}{1.071232in}}%
\pgfpathlineto{\pgfqpoint{5.331839in}{0.956096in}}%
\pgfpathlineto{\pgfqpoint{5.332073in}{1.013565in}}%
\pgfpathlineto{\pgfqpoint{5.333009in}{1.224166in}}%
\pgfpathlineto{\pgfqpoint{5.333944in}{1.189425in}}%
\pgfpathlineto{\pgfqpoint{5.334178in}{1.191590in}}%
\pgfpathlineto{\pgfqpoint{5.334880in}{1.254721in}}%
\pgfpathlineto{\pgfqpoint{5.335113in}{1.179948in}}%
\pgfpathlineto{\pgfqpoint{5.336283in}{1.019590in}}%
\pgfpathlineto{\pgfqpoint{5.336517in}{1.041282in}}%
\pgfpathlineto{\pgfqpoint{5.336750in}{1.077260in}}%
\pgfpathlineto{\pgfqpoint{5.337218in}{1.026917in}}%
\pgfpathlineto{\pgfqpoint{5.337686in}{0.927436in}}%
\pgfpathlineto{\pgfqpoint{5.338388in}{1.009240in}}%
\pgfpathlineto{\pgfqpoint{5.340726in}{1.186308in}}%
\pgfpathlineto{\pgfqpoint{5.342129in}{1.105979in}}%
\pgfpathlineto{\pgfqpoint{5.343299in}{1.215222in}}%
\pgfpathlineto{\pgfqpoint{5.343766in}{1.172617in}}%
\pgfpathlineto{\pgfqpoint{5.345169in}{0.999658in}}%
\pgfpathlineto{\pgfqpoint{5.345403in}{1.010187in}}%
\pgfpathlineto{\pgfqpoint{5.346339in}{0.917458in}}%
\pgfpathlineto{\pgfqpoint{5.346573in}{0.958574in}}%
\pgfpathlineto{\pgfqpoint{5.348210in}{1.104822in}}%
\pgfpathlineto{\pgfqpoint{5.348677in}{1.064421in}}%
\pgfpathlineto{\pgfqpoint{5.348911in}{1.059797in}}%
\pgfpathlineto{\pgfqpoint{5.350548in}{1.153173in}}%
\pgfpathlineto{\pgfqpoint{5.351250in}{1.061366in}}%
\pgfpathlineto{\pgfqpoint{5.351717in}{1.093184in}}%
\pgfpathlineto{\pgfqpoint{5.352419in}{1.080121in}}%
\pgfpathlineto{\pgfqpoint{5.352887in}{1.162372in}}%
\pgfpathlineto{\pgfqpoint{5.354290in}{1.078767in}}%
\pgfpathlineto{\pgfqpoint{5.355225in}{1.096618in}}%
\pgfpathlineto{\pgfqpoint{5.355693in}{1.031733in}}%
\pgfpathlineto{\pgfqpoint{5.356395in}{1.028507in}}%
\pgfpathlineto{\pgfqpoint{5.356629in}{1.044611in}}%
\pgfpathlineto{\pgfqpoint{5.357564in}{0.930014in}}%
\pgfpathlineto{\pgfqpoint{5.358266in}{0.946844in}}%
\pgfpathlineto{\pgfqpoint{5.361072in}{1.186552in}}%
\pgfpathlineto{\pgfqpoint{5.361773in}{1.228761in}}%
\pgfpathlineto{\pgfqpoint{5.362007in}{1.214123in}}%
\pgfpathlineto{\pgfqpoint{5.363878in}{0.923367in}}%
\pgfpathlineto{\pgfqpoint{5.364346in}{0.902701in}}%
\pgfpathlineto{\pgfqpoint{5.364580in}{0.921161in}}%
\pgfpathlineto{\pgfqpoint{5.367386in}{1.173438in}}%
\pgfpathlineto{\pgfqpoint{5.367620in}{1.137487in}}%
\pgfpathlineto{\pgfqpoint{5.367854in}{1.060955in}}%
\pgfpathlineto{\pgfqpoint{5.368555in}{1.106473in}}%
\pgfpathlineto{\pgfqpoint{5.369959in}{1.205604in}}%
\pgfpathlineto{\pgfqpoint{5.370894in}{1.069579in}}%
\pgfpathlineto{\pgfqpoint{5.372765in}{0.889375in}}%
\pgfpathlineto{\pgfqpoint{5.372999in}{0.889968in}}%
\pgfpathlineto{\pgfqpoint{5.374402in}{1.239798in}}%
\pgfpathlineto{\pgfqpoint{5.375103in}{1.127447in}}%
\pgfpathlineto{\pgfqpoint{5.376507in}{0.943912in}}%
\pgfpathlineto{\pgfqpoint{5.379313in}{1.235373in}}%
\pgfpathlineto{\pgfqpoint{5.379547in}{1.201142in}}%
\pgfpathlineto{\pgfqpoint{5.381885in}{0.871511in}}%
\pgfpathlineto{\pgfqpoint{5.382353in}{0.946316in}}%
\pgfpathlineto{\pgfqpoint{5.383288in}{1.014239in}}%
\pgfpathlineto{\pgfqpoint{5.383990in}{1.204387in}}%
\pgfpathlineto{\pgfqpoint{5.385159in}{1.165899in}}%
\pgfpathlineto{\pgfqpoint{5.387732in}{0.989961in}}%
\pgfpathlineto{\pgfqpoint{5.387966in}{1.040161in}}%
\pgfpathlineto{\pgfqpoint{5.388433in}{0.967098in}}%
\pgfpathlineto{\pgfqpoint{5.388667in}{0.947254in}}%
\pgfpathlineto{\pgfqpoint{5.388901in}{0.995037in}}%
\pgfpathlineto{\pgfqpoint{5.389837in}{1.142493in}}%
\pgfpathlineto{\pgfqpoint{5.390304in}{1.068025in}}%
\pgfpathlineto{\pgfqpoint{5.391006in}{1.068057in}}%
\pgfpathlineto{\pgfqpoint{5.391474in}{1.013408in}}%
\pgfpathlineto{\pgfqpoint{5.392175in}{1.205348in}}%
\pgfpathlineto{\pgfqpoint{5.392643in}{1.106006in}}%
\pgfpathlineto{\pgfqpoint{5.393111in}{0.995375in}}%
\pgfpathlineto{\pgfqpoint{5.393812in}{1.062972in}}%
\pgfpathlineto{\pgfqpoint{5.395449in}{0.957370in}}%
\pgfpathlineto{\pgfqpoint{5.397554in}{1.232343in}}%
\pgfpathlineto{\pgfqpoint{5.398957in}{0.971103in}}%
\pgfpathlineto{\pgfqpoint{5.399191in}{0.980673in}}%
\pgfpathlineto{\pgfqpoint{5.400126in}{1.116476in}}%
\pgfpathlineto{\pgfqpoint{5.400828in}{1.090001in}}%
\pgfpathlineto{\pgfqpoint{5.401997in}{0.872559in}}%
\pgfpathlineto{\pgfqpoint{5.402465in}{0.952013in}}%
\pgfpathlineto{\pgfqpoint{5.403868in}{1.128665in}}%
\pgfpathlineto{\pgfqpoint{5.404102in}{1.118753in}}%
\pgfpathlineto{\pgfqpoint{5.405037in}{1.060111in}}%
\pgfpathlineto{\pgfqpoint{5.405271in}{1.087781in}}%
\pgfpathlineto{\pgfqpoint{5.405739in}{1.198412in}}%
\pgfpathlineto{\pgfqpoint{5.406207in}{1.144086in}}%
\pgfpathlineto{\pgfqpoint{5.407376in}{0.961705in}}%
\pgfpathlineto{\pgfqpoint{5.407610in}{0.975347in}}%
\pgfpathlineto{\pgfqpoint{5.407844in}{1.001208in}}%
\pgfpathlineto{\pgfqpoint{5.408311in}{0.957619in}}%
\pgfpathlineto{\pgfqpoint{5.408545in}{0.891051in}}%
\pgfpathlineto{\pgfqpoint{5.409247in}{0.970938in}}%
\pgfpathlineto{\pgfqpoint{5.410650in}{1.046456in}}%
\pgfpathlineto{\pgfqpoint{5.410884in}{1.018294in}}%
\pgfpathlineto{\pgfqpoint{5.411819in}{1.025721in}}%
\pgfpathlineto{\pgfqpoint{5.412989in}{1.236763in}}%
\pgfpathlineto{\pgfqpoint{5.413456in}{1.179699in}}%
\pgfpathlineto{\pgfqpoint{5.414859in}{0.985835in}}%
\pgfpathlineto{\pgfqpoint{5.415093in}{0.980322in}}%
\pgfpathlineto{\pgfqpoint{5.416263in}{1.067736in}}%
\pgfpathlineto{\pgfqpoint{5.417666in}{0.869441in}}%
\pgfpathlineto{\pgfqpoint{5.417900in}{0.872454in}}%
\pgfpathlineto{\pgfqpoint{5.418601in}{1.002785in}}%
\pgfpathlineto{\pgfqpoint{5.419537in}{1.213541in}}%
\pgfpathlineto{\pgfqpoint{5.420004in}{1.205472in}}%
\pgfpathlineto{\pgfqpoint{5.422343in}{1.040494in}}%
\pgfpathlineto{\pgfqpoint{5.422811in}{1.022099in}}%
\pgfpathlineto{\pgfqpoint{5.423746in}{1.073237in}}%
\pgfpathlineto{\pgfqpoint{5.424214in}{0.975089in}}%
\pgfpathlineto{\pgfqpoint{5.424915in}{0.994353in}}%
\pgfpathlineto{\pgfqpoint{5.425149in}{1.040052in}}%
\pgfpathlineto{\pgfqpoint{5.425851in}{0.989654in}}%
\pgfpathlineto{\pgfqpoint{5.426085in}{0.956458in}}%
\pgfpathlineto{\pgfqpoint{5.426552in}{0.965394in}}%
\pgfpathlineto{\pgfqpoint{5.427488in}{1.161185in}}%
\pgfpathlineto{\pgfqpoint{5.428189in}{1.109935in}}%
\pgfpathlineto{\pgfqpoint{5.428891in}{1.182702in}}%
\pgfpathlineto{\pgfqpoint{5.429593in}{1.125332in}}%
\pgfpathlineto{\pgfqpoint{5.429826in}{1.113080in}}%
\pgfpathlineto{\pgfqpoint{5.430294in}{1.198654in}}%
\pgfpathlineto{\pgfqpoint{5.430762in}{1.137748in}}%
\pgfpathlineto{\pgfqpoint{5.431931in}{1.016209in}}%
\pgfpathlineto{\pgfqpoint{5.432399in}{1.047097in}}%
\pgfpathlineto{\pgfqpoint{5.433802in}{1.228777in}}%
\pgfpathlineto{\pgfqpoint{5.434504in}{1.206309in}}%
\pgfpathlineto{\pgfqpoint{5.436141in}{0.929827in}}%
\pgfpathlineto{\pgfqpoint{5.436608in}{1.004267in}}%
\pgfpathlineto{\pgfqpoint{5.437310in}{1.132164in}}%
\pgfpathlineto{\pgfqpoint{5.438713in}{1.236220in}}%
\pgfpathlineto{\pgfqpoint{5.438947in}{1.241307in}}%
\pgfpathlineto{\pgfqpoint{5.439181in}{1.239764in}}%
\pgfpathlineto{\pgfqpoint{5.440350in}{1.179334in}}%
\pgfpathlineto{\pgfqpoint{5.439649in}{1.242520in}}%
\pgfpathlineto{\pgfqpoint{5.440584in}{1.184531in}}%
\pgfpathlineto{\pgfqpoint{5.440818in}{1.212871in}}%
\pgfpathlineto{\pgfqpoint{5.441052in}{1.172018in}}%
\pgfpathlineto{\pgfqpoint{5.442455in}{1.062028in}}%
\pgfpathlineto{\pgfqpoint{5.442923in}{1.098209in}}%
\pgfpathlineto{\pgfqpoint{5.443390in}{1.097348in}}%
\pgfpathlineto{\pgfqpoint{5.444092in}{1.051358in}}%
\pgfpathlineto{\pgfqpoint{5.444326in}{1.056936in}}%
\pgfpathlineto{\pgfqpoint{5.444793in}{1.166856in}}%
\pgfpathlineto{\pgfqpoint{5.445729in}{1.154847in}}%
\pgfpathlineto{\pgfqpoint{5.446430in}{1.169070in}}%
\pgfpathlineto{\pgfqpoint{5.446664in}{1.132747in}}%
\pgfpathlineto{\pgfqpoint{5.448301in}{1.290776in}}%
\pgfpathlineto{\pgfqpoint{5.451809in}{1.024283in}}%
\pgfpathlineto{\pgfqpoint{5.452745in}{1.109380in}}%
\pgfpathlineto{\pgfqpoint{5.453914in}{1.309983in}}%
\pgfpathlineto{\pgfqpoint{5.454382in}{1.289939in}}%
\pgfpathlineto{\pgfqpoint{5.455317in}{1.311579in}}%
\pgfpathlineto{\pgfqpoint{5.455551in}{1.310917in}}%
\pgfpathlineto{\pgfqpoint{5.457422in}{1.012397in}}%
\pgfpathlineto{\pgfqpoint{5.457656in}{0.981362in}}%
\pgfpathlineto{\pgfqpoint{5.458123in}{1.025426in}}%
\pgfpathlineto{\pgfqpoint{5.459760in}{1.342498in}}%
\pgfpathlineto{\pgfqpoint{5.459994in}{1.322496in}}%
\pgfpathlineto{\pgfqpoint{5.461631in}{1.169105in}}%
\pgfpathlineto{\pgfqpoint{5.462801in}{1.324717in}}%
\pgfpathlineto{\pgfqpoint{5.463268in}{1.273080in}}%
\pgfpathlineto{\pgfqpoint{5.465139in}{0.979557in}}%
\pgfpathlineto{\pgfqpoint{5.465607in}{1.042361in}}%
\pgfpathlineto{\pgfqpoint{5.467010in}{1.189018in}}%
\pgfpathlineto{\pgfqpoint{5.467946in}{1.255824in}}%
\pgfpathlineto{\pgfqpoint{5.468881in}{1.371745in}}%
\pgfpathlineto{\pgfqpoint{5.469115in}{1.340818in}}%
\pgfpathlineto{\pgfqpoint{5.470986in}{1.127842in}}%
\pgfpathlineto{\pgfqpoint{5.471687in}{1.015797in}}%
\pgfpathlineto{\pgfqpoint{5.472155in}{1.068237in}}%
\pgfpathlineto{\pgfqpoint{5.472389in}{1.056266in}}%
\pgfpathlineto{\pgfqpoint{5.472623in}{1.064960in}}%
\pgfpathlineto{\pgfqpoint{5.474026in}{1.308259in}}%
\pgfpathlineto{\pgfqpoint{5.474260in}{1.291277in}}%
\pgfpathlineto{\pgfqpoint{5.474494in}{1.328488in}}%
\pgfpathlineto{\pgfqpoint{5.474961in}{1.278269in}}%
\pgfpathlineto{\pgfqpoint{5.477066in}{1.076314in}}%
\pgfpathlineto{\pgfqpoint{5.478002in}{1.090437in}}%
\pgfpathlineto{\pgfqpoint{5.479171in}{1.308553in}}%
\pgfpathlineto{\pgfqpoint{5.479639in}{1.245316in}}%
\pgfpathlineto{\pgfqpoint{5.479872in}{1.270307in}}%
\pgfpathlineto{\pgfqpoint{5.480106in}{1.250048in}}%
\pgfpathlineto{\pgfqpoint{5.480340in}{1.188696in}}%
\pgfpathlineto{\pgfqpoint{5.481042in}{1.235319in}}%
\pgfpathlineto{\pgfqpoint{5.481276in}{1.240573in}}%
\pgfpathlineto{\pgfqpoint{5.481509in}{1.222219in}}%
\pgfpathlineto{\pgfqpoint{5.481743in}{1.224619in}}%
\pgfpathlineto{\pgfqpoint{5.481977in}{1.253635in}}%
\pgfpathlineto{\pgfqpoint{5.482445in}{1.240210in}}%
\pgfpathlineto{\pgfqpoint{5.483146in}{1.145065in}}%
\pgfpathlineto{\pgfqpoint{5.483848in}{1.162492in}}%
\pgfpathlineto{\pgfqpoint{5.484082in}{1.165432in}}%
\pgfpathlineto{\pgfqpoint{5.484316in}{1.136632in}}%
\pgfpathlineto{\pgfqpoint{5.485017in}{1.155611in}}%
\pgfpathlineto{\pgfqpoint{5.486187in}{1.361609in}}%
\pgfpathlineto{\pgfqpoint{5.486420in}{1.297168in}}%
\pgfpathlineto{\pgfqpoint{5.487356in}{1.148472in}}%
\pgfpathlineto{\pgfqpoint{5.487590in}{1.241986in}}%
\pgfpathlineto{\pgfqpoint{5.488057in}{1.304350in}}%
\pgfpathlineto{\pgfqpoint{5.488759in}{1.283850in}}%
\pgfpathlineto{\pgfqpoint{5.488993in}{1.269108in}}%
\pgfpathlineto{\pgfqpoint{5.490396in}{1.104523in}}%
\pgfpathlineto{\pgfqpoint{5.491098in}{1.141591in}}%
\pgfpathlineto{\pgfqpoint{5.491565in}{1.126783in}}%
\pgfpathlineto{\pgfqpoint{5.492267in}{1.107531in}}%
\pgfpathlineto{\pgfqpoint{5.492501in}{1.113615in}}%
\pgfpathlineto{\pgfqpoint{5.494839in}{1.409170in}}%
\pgfpathlineto{\pgfqpoint{5.495541in}{1.358811in}}%
\pgfpathlineto{\pgfqpoint{5.497646in}{1.125541in}}%
\pgfpathlineto{\pgfqpoint{5.497880in}{1.117682in}}%
\pgfpathlineto{\pgfqpoint{5.498113in}{1.132760in}}%
\pgfpathlineto{\pgfqpoint{5.500218in}{1.258808in}}%
\pgfpathlineto{\pgfqpoint{5.500920in}{1.134959in}}%
\pgfpathlineto{\pgfqpoint{5.502089in}{1.161315in}}%
\pgfpathlineto{\pgfqpoint{5.502557in}{1.116046in}}%
\pgfpathlineto{\pgfqpoint{5.502791in}{1.169069in}}%
\pgfpathlineto{\pgfqpoint{5.504895in}{1.348810in}}%
\pgfpathlineto{\pgfqpoint{5.505129in}{1.313461in}}%
\pgfpathlineto{\pgfqpoint{5.505597in}{1.373392in}}%
\pgfpathlineto{\pgfqpoint{5.505831in}{1.410426in}}%
\pgfpathlineto{\pgfqpoint{5.506065in}{1.363065in}}%
\pgfpathlineto{\pgfqpoint{5.508169in}{0.997961in}}%
\pgfpathlineto{\pgfqpoint{5.510742in}{1.448146in}}%
\pgfpathlineto{\pgfqpoint{5.510976in}{1.447136in}}%
\pgfpathlineto{\pgfqpoint{5.511443in}{1.377610in}}%
\pgfpathlineto{\pgfqpoint{5.512145in}{1.071599in}}%
\pgfpathlineto{\pgfqpoint{5.512847in}{1.124166in}}%
\pgfpathlineto{\pgfqpoint{5.513314in}{1.113400in}}%
\pgfpathlineto{\pgfqpoint{5.514016in}{1.237120in}}%
\pgfpathlineto{\pgfqpoint{5.514951in}{1.189418in}}%
\pgfpathlineto{\pgfqpoint{5.515419in}{1.233277in}}%
\pgfpathlineto{\pgfqpoint{5.515887in}{1.322200in}}%
\pgfpathlineto{\pgfqpoint{5.516588in}{1.290283in}}%
\pgfpathlineto{\pgfqpoint{5.517290in}{1.234876in}}%
\pgfpathlineto{\pgfqpoint{5.517524in}{1.257168in}}%
\pgfpathlineto{\pgfqpoint{5.518225in}{1.328488in}}%
\pgfpathlineto{\pgfqpoint{5.518459in}{1.280041in}}%
\pgfpathlineto{\pgfqpoint{5.518927in}{1.185886in}}%
\pgfpathlineto{\pgfqpoint{5.519628in}{1.236088in}}%
\pgfpathlineto{\pgfqpoint{5.519862in}{1.303277in}}%
\pgfpathlineto{\pgfqpoint{5.520798in}{1.282599in}}%
\pgfpathlineto{\pgfqpoint{5.521032in}{1.229049in}}%
\pgfpathlineto{\pgfqpoint{5.521499in}{1.317357in}}%
\pgfpathlineto{\pgfqpoint{5.521733in}{1.298941in}}%
\pgfpathlineto{\pgfqpoint{5.523838in}{1.122856in}}%
\pgfpathlineto{\pgfqpoint{5.524072in}{1.147853in}}%
\pgfpathlineto{\pgfqpoint{5.525241in}{1.221443in}}%
\pgfpathlineto{\pgfqpoint{5.525475in}{1.207331in}}%
\pgfpathlineto{\pgfqpoint{5.525709in}{1.164786in}}%
\pgfpathlineto{\pgfqpoint{5.526644in}{1.169509in}}%
\pgfpathlineto{\pgfqpoint{5.528983in}{1.427147in}}%
\pgfpathlineto{\pgfqpoint{5.532023in}{1.068924in}}%
\pgfpathlineto{\pgfqpoint{5.532958in}{1.177332in}}%
\pgfpathlineto{\pgfqpoint{5.533894in}{1.348294in}}%
\pgfpathlineto{\pgfqpoint{5.534362in}{1.278933in}}%
\pgfpathlineto{\pgfqpoint{5.534595in}{1.245132in}}%
\pgfpathlineto{\pgfqpoint{5.534829in}{1.281312in}}%
\pgfpathlineto{\pgfqpoint{5.535063in}{1.278035in}}%
\pgfpathlineto{\pgfqpoint{5.536232in}{1.444684in}}%
\pgfpathlineto{\pgfqpoint{5.536466in}{1.419718in}}%
\pgfpathlineto{\pgfqpoint{5.539506in}{1.056067in}}%
\pgfpathlineto{\pgfqpoint{5.540442in}{1.203194in}}%
\pgfpathlineto{\pgfqpoint{5.541845in}{1.483457in}}%
\pgfpathlineto{\pgfqpoint{5.542313in}{1.423303in}}%
\pgfpathlineto{\pgfqpoint{5.543950in}{1.147653in}}%
\pgfpathlineto{\pgfqpoint{5.544418in}{1.162811in}}%
\pgfpathlineto{\pgfqpoint{5.545353in}{1.369233in}}%
\pgfpathlineto{\pgfqpoint{5.546055in}{1.306646in}}%
\pgfpathlineto{\pgfqpoint{5.546288in}{1.333347in}}%
\pgfpathlineto{\pgfqpoint{5.546990in}{1.298040in}}%
\pgfpathlineto{\pgfqpoint{5.547458in}{1.260093in}}%
\pgfpathlineto{\pgfqpoint{5.547925in}{1.300855in}}%
\pgfpathlineto{\pgfqpoint{5.548393in}{1.307788in}}%
\pgfpathlineto{\pgfqpoint{5.549095in}{1.351574in}}%
\pgfpathlineto{\pgfqpoint{5.549329in}{1.319912in}}%
\pgfpathlineto{\pgfqpoint{5.550732in}{1.223810in}}%
\pgfpathlineto{\pgfqpoint{5.550966in}{1.229455in}}%
\pgfpathlineto{\pgfqpoint{5.551901in}{1.387905in}}%
\pgfpathlineto{\pgfqpoint{5.552836in}{1.379019in}}%
\pgfpathlineto{\pgfqpoint{5.554240in}{1.201620in}}%
\pgfpathlineto{\pgfqpoint{5.556111in}{1.407122in}}%
\pgfpathlineto{\pgfqpoint{5.556344in}{1.388304in}}%
\pgfpathlineto{\pgfqpoint{5.556578in}{1.392091in}}%
\pgfpathlineto{\pgfqpoint{5.557280in}{1.457390in}}%
\pgfpathlineto{\pgfqpoint{5.557748in}{1.420916in}}%
\pgfpathlineto{\pgfqpoint{5.559385in}{1.305018in}}%
\pgfpathlineto{\pgfqpoint{5.559852in}{1.365849in}}%
\pgfpathlineto{\pgfqpoint{5.560086in}{1.351311in}}%
\pgfpathlineto{\pgfqpoint{5.561022in}{1.193603in}}%
\pgfpathlineto{\pgfqpoint{5.561255in}{1.203720in}}%
\pgfpathlineto{\pgfqpoint{5.561957in}{1.272466in}}%
\pgfpathlineto{\pgfqpoint{5.562425in}{1.271549in}}%
\pgfpathlineto{\pgfqpoint{5.562659in}{1.239488in}}%
\pgfpathlineto{\pgfqpoint{5.563126in}{1.269192in}}%
\pgfpathlineto{\pgfqpoint{5.565699in}{1.530622in}}%
\pgfpathlineto{\pgfqpoint{5.565933in}{1.510036in}}%
\pgfpathlineto{\pgfqpoint{5.567336in}{1.365220in}}%
\pgfpathlineto{\pgfqpoint{5.567570in}{1.372361in}}%
\pgfpathlineto{\pgfqpoint{5.567803in}{1.354741in}}%
\pgfpathlineto{\pgfqpoint{5.568739in}{1.359837in}}%
\pgfpathlineto{\pgfqpoint{5.569207in}{1.275667in}}%
\pgfpathlineto{\pgfqpoint{5.569440in}{1.298226in}}%
\pgfpathlineto{\pgfqpoint{5.569674in}{1.246106in}}%
\pgfpathlineto{\pgfqpoint{5.570376in}{1.188807in}}%
\pgfpathlineto{\pgfqpoint{5.570610in}{1.239549in}}%
\pgfpathlineto{\pgfqpoint{5.571779in}{1.359826in}}%
\pgfpathlineto{\pgfqpoint{5.572247in}{1.334924in}}%
\pgfpathlineto{\pgfqpoint{5.572715in}{1.293234in}}%
\pgfpathlineto{\pgfqpoint{5.572948in}{1.336988in}}%
\pgfpathlineto{\pgfqpoint{5.574585in}{1.560118in}}%
\pgfpathlineto{\pgfqpoint{5.575053in}{1.480050in}}%
\pgfpathlineto{\pgfqpoint{5.575755in}{1.304176in}}%
\pgfpathlineto{\pgfqpoint{5.576456in}{1.311000in}}%
\pgfpathlineto{\pgfqpoint{5.577626in}{1.184468in}}%
\pgfpathlineto{\pgfqpoint{5.577859in}{1.188561in}}%
\pgfpathlineto{\pgfqpoint{5.579964in}{1.520131in}}%
\pgfpathlineto{\pgfqpoint{5.580198in}{1.512520in}}%
\pgfpathlineto{\pgfqpoint{5.582770in}{1.318336in}}%
\pgfpathlineto{\pgfqpoint{5.583004in}{1.316800in}}%
\pgfpathlineto{\pgfqpoint{5.583472in}{1.183611in}}%
\pgfpathlineto{\pgfqpoint{5.584174in}{1.282300in}}%
\pgfpathlineto{\pgfqpoint{5.584407in}{1.265228in}}%
\pgfpathlineto{\pgfqpoint{5.584641in}{1.297884in}}%
\pgfpathlineto{\pgfqpoint{5.586980in}{1.587598in}}%
\pgfpathlineto{\pgfqpoint{5.587214in}{1.542435in}}%
\pgfpathlineto{\pgfqpoint{5.589085in}{1.133392in}}%
\pgfpathlineto{\pgfqpoint{5.589786in}{1.218170in}}%
\pgfpathlineto{\pgfqpoint{5.591189in}{1.521230in}}%
\pgfpathlineto{\pgfqpoint{5.591423in}{1.476676in}}%
\pgfpathlineto{\pgfqpoint{5.591891in}{1.537659in}}%
\pgfpathlineto{\pgfqpoint{5.592125in}{1.523128in}}%
\pgfpathlineto{\pgfqpoint{5.593762in}{1.347858in}}%
\pgfpathlineto{\pgfqpoint{5.593996in}{1.350421in}}%
\pgfpathlineto{\pgfqpoint{5.595399in}{1.263046in}}%
\pgfpathlineto{\pgfqpoint{5.594697in}{1.350870in}}%
\pgfpathlineto{\pgfqpoint{5.595633in}{1.277097in}}%
\pgfpathlineto{\pgfqpoint{5.596334in}{1.177284in}}%
\pgfpathlineto{\pgfqpoint{5.597270in}{1.226093in}}%
\pgfpathlineto{\pgfqpoint{5.597971in}{1.368688in}}%
\pgfpathlineto{\pgfqpoint{5.598205in}{1.364430in}}%
\pgfpathlineto{\pgfqpoint{5.599608in}{1.480077in}}%
\pgfpathlineto{\pgfqpoint{5.601245in}{1.372191in}}%
\pgfpathlineto{\pgfqpoint{5.601479in}{1.371379in}}%
\pgfpathlineto{\pgfqpoint{5.601713in}{1.372647in}}%
\pgfpathlineto{\pgfqpoint{5.603584in}{1.271957in}}%
\pgfpathlineto{\pgfqpoint{5.602415in}{1.382051in}}%
\pgfpathlineto{\pgfqpoint{5.604052in}{1.283408in}}%
\pgfpathlineto{\pgfqpoint{5.604286in}{1.295336in}}%
\pgfpathlineto{\pgfqpoint{5.605221in}{1.117491in}}%
\pgfpathlineto{\pgfqpoint{5.605689in}{1.149550in}}%
\pgfpathlineto{\pgfqpoint{5.607326in}{1.429150in}}%
\pgfpathlineto{\pgfqpoint{5.608729in}{1.401913in}}%
\pgfpathlineto{\pgfqpoint{5.609898in}{1.276986in}}%
\pgfpathlineto{\pgfqpoint{5.610366in}{1.301134in}}%
\pgfpathlineto{\pgfqpoint{5.610600in}{1.341763in}}%
\pgfpathlineto{\pgfqpoint{5.610834in}{1.256069in}}%
\pgfpathlineto{\pgfqpoint{5.611067in}{1.275933in}}%
\pgfpathlineto{\pgfqpoint{5.611535in}{1.186534in}}%
\pgfpathlineto{\pgfqpoint{5.612237in}{1.237337in}}%
\pgfpathlineto{\pgfqpoint{5.612471in}{1.249949in}}%
\pgfpathlineto{\pgfqpoint{5.612704in}{1.203365in}}%
\pgfpathlineto{\pgfqpoint{5.613640in}{1.087764in}}%
\pgfpathlineto{\pgfqpoint{5.613874in}{1.142968in}}%
\pgfpathlineto{\pgfqpoint{5.614809in}{1.239146in}}%
\pgfpathlineto{\pgfqpoint{5.615043in}{1.231421in}}%
\pgfpathlineto{\pgfqpoint{5.615277in}{1.183431in}}%
\pgfpathlineto{\pgfqpoint{5.615745in}{1.260558in}}%
\pgfpathlineto{\pgfqpoint{5.615978in}{1.272744in}}%
\pgfpathlineto{\pgfqpoint{5.616680in}{1.401063in}}%
\pgfpathlineto{\pgfqpoint{5.617148in}{1.385707in}}%
\pgfpathlineto{\pgfqpoint{5.618551in}{1.219291in}}%
\pgfpathlineto{\pgfqpoint{5.619019in}{1.287470in}}%
\pgfpathlineto{\pgfqpoint{5.619253in}{1.321347in}}%
\pgfpathlineto{\pgfqpoint{5.619720in}{1.293016in}}%
\pgfpathlineto{\pgfqpoint{5.620890in}{1.140693in}}%
\pgfpathlineto{\pgfqpoint{5.621357in}{1.173985in}}%
\pgfpathlineto{\pgfqpoint{5.621591in}{1.185312in}}%
\pgfpathlineto{\pgfqpoint{5.622059in}{1.172348in}}%
\pgfpathlineto{\pgfqpoint{5.623228in}{1.127394in}}%
\pgfpathlineto{\pgfqpoint{5.625099in}{1.342285in}}%
\pgfpathlineto{\pgfqpoint{5.625333in}{1.334039in}}%
\pgfpathlineto{\pgfqpoint{5.627905in}{1.185888in}}%
\pgfpathlineto{\pgfqpoint{5.628373in}{1.228103in}}%
\pgfpathlineto{\pgfqpoint{5.628607in}{1.259716in}}%
\pgfpathlineto{\pgfqpoint{5.629075in}{1.200772in}}%
\pgfpathlineto{\pgfqpoint{5.629542in}{1.095666in}}%
\pgfpathlineto{\pgfqpoint{5.630244in}{1.111524in}}%
\pgfpathlineto{\pgfqpoint{5.630945in}{1.209315in}}%
\pgfpathlineto{\pgfqpoint{5.631413in}{1.140854in}}%
\pgfpathlineto{\pgfqpoint{5.631881in}{1.114840in}}%
\pgfpathlineto{\pgfqpoint{5.632115in}{1.148985in}}%
\pgfpathlineto{\pgfqpoint{5.632349in}{1.155729in}}%
\pgfpathlineto{\pgfqpoint{5.633518in}{1.252028in}}%
\pgfpathlineto{\pgfqpoint{5.633752in}{1.245458in}}%
\pgfpathlineto{\pgfqpoint{5.633986in}{1.244831in}}%
\pgfpathlineto{\pgfqpoint{5.635857in}{1.084804in}}%
\pgfpathlineto{\pgfqpoint{5.636090in}{1.049583in}}%
\pgfpathlineto{\pgfqpoint{5.636558in}{1.104118in}}%
\pgfpathlineto{\pgfqpoint{5.637727in}{1.181866in}}%
\pgfpathlineto{\pgfqpoint{5.638663in}{1.383463in}}%
\pgfpathlineto{\pgfqpoint{5.639131in}{1.294134in}}%
\pgfpathlineto{\pgfqpoint{5.640300in}{0.954883in}}%
\pgfpathlineto{\pgfqpoint{5.640768in}{1.038400in}}%
\pgfpathlineto{\pgfqpoint{5.641703in}{1.173523in}}%
\pgfpathlineto{\pgfqpoint{5.642405in}{1.161929in}}%
\pgfpathlineto{\pgfqpoint{5.642638in}{1.161207in}}%
\pgfpathlineto{\pgfqpoint{5.642872in}{1.123898in}}%
\pgfpathlineto{\pgfqpoint{5.643340in}{1.210396in}}%
\pgfpathlineto{\pgfqpoint{5.643574in}{1.263891in}}%
\pgfpathlineto{\pgfqpoint{5.644275in}{1.262490in}}%
\pgfpathlineto{\pgfqpoint{5.645679in}{1.115186in}}%
\pgfpathlineto{\pgfqpoint{5.646380in}{1.049054in}}%
\pgfpathlineto{\pgfqpoint{5.646614in}{1.018978in}}%
\pgfpathlineto{\pgfqpoint{5.647082in}{1.062017in}}%
\pgfpathlineto{\pgfqpoint{5.648251in}{1.264610in}}%
\pgfpathlineto{\pgfqpoint{5.648719in}{1.228351in}}%
\pgfpathlineto{\pgfqpoint{5.649654in}{1.044290in}}%
\pgfpathlineto{\pgfqpoint{5.650356in}{1.063481in}}%
\pgfpathlineto{\pgfqpoint{5.650590in}{1.075582in}}%
\pgfpathlineto{\pgfqpoint{5.650824in}{1.036435in}}%
\pgfpathlineto{\pgfqpoint{5.651525in}{0.951161in}}%
\pgfpathlineto{\pgfqpoint{5.651993in}{1.019866in}}%
\pgfpathlineto{\pgfqpoint{5.653162in}{1.196450in}}%
\pgfpathlineto{\pgfqpoint{5.653396in}{1.147526in}}%
\pgfpathlineto{\pgfqpoint{5.654331in}{1.058916in}}%
\pgfpathlineto{\pgfqpoint{5.654565in}{1.103174in}}%
\pgfpathlineto{\pgfqpoint{5.655501in}{1.209163in}}%
\pgfpathlineto{\pgfqpoint{5.655735in}{1.193569in}}%
\pgfpathlineto{\pgfqpoint{5.659710in}{0.948849in}}%
\pgfpathlineto{\pgfqpoint{5.661581in}{1.120481in}}%
\pgfpathlineto{\pgfqpoint{5.662516in}{1.248189in}}%
\pgfpathlineto{\pgfqpoint{5.662750in}{1.190668in}}%
\pgfpathlineto{\pgfqpoint{5.664855in}{0.914349in}}%
\pgfpathlineto{\pgfqpoint{5.665089in}{0.928336in}}%
\pgfpathlineto{\pgfqpoint{5.665323in}{0.932275in}}%
\pgfpathlineto{\pgfqpoint{5.666960in}{1.146149in}}%
\pgfpathlineto{\pgfqpoint{5.667194in}{1.108836in}}%
\pgfpathlineto{\pgfqpoint{5.668363in}{1.005427in}}%
\pgfpathlineto{\pgfqpoint{5.668597in}{1.009969in}}%
\pgfpathlineto{\pgfqpoint{5.668831in}{0.988933in}}%
\pgfpathlineto{\pgfqpoint{5.669298in}{0.994998in}}%
\pgfpathlineto{\pgfqpoint{5.670000in}{1.125548in}}%
\pgfpathlineto{\pgfqpoint{5.670702in}{1.050273in}}%
\pgfpathlineto{\pgfqpoint{5.670935in}{1.048875in}}%
\pgfpathlineto{\pgfqpoint{5.671169in}{1.052568in}}%
\pgfpathlineto{\pgfqpoint{5.672105in}{0.880553in}}%
\pgfpathlineto{\pgfqpoint{5.672806in}{0.961805in}}%
\pgfpathlineto{\pgfqpoint{5.673976in}{1.088624in}}%
\pgfpathlineto{\pgfqpoint{5.674209in}{1.053644in}}%
\pgfpathlineto{\pgfqpoint{5.675613in}{1.159856in}}%
\pgfpathlineto{\pgfqpoint{5.675846in}{1.149627in}}%
\pgfpathlineto{\pgfqpoint{5.676314in}{1.098703in}}%
\pgfpathlineto{\pgfqpoint{5.677951in}{0.886242in}}%
\pgfpathlineto{\pgfqpoint{5.680056in}{0.984132in}}%
\pgfpathlineto{\pgfqpoint{5.680290in}{0.951897in}}%
\pgfpathlineto{\pgfqpoint{5.680758in}{1.016447in}}%
\pgfpathlineto{\pgfqpoint{5.681459in}{1.058149in}}%
\pgfpathlineto{\pgfqpoint{5.681693in}{1.048989in}}%
\pgfpathlineto{\pgfqpoint{5.681927in}{1.016053in}}%
\pgfpathlineto{\pgfqpoint{5.682628in}{1.060287in}}%
\pgfpathlineto{\pgfqpoint{5.684265in}{1.143268in}}%
\pgfpathlineto{\pgfqpoint{5.683096in}{1.053408in}}%
\pgfpathlineto{\pgfqpoint{5.684499in}{1.122748in}}%
\pgfpathlineto{\pgfqpoint{5.684967in}{1.059950in}}%
\pgfpathlineto{\pgfqpoint{5.685201in}{1.061826in}}%
\pgfpathlineto{\pgfqpoint{5.685669in}{0.936996in}}%
\pgfpathlineto{\pgfqpoint{5.686370in}{1.014135in}}%
\pgfpathlineto{\pgfqpoint{5.686604in}{1.035728in}}%
\pgfpathlineto{\pgfqpoint{5.687072in}{1.021663in}}%
\pgfpathlineto{\pgfqpoint{5.687773in}{0.838091in}}%
\pgfpathlineto{\pgfqpoint{5.688475in}{0.894172in}}%
\pgfpathlineto{\pgfqpoint{5.688709in}{0.891234in}}%
\pgfpathlineto{\pgfqpoint{5.689644in}{0.991499in}}%
\pgfpathlineto{\pgfqpoint{5.690580in}{1.277289in}}%
\pgfpathlineto{\pgfqpoint{5.691047in}{1.180564in}}%
\pgfpathlineto{\pgfqpoint{5.693152in}{0.950543in}}%
\pgfpathlineto{\pgfqpoint{5.693386in}{0.914658in}}%
\pgfpathlineto{\pgfqpoint{5.693620in}{0.963353in}}%
\pgfpathlineto{\pgfqpoint{5.694087in}{0.953836in}}%
\pgfpathlineto{\pgfqpoint{5.694321in}{0.971792in}}%
\pgfpathlineto{\pgfqpoint{5.694555in}{0.946418in}}%
\pgfpathlineto{\pgfqpoint{5.695023in}{0.879693in}}%
\pgfpathlineto{\pgfqpoint{5.695257in}{0.920696in}}%
\pgfpathlineto{\pgfqpoint{5.695958in}{1.082576in}}%
\pgfpathlineto{\pgfqpoint{5.696660in}{1.051112in}}%
\pgfpathlineto{\pgfqpoint{5.697128in}{1.089457in}}%
\pgfpathlineto{\pgfqpoint{5.697362in}{1.072009in}}%
\pgfpathlineto{\pgfqpoint{5.698297in}{0.962208in}}%
\pgfpathlineto{\pgfqpoint{5.698765in}{0.973668in}}%
\pgfpathlineto{\pgfqpoint{5.699934in}{1.161321in}}%
\pgfpathlineto{\pgfqpoint{5.700636in}{1.136300in}}%
\pgfpathlineto{\pgfqpoint{5.700869in}{1.082002in}}%
\pgfpathlineto{\pgfqpoint{5.701805in}{1.111810in}}%
\pgfpathlineto{\pgfqpoint{5.704845in}{0.822236in}}%
\pgfpathlineto{\pgfqpoint{5.706716in}{1.193761in}}%
\pgfpathlineto{\pgfqpoint{5.707417in}{1.162087in}}%
\pgfpathlineto{\pgfqpoint{5.707885in}{1.169583in}}%
\pgfpathlineto{\pgfqpoint{5.708119in}{1.158242in}}%
\pgfpathlineto{\pgfqpoint{5.709522in}{0.984324in}}%
\pgfpathlineto{\pgfqpoint{5.710925in}{0.877113in}}%
\pgfpathlineto{\pgfqpoint{5.713498in}{1.120203in}}%
\pgfpathlineto{\pgfqpoint{5.713732in}{1.100254in}}%
\pgfpathlineto{\pgfqpoint{5.715603in}{0.988645in}}%
\pgfpathlineto{\pgfqpoint{5.716304in}{1.148763in}}%
\pgfpathlineto{\pgfqpoint{5.717240in}{1.106398in}}%
\pgfpathlineto{\pgfqpoint{5.718409in}{1.042578in}}%
\pgfpathlineto{\pgfqpoint{5.717941in}{1.124534in}}%
\pgfpathlineto{\pgfqpoint{5.718643in}{1.057475in}}%
\pgfpathlineto{\pgfqpoint{5.719110in}{1.108610in}}%
\pgfpathlineto{\pgfqpoint{5.719578in}{1.058820in}}%
\pgfpathlineto{\pgfqpoint{5.719812in}{1.057620in}}%
\pgfpathlineto{\pgfqpoint{5.721215in}{0.932365in}}%
\pgfpathlineto{\pgfqpoint{5.721683in}{0.900670in}}%
\pgfpathlineto{\pgfqpoint{5.721917in}{0.933679in}}%
\pgfpathlineto{\pgfqpoint{5.724255in}{1.185476in}}%
\pgfpathlineto{\pgfqpoint{5.724489in}{1.164178in}}%
\pgfpathlineto{\pgfqpoint{5.724957in}{1.138981in}}%
\pgfpathlineto{\pgfqpoint{5.725425in}{1.085718in}}%
\pgfpathlineto{\pgfqpoint{5.726126in}{1.115858in}}%
\pgfpathlineto{\pgfqpoint{5.726360in}{1.113114in}}%
\pgfpathlineto{\pgfqpoint{5.726594in}{1.131769in}}%
\pgfpathlineto{\pgfqpoint{5.727296in}{1.105227in}}%
\pgfpathlineto{\pgfqpoint{5.728933in}{0.887803in}}%
\pgfpathlineto{\pgfqpoint{5.729166in}{0.893605in}}%
\pgfpathlineto{\pgfqpoint{5.730570in}{1.013117in}}%
\pgfpathlineto{\pgfqpoint{5.731739in}{1.072615in}}%
\pgfpathlineto{\pgfqpoint{5.732674in}{1.259919in}}%
\pgfpathlineto{\pgfqpoint{5.733376in}{1.205363in}}%
\pgfpathlineto{\pgfqpoint{5.735481in}{0.866907in}}%
\pgfpathlineto{\pgfqpoint{5.738521in}{1.168025in}}%
\pgfpathlineto{\pgfqpoint{5.738755in}{1.129184in}}%
\pgfpathlineto{\pgfqpoint{5.739456in}{1.071245in}}%
\pgfpathlineto{\pgfqpoint{5.739690in}{1.105835in}}%
\pgfpathlineto{\pgfqpoint{5.740625in}{1.155260in}}%
\pgfpathlineto{\pgfqpoint{5.740859in}{1.141228in}}%
\pgfpathlineto{\pgfqpoint{5.741093in}{1.134638in}}%
\pgfpathlineto{\pgfqpoint{5.742263in}{1.007675in}}%
\pgfpathlineto{\pgfqpoint{5.742496in}{1.008470in}}%
\pgfpathlineto{\pgfqpoint{5.742730in}{1.015184in}}%
\pgfpathlineto{\pgfqpoint{5.742964in}{0.960790in}}%
\pgfpathlineto{\pgfqpoint{5.743666in}{1.037573in}}%
\pgfpathlineto{\pgfqpoint{5.745537in}{1.212858in}}%
\pgfpathlineto{\pgfqpoint{5.746706in}{0.926339in}}%
\pgfpathlineto{\pgfqpoint{5.747174in}{0.963719in}}%
\pgfpathlineto{\pgfqpoint{5.748343in}{0.992807in}}%
\pgfpathlineto{\pgfqpoint{5.750214in}{1.234230in}}%
\pgfpathlineto{\pgfqpoint{5.751617in}{0.992692in}}%
\pgfpathlineto{\pgfqpoint{5.752318in}{1.075767in}}%
\pgfpathlineto{\pgfqpoint{5.753722in}{1.219531in}}%
\pgfpathlineto{\pgfqpoint{5.753955in}{1.205568in}}%
\pgfpathlineto{\pgfqpoint{5.755359in}{0.910908in}}%
\pgfpathlineto{\pgfqpoint{5.756294in}{0.940946in}}%
\pgfpathlineto{\pgfqpoint{5.758165in}{1.081827in}}%
\pgfpathlineto{\pgfqpoint{5.758399in}{1.074955in}}%
\pgfpathlineto{\pgfqpoint{5.758867in}{1.070596in}}%
\pgfpathlineto{\pgfqpoint{5.759100in}{1.077840in}}%
\pgfpathlineto{\pgfqpoint{5.760737in}{1.175527in}}%
\pgfpathlineto{\pgfqpoint{5.760971in}{1.136818in}}%
\pgfpathlineto{\pgfqpoint{5.761907in}{1.149320in}}%
\pgfpathlineto{\pgfqpoint{5.762141in}{1.160514in}}%
\pgfpathlineto{\pgfqpoint{5.764245in}{0.891588in}}%
\pgfpathlineto{\pgfqpoint{5.765415in}{1.178105in}}%
\pgfpathlineto{\pgfqpoint{5.766350in}{1.169328in}}%
\pgfpathlineto{\pgfqpoint{5.766584in}{1.169527in}}%
\pgfpathlineto{\pgfqpoint{5.767285in}{1.198332in}}%
\pgfpathlineto{\pgfqpoint{5.769156in}{0.918695in}}%
\pgfpathlineto{\pgfqpoint{5.769390in}{0.921075in}}%
\pgfpathlineto{\pgfqpoint{5.770092in}{1.072467in}}%
\pgfpathlineto{\pgfqpoint{5.770793in}{1.006336in}}%
\pgfpathlineto{\pgfqpoint{5.771495in}{0.956391in}}%
\pgfpathlineto{\pgfqpoint{5.771729in}{0.968740in}}%
\pgfpathlineto{\pgfqpoint{5.773366in}{1.264505in}}%
\pgfpathlineto{\pgfqpoint{5.774301in}{1.197626in}}%
\pgfpathlineto{\pgfqpoint{5.775003in}{1.125268in}}%
\pgfpathlineto{\pgfqpoint{5.776406in}{0.866157in}}%
\pgfpathlineto{\pgfqpoint{5.776874in}{0.898575in}}%
\pgfpathlineto{\pgfqpoint{5.777575in}{0.976100in}}%
\pgfpathlineto{\pgfqpoint{5.779446in}{1.223054in}}%
\pgfpathlineto{\pgfqpoint{5.779680in}{1.214131in}}%
\pgfpathlineto{\pgfqpoint{5.780382in}{1.075363in}}%
\pgfpathlineto{\pgfqpoint{5.780849in}{1.136730in}}%
\pgfpathlineto{\pgfqpoint{5.781317in}{1.123314in}}%
\pgfpathlineto{\pgfqpoint{5.781551in}{1.142348in}}%
\pgfpathlineto{\pgfqpoint{5.782019in}{1.155892in}}%
\pgfpathlineto{\pgfqpoint{5.782252in}{1.138146in}}%
\pgfpathlineto{\pgfqpoint{5.782486in}{1.140021in}}%
\pgfpathlineto{\pgfqpoint{5.784591in}{0.906500in}}%
\pgfpathlineto{\pgfqpoint{5.785293in}{0.987165in}}%
\pgfpathlineto{\pgfqpoint{5.786228in}{1.082249in}}%
\pgfpathlineto{\pgfqpoint{5.786462in}{1.029742in}}%
\pgfpathlineto{\pgfqpoint{5.786696in}{1.005006in}}%
\pgfpathlineto{\pgfqpoint{5.786930in}{1.027514in}}%
\pgfpathlineto{\pgfqpoint{5.788099in}{1.306429in}}%
\pgfpathlineto{\pgfqpoint{5.788567in}{1.217261in}}%
\pgfpathlineto{\pgfqpoint{5.790671in}{0.934472in}}%
\pgfpathlineto{\pgfqpoint{5.791139in}{0.988320in}}%
\pgfpathlineto{\pgfqpoint{5.791607in}{0.978184in}}%
\pgfpathlineto{\pgfqpoint{5.791841in}{1.015407in}}%
\pgfpathlineto{\pgfqpoint{5.792075in}{1.068034in}}%
\pgfpathlineto{\pgfqpoint{5.792776in}{1.025198in}}%
\pgfpathlineto{\pgfqpoint{5.793010in}{1.027842in}}%
\pgfpathlineto{\pgfqpoint{5.794647in}{1.247049in}}%
\pgfpathlineto{\pgfqpoint{5.795349in}{1.199593in}}%
\pgfpathlineto{\pgfqpoint{5.796752in}{1.111687in}}%
\pgfpathlineto{\pgfqpoint{5.797219in}{1.115339in}}%
\pgfpathlineto{\pgfqpoint{5.797687in}{1.053802in}}%
\pgfpathlineto{\pgfqpoint{5.798623in}{0.888119in}}%
\pgfpathlineto{\pgfqpoint{5.799090in}{0.963366in}}%
\pgfpathlineto{\pgfqpoint{5.800727in}{1.264476in}}%
\pgfpathlineto{\pgfqpoint{5.801195in}{1.188308in}}%
\pgfpathlineto{\pgfqpoint{5.801663in}{1.176207in}}%
\pgfpathlineto{\pgfqpoint{5.801897in}{1.200943in}}%
\pgfpathlineto{\pgfqpoint{5.802598in}{1.162925in}}%
\pgfpathlineto{\pgfqpoint{5.803066in}{1.068835in}}%
\pgfpathlineto{\pgfqpoint{5.803767in}{0.952260in}}%
\pgfpathlineto{\pgfqpoint{5.804469in}{0.977378in}}%
\pgfpathlineto{\pgfqpoint{5.806106in}{1.106215in}}%
\pgfpathlineto{\pgfqpoint{5.806340in}{1.091999in}}%
\pgfpathlineto{\pgfqpoint{5.807275in}{1.020965in}}%
\pgfpathlineto{\pgfqpoint{5.807509in}{1.033258in}}%
\pgfpathlineto{\pgfqpoint{5.808445in}{1.200571in}}%
\pgfpathlineto{\pgfqpoint{5.808912in}{1.186903in}}%
\pgfpathlineto{\pgfqpoint{5.809614in}{1.154662in}}%
\pgfpathlineto{\pgfqpoint{5.809848in}{1.190205in}}%
\pgfpathlineto{\pgfqpoint{5.810082in}{1.177058in}}%
\pgfpathlineto{\pgfqpoint{5.810549in}{1.183614in}}%
\pgfpathlineto{\pgfqpoint{5.810783in}{1.201684in}}%
\pgfpathlineto{\pgfqpoint{5.811017in}{1.166732in}}%
\pgfpathlineto{\pgfqpoint{5.812654in}{0.903808in}}%
\pgfpathlineto{\pgfqpoint{5.813356in}{1.006707in}}%
\pgfpathlineto{\pgfqpoint{5.815928in}{1.245446in}}%
\pgfpathlineto{\pgfqpoint{5.816162in}{1.239244in}}%
\pgfpathlineto{\pgfqpoint{5.817331in}{1.086006in}}%
\pgfpathlineto{\pgfqpoint{5.817565in}{1.144711in}}%
\pgfpathlineto{\pgfqpoint{5.818033in}{1.261682in}}%
\pgfpathlineto{\pgfqpoint{5.818734in}{1.193495in}}%
\pgfpathlineto{\pgfqpoint{5.820138in}{0.940526in}}%
\pgfpathlineto{\pgfqpoint{5.823178in}{1.288980in}}%
\pgfpathlineto{\pgfqpoint{5.823646in}{1.272669in}}%
\pgfpathlineto{\pgfqpoint{5.824815in}{1.092006in}}%
\pgfpathlineto{\pgfqpoint{5.825516in}{1.136113in}}%
\pgfpathlineto{\pgfqpoint{5.826920in}{1.052059in}}%
\pgfpathlineto{\pgfqpoint{5.828089in}{1.096633in}}%
\pgfpathlineto{\pgfqpoint{5.828323in}{1.078267in}}%
\pgfpathlineto{\pgfqpoint{5.829258in}{1.000108in}}%
\pgfpathlineto{\pgfqpoint{5.829492in}{1.051299in}}%
\pgfpathlineto{\pgfqpoint{5.832298in}{1.274994in}}%
\pgfpathlineto{\pgfqpoint{5.833701in}{1.203108in}}%
\pgfpathlineto{\pgfqpoint{5.833935in}{1.199028in}}%
\pgfpathlineto{\pgfqpoint{5.834169in}{1.215233in}}%
\pgfpathlineto{\pgfqpoint{5.834871in}{1.211367in}}%
\pgfpathlineto{\pgfqpoint{5.836040in}{1.055244in}}%
\pgfpathlineto{\pgfqpoint{5.836274in}{1.056220in}}%
\pgfpathlineto{\pgfqpoint{5.837443in}{1.097428in}}%
\pgfpathlineto{\pgfqpoint{5.836976in}{1.032835in}}%
\pgfpathlineto{\pgfqpoint{5.837677in}{1.092693in}}%
\pgfpathlineto{\pgfqpoint{5.838145in}{1.040864in}}%
\pgfpathlineto{\pgfqpoint{5.838846in}{1.062573in}}%
\pgfpathlineto{\pgfqpoint{5.841185in}{1.340470in}}%
\pgfpathlineto{\pgfqpoint{5.841653in}{1.298378in}}%
\pgfpathlineto{\pgfqpoint{5.843991in}{1.022747in}}%
\pgfpathlineto{\pgfqpoint{5.844459in}{1.088561in}}%
\pgfpathlineto{\pgfqpoint{5.844927in}{1.006561in}}%
\pgfpathlineto{\pgfqpoint{5.845161in}{1.042027in}}%
\pgfpathlineto{\pgfqpoint{5.845628in}{1.011632in}}%
\pgfpathlineto{\pgfqpoint{5.845862in}{1.059623in}}%
\pgfpathlineto{\pgfqpoint{5.847967in}{1.292039in}}%
\pgfpathlineto{\pgfqpoint{5.848201in}{1.276131in}}%
\pgfpathlineto{\pgfqpoint{5.849136in}{1.109887in}}%
\pgfpathlineto{\pgfqpoint{5.849838in}{1.128834in}}%
\pgfpathlineto{\pgfqpoint{5.850305in}{1.192183in}}%
\pgfpathlineto{\pgfqpoint{5.850773in}{1.160385in}}%
\pgfpathlineto{\pgfqpoint{5.851007in}{1.107645in}}%
\pgfpathlineto{\pgfqpoint{5.851475in}{1.185062in}}%
\pgfpathlineto{\pgfqpoint{5.851709in}{1.180030in}}%
\pgfpathlineto{\pgfqpoint{5.851943in}{1.177591in}}%
\pgfpathlineto{\pgfqpoint{5.852176in}{1.149132in}}%
\pgfpathlineto{\pgfqpoint{5.853112in}{1.153725in}}%
\pgfpathlineto{\pgfqpoint{5.853813in}{1.195262in}}%
\pgfpathlineto{\pgfqpoint{5.854047in}{1.182560in}}%
\pgfpathlineto{\pgfqpoint{5.855217in}{1.077488in}}%
\pgfpathlineto{\pgfqpoint{5.855450in}{1.081889in}}%
\pgfpathlineto{\pgfqpoint{5.856854in}{1.191639in}}%
\pgfpathlineto{\pgfqpoint{5.857087in}{1.186999in}}%
\pgfpathlineto{\pgfqpoint{5.857321in}{1.210427in}}%
\pgfpathlineto{\pgfqpoint{5.857555in}{1.177428in}}%
\pgfpathlineto{\pgfqpoint{5.858257in}{1.027982in}}%
\pgfpathlineto{\pgfqpoint{5.858724in}{1.093897in}}%
\pgfpathlineto{\pgfqpoint{5.859894in}{1.172939in}}%
\pgfpathlineto{\pgfqpoint{5.860128in}{1.171675in}}%
\pgfpathlineto{\pgfqpoint{5.860595in}{1.212245in}}%
\pgfpathlineto{\pgfqpoint{5.860829in}{1.211212in}}%
\pgfpathlineto{\pgfqpoint{5.861531in}{1.326534in}}%
\pgfpathlineto{\pgfqpoint{5.862232in}{1.299627in}}%
\pgfpathlineto{\pgfqpoint{5.862700in}{1.243487in}}%
\pgfpathlineto{\pgfqpoint{5.863635in}{1.033972in}}%
\pgfpathlineto{\pgfqpoint{5.864103in}{1.105014in}}%
\pgfpathlineto{\pgfqpoint{5.864337in}{1.121584in}}%
\pgfpathlineto{\pgfqpoint{5.864805in}{1.095465in}}%
\pgfpathlineto{\pgfqpoint{5.865039in}{1.080519in}}%
\pgfpathlineto{\pgfqpoint{5.865272in}{1.094654in}}%
\pgfpathlineto{\pgfqpoint{5.866208in}{1.330648in}}%
\pgfpathlineto{\pgfqpoint{5.866910in}{1.322814in}}%
\pgfpathlineto{\pgfqpoint{5.867143in}{1.319623in}}%
\pgfpathlineto{\pgfqpoint{5.869014in}{0.995662in}}%
\pgfpathlineto{\pgfqpoint{5.869248in}{0.999217in}}%
\pgfpathlineto{\pgfqpoint{5.871353in}{1.315644in}}%
\pgfpathlineto{\pgfqpoint{5.871821in}{1.247023in}}%
\pgfpathlineto{\pgfqpoint{5.872054in}{1.236826in}}%
\pgfpathlineto{\pgfqpoint{5.872288in}{1.241426in}}%
\pgfpathlineto{\pgfqpoint{5.873458in}{1.317192in}}%
\pgfpathlineto{\pgfqpoint{5.873691in}{1.300653in}}%
\pgfpathlineto{\pgfqpoint{5.875095in}{1.131227in}}%
\pgfpathlineto{\pgfqpoint{5.875796in}{1.153587in}}%
\pgfpathlineto{\pgfqpoint{5.877199in}{1.109526in}}%
\pgfpathlineto{\pgfqpoint{5.876264in}{1.173925in}}%
\pgfpathlineto{\pgfqpoint{5.877433in}{1.116260in}}%
\pgfpathlineto{\pgfqpoint{5.878836in}{1.224453in}}%
\pgfpathlineto{\pgfqpoint{5.881175in}{1.421195in}}%
\pgfpathlineto{\pgfqpoint{5.879538in}{1.206233in}}%
\pgfpathlineto{\pgfqpoint{5.881643in}{1.371748in}}%
\pgfpathlineto{\pgfqpoint{5.883046in}{1.208118in}}%
\pgfpathlineto{\pgfqpoint{5.883280in}{1.208453in}}%
\pgfpathlineto{\pgfqpoint{5.884215in}{1.262104in}}%
\pgfpathlineto{\pgfqpoint{5.884449in}{1.239298in}}%
\pgfpathlineto{\pgfqpoint{5.885618in}{1.117993in}}%
\pgfpathlineto{\pgfqpoint{5.885852in}{1.144709in}}%
\pgfpathlineto{\pgfqpoint{5.887255in}{1.347014in}}%
\pgfpathlineto{\pgfqpoint{5.888658in}{1.166701in}}%
\pgfpathlineto{\pgfqpoint{5.888892in}{1.183874in}}%
\pgfpathlineto{\pgfqpoint{5.890763in}{1.377372in}}%
\pgfpathlineto{\pgfqpoint{5.890997in}{1.367440in}}%
\pgfpathlineto{\pgfqpoint{5.893102in}{1.116435in}}%
\pgfpathlineto{\pgfqpoint{5.891699in}{1.373523in}}%
\pgfpathlineto{\pgfqpoint{5.893336in}{1.195061in}}%
\pgfpathlineto{\pgfqpoint{5.894973in}{1.321706in}}%
\pgfpathlineto{\pgfqpoint{5.896610in}{1.112817in}}%
\pgfpathlineto{\pgfqpoint{5.898481in}{1.429617in}}%
\pgfpathlineto{\pgfqpoint{5.899416in}{1.365453in}}%
\pgfpathlineto{\pgfqpoint{5.900351in}{1.192958in}}%
\pgfpathlineto{\pgfqpoint{5.900819in}{1.238628in}}%
\pgfpathlineto{\pgfqpoint{5.901053in}{1.253026in}}%
\pgfpathlineto{\pgfqpoint{5.901287in}{1.213879in}}%
\pgfpathlineto{\pgfqpoint{5.901988in}{1.182811in}}%
\pgfpathlineto{\pgfqpoint{5.901755in}{1.231674in}}%
\pgfpathlineto{\pgfqpoint{5.902222in}{1.216046in}}%
\pgfpathlineto{\pgfqpoint{5.902690in}{1.305474in}}%
\pgfpathlineto{\pgfqpoint{5.903158in}{1.250939in}}%
\pgfpathlineto{\pgfqpoint{5.903392in}{1.220883in}}%
\pgfpathlineto{\pgfqpoint{5.903625in}{1.277678in}}%
\pgfpathlineto{\pgfqpoint{5.904795in}{1.381224in}}%
\pgfpathlineto{\pgfqpoint{5.905029in}{1.344386in}}%
\pgfpathlineto{\pgfqpoint{5.905496in}{1.237490in}}%
\pgfpathlineto{\pgfqpoint{5.905964in}{1.163253in}}%
\pgfpathlineto{\pgfqpoint{5.906666in}{1.170383in}}%
\pgfpathlineto{\pgfqpoint{5.907835in}{1.241158in}}%
\pgfpathlineto{\pgfqpoint{5.908069in}{1.225949in}}%
\pgfpathlineto{\pgfqpoint{5.909004in}{1.371754in}}%
\pgfpathlineto{\pgfqpoint{5.909238in}{1.397135in}}%
\pgfpathlineto{\pgfqpoint{5.909472in}{1.371163in}}%
\pgfpathlineto{\pgfqpoint{5.909940in}{1.372855in}}%
\pgfpathlineto{\pgfqpoint{5.912044in}{1.181863in}}%
\pgfpathlineto{\pgfqpoint{5.912278in}{1.236120in}}%
\pgfpathlineto{\pgfqpoint{5.912512in}{1.256467in}}%
\pgfpathlineto{\pgfqpoint{5.912980in}{1.200902in}}%
\pgfpathlineto{\pgfqpoint{5.914851in}{1.322465in}}%
\pgfpathlineto{\pgfqpoint{5.915318in}{1.306081in}}%
\pgfpathlineto{\pgfqpoint{5.916020in}{1.198176in}}%
\pgfpathlineto{\pgfqpoint{5.916488in}{1.256083in}}%
\pgfpathlineto{\pgfqpoint{5.916722in}{1.261694in}}%
\pgfpathlineto{\pgfqpoint{5.917891in}{1.152584in}}%
\pgfpathlineto{\pgfqpoint{5.918125in}{1.159811in}}%
\pgfpathlineto{\pgfqpoint{5.920463in}{1.379044in}}%
\pgfpathlineto{\pgfqpoint{5.920697in}{1.335813in}}%
\pgfpathlineto{\pgfqpoint{5.921866in}{1.375245in}}%
\pgfpathlineto{\pgfqpoint{5.922100in}{1.370899in}}%
\pgfpathlineto{\pgfqpoint{5.922334in}{1.370884in}}%
\pgfpathlineto{\pgfqpoint{5.923737in}{1.046702in}}%
\pgfpathlineto{\pgfqpoint{5.924205in}{1.121796in}}%
\pgfpathlineto{\pgfqpoint{5.924673in}{1.140841in}}%
\pgfpathlineto{\pgfqpoint{5.926310in}{1.338653in}}%
\pgfpathlineto{\pgfqpoint{5.926777in}{1.305287in}}%
\pgfpathlineto{\pgfqpoint{5.927245in}{1.240685in}}%
\pgfpathlineto{\pgfqpoint{5.927947in}{1.262468in}}%
\pgfpathlineto{\pgfqpoint{5.928648in}{1.417626in}}%
\pgfpathlineto{\pgfqpoint{5.929116in}{1.336264in}}%
\pgfpathlineto{\pgfqpoint{5.930753in}{1.166077in}}%
\pgfpathlineto{\pgfqpoint{5.930987in}{1.173298in}}%
\pgfpathlineto{\pgfqpoint{5.931221in}{1.170047in}}%
\pgfpathlineto{\pgfqpoint{5.931455in}{1.195942in}}%
\pgfpathlineto{\pgfqpoint{5.931689in}{1.171181in}}%
\pgfpathlineto{\pgfqpoint{5.932624in}{1.035731in}}%
\pgfpathlineto{\pgfqpoint{5.932858in}{1.037520in}}%
\pgfpathlineto{\pgfqpoint{5.934495in}{1.384038in}}%
\pgfpathlineto{\pgfqpoint{5.934729in}{1.337538in}}%
\pgfpathlineto{\pgfqpoint{5.936833in}{1.089763in}}%
\pgfpathlineto{\pgfqpoint{5.937301in}{1.069674in}}%
\pgfpathlineto{\pgfqpoint{5.938237in}{1.284882in}}%
\pgfpathlineto{\pgfqpoint{5.938938in}{1.224843in}}%
\pgfpathlineto{\pgfqpoint{5.939172in}{1.220199in}}%
\pgfpathlineto{\pgfqpoint{5.940107in}{1.291758in}}%
\pgfpathlineto{\pgfqpoint{5.940341in}{1.240331in}}%
\pgfpathlineto{\pgfqpoint{5.941744in}{1.142964in}}%
\pgfpathlineto{\pgfqpoint{5.942212in}{1.077878in}}%
\pgfpathlineto{\pgfqpoint{5.942680in}{1.145017in}}%
\pgfpathlineto{\pgfqpoint{5.942914in}{1.150276in}}%
\pgfpathlineto{\pgfqpoint{5.944551in}{1.055860in}}%
\pgfpathlineto{\pgfqpoint{5.945252in}{1.101586in}}%
\pgfpathlineto{\pgfqpoint{5.947123in}{1.340876in}}%
\pgfpathlineto{\pgfqpoint{5.947591in}{1.245369in}}%
\pgfpathlineto{\pgfqpoint{5.949462in}{0.998757in}}%
\pgfpathlineto{\pgfqpoint{5.951567in}{1.199152in}}%
\pgfpathlineto{\pgfqpoint{5.951800in}{1.160466in}}%
\pgfpathlineto{\pgfqpoint{5.952736in}{1.062797in}}%
\pgfpathlineto{\pgfqpoint{5.952970in}{1.062961in}}%
\pgfpathlineto{\pgfqpoint{5.954841in}{1.161212in}}%
\pgfpathlineto{\pgfqpoint{5.955074in}{1.143914in}}%
\pgfpathlineto{\pgfqpoint{5.955308in}{1.194678in}}%
\pgfpathlineto{\pgfqpoint{5.956244in}{1.228919in}}%
\pgfpathlineto{\pgfqpoint{5.956711in}{1.156544in}}%
\pgfpathlineto{\pgfqpoint{5.957413in}{1.197794in}}%
\pgfpathlineto{\pgfqpoint{5.958582in}{0.937837in}}%
\pgfpathlineto{\pgfqpoint{5.959518in}{1.014753in}}%
\pgfpathlineto{\pgfqpoint{5.959986in}{1.038790in}}%
\pgfpathlineto{\pgfqpoint{5.961389in}{1.316231in}}%
\pgfpathlineto{\pgfqpoint{5.962324in}{1.251184in}}%
\pgfpathlineto{\pgfqpoint{5.964429in}{1.020555in}}%
\pgfpathlineto{\pgfqpoint{5.965130in}{0.904612in}}%
\pgfpathlineto{\pgfqpoint{5.965598in}{0.965908in}}%
\pgfpathlineto{\pgfqpoint{5.966300in}{1.163516in}}%
\pgfpathlineto{\pgfqpoint{5.967001in}{1.109035in}}%
\pgfpathlineto{\pgfqpoint{5.969574in}{1.264094in}}%
\pgfpathlineto{\pgfqpoint{5.969808in}{1.257486in}}%
\pgfpathlineto{\pgfqpoint{5.972614in}{0.924684in}}%
\pgfpathlineto{\pgfqpoint{5.974017in}{1.167157in}}%
\pgfpathlineto{\pgfqpoint{5.974719in}{1.363447in}}%
\pgfpathlineto{\pgfqpoint{5.975186in}{1.217129in}}%
\pgfpathlineto{\pgfqpoint{5.976590in}{1.018706in}}%
\pgfpathlineto{\pgfqpoint{5.977057in}{1.035961in}}%
\pgfpathlineto{\pgfqpoint{5.978227in}{1.130194in}}%
\pgfpathlineto{\pgfqpoint{5.978460in}{1.091145in}}%
\pgfpathlineto{\pgfqpoint{5.978928in}{0.964831in}}%
\pgfpathlineto{\pgfqpoint{5.979630in}{1.012716in}}%
\pgfpathlineto{\pgfqpoint{5.980097in}{1.074331in}}%
\pgfpathlineto{\pgfqpoint{5.981734in}{1.292470in}}%
\pgfpathlineto{\pgfqpoint{5.981968in}{1.291219in}}%
\pgfpathlineto{\pgfqpoint{5.982202in}{1.293322in}}%
\pgfpathlineto{\pgfqpoint{5.983371in}{1.038359in}}%
\pgfpathlineto{\pgfqpoint{5.984541in}{0.952388in}}%
\pgfpathlineto{\pgfqpoint{5.984775in}{0.974359in}}%
\pgfpathlineto{\pgfqpoint{5.986178in}{1.098437in}}%
\pgfpathlineto{\pgfqpoint{5.988049in}{1.257402in}}%
\pgfpathlineto{\pgfqpoint{5.988750in}{1.160377in}}%
\pgfpathlineto{\pgfqpoint{5.989218in}{1.222088in}}%
\pgfpathlineto{\pgfqpoint{5.989452in}{1.251024in}}%
\pgfpathlineto{\pgfqpoint{5.990153in}{1.213436in}}%
\pgfpathlineto{\pgfqpoint{5.992024in}{0.926604in}}%
\pgfpathlineto{\pgfqpoint{5.992258in}{0.957923in}}%
\pgfpathlineto{\pgfqpoint{5.994597in}{1.274573in}}%
\pgfpathlineto{\pgfqpoint{5.995064in}{1.260500in}}%
\pgfpathlineto{\pgfqpoint{5.995298in}{1.275151in}}%
\pgfpathlineto{\pgfqpoint{5.995532in}{1.299595in}}%
\pgfpathlineto{\pgfqpoint{5.995766in}{1.266517in}}%
\pgfpathlineto{\pgfqpoint{5.997871in}{1.034103in}}%
\pgfpathlineto{\pgfqpoint{5.998572in}{1.137837in}}%
\pgfpathlineto{\pgfqpoint{5.999040in}{1.067463in}}%
\pgfpathlineto{\pgfqpoint{6.000677in}{0.994754in}}%
\pgfpathlineto{\pgfqpoint{6.003249in}{1.272171in}}%
\pgfpathlineto{\pgfqpoint{6.003717in}{1.222012in}}%
\pgfpathlineto{\pgfqpoint{6.005588in}{0.995025in}}%
\pgfpathlineto{\pgfqpoint{6.007225in}{1.204782in}}%
\pgfpathlineto{\pgfqpoint{6.006056in}{0.984147in}}%
\pgfpathlineto{\pgfqpoint{6.007927in}{1.132771in}}%
\pgfpathlineto{\pgfqpoint{6.008394in}{1.150193in}}%
\pgfpathlineto{\pgfqpoint{6.009096in}{1.116670in}}%
\pgfpathlineto{\pgfqpoint{6.009564in}{1.145831in}}%
\pgfpathlineto{\pgfqpoint{6.010031in}{1.122598in}}%
\pgfpathlineto{\pgfqpoint{6.011902in}{0.973264in}}%
\pgfpathlineto{\pgfqpoint{6.013072in}{1.197264in}}%
\pgfpathlineto{\pgfqpoint{6.013539in}{1.133997in}}%
\pgfpathlineto{\pgfqpoint{6.013773in}{1.075567in}}%
\pgfpathlineto{\pgfqpoint{6.014475in}{1.092577in}}%
\pgfpathlineto{\pgfqpoint{6.016112in}{1.298843in}}%
\pgfpathlineto{\pgfqpoint{6.016579in}{1.289355in}}%
\pgfpathlineto{\pgfqpoint{6.018684in}{0.936476in}}%
\pgfpathlineto{\pgfqpoint{6.020789in}{1.228177in}}%
\pgfpathlineto{\pgfqpoint{6.021257in}{1.148016in}}%
\pgfpathlineto{\pgfqpoint{6.021958in}{1.205256in}}%
\pgfpathlineto{\pgfqpoint{6.022192in}{1.203200in}}%
\pgfpathlineto{\pgfqpoint{6.023361in}{1.252672in}}%
\pgfpathlineto{\pgfqpoint{6.023595in}{1.256613in}}%
\pgfpathlineto{\pgfqpoint{6.024063in}{1.296152in}}%
\pgfpathlineto{\pgfqpoint{6.024531in}{1.262942in}}%
\pgfpathlineto{\pgfqpoint{6.028506in}{1.041804in}}%
\pgfpathlineto{\pgfqpoint{6.028974in}{1.080118in}}%
\pgfpathlineto{\pgfqpoint{6.030377in}{1.451815in}}%
\pgfpathlineto{\pgfqpoint{6.031079in}{1.346183in}}%
\pgfpathlineto{\pgfqpoint{6.032014in}{1.206718in}}%
\pgfpathlineto{\pgfqpoint{6.032482in}{1.245656in}}%
\pgfpathlineto{\pgfqpoint{6.032716in}{1.279325in}}%
\pgfpathlineto{\pgfqpoint{6.033417in}{1.229234in}}%
\pgfpathlineto{\pgfqpoint{6.033651in}{1.243717in}}%
\pgfpathlineto{\pgfqpoint{6.033885in}{1.229982in}}%
\pgfpathlineto{\pgfqpoint{6.035288in}{1.122214in}}%
\pgfpathlineto{\pgfqpoint{6.035522in}{1.170274in}}%
\pgfpathlineto{\pgfqpoint{6.035756in}{1.119031in}}%
\pgfpathlineto{\pgfqpoint{6.036224in}{1.120271in}}%
\pgfpathlineto{\pgfqpoint{6.036691in}{1.110921in}}%
\pgfpathlineto{\pgfqpoint{6.036925in}{1.117566in}}%
\pgfpathlineto{\pgfqpoint{6.038562in}{1.434923in}}%
\pgfpathlineto{\pgfqpoint{6.039498in}{1.325047in}}%
\pgfpathlineto{\pgfqpoint{6.039965in}{1.318064in}}%
\pgfpathlineto{\pgfqpoint{6.040667in}{1.360421in}}%
\pgfpathlineto{\pgfqpoint{6.040901in}{1.342982in}}%
\pgfpathlineto{\pgfqpoint{6.042538in}{1.120706in}}%
\pgfpathlineto{\pgfqpoint{6.043239in}{1.192588in}}%
\pgfpathlineto{\pgfqpoint{6.043473in}{1.219810in}}%
\pgfpathlineto{\pgfqpoint{6.044175in}{1.177410in}}%
\pgfpathlineto{\pgfqpoint{6.044643in}{1.116004in}}%
\pgfpathlineto{\pgfqpoint{6.045110in}{1.140414in}}%
\pgfpathlineto{\pgfqpoint{6.047215in}{1.364627in}}%
\pgfpathlineto{\pgfqpoint{6.048384in}{1.329436in}}%
\pgfpathlineto{\pgfqpoint{6.049320in}{1.379380in}}%
\pgfpathlineto{\pgfqpoint{6.051191in}{1.161811in}}%
\pgfpathlineto{\pgfqpoint{6.051892in}{1.252272in}}%
\pgfpathlineto{\pgfqpoint{6.052360in}{1.191602in}}%
\pgfpathlineto{\pgfqpoint{6.052594in}{1.147274in}}%
\pgfpathlineto{\pgfqpoint{6.053062in}{1.260496in}}%
\pgfpathlineto{\pgfqpoint{6.053295in}{1.173136in}}%
\pgfpathlineto{\pgfqpoint{6.053529in}{1.181759in}}%
\pgfpathlineto{\pgfqpoint{6.053763in}{1.167621in}}%
\pgfpathlineto{\pgfqpoint{6.054231in}{1.120317in}}%
\pgfpathlineto{\pgfqpoint{6.054699in}{1.180911in}}%
\pgfpathlineto{\pgfqpoint{6.056803in}{1.422850in}}%
\pgfpathlineto{\pgfqpoint{6.057037in}{1.368561in}}%
\pgfpathlineto{\pgfqpoint{6.060311in}{1.049007in}}%
\pgfpathlineto{\pgfqpoint{6.060779in}{1.099620in}}%
\pgfpathlineto{\pgfqpoint{6.062182in}{1.223669in}}%
\pgfpathlineto{\pgfqpoint{6.064754in}{1.383509in}}%
\pgfpathlineto{\pgfqpoint{6.065222in}{1.282555in}}%
\pgfpathlineto{\pgfqpoint{6.067093in}{0.994207in}}%
\pgfpathlineto{\pgfqpoint{6.067561in}{1.069477in}}%
\pgfpathlineto{\pgfqpoint{6.068730in}{1.400743in}}%
\pgfpathlineto{\pgfqpoint{6.068964in}{1.342262in}}%
\pgfpathlineto{\pgfqpoint{6.069432in}{1.250107in}}%
\pgfpathlineto{\pgfqpoint{6.070133in}{1.256903in}}%
\pgfpathlineto{\pgfqpoint{6.070367in}{1.308147in}}%
\pgfpathlineto{\pgfqpoint{6.071069in}{1.282216in}}%
\pgfpathlineto{\pgfqpoint{6.072940in}{1.064599in}}%
\pgfpathlineto{\pgfqpoint{6.073875in}{1.065751in}}%
\pgfpathlineto{\pgfqpoint{6.075044in}{1.222788in}}%
\pgfpathlineto{\pgfqpoint{6.075746in}{1.171885in}}%
\pgfpathlineto{\pgfqpoint{6.075980in}{1.156625in}}%
\pgfpathlineto{\pgfqpoint{6.076214in}{1.192993in}}%
\pgfpathlineto{\pgfqpoint{6.076915in}{1.164934in}}%
\pgfpathlineto{\pgfqpoint{6.077149in}{1.191697in}}%
\pgfpathlineto{\pgfqpoint{6.077617in}{1.141984in}}%
\pgfpathlineto{\pgfqpoint{6.078084in}{1.097869in}}%
\pgfpathlineto{\pgfqpoint{6.078318in}{1.132596in}}%
\pgfpathlineto{\pgfqpoint{6.079721in}{1.272760in}}%
\pgfpathlineto{\pgfqpoint{6.079955in}{1.260688in}}%
\pgfpathlineto{\pgfqpoint{6.080189in}{1.261867in}}%
\pgfpathlineto{\pgfqpoint{6.080657in}{1.307069in}}%
\pgfpathlineto{\pgfqpoint{6.081125in}{1.367500in}}%
\pgfpathlineto{\pgfqpoint{6.081592in}{1.308325in}}%
\pgfpathlineto{\pgfqpoint{6.082528in}{1.069135in}}%
\pgfpathlineto{\pgfqpoint{6.082762in}{0.993992in}}%
\pgfpathlineto{\pgfqpoint{6.083229in}{1.080112in}}%
\pgfpathlineto{\pgfqpoint{6.084633in}{1.265699in}}%
\pgfpathlineto{\pgfqpoint{6.084866in}{1.199756in}}%
\pgfpathlineto{\pgfqpoint{6.086270in}{1.171160in}}%
\pgfpathlineto{\pgfqpoint{6.086503in}{1.170266in}}%
\pgfpathlineto{\pgfqpoint{6.086737in}{1.118843in}}%
\pgfpathlineto{\pgfqpoint{6.087673in}{1.134427in}}%
\pgfpathlineto{\pgfqpoint{6.088374in}{1.216896in}}%
\pgfpathlineto{\pgfqpoint{6.089310in}{1.297355in}}%
\pgfpathlineto{\pgfqpoint{6.089544in}{1.288120in}}%
\pgfpathlineto{\pgfqpoint{6.090245in}{1.172879in}}%
\pgfpathlineto{\pgfqpoint{6.090713in}{1.239691in}}%
\pgfpathlineto{\pgfqpoint{6.090947in}{1.244381in}}%
\pgfpathlineto{\pgfqpoint{6.091414in}{1.313087in}}%
\pgfpathlineto{\pgfqpoint{6.091882in}{1.277651in}}%
\pgfpathlineto{\pgfqpoint{6.093987in}{1.119391in}}%
\pgfpathlineto{\pgfqpoint{6.092584in}{1.278743in}}%
\pgfpathlineto{\pgfqpoint{6.094221in}{1.152565in}}%
\pgfpathlineto{\pgfqpoint{6.094455in}{1.192813in}}%
\pgfpathlineto{\pgfqpoint{6.094922in}{1.135650in}}%
\pgfpathlineto{\pgfqpoint{6.095156in}{1.063986in}}%
\pgfpathlineto{\pgfqpoint{6.096092in}{1.112387in}}%
\pgfpathlineto{\pgfqpoint{6.096325in}{1.061302in}}%
\pgfpathlineto{\pgfqpoint{6.096793in}{1.123022in}}%
\pgfpathlineto{\pgfqpoint{6.097027in}{1.175045in}}%
\pgfpathlineto{\pgfqpoint{6.097729in}{1.102070in}}%
\pgfpathlineto{\pgfqpoint{6.097962in}{1.150375in}}%
\pgfpathlineto{\pgfqpoint{6.099366in}{1.317016in}}%
\pgfpathlineto{\pgfqpoint{6.099600in}{1.265857in}}%
\pgfpathlineto{\pgfqpoint{6.100535in}{1.138594in}}%
\pgfpathlineto{\pgfqpoint{6.101003in}{1.172197in}}%
\pgfpathlineto{\pgfqpoint{6.102640in}{1.288775in}}%
\pgfpathlineto{\pgfqpoint{6.104277in}{1.197980in}}%
\pgfpathlineto{\pgfqpoint{6.104744in}{1.245649in}}%
\pgfpathlineto{\pgfqpoint{6.104978in}{1.220873in}}%
\pgfpathlineto{\pgfqpoint{6.106381in}{1.025098in}}%
\pgfpathlineto{\pgfqpoint{6.106615in}{1.060191in}}%
\pgfpathlineto{\pgfqpoint{6.106849in}{1.056938in}}%
\pgfpathlineto{\pgfqpoint{6.109655in}{1.282334in}}%
\pgfpathlineto{\pgfqpoint{6.110825in}{1.262224in}}%
\pgfpathlineto{\pgfqpoint{6.111526in}{1.157783in}}%
\pgfpathlineto{\pgfqpoint{6.112929in}{1.166448in}}%
\pgfpathlineto{\pgfqpoint{6.113397in}{1.239094in}}%
\pgfpathlineto{\pgfqpoint{6.114099in}{1.190007in}}%
\pgfpathlineto{\pgfqpoint{6.115736in}{0.961319in}}%
\pgfpathlineto{\pgfqpoint{6.116437in}{1.016458in}}%
\pgfpathlineto{\pgfqpoint{6.118308in}{1.227795in}}%
\pgfpathlineto{\pgfqpoint{6.118776in}{1.322564in}}%
\pgfpathlineto{\pgfqpoint{6.119244in}{1.270961in}}%
\pgfpathlineto{\pgfqpoint{6.121115in}{0.979291in}}%
\pgfpathlineto{\pgfqpoint{6.121348in}{1.024913in}}%
\pgfpathlineto{\pgfqpoint{6.121582in}{1.044467in}}%
\pgfpathlineto{\pgfqpoint{6.121816in}{1.030383in}}%
\pgfpathlineto{\pgfqpoint{6.122050in}{0.976137in}}%
\pgfpathlineto{\pgfqpoint{6.122752in}{1.010959in}}%
\pgfpathlineto{\pgfqpoint{6.122985in}{1.026194in}}%
\pgfpathlineto{\pgfqpoint{6.123219in}{0.963279in}}%
\pgfpathlineto{\pgfqpoint{6.123921in}{0.986606in}}%
\pgfpathlineto{\pgfqpoint{6.126259in}{1.206540in}}%
\pgfpathlineto{\pgfqpoint{6.126727in}{1.096010in}}%
\pgfpathlineto{\pgfqpoint{6.127429in}{1.147535in}}%
\pgfpathlineto{\pgfqpoint{6.127896in}{1.197158in}}%
\pgfpathlineto{\pgfqpoint{6.128364in}{1.140770in}}%
\pgfpathlineto{\pgfqpoint{6.130469in}{0.835389in}}%
\pgfpathlineto{\pgfqpoint{6.130937in}{0.947831in}}%
\pgfpathlineto{\pgfqpoint{6.131171in}{0.930432in}}%
\pgfpathlineto{\pgfqpoint{6.131404in}{0.973539in}}%
\pgfpathlineto{\pgfqpoint{6.132574in}{1.195946in}}%
\pgfpathlineto{\pgfqpoint{6.132808in}{1.186995in}}%
\pgfpathlineto{\pgfqpoint{6.135614in}{0.854681in}}%
\pgfpathlineto{\pgfqpoint{6.135848in}{0.854900in}}%
\pgfpathlineto{\pgfqpoint{6.138186in}{1.085776in}}%
\pgfpathlineto{\pgfqpoint{6.138888in}{1.077390in}}%
\pgfpathlineto{\pgfqpoint{6.140057in}{0.964080in}}%
\pgfpathlineto{\pgfqpoint{6.140759in}{0.995114in}}%
\pgfpathlineto{\pgfqpoint{6.141226in}{1.057787in}}%
\pgfpathlineto{\pgfqpoint{6.141694in}{0.995944in}}%
\pgfpathlineto{\pgfqpoint{6.142863in}{0.852014in}}%
\pgfpathlineto{\pgfqpoint{6.143331in}{0.935812in}}%
\pgfpathlineto{\pgfqpoint{6.143799in}{0.970939in}}%
\pgfpathlineto{\pgfqpoint{6.144033in}{0.935700in}}%
\pgfpathlineto{\pgfqpoint{6.144267in}{0.960104in}}%
\pgfpathlineto{\pgfqpoint{6.146371in}{0.851686in}}%
\pgfpathlineto{\pgfqpoint{6.147541in}{0.908921in}}%
\pgfpathlineto{\pgfqpoint{6.148242in}{0.871835in}}%
\pgfpathlineto{\pgfqpoint{6.148476in}{0.906534in}}%
\pgfpathlineto{\pgfqpoint{6.149645in}{1.059432in}}%
\pgfpathlineto{\pgfqpoint{6.150581in}{1.050293in}}%
\pgfpathlineto{\pgfqpoint{6.152218in}{0.853154in}}%
\pgfpathlineto{\pgfqpoint{6.152919in}{0.747048in}}%
\pgfpathlineto{\pgfqpoint{6.153153in}{0.820008in}}%
\pgfpathlineto{\pgfqpoint{6.154790in}{1.016220in}}%
\pgfpathlineto{\pgfqpoint{6.157363in}{0.762817in}}%
\pgfpathlineto{\pgfqpoint{6.158064in}{0.832295in}}%
\pgfpathlineto{\pgfqpoint{6.160169in}{0.982939in}}%
\pgfpathlineto{\pgfqpoint{6.160403in}{0.981998in}}%
\pgfpathlineto{\pgfqpoint{6.161338in}{1.087164in}}%
\pgfpathlineto{\pgfqpoint{6.161572in}{1.037739in}}%
\pgfpathlineto{\pgfqpoint{6.162742in}{0.731829in}}%
\pgfpathlineto{\pgfqpoint{6.162742in}{0.731829in}}%
\pgfusepath{stroke}%
\end{pgfscope}%
\begin{pgfscope}%
\pgfpathrectangle{\pgfqpoint{1.116292in}{0.549691in}}{\pgfqpoint{5.298772in}{4.007027in}}%
\pgfusepath{clip}%
\pgfsetrectcap%
\pgfsetroundjoin%
\pgfsetlinewidth{1.505625pt}%
\definecolor{currentstroke}{rgb}{1.000000,0.498039,0.054902}%
\pgfsetstrokecolor{currentstroke}%
\pgfsetdash{}{0pt}%
\pgfpathmoveto{\pgfqpoint{1.116292in}{3.624783in}}%
\pgfpathlineto{\pgfqpoint{1.118397in}{3.444771in}}%
\pgfpathlineto{\pgfqpoint{1.121437in}{3.356650in}}%
\pgfpathlineto{\pgfqpoint{1.128686in}{3.252557in}}%
\pgfpathlineto{\pgfqpoint{1.133597in}{3.113215in}}%
\pgfpathlineto{\pgfqpoint{1.143887in}{3.012598in}}%
\pgfpathlineto{\pgfqpoint{1.145992in}{2.999113in}}%
\pgfpathlineto{\pgfqpoint{1.151371in}{2.960113in}}%
\pgfpathlineto{\pgfqpoint{1.154411in}{2.934501in}}%
\pgfpathlineto{\pgfqpoint{1.162596in}{2.859118in}}%
\pgfpathlineto{\pgfqpoint{1.165636in}{2.834077in}}%
\pgfpathlineto{\pgfqpoint{1.172886in}{2.761459in}}%
\pgfpathlineto{\pgfqpoint{1.174990in}{2.749640in}}%
\pgfpathlineto{\pgfqpoint{1.178966in}{2.733796in}}%
\pgfpathlineto{\pgfqpoint{1.180603in}{2.726109in}}%
\pgfpathlineto{\pgfqpoint{1.185982in}{2.693186in}}%
\pgfpathlineto{\pgfqpoint{1.187385in}{2.687544in}}%
\pgfpathlineto{\pgfqpoint{1.197207in}{2.635492in}}%
\pgfpathlineto{\pgfqpoint{1.198376in}{2.633276in}}%
\pgfpathlineto{\pgfqpoint{1.202820in}{2.612054in}}%
\pgfpathlineto{\pgfqpoint{1.205860in}{2.602162in}}%
\pgfpathlineto{\pgfqpoint{1.211005in}{2.590825in}}%
\pgfpathlineto{\pgfqpoint{1.214279in}{2.583054in}}%
\pgfpathlineto{\pgfqpoint{1.220359in}{2.563602in}}%
\pgfpathlineto{\pgfqpoint{1.221762in}{2.560190in}}%
\pgfpathlineto{\pgfqpoint{1.232286in}{2.523545in}}%
\pgfpathlineto{\pgfqpoint{1.235794in}{2.517935in}}%
\pgfpathlineto{\pgfqpoint{1.238600in}{2.510068in}}%
\pgfpathlineto{\pgfqpoint{1.241640in}{2.499715in}}%
\pgfpathlineto{\pgfqpoint{1.243745in}{2.495748in}}%
\pgfpathlineto{\pgfqpoint{1.251462in}{2.466444in}}%
\pgfpathlineto{\pgfqpoint{1.252866in}{2.462590in}}%
\pgfpathlineto{\pgfqpoint{1.254503in}{2.450748in}}%
\pgfpathlineto{\pgfqpoint{1.257075in}{2.437042in}}%
\pgfpathlineto{\pgfqpoint{1.259414in}{2.430942in}}%
\pgfpathlineto{\pgfqpoint{1.261285in}{2.424632in}}%
\pgfpathlineto{\pgfqpoint{1.264325in}{2.416524in}}%
\pgfpathlineto{\pgfqpoint{1.267131in}{2.411492in}}%
\pgfpathlineto{\pgfqpoint{1.271107in}{2.397925in}}%
\pgfpathlineto{\pgfqpoint{1.276719in}{2.385010in}}%
\pgfpathlineto{\pgfqpoint{1.279292in}{2.376570in}}%
\pgfpathlineto{\pgfqpoint{1.284904in}{2.368226in}}%
\pgfpathlineto{\pgfqpoint{1.287243in}{2.362060in}}%
\pgfpathlineto{\pgfqpoint{1.291452in}{2.358753in}}%
\pgfpathlineto{\pgfqpoint{1.294493in}{2.353773in}}%
\pgfpathlineto{\pgfqpoint{1.298000in}{2.351782in}}%
\pgfpathlineto{\pgfqpoint{1.301508in}{2.347491in}}%
\pgfpathlineto{\pgfqpoint{1.303847in}{2.346207in}}%
\pgfpathlineto{\pgfqpoint{1.306887in}{2.342211in}}%
\pgfpathlineto{\pgfqpoint{1.310629in}{2.340457in}}%
\pgfpathlineto{\pgfqpoint{1.313669in}{2.337846in}}%
\pgfpathlineto{\pgfqpoint{1.317177in}{2.336247in}}%
\pgfpathlineto{\pgfqpoint{1.319749in}{2.333082in}}%
\pgfpathlineto{\pgfqpoint{1.321854in}{2.331827in}}%
\pgfpathlineto{\pgfqpoint{1.324427in}{2.328408in}}%
\pgfpathlineto{\pgfqpoint{1.327934in}{2.326803in}}%
\pgfpathlineto{\pgfqpoint{1.330507in}{2.325038in}}%
\pgfpathlineto{\pgfqpoint{1.333079in}{2.324007in}}%
\pgfpathlineto{\pgfqpoint{1.335652in}{2.321851in}}%
\pgfpathlineto{\pgfqpoint{1.337757in}{2.320754in}}%
\pgfpathlineto{\pgfqpoint{1.339861in}{2.318783in}}%
\pgfpathlineto{\pgfqpoint{1.342668in}{2.317629in}}%
\pgfpathlineto{\pgfqpoint{1.344772in}{2.316170in}}%
\pgfpathlineto{\pgfqpoint{1.350853in}{2.314116in}}%
\pgfpathlineto{\pgfqpoint{1.355296in}{2.311504in}}%
\pgfpathlineto{\pgfqpoint{1.359739in}{2.309961in}}%
\pgfpathlineto{\pgfqpoint{1.362546in}{2.308240in}}%
\pgfpathlineto{\pgfqpoint{1.367223in}{2.306251in}}%
\pgfpathlineto{\pgfqpoint{1.369094in}{2.305121in}}%
\pgfpathlineto{\pgfqpoint{1.372835in}{2.302167in}}%
\pgfpathlineto{\pgfqpoint{1.377980in}{2.300278in}}%
\pgfpathlineto{\pgfqpoint{1.380787in}{2.297410in}}%
\pgfpathlineto{\pgfqpoint{1.383827in}{2.296202in}}%
\pgfpathlineto{\pgfqpoint{1.386165in}{2.294627in}}%
\pgfpathlineto{\pgfqpoint{1.399262in}{2.291176in}}%
\pgfpathlineto{\pgfqpoint{1.413059in}{2.287763in}}%
\pgfpathlineto{\pgfqpoint{1.416333in}{2.287052in}}%
\pgfpathlineto{\pgfqpoint{1.421712in}{2.285462in}}%
\pgfpathlineto{\pgfqpoint{1.428260in}{2.283984in}}%
\pgfpathlineto{\pgfqpoint{1.439251in}{2.280226in}}%
\pgfpathlineto{\pgfqpoint{1.442759in}{2.278470in}}%
\pgfpathlineto{\pgfqpoint{1.448138in}{2.276844in}}%
\pgfpathlineto{\pgfqpoint{1.458428in}{2.271817in}}%
\pgfpathlineto{\pgfqpoint{1.468952in}{2.266682in}}%
\pgfpathlineto{\pgfqpoint{1.486023in}{2.249816in}}%
\pgfpathlineto{\pgfqpoint{1.488830in}{2.245482in}}%
\pgfpathlineto{\pgfqpoint{1.492338in}{2.243171in}}%
\pgfpathlineto{\pgfqpoint{1.495378in}{2.237667in}}%
\pgfpathlineto{\pgfqpoint{1.497950in}{2.236233in}}%
\pgfpathlineto{\pgfqpoint{1.501692in}{2.232694in}}%
\pgfpathlineto{\pgfqpoint{1.503563in}{2.231293in}}%
\pgfpathlineto{\pgfqpoint{1.506369in}{2.226623in}}%
\pgfpathlineto{\pgfqpoint{1.510111in}{2.224161in}}%
\pgfpathlineto{\pgfqpoint{1.512216in}{2.221035in}}%
\pgfpathlineto{\pgfqpoint{1.515957in}{2.219063in}}%
\pgfpathlineto{\pgfqpoint{1.518530in}{2.216516in}}%
\pgfpathlineto{\pgfqpoint{1.521102in}{2.214902in}}%
\pgfpathlineto{\pgfqpoint{1.523441in}{2.213356in}}%
\pgfpathlineto{\pgfqpoint{1.526247in}{2.212176in}}%
\pgfpathlineto{\pgfqpoint{1.528586in}{2.210026in}}%
\pgfpathlineto{\pgfqpoint{1.532327in}{2.208578in}}%
\pgfpathlineto{\pgfqpoint{1.534900in}{2.205874in}}%
\pgfpathlineto{\pgfqpoint{1.538174in}{2.204249in}}%
\pgfpathlineto{\pgfqpoint{1.540513in}{2.202714in}}%
\pgfpathlineto{\pgfqpoint{1.542851in}{2.201307in}}%
\pgfpathlineto{\pgfqpoint{1.545657in}{2.199399in}}%
\pgfpathlineto{\pgfqpoint{1.548464in}{2.198013in}}%
\pgfpathlineto{\pgfqpoint{1.550802in}{2.196930in}}%
\pgfpathlineto{\pgfqpoint{1.556649in}{2.194938in}}%
\pgfpathlineto{\pgfqpoint{1.558520in}{2.193515in}}%
\pgfpathlineto{\pgfqpoint{1.560157in}{2.192519in}}%
\pgfpathlineto{\pgfqpoint{1.563197in}{2.191212in}}%
\pgfpathlineto{\pgfqpoint{1.565536in}{2.189955in}}%
\pgfpathlineto{\pgfqpoint{1.580503in}{2.184348in}}%
\pgfpathlineto{\pgfqpoint{1.584712in}{2.181559in}}%
\pgfpathlineto{\pgfqpoint{1.591494in}{2.177946in}}%
\pgfpathlineto{\pgfqpoint{1.593832in}{2.177385in}}%
\pgfpathlineto{\pgfqpoint{1.595469in}{2.175899in}}%
\pgfpathlineto{\pgfqpoint{1.597574in}{2.174217in}}%
\pgfpathlineto{\pgfqpoint{1.619323in}{2.162951in}}%
\pgfpathlineto{\pgfqpoint{1.620960in}{2.161351in}}%
\pgfpathlineto{\pgfqpoint{1.623766in}{2.159295in}}%
\pgfpathlineto{\pgfqpoint{1.629613in}{2.156872in}}%
\pgfpathlineto{\pgfqpoint{1.634758in}{2.153796in}}%
\pgfpathlineto{\pgfqpoint{1.637564in}{2.152304in}}%
\pgfpathlineto{\pgfqpoint{1.642475in}{2.148682in}}%
\pgfpathlineto{\pgfqpoint{1.645749in}{2.147327in}}%
\pgfpathlineto{\pgfqpoint{1.663055in}{2.136750in}}%
\pgfpathlineto{\pgfqpoint{1.666329in}{2.134580in}}%
\pgfpathlineto{\pgfqpoint{1.681997in}{2.119514in}}%
\pgfpathlineto{\pgfqpoint{1.684570in}{2.113260in}}%
\pgfpathlineto{\pgfqpoint{1.688312in}{2.109877in}}%
\pgfpathlineto{\pgfqpoint{1.695094in}{2.099052in}}%
\pgfpathlineto{\pgfqpoint{1.697198in}{2.096056in}}%
\pgfpathlineto{\pgfqpoint{1.701875in}{2.085062in}}%
\pgfpathlineto{\pgfqpoint{1.703279in}{2.082802in}}%
\pgfpathlineto{\pgfqpoint{1.707488in}{2.071959in}}%
\pgfpathlineto{\pgfqpoint{1.709125in}{2.070644in}}%
\pgfpathlineto{\pgfqpoint{1.712867in}{2.059323in}}%
\pgfpathlineto{\pgfqpoint{1.715673in}{2.056900in}}%
\pgfpathlineto{\pgfqpoint{1.716609in}{2.053606in}}%
\pgfpathlineto{\pgfqpoint{1.719415in}{2.046566in}}%
\pgfpathlineto{\pgfqpoint{1.721987in}{2.043806in}}%
\pgfpathlineto{\pgfqpoint{1.724092in}{2.038235in}}%
\pgfpathlineto{\pgfqpoint{1.732979in}{2.024459in}}%
\pgfpathlineto{\pgfqpoint{1.735785in}{2.014680in}}%
\pgfpathlineto{\pgfqpoint{1.737890in}{2.013170in}}%
\pgfpathlineto{\pgfqpoint{1.742333in}{2.001073in}}%
\pgfpathlineto{\pgfqpoint{1.744438in}{1.999313in}}%
\pgfpathlineto{\pgfqpoint{1.749349in}{1.990255in}}%
\pgfpathlineto{\pgfqpoint{1.752389in}{1.987230in}}%
\pgfpathlineto{\pgfqpoint{1.759873in}{1.976564in}}%
\pgfpathlineto{\pgfqpoint{1.761743in}{1.974518in}}%
\pgfpathlineto{\pgfqpoint{1.764082in}{1.971949in}}%
\pgfpathlineto{\pgfqpoint{1.766187in}{1.970233in}}%
\pgfpathlineto{\pgfqpoint{1.771799in}{1.963194in}}%
\pgfpathlineto{\pgfqpoint{1.777178in}{1.957020in}}%
\pgfpathlineto{\pgfqpoint{1.779283in}{1.953074in}}%
\pgfpathlineto{\pgfqpoint{1.783960in}{1.949768in}}%
\pgfpathlineto{\pgfqpoint{1.786299in}{1.945540in}}%
\pgfpathlineto{\pgfqpoint{1.792379in}{1.941850in}}%
\pgfpathlineto{\pgfqpoint{1.796355in}{1.939012in}}%
\pgfpathlineto{\pgfqpoint{1.799161in}{1.936666in}}%
\pgfpathlineto{\pgfqpoint{1.801266in}{1.935556in}}%
\pgfpathlineto{\pgfqpoint{1.803137in}{1.934035in}}%
\pgfpathlineto{\pgfqpoint{1.805241in}{1.931181in}}%
\pgfpathlineto{\pgfqpoint{1.825353in}{1.921547in}}%
\pgfpathlineto{\pgfqpoint{1.830966in}{1.919272in}}%
\pgfpathlineto{\pgfqpoint{1.833071in}{1.918622in}}%
\pgfpathlineto{\pgfqpoint{1.837748in}{1.917234in}}%
\pgfpathlineto{\pgfqpoint{1.840788in}{1.916233in}}%
\pgfpathlineto{\pgfqpoint{1.844764in}{1.914709in}}%
\pgfpathlineto{\pgfqpoint{1.848271in}{1.914185in}}%
\pgfpathlineto{\pgfqpoint{1.858795in}{1.911927in}}%
\pgfpathlineto{\pgfqpoint{1.864875in}{1.910991in}}%
\pgfpathlineto{\pgfqpoint{1.908139in}{1.904045in}}%
\pgfpathlineto{\pgfqpoint{1.911180in}{1.903472in}}%
\pgfpathlineto{\pgfqpoint{1.921703in}{1.901897in}}%
\pgfpathlineto{\pgfqpoint{1.927550in}{1.901158in}}%
\pgfpathlineto{\pgfqpoint{1.945791in}{1.899273in}}%
\pgfpathlineto{\pgfqpoint{1.980636in}{1.896084in}}%
\pgfpathlineto{\pgfqpoint{2.045882in}{1.884037in}}%
\pgfpathlineto{\pgfqpoint{2.048455in}{1.883034in}}%
\pgfpathlineto{\pgfqpoint{2.052431in}{1.881820in}}%
\pgfpathlineto{\pgfqpoint{2.055705in}{1.881032in}}%
\pgfpathlineto{\pgfqpoint{2.060849in}{1.879584in}}%
\pgfpathlineto{\pgfqpoint{2.063890in}{1.878030in}}%
\pgfpathlineto{\pgfqpoint{2.074881in}{1.875311in}}%
\pgfpathlineto{\pgfqpoint{2.080728in}{1.873493in}}%
\pgfpathlineto{\pgfqpoint{2.087276in}{1.871771in}}%
\pgfpathlineto{\pgfqpoint{2.113702in}{1.861939in}}%
\pgfpathlineto{\pgfqpoint{2.116274in}{1.860646in}}%
\pgfpathlineto{\pgfqpoint{2.119080in}{1.859520in}}%
\pgfpathlineto{\pgfqpoint{2.122121in}{1.857732in}}%
\pgfpathlineto{\pgfqpoint{2.126798in}{1.856280in}}%
\pgfpathlineto{\pgfqpoint{2.129370in}{1.855048in}}%
\pgfpathlineto{\pgfqpoint{2.133580in}{1.853333in}}%
\pgfpathlineto{\pgfqpoint{2.135451in}{1.852492in}}%
\pgfpathlineto{\pgfqpoint{2.139192in}{1.851142in}}%
\pgfpathlineto{\pgfqpoint{2.141531in}{1.849572in}}%
\pgfpathlineto{\pgfqpoint{2.146910in}{1.847375in}}%
\pgfpathlineto{\pgfqpoint{2.149950in}{1.846404in}}%
\pgfpathlineto{\pgfqpoint{2.152756in}{1.845306in}}%
\pgfpathlineto{\pgfqpoint{2.155095in}{1.844197in}}%
\pgfpathlineto{\pgfqpoint{2.168191in}{1.838036in}}%
\pgfpathlineto{\pgfqpoint{2.171699in}{1.836451in}}%
\pgfpathlineto{\pgfqpoint{2.173804in}{1.835591in}}%
\pgfpathlineto{\pgfqpoint{2.178481in}{1.834175in}}%
\pgfpathlineto{\pgfqpoint{2.182222in}{1.833127in}}%
\pgfpathlineto{\pgfqpoint{2.192278in}{1.831225in}}%
\pgfpathlineto{\pgfqpoint{2.214027in}{1.827507in}}%
\pgfpathlineto{\pgfqpoint{2.223616in}{1.826166in}}%
\pgfpathlineto{\pgfqpoint{2.228994in}{1.825532in}}%
\pgfpathlineto{\pgfqpoint{2.255888in}{1.822173in}}%
\pgfpathlineto{\pgfqpoint{2.261735in}{1.821148in}}%
\pgfpathlineto{\pgfqpoint{2.413275in}{1.805740in}}%
\pgfpathlineto{\pgfqpoint{2.418654in}{1.805081in}}%
\pgfpathlineto{\pgfqpoint{2.431049in}{1.803497in}}%
\pgfpathlineto{\pgfqpoint{2.435492in}{1.803044in}}%
\pgfpathlineto{\pgfqpoint{2.447653in}{1.801398in}}%
\pgfpathlineto{\pgfqpoint{2.453265in}{1.800822in}}%
\pgfpathlineto{\pgfqpoint{2.469168in}{1.798437in}}%
\pgfpathlineto{\pgfqpoint{2.488344in}{1.795342in}}%
\pgfpathlineto{\pgfqpoint{2.501207in}{1.793665in}}%
\pgfpathlineto{\pgfqpoint{2.516174in}{1.792045in}}%
\pgfpathlineto{\pgfqpoint{2.530205in}{1.790573in}}%
\pgfpathlineto{\pgfqpoint{2.550083in}{1.788700in}}%
\pgfpathlineto{\pgfqpoint{2.562945in}{1.787181in}}%
\pgfpathlineto{\pgfqpoint{2.589839in}{1.783732in}}%
\pgfpathlineto{\pgfqpoint{2.612757in}{1.780188in}}%
\pgfpathlineto{\pgfqpoint{2.632402in}{1.773682in}}%
\pgfpathlineto{\pgfqpoint{2.635442in}{1.771454in}}%
\pgfpathlineto{\pgfqpoint{2.641522in}{1.768825in}}%
\pgfpathlineto{\pgfqpoint{2.645030in}{1.765133in}}%
\pgfpathlineto{\pgfqpoint{2.647836in}{1.763733in}}%
\pgfpathlineto{\pgfqpoint{2.653215in}{1.759239in}}%
\pgfpathlineto{\pgfqpoint{2.658828in}{1.756694in}}%
\pgfpathlineto{\pgfqpoint{2.662803in}{1.752985in}}%
\pgfpathlineto{\pgfqpoint{2.667948in}{1.750093in}}%
\pgfpathlineto{\pgfqpoint{2.670755in}{1.746986in}}%
\pgfpathlineto{\pgfqpoint{2.673795in}{1.745627in}}%
\pgfpathlineto{\pgfqpoint{2.676601in}{1.742210in}}%
\pgfpathlineto{\pgfqpoint{2.681044in}{1.740339in}}%
\pgfpathlineto{\pgfqpoint{2.682915in}{1.737759in}}%
\pgfpathlineto{\pgfqpoint{2.708874in}{1.727471in}}%
\pgfpathlineto{\pgfqpoint{2.762661in}{1.720398in}}%
\pgfpathlineto{\pgfqpoint{2.815981in}{1.718575in}}%
\pgfpathlineto{\pgfqpoint{2.846149in}{1.716949in}}%
\pgfpathlineto{\pgfqpoint{2.861116in}{1.715256in}}%
\pgfpathlineto{\pgfqpoint{2.877486in}{1.711832in}}%
\pgfpathlineto{\pgfqpoint{2.880994in}{1.711207in}}%
\pgfpathlineto{\pgfqpoint{2.885437in}{1.709836in}}%
\pgfpathlineto{\pgfqpoint{2.909291in}{1.703548in}}%
\pgfpathlineto{\pgfqpoint{2.912331in}{1.702023in}}%
\pgfpathlineto{\pgfqpoint{2.915137in}{1.701451in}}%
\pgfpathlineto{\pgfqpoint{2.917242in}{1.700186in}}%
\pgfpathlineto{\pgfqpoint{2.919113in}{1.698887in}}%
\pgfpathlineto{\pgfqpoint{2.924492in}{1.697382in}}%
\pgfpathlineto{\pgfqpoint{2.937588in}{1.692544in}}%
\pgfpathlineto{\pgfqpoint{2.943201in}{1.691083in}}%
\pgfpathlineto{\pgfqpoint{2.946475in}{1.690022in}}%
\pgfpathlineto{\pgfqpoint{2.952789in}{1.688472in}}%
\pgfpathlineto{\pgfqpoint{2.961675in}{1.686392in}}%
\pgfpathlineto{\pgfqpoint{2.978747in}{1.683690in}}%
\pgfpathlineto{\pgfqpoint{2.995585in}{1.682414in}}%
\pgfpathlineto{\pgfqpoint{3.049607in}{1.681055in}}%
\pgfpathlineto{\pgfqpoint{3.112281in}{1.679710in}}%
\pgfpathlineto{\pgfqpoint{3.154376in}{1.675054in}}%
\pgfpathlineto{\pgfqpoint{3.158351in}{1.674044in}}%
\pgfpathlineto{\pgfqpoint{3.163964in}{1.672168in}}%
\pgfpathlineto{\pgfqpoint{3.168875in}{1.671210in}}%
\pgfpathlineto{\pgfqpoint{3.170980in}{1.669917in}}%
\pgfpathlineto{\pgfqpoint{3.176124in}{1.668268in}}%
\pgfpathlineto{\pgfqpoint{3.178229in}{1.666848in}}%
\pgfpathlineto{\pgfqpoint{3.180334in}{1.665833in}}%
\pgfpathlineto{\pgfqpoint{3.190390in}{1.662428in}}%
\pgfpathlineto{\pgfqpoint{3.198107in}{1.656649in}}%
\pgfpathlineto{\pgfqpoint{3.201615in}{1.655109in}}%
\pgfpathlineto{\pgfqpoint{3.204655in}{1.653705in}}%
\pgfpathlineto{\pgfqpoint{3.206760in}{1.651659in}}%
\pgfpathlineto{\pgfqpoint{3.229912in}{1.637419in}}%
\pgfpathlineto{\pgfqpoint{3.233420in}{1.635973in}}%
\pgfpathlineto{\pgfqpoint{3.236928in}{1.634540in}}%
\pgfpathlineto{\pgfqpoint{3.248153in}{1.632126in}}%
\pgfpathlineto{\pgfqpoint{3.290949in}{1.628278in}}%
\pgfpathlineto{\pgfqpoint{3.378179in}{1.626541in}}%
\pgfpathlineto{\pgfqpoint{3.392678in}{1.625048in}}%
\pgfpathlineto{\pgfqpoint{3.406242in}{1.623109in}}%
\pgfpathlineto{\pgfqpoint{3.411387in}{1.622136in}}%
\pgfpathlineto{\pgfqpoint{3.414193in}{1.620825in}}%
\pgfpathlineto{\pgfqpoint{3.419806in}{1.619924in}}%
\pgfpathlineto{\pgfqpoint{3.438047in}{1.613561in}}%
\pgfpathlineto{\pgfqpoint{3.441555in}{1.611797in}}%
\pgfpathlineto{\pgfqpoint{3.445063in}{1.610326in}}%
\pgfpathlineto{\pgfqpoint{3.449506in}{1.608126in}}%
\pgfpathlineto{\pgfqpoint{3.451611in}{1.606788in}}%
\pgfpathlineto{\pgfqpoint{3.455586in}{1.604700in}}%
\pgfpathlineto{\pgfqpoint{3.462368in}{1.601385in}}%
\pgfpathlineto{\pgfqpoint{3.467513in}{1.598445in}}%
\pgfpathlineto{\pgfqpoint{3.470319in}{1.597071in}}%
\pgfpathlineto{\pgfqpoint{3.472658in}{1.595820in}}%
\pgfpathlineto{\pgfqpoint{3.480375in}{1.593120in}}%
\pgfpathlineto{\pgfqpoint{3.484351in}{1.591209in}}%
\pgfpathlineto{\pgfqpoint{3.489262in}{1.589896in}}%
\pgfpathlineto{\pgfqpoint{3.491601in}{1.589209in}}%
\pgfpathlineto{\pgfqpoint{3.497447in}{1.587750in}}%
\pgfpathlineto{\pgfqpoint{3.511713in}{1.585395in}}%
\pgfpathlineto{\pgfqpoint{3.529486in}{1.583685in}}%
\pgfpathlineto{\pgfqpoint{3.542816in}{1.582833in}}%
\pgfpathlineto{\pgfqpoint{3.594733in}{1.581829in}}%
\pgfpathlineto{\pgfqpoint{3.645948in}{1.579686in}}%
\pgfpathlineto{\pgfqpoint{3.665124in}{1.576289in}}%
\pgfpathlineto{\pgfqpoint{3.667931in}{1.575040in}}%
\pgfpathlineto{\pgfqpoint{3.673543in}{1.573621in}}%
\pgfpathlineto{\pgfqpoint{3.676116in}{1.572386in}}%
\pgfpathlineto{\pgfqpoint{3.682196in}{1.570195in}}%
\pgfpathlineto{\pgfqpoint{3.722887in}{1.548579in}}%
\pgfpathlineto{\pgfqpoint{3.729669in}{1.547011in}}%
\pgfpathlineto{\pgfqpoint{3.742999in}{1.544751in}}%
\pgfpathlineto{\pgfqpoint{3.761474in}{1.542717in}}%
\pgfpathlineto{\pgfqpoint{3.864372in}{1.539918in}}%
\pgfpathlineto{\pgfqpoint{3.892202in}{1.538517in}}%
\pgfpathlineto{\pgfqpoint{3.906467in}{1.537018in}}%
\pgfpathlineto{\pgfqpoint{3.935232in}{1.531124in}}%
\pgfpathlineto{\pgfqpoint{3.951368in}{1.527029in}}%
\pgfpathlineto{\pgfqpoint{3.959319in}{1.525238in}}%
\pgfpathlineto{\pgfqpoint{3.973819in}{1.523191in}}%
\pgfpathlineto{\pgfqpoint{3.986915in}{1.521905in}}%
\pgfpathlineto{\pgfqpoint{4.046549in}{1.520086in}}%
\pgfpathlineto{\pgfqpoint{4.142899in}{1.518641in}}%
\pgfpathlineto{\pgfqpoint{4.150616in}{1.518242in}}%
\pgfpathlineto{\pgfqpoint{4.168857in}{1.516643in}}%
\pgfpathlineto{\pgfqpoint{4.228959in}{1.508701in}}%
\pgfpathlineto{\pgfqpoint{4.253748in}{1.507204in}}%
\pgfpathlineto{\pgfqpoint{4.260296in}{1.506940in}}%
\pgfpathlineto{\pgfqpoint{4.389854in}{1.505308in}}%
\pgfpathlineto{\pgfqpoint{4.435924in}{1.500279in}}%
\pgfpathlineto{\pgfqpoint{4.456036in}{1.496749in}}%
\pgfpathlineto{\pgfqpoint{4.484567in}{1.493556in}}%
\pgfpathlineto{\pgfqpoint{4.669316in}{1.489312in}}%
\pgfpathlineto{\pgfqpoint{4.697379in}{1.485191in}}%
\pgfpathlineto{\pgfqpoint{4.768940in}{1.482511in}}%
\pgfpathlineto{\pgfqpoint{4.902240in}{1.479623in}}%
\pgfpathlineto{\pgfqpoint{4.920481in}{1.477191in}}%
\pgfpathlineto{\pgfqpoint{4.939657in}{1.474554in}}%
\pgfpathlineto{\pgfqpoint{4.950415in}{1.473309in}}%
\pgfpathlineto{\pgfqpoint{4.985494in}{1.470179in}}%
\pgfpathlineto{\pgfqpoint{5.140543in}{1.466156in}}%
\pgfpathlineto{\pgfqpoint{5.170944in}{1.461515in}}%
\pgfpathlineto{\pgfqpoint{5.173283in}{1.460958in}}%
\pgfpathlineto{\pgfqpoint{5.177726in}{1.459730in}}%
\pgfpathlineto{\pgfqpoint{5.210700in}{1.456786in}}%
\pgfpathlineto{\pgfqpoint{5.232683in}{1.455922in}}%
\pgfpathlineto{\pgfqpoint{5.293954in}{1.452975in}}%
\pgfpathlineto{\pgfqpoint{5.312195in}{1.452332in}}%
\pgfpathlineto{\pgfqpoint{5.449471in}{1.450528in}}%
\pgfpathlineto{\pgfqpoint{5.541845in}{1.446599in}}%
\pgfpathlineto{\pgfqpoint{5.547224in}{1.446144in}}%
\pgfpathlineto{\pgfqpoint{5.566868in}{1.444239in}}%
\pgfpathlineto{\pgfqpoint{5.577392in}{1.442854in}}%
\pgfpathlineto{\pgfqpoint{5.586044in}{1.441490in}}%
\pgfpathlineto{\pgfqpoint{5.593762in}{1.439619in}}%
\pgfpathlineto{\pgfqpoint{5.643340in}{1.436478in}}%
\pgfpathlineto{\pgfqpoint{5.679588in}{1.436013in}}%
\pgfpathlineto{\pgfqpoint{5.823646in}{1.434118in}}%
\pgfpathlineto{\pgfqpoint{6.005588in}{1.427237in}}%
\pgfpathlineto{\pgfqpoint{6.039498in}{1.425944in}}%
\pgfpathlineto{\pgfqpoint{6.075044in}{1.423815in}}%
\pgfpathlineto{\pgfqpoint{6.162742in}{1.421890in}}%
\pgfpathlineto{\pgfqpoint{6.162742in}{1.421890in}}%
\pgfusepath{stroke}%
\end{pgfscope}%
\begin{pgfscope}%
\pgfsetrectcap%
\pgfsetmiterjoin%
\pgfsetlinewidth{0.803000pt}%
\definecolor{currentstroke}{rgb}{0.000000,0.000000,0.000000}%
\pgfsetstrokecolor{currentstroke}%
\pgfsetdash{}{0pt}%
\pgfpathmoveto{\pgfqpoint{1.116292in}{0.549691in}}%
\pgfpathlineto{\pgfqpoint{1.116292in}{4.556718in}}%
\pgfusepath{stroke}%
\end{pgfscope}%
\begin{pgfscope}%
\pgfsetrectcap%
\pgfsetmiterjoin%
\pgfsetlinewidth{0.803000pt}%
\definecolor{currentstroke}{rgb}{0.000000,0.000000,0.000000}%
\pgfsetstrokecolor{currentstroke}%
\pgfsetdash{}{0pt}%
\pgfpathmoveto{\pgfqpoint{6.415064in}{0.549691in}}%
\pgfpathlineto{\pgfqpoint{6.415064in}{4.556718in}}%
\pgfusepath{stroke}%
\end{pgfscope}%
\begin{pgfscope}%
\pgfsetrectcap%
\pgfsetmiterjoin%
\pgfsetlinewidth{0.803000pt}%
\definecolor{currentstroke}{rgb}{0.000000,0.000000,0.000000}%
\pgfsetstrokecolor{currentstroke}%
\pgfsetdash{}{0pt}%
\pgfpathmoveto{\pgfqpoint{1.116292in}{0.549691in}}%
\pgfpathlineto{\pgfqpoint{6.415064in}{0.549691in}}%
\pgfusepath{stroke}%
\end{pgfscope}%
\begin{pgfscope}%
\pgfsetrectcap%
\pgfsetmiterjoin%
\pgfsetlinewidth{0.803000pt}%
\definecolor{currentstroke}{rgb}{0.000000,0.000000,0.000000}%
\pgfsetstrokecolor{currentstroke}%
\pgfsetdash{}{0pt}%
\pgfpathmoveto{\pgfqpoint{1.116292in}{4.556718in}}%
\pgfpathlineto{\pgfqpoint{6.415064in}{4.556718in}}%
\pgfusepath{stroke}%
\end{pgfscope}%
\begin{pgfscope}%
\pgfsetbuttcap%
\pgfsetmiterjoin%
\definecolor{currentfill}{rgb}{1.000000,1.000000,1.000000}%
\pgfsetfillcolor{currentfill}%
\pgfsetlinewidth{1.003750pt}%
\definecolor{currentstroke}{rgb}{0.000000,0.000000,0.000000}%
\pgfsetstrokecolor{currentstroke}%
\pgfsetdash{}{0pt}%
\pgfpathmoveto{\pgfqpoint{5.398955in}{3.923385in}}%
\pgfpathlineto{\pgfqpoint{6.298397in}{3.923385in}}%
\pgfpathquadraticcurveto{\pgfqpoint{6.331731in}{3.923385in}}{\pgfqpoint{6.331731in}{3.956718in}}%
\pgfpathlineto{\pgfqpoint{6.331731in}{4.440051in}}%
\pgfpathquadraticcurveto{\pgfqpoint{6.331731in}{4.473385in}}{\pgfqpoint{6.298397in}{4.473385in}}%
\pgfpathlineto{\pgfqpoint{5.398955in}{4.473385in}}%
\pgfpathquadraticcurveto{\pgfqpoint{5.365622in}{4.473385in}}{\pgfqpoint{5.365622in}{4.440051in}}%
\pgfpathlineto{\pgfqpoint{5.365622in}{3.956718in}}%
\pgfpathquadraticcurveto{\pgfqpoint{5.365622in}{3.923385in}}{\pgfqpoint{5.398955in}{3.923385in}}%
\pgfpathlineto{\pgfqpoint{5.398955in}{3.923385in}}%
\pgfpathclose%
\pgfusepath{stroke,fill}%
\end{pgfscope}%
\begin{pgfscope}%
\pgfsetrectcap%
\pgfsetroundjoin%
\pgfsetlinewidth{1.505625pt}%
\definecolor{currentstroke}{rgb}{0.121569,0.466667,0.705882}%
\pgfsetstrokecolor{currentstroke}%
\pgfsetdash{}{0pt}%
\pgfpathmoveto{\pgfqpoint{5.432288in}{4.340051in}}%
\pgfpathlineto{\pgfqpoint{5.598955in}{4.340051in}}%
\pgfpathlineto{\pgfqpoint{5.765622in}{4.340051in}}%
\pgfusepath{stroke}%
\end{pgfscope}%
\begin{pgfscope}%
\definecolor{textcolor}{rgb}{0.000000,0.000000,0.000000}%
\pgfsetstrokecolor{textcolor}%
\pgfsetfillcolor{textcolor}%
\pgftext[x=5.898955in,y=4.281718in,left,base]{\color{textcolor}\rmfamily\fontsize{12.000000}{14.400000}\selectfont \(\displaystyle ||\mathbf{r}||_2\)}%
\end{pgfscope}%
\begin{pgfscope}%
\pgfsetrectcap%
\pgfsetroundjoin%
\pgfsetlinewidth{1.505625pt}%
\definecolor{currentstroke}{rgb}{1.000000,0.498039,0.054902}%
\pgfsetstrokecolor{currentstroke}%
\pgfsetdash{}{0pt}%
\pgfpathmoveto{\pgfqpoint{5.432288in}{4.090051in}}%
\pgfpathlineto{\pgfqpoint{5.598955in}{4.090051in}}%
\pgfpathlineto{\pgfqpoint{5.765622in}{4.090051in}}%
\pgfusepath{stroke}%
\end{pgfscope}%
\begin{pgfscope}%
\definecolor{textcolor}{rgb}{0.000000,0.000000,0.000000}%
\pgfsetstrokecolor{textcolor}%
\pgfsetfillcolor{textcolor}%
\pgftext[x=5.898955in,y=4.031718in,left,base]{\color{textcolor}\rmfamily\fontsize{12.000000}{14.400000}\selectfont \(\displaystyle ||\mathbf{e}||_A\)}%
\end{pgfscope}%
\end{pgfpicture}%
\makeatother%
\endgroup%
}
	\caption{Error $\mathbf{e}_k = \mathbf{x}_k - \mathbf{x} $ in A-Norm $||\mathbf{e}||_A = \sqrt{(A\mathbf{e}, \mathbf{e})}$ and the residual in 2-Norm $||\mathbf{e}||_2 = \sqrt{(\mathbf{r}, \mathbf{r})}$ against the iteration index $k$ for CG-method}
	\label{fig::CGNorms}
\end{figure}

\end{document}